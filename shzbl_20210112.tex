
% main.tex
% Encoding: UTF-8


% preamble.tex
% Encoding: UTF-8

%“em”是相對长度單位,相当於所用字体中大寫“M”的寬度。
%“ex”是相對长度單位,相当於所用字体中小寫“x”的高度,此高度通常为字体尺寸的一半。

\documentclass[b5paper,twoside,zihao=-4,openany]{ctexbook}

%\usepackage{syntonly}
%	\syntaxonly

\usepackage{ctex}
	\ctexset{
		section = {
			format += \zihao{-4} \heiti \raggedright,
			name = {,.},
			number = \arabic{section},
			beforeskip = 1.0ex plus 0.2ex minus .2ex,
			afterskip = 1.0ex plus 0.2ex minus .2ex,
			aftername = \hspace{0.1em}
		},
		chapter = {
			format += \zihao{-3} \heiti,
			name = {,.},
			number = \arabic{chapter},
			beforeskip = 1.0ex plus 0.2ex minus .2ex,
			afterskip = 1.0ex plus 0.2ex minus .2ex,
			aftername = \hspace{0.1em}
		}
	}

\usepackage{xeCJK}
	\xeCJKsetup{
		PunctStyle = kaiming
	}
	\setCJKmainfont[Path=fonts/]{TH-Sung-TP0.ttf}
	\setCJKsansfont[Path=fonts/]{TH-Sung-TP0.ttf}
	\setCJKmonofont[Path=fonts/]{TH-Sung-TP0.ttf}
	\setCJKfamilyfont{sungtpii}[Path=fonts/]{TH-Sung-TP2.ttf}
	\setCJKfamilyfont{khaaitp}[Path=fonts/]{TH-Khaai-TP0.ttf}
	\setCJKfamilyfont{khaaitpii}[Path=fonts/]{TH-Khaai-TP2.ttf}
	
%	\setCJKmainfont{SimSun}
%	\setCJKsansfont{SimSun}
%	\setCJKmonofont{SimSun}

%\usepackage{fontspec}
%	\setmainfont{SimSun}
%	\setmainfont[Path=fonts/]{TH-Sy-P0}
%	\setsansfont{SimSun}

\usepackage{geometry}
	\geometry{left=2.5cm,right=2.5cm,top=2cm,bottom=2cm}

\usepackage[stable]{footmisc}

\usepackage{graphicx}
	\graphicspath{{pics/}}

\usepackage{xcolor}

\usepackage{tocloft}
	\cftbeforesecskip 0.5ex
	\cftbeforechapskip 0.5ex
	\cftbeforepartskip 1ex
	\cftsecindent 4em
	\cftchapindent 2em
	\cftpartindent 0em

\usepackage{makeidx}
	\makeindex

\usepackage{tcolorbox}
	\tcbuselibrary{skins, vignette, breakable, theorems, fitting}

\usepackage{lineno}

%\nofiles
%\punctstyle{kaiming}
%\raggedright
\pagestyle{headings}
\CJKsetecglue{}%完全禁用汉字与其他内容间的空格
\setlength{\parindent}{0em}%段落中第一行缩进量
\setlength{\parskip}{3ex}%段落间距
\setcounter{secnumdepth}{4}
\setcounter{tocdepth}{4}

\title{傷寒雜病論匯校本
%	\thanks{
%		
%	}
	\footnote{
		唐弘宇按:本論的原名現已不可考,可能本來就沒有名字。至於「张仲景方」、「张仲景辨傷寒」、「仲景全書」、「傷寒卒病論」等名,与「傷寒雜病論」一樣,皆是後人杜撰,沒有優劣之分。我選取「傷寒雜病論」作为書名的原因有二,一是這個名字已被大乑所熟知,二是這個書名概括了本論的兩大部分,即傷寒部分和雜病部分。有很多老師和同學給我提意見,説這個名字不好,應改成XXX。我則認为名字叫什麼不必太過追究,陸淵雷説的好:「其精華,在於諸方之症候用法」。
	}
}

\author{
	张仲景{\hfill}述\\
	王叔和{\hfill}撰次\\
	唐弘宇{\hfill}校訂
%	{\includegraphics[width=3em]{tanghongyu.jpg}}校訂
%\and
%	【編{\hfill}委】
%		\footnote{
%			按貢獻排序。
%		}\\
%	枼建橋\\
%	毛{\hfill}敏\\
%	胡曉帆
%	{\includegraphics[width=3em]{yejianqiao.jpg}}\\
%	{\includegraphics[width=3em]{maomin.jpg}}\\
%	{\includegraphics[width=3em]{huxiaofan.jpg}}
}

\date{\today}

\endinput

% newcommand.tex
% Encoding: UTF-8

%\renewcommand*{\sc}{\scriptsize}
%\renewcommand*{\fn}{\footnote}

\newcommand{\jing}{\hbox{\scalebox{0.6}[1]{纟}\kern-0.3em\scalebox{0.7}[1]{巠}}}%經
\newcommand{\qing}{\hbox{\scalebox{0.5}[1]{车}\kern-0.15em\scalebox{0.65}[1]{巠}}}%輕
\newcommand{\xu}{\hbox{\scalebox{0.6}[1]{纟}\kern-0.3em\scalebox{0.7}[1]{賣}}}%續
\newcommand{\rao}{\hbox{\scalebox{0.6}[1]{纟}\kern-0.3em\scalebox{0.7}[1]{堯}}}%繞

\newcommand{\wuben}{\colorbox{black}{\textcolor{white}{吳}}}
\newcommand{\dengben}{\colorbox{black}{\textcolor{white}{鄧}}}
\newcommand{\zhaoben}{\colorbox{black}{\textcolor{white}{趙}}}
\newcommand{\shenghui}{\colorbox{black}{\textcolor{white}{淳}}}
\newcommand{\qianjin}{\colorbox{black}{\textcolor{white}{千}}}
\newcommand{\yifang}{\colorbox{black}{\textcolor{white}{翼}}}
\newcommand{\chenben}{\colorbox{black}{\textcolor{white}{成}}}
\newcommand{\yuhan}{\colorbox{black}{\textcolor{white}{函}}}
\newcommand{\maijing}{\colorbox{black}{\textcolor{white}{脉}}}
\newcommand{\yixin}{\colorbox{black}{\textcolor{white}{心}}}

%\newcommand{\sungtpii}{\CJKfontspec[Path=fonts/]{TH-Sung-TP2.ttf}}
%\newcommand{\tshynpii}{\CJKfontspec[Path=fonts/]{TH-Tshyn-P2.ttf}}
%\newcommand{\tshynpxvi}{\CJKfontspec[Path=fonts/]{TH-Tshyn-P16.ttf}}
%\newcommand{\khaaitp}{\CJKfontspec[Path=fonts/]{TH-Khaai-TP0.ttf}}
%\newcommand{\khaaitpii}{\CJKfontspec[Path=fonts/]{TH-Khaai-TP2.ttf}}

\newcommand*{\khaai}{\CJKfamily{khaai}}
\newcommand*{\khaaiii}{\CJKfamily{khaaiii}}
\newcommand*{\sungii}{\CJKfamily{sungii}}

\endinput

\begin{document}

% frontmatter.tex
% Encoding: UTF-8

\frontmatter
\maketitle
\tableofcontents

\chapter{凡例}

\begin{itemize}
	
	\item 《傷寒雜病論匯校本》分为三個部分:第一部分,類聚方部分;第二部分,傷寒部分;第三部分,雜病部分。\\
	《類聚方》一書,是日本漢方醫學古方派代表人物吉益東洞的著作,此書分解《傷寒論》、《金匱要略》條文,以方剂为類目,匯集仲景相関論述,着意凸顯其「方証相對」之學術主張。本書的第一部分,即是以吉益東洞《類聚方》的形式,重新排列第二、第三部分的條文。方剂的先後次序基本按照《類聚方》,《類聚方》中缺失的方剂,則按照其在《傷寒雜病論》中出現的次序,排在最後。\\
	第一部分中的所有條文,与第二、第三部分的條文完全相同,但條文的註釋和校改記錄,僅在第二、第三部分有,第一部分不再出現。\\
	第二、第三部分僅有條文,无方剂,所有方剂僅出現在第一部分,所以對方剂的註釋,也僅出現在第一部分。
	
	\item 本書傷寒論部分所用的底本,是臺北故宮博物院文獻大樓藏明趙開美翻刻北宋元祐三年小字本《傷寒論》,雜病論部分所用的底本,是明吳遷钞本《金匱要略方》。本書所用的主校本有《千金翼方》、《金匱玉函經》、《脉經》。参校本有《千金要方》、《外臺祕要》、《聖惠方》、《醫心方》等。不用《康平本》、《康治本》、《桂林古本》。
	
	\item 主要參考書目:
		\begin{itemize}
			\item 《影印孫思邈本傷寒論校注考證》
			\item 《宋本傷寒論文獻史論》
			\item 《影印南朝祕本敦煌祕卷傷寒論校注考證》
			\item 《影印金匱玉函經校注考證》
			\item 《校勘元本影印明本金匱要略集》
			\item 《明洪武鈔本金匱要略方》
		\end{itemize}

	\item 古人校訂書籍,不修改原文,只在疑似有錯処出校注,這樣做是为了保存文獻原貌,防止由於自己理解錯誤而妄改原文。這些謹慎的作法,都是因为古代印刷技術、保存手段落後等原因,而採取的無奈之舉。在現代,保存古籍的任務有政府建立圖書館負責,而且以現代的保存手段,不太可能再有古籍消失。所以,我在校訂此書時,不考慮保存原貌的問題,直接修改原文,必要時出校記,這樣做是为了使學習、閲讀更加方便。
	
	\item 規範字形,將古書中的異體字、通假字,一律改为正字,部分字使用簡化字。請注意,「正字」不是臺灣正體字,「簡化字」也不是現行的大陸簡體字,宋本書籍中已有使用簡化字的先例。
	
	\item 《太平聖惠方》卷二論合和曰:「古方藥味,多以銖兩,及用水,皆言升數。年代綿歷浸遠,傳寫轉見乖訛。」(待補充)。大柴胡湯、柴胡桂枝乾薑湯在《傷寒論》和《金匱要略》中重出,其中柴胡的份量,趙本《傷寒論》作「半斤」,吳本《金匱要略》作「八兩」。为盡量統一全書單位,書中所有的「斤」單位,全部依此例轉換为「兩」單位,「一斤」合「十六兩」。
	
	\item 相比宋本系統,本書在行文上更簡潔,風格更接近《千金翼方》和《金匱玉函經》。舉一個例子,趙本第282條中的「下焦虗有寒」,《千金翼》作「下焦虗寒」,《聖惠方》作「下焦有虗寒」,它們表達的意思相同,但《千金翼》更簡潔,所以我選取了《千金翼》的説法。
	
	\item 
	
	\item 
	
	\item 
	
\end{itemize}























\endinput

\mainmatter

\part{類聚方}

\section{桂枝湯}

桂枝{\scriptsize 三兩} 芍藥{\scriptsize 三兩} 生薑{\scriptsize 三兩} 甘草{\scriptsize 二兩炙} 大棗{\scriptsize 十二枚擘}\\
右五味。㕮咀三味。以水七升。微火煮取三升。去滓。温服一升。須臾。飲熱粥一升餘。以助藥力。温覆令一时許。遍身漐漐微似有汗者益佳。不可令如水流漓。病必不除。若一服汗出病差。停後服。不必{\sungtpii 𥁞}剂。若不汗。再服如前。又不汗。後服小促其間。令半日許三服{\sungtpii 𥁞}。病重者。一日一夜服。晬时觀之。服一剂{\sungtpii 𥁞}。病証猶在者。復作服。若汗不出者。服之二三剂乃解。{\khaaitp 禁生冷。粘滑。肉面。五辛。酒酪。臭惡等物。}

太陽中風。{\khaaitp 脉}陽浮而陰弱。陽浮者熱自发。陰弱者汗自出。嗇嗇惡寒。淅淅惡風。翕翕发熱。鼻鳴。乾嘔。桂枝湯主之。12

太陽病。发熱。汗出。此为榮弱衛强。故使汗出。欲救邪風。宜桂枝湯。95

太陽病。頭痛。发熱。汗出。惡風。桂枝湯主之。13

太陽病。下之。其气上衝者。可与桂枝湯。不衝者。不可与之。15

太陽病三日。已发汗吐下温針而不解。此为壞病。桂枝湯不復中与也。觀其脉証。知犯何逆。隨証治之。16

桂枝湯本为解肌。若其人脉浮緊。发熱。无汗。不可与也。常須識此。勿令誤也。16

酒客不可与桂枝湯。得之則嘔。以酒客不喜甘故也。17

服桂枝湯吐者。其後必吐膿血。19

太陽病。初服桂枝湯。反煩不解者。当先刺風池風府。卻与桂枝湯即愈。24

服桂枝湯。大汗出。脉洪大者。与桂枝湯。若形如瘧。一日再发者。汗出便解。宜桂枝二麻黄一湯。25

太陽病。外証未解。其脉浮弱。当以汗解。宜桂枝湯。42

太陽病。外証未解者。不可下。下之为逆。欲解外者。宜桂枝湯。44

太陽病。先发汗不解而下之。其脉浮者不愈。浮为在外。而反下之。故令不愈。今脉浮。故在外。当解其外則愈。宜桂枝湯。45

病常自汗出者。此为榮气和。衛气不和也。榮行脉中。衛行脉外。復发其汗。衛和則愈。宜桂枝湯。53

病人臓无他病。时发熱。自汗出。而不愈者。此衛气不和也。先其时发汗則愈。宜桂枝湯。54

傷寒。不大便六七日。頭痛。有熱者。与承气湯。其小便清者。此为不在裏。續在表也。当发其汗。頭痛者必衄。宜桂枝湯。56

傷寒。发汗已解。半日許復煩。脉浮數者。可復发汗。宜桂枝湯。57

傷寒。醫下之。續得下利。清穀不止。身体疼痛。急当救裏。後身体疼痛。清便自調。急当救表。救裏宜四逆湯。救表宜桂枝湯。91

傷寒。大下後。復发汗。心下痞。惡寒者。表未解也。不可攻痞。当先解表。表解乃可攻痞。解表宜桂枝湯。攻痞宜大黄黄連瀉心湯。164

陽明病。脉遲。雖汗出。不惡寒。其身必重。短气。腹滿而喘。有潮熱。如此者。其外为解。可攻其裏。若手足濈然汗出者。此大便已堅。{\khaaitp 大}承气湯主之。若汗多。微发熱。惡寒者。为外未解。{\khaaitp 桂枝湯主之。}其熱不潮。未可与承气湯。若腹大滿。而不大便者。可与小承气湯。微和其胃气。勿令至大下。208

陽明病。脉遲。汗出多。微惡寒者。表未解也。可发汗。宜桂枝湯。234

病者煩熱。汗出即解。復如瘧狀。日晡所发者。屬陽明。脉実者。当下之。脉浮虗者。当发其汗。下之宜{\khaaitp 大}承气湯。发汗宜桂枝湯。240

太陰病。脉浮者。可发汗。宜桂枝湯。276

下利。腹{\khaaitp 胀}滿。身体疼痛者。先温其裏。乃攻其表。温裏宜四逆湯。攻表宜桂枝湯。372

吐利止。而身痛不休者。当消息和解其外。宜桂枝湯小和之。387

師曰。脉婦人得平脉。陰脉小弱。其人渴。不能食。无寒熱。名为軀。桂枝湯主之。法六十日当有娠。設有醫治逆者。卻一月。加吐下者。則絕之。{\wuben}

師曰。婦人得平脉。陰脉小弱。其人渴。不能食。无寒熱。名妊娠。桂枝湯主之。於法六十日当有此証。設有醫治逆者。卻一月。加吐下者。則絕之。{\dengben}

婦人產得風。續之數十日不解。頭微痛。惡寒。时时有熱。心下堅。乾嘔。汗出。雖久。陽旦証續在耳。可与陽旦湯。{\wuben}

產後風。續之數十日不解。頭微痛。惡寒。时时有熱。心下悶。乾嘔。汗出。雖久。陽旦証續在耳。可与陽旦湯。{\dengben}

\section{桂枝加桂湯}

桂枝{\scriptsize 五兩} 芍藥{\scriptsize 三兩} 生薑{\scriptsize 三兩切} 甘草{\scriptsize 二兩炙} 大棗{\scriptsize 十二枚擘}\\
右五味。以水七升。煮取三升。去滓。温服一升。本云桂枝湯。今加桂滿五兩。所以加桂者,以能瀉奔豚气也。

燒針令其汗。針処被寒。核起而赤者。必发奔豚。气從少腹上衝心者。灸其核上各一壯。与桂枝加桂湯。117

\section{桂枝加芍藥湯}

桂枝{\scriptsize 三兩} 芍藥{\scriptsize 六兩} 甘草{\scriptsize 二兩炙} 大棗{\scriptsize 十二枚擘} 生薑{\scriptsize 三兩切}\\
右五味。以水七升。煮取三升。去滓。分温三服。本云桂枝湯。今加芍藥。

{\khaaitp 本}太陽病。醫反下之。因尔腹滿时痛者。屬太陰。桂枝加芍藥湯主之。大実痛者。桂枝加大黄湯主之。279

\section{桂枝去芍藥湯}

桂枝{\scriptsize 三兩} 甘草{\scriptsize 二兩炙} 生薑{\scriptsize 三兩切} 大棗{\scriptsize 十二枚擘}\\
右四味。以水七升。煮取三升。去滓。温服一升。本云桂枝湯。今去芍藥。

太陽病。下之。脉促。胸滿者。桂枝去芍藥湯主之。若微{\khaaitp 惡}寒者。桂枝去芍藥加附子湯主之。{\scriptsize 促。一作縱。}21.22

\section{桂枝加葛根湯}

葛根{\scriptsize 四兩} 芍藥{\scriptsize 二兩} 生薑{\scriptsize 三兩切} 甘草{\scriptsize 二兩炙} 大棗{\scriptsize 十二枚擘} 桂枝{\scriptsize 二兩去皮}\\
右七味。以水一斗。先煮麻黄。葛根。減二升。去上沫。内諸藥。煮取三升。去滓。温服一升。覆取微似汗。不需啜粥。餘如桂枝湯法將息及禁忌。{\scriptsize 臣億等謹按。仲景本論。太陽中風自汗用桂枝。傷寒无汗用麻黄。今証云汗出惡風。而方中有麻黄。恐非本意也。第三卷有葛根湯証云无汗惡風。正与此方同是合用麻黄也。此云桂枝加葛根湯。恐是桂枝中但加葛根耳。}

太陽病。項背强几几。反汗出。惡風。桂枝{\khaaitp 加葛根}湯主之。14

\section{栝蔞桂枝湯}

栝蔞根{\scriptsize 二兩} 桂枝{\scriptsize 三兩去皮} 芍藥{\scriptsize 三兩} 甘草{\scriptsize 二兩炙} 生薑{\scriptsize 三兩切} 大棗{\scriptsize 十二枚擘}\\
右六味。㕮咀。以水九升。煮取三升。去滓。分温三服。取微汗。汗不出。食頃。啜熱粥发之。

太陽病。其証備。身体强。几几然。脉反沈遲。此为痙。栝蔞桂枝湯主之。

\section{桂枝加黄耆湯}

桂枝{\scriptsize 三兩去皮} 生薑{\scriptsize 三兩} 芍藥{\scriptsize 三兩} 甘草{\scriptsize 二兩炙} 大棗{\scriptsize 十二枚擘} 黄耆{\scriptsize 二兩}\\
右六味。㕮咀。以水八升。煮取三升。去滓。温服一升。須臾。飲熱稀粥一升餘。以助藥力。温覆取微汗。若不汗者更服。

黄汗之病。兩脛自冷。假令发熱。此屬歷節。食已汗出。又身常暮{\khaaitp 卧}盜汗出者。此勞气也。若汗出已。反发熱者。久久其身必甲錯。发熱不止者。必生惡瘡。若身重。汗出已輒輕者。久久必身瞤。即胸中痛。又從腰以上必汗出。下无汗。腰髖弛痛。如有物在皮中狀。劇者不能食。身疼重。煩躁。小便不利。此为黄汗。桂枝加黄耆湯主之。

諸病黄家。但利其小便。假令脉浮。当以汗解之。宜桂枝加黄耆湯主之。

\section{桂枝加大黄湯}

桂枝{\scriptsize 三兩去皮} 大黄{\scriptsize 二兩} 芍藥{\scriptsize 六兩} 生薑{\scriptsize 三兩切} 甘草{\scriptsize 二兩炙} 大棗{\scriptsize 十二枚擘}\\
右六味。以水七升。煮取三升。去滓。温服一升。日三服。

{\khaaitp 本}太陽病。醫反下之。因尔腹滿时痛者。屬太陰。桂枝加芍藥湯主之。大実痛者。桂枝加大黄湯主之。279

\section{桂枝加芍藥生薑人参湯}

桂枝{\scriptsize 三兩去皮} 芍藥{\scriptsize 四兩} 甘草{\scriptsize 二兩炙} 人参{\scriptsize 三兩} 大棗{\scriptsize 十二枚擘} 生薑{\scriptsize 四兩切}\\
右六味。以水一斗二升。煮取三升。去滓。温服一升。本云桂枝湯。今加芍藥。生薑。人参。

发汗後。身体疼痛。其脉沈遲。桂枝加芍藥生薑人参湯主之。62

\section{桂枝加厚朴杏仁湯}

喘家作桂枝湯。加厚朴杏仁佳。18

太陽病。下之。微喘者。表未解也。桂枝{\khaaitp 加厚朴杏仁}湯主之。43

\section{烏頭桂枝湯}

烏頭{\scriptsize 五枚実者去角}\\
右一味。以蜜二斤。煎減半。去滓。以桂枝湯五合解之。令得一升許。初服二合。不知。即服三合。又不知。復更加至五合。其知者如醉狀。得吐者为中病。

寒疝。腹中痛。逆冷。手足不仁。若身疼痛。灸刺諸藥不能治。烏頭桂枝湯主之。

\section{桂枝加附子湯}

桂枝{\scriptsize 三兩} 芍藥{\scriptsize 三兩} 甘草{\scriptsize 三兩炙} 生薑{\scriptsize 三兩切} 大棗{\scriptsize 十二枚擘} 附子{\scriptsize 一枚炮去皮破八片}\\
右六味。以水七升。煮取三升。去滓。温服一升。本云桂枝湯。今加附子。

太陽病。发汗。遂漏不止。其人惡風。小便難。四肢微急。難以屈伸。桂枝加附子湯主之。20

\section{桂枝去芍藥加附子湯}

桂枝{\scriptsize 三兩} 甘草{\scriptsize 二兩炙} 生薑{\scriptsize 三兩切} 大棗{\scriptsize 十二枚擘} 附子{\scriptsize 一枚炮去皮破八片}\\
右五味。以水七升。煮取三升。去滓。温服一升。本云桂枝湯。今去芍藥。加附子。

太陽病。下之。脉促。胸滿者。桂枝去芍藥湯主之。若微{\khaaitp 惡}寒者。桂枝去芍藥加附子湯主之。{\scriptsize 促。一作縱。}21.22

\section{桂枝附子湯}

桂枝{\scriptsize 四兩去皮} 附子{\scriptsize 三枚炮去皮破} 生薑{\scriptsize 三兩切} 大棗{\scriptsize 十二枚擘} 甘草{\scriptsize 二兩炙}\\
右五味。以水六升。煮取二升。去滓。分温三服。

傷寒八九日。風濕相摶。身体疼煩。不能自轉側。不嘔。不渴。脉浮虗而濇者。桂枝附子湯主之。若其人大便堅。小便自利者。术附子湯主之。174

\section{术附子湯}

附子{\scriptsize 三枚炮去皮破} 白术{\scriptsize 四兩} 生薑{\scriptsize 三兩切} 甘草{\scriptsize 二兩炙} 大棗{\scriptsize 十二枚擘}\\
右五味。以水六升。煮取二升。去滓。分温三服。\\
初一服。其人身如痹。半日許。復服之。三服都{\sungtpii 𥁞}。其人如冒狀。勿怪。此以附子术并走皮内。逐水气。未得除。故使之耳。法当加桂四兩。此本一方二法。以大便堅。小便自利。故去桂也。以大便不堅。小便不利。当加桂。附子三枚恐多也。虗弱家及產婦宜減服之。

傷寒八九日。風濕相摶。身体疼煩。不能自轉側。不嘔。不渴。脉浮虗而濇者。桂枝附子湯主之。若其人大便堅。小便自利者。术附子湯主之。174

术附子湯。治風虗。頭重眩。苦極。不知食味。暖肌補中。益精气。

\section{甘草附子湯}

甘草{\scriptsize 二兩炙} 附子{\scriptsize 二枚炮} 白术{\scriptsize 三兩} 桂枝{\scriptsize 四兩}\\ 
右四味。以水六升。煮取三升。去滓。温服一升。日三服。初服得微汗即止。能食。汗止復煩者。將服五合。恐一升多者。後服六七合愈。

風濕相摶。骨節疼煩。掣痛。不得屈伸。近之則痛劇。汗出。短气。小便不利。惡風。不欲去衣。或身微腫。甘草附子湯主之。175

\section{桂枝去桂加茯苓术湯}

芍藥{\scriptsize 三兩} 甘草{\scriptsize 二兩炙} 生薑{\scriptsize 三兩切} 白术{\scriptsize 三兩} 茯苓{\scriptsize 三兩} 大棗{\scriptsize 十二枚}\\
右六味。以水八升。煮取三升。去滓。温服一升。小便利則愈。本云桂枝湯。今去桂枝。加茯苓。白术。

服桂枝湯。{\khaaitp 或}下之。仍頭項强痛。翕翕发熱。无汗。心下滿。微痛。小便不利。桂枝去桂加茯苓术湯主之。28

\section{桂枝去芍藥加麻黄細辛附子湯}

桂枝{\scriptsize 三兩去皮} 生薑{\scriptsize 三兩切} 甘草{\scriptsize 二兩炙} 大棗{\scriptsize 十二枚擘} 麻黄{\scriptsize 二兩去節} 細辛{\scriptsize 二兩} 附子{\scriptsize 一枚炮去皮破八片}\\
右七味。㕮咀。以水七升。先煮麻黄再沸。去上沫。内諸藥。煮取二升。去滓。分温三服。当汗出。如虫行皮中即愈。

气分。心下堅。大如盤。邊如旋杯。水飲所作。桂枝去芍藥加麻黄細辛附子湯主之。

\section{桂枝去芍藥加皂莢湯}

桂枝{\scriptsize 三兩去皮} 生薑{\scriptsize 三兩切} 甘草{\scriptsize 二兩炙} 大棗{\scriptsize 十二枚擘} 皂莢{\scriptsize 一枚去皮子炙焦}\\
右五味。㕮咀。以水七升。微微火煮取三升。去滓。分温三服。

肺痿。吐涎沫。桂枝去芍藥加皂莢湯主之。

\section{桂枝加龙骨牡蛎湯}

桂枝{\scriptsize 三兩去皮} 芍藥{\scriptsize 三兩} 生薑{\scriptsize 三兩切} 甘草{\scriptsize 二兩炙} 大棗{\scriptsize 十二枚擘} 龍骨{\scriptsize 二兩熬} 牡蛎{\scriptsize 二兩熬}\\
右七味。㕮咀。以水七升。煮取三升。去滓。分温三服。

夫失精家。少腹弦急。陰頭寒。目眩。髮落。脉極虗芤遲。为清穀。亡血。失精。脉得諸芤動微緊。男子失精。女子夢交。桂枝加龙骨牡蛎湯主之。天雄散亦主之。

\section{桂枝去芍藥加蜀漆牡蛎龙骨救逆湯}

桂枝{\scriptsize 三兩去皮} 甘草{\scriptsize 二兩炙} 生薑{\scriptsize 三兩切} 大棗{\scriptsize 十二枚擘} 牡蛎{\scriptsize 五兩熬} 蜀漆{\scriptsize 三兩洗去腥} 龙骨{\scriptsize 四兩}\\
右七味。以水一斗二升。先煮蜀漆。減二升。内諸藥。煮取三升。去滓。温服一升。本云桂枝湯。今去芍藥。加蜀漆牡蛎龙骨。

傷寒。脉浮。醫以火迫劫之。亡陽。{\khaaitp 必}驚狂。卧起不安。桂枝去芍藥加蜀漆牡蛎龙骨救逆湯主之。112

火邪者。桂枝去芍藥加蜀漆牡蛎龙骨救逆湯主之。

\section{桂枝甘草龙骨牡蛎湯}

桂枝{\scriptsize 一兩去皮} 甘草{\scriptsize 二兩炙} 牡蛎{\scriptsize 二兩熬} 龙骨{\scriptsize 二兩}\\
右四味。以水五升。煮取二升半。去滓。温服八合。日三服。

火逆。下之。因燒針。煩躁者。桂枝甘草龙骨牡蛎湯主之。118

\section{桂枝二麻黄一湯}

桂枝{\scriptsize 一兩十七銖去皮} 芍藥{\scriptsize 一兩六銖} 麻黄{\scriptsize 十六銖去節} 生薑{\scriptsize 一兩六銖切} 杏人{\scriptsize 十六個去皮尖} 甘草{\scriptsize 一兩二銖炙} 大棗{\scriptsize 五枚擘}\\
右七味。以水五升。先煮麻黄一二沸。去上沫。内諸藥。煮取二升。去滓。温服一升。日再服。本云桂枝湯二分。麻黄湯一分。合为二升。分再服。今合为一方。將息如前法。{\scriptsize 臣億等謹按。桂枝湯方。桂枝芍藥生薑各三兩。甘草二兩。大棗十二枚。麻黄湯方。麻黄三兩。桂枝二兩。甘草一兩。杏仁七十個。今以算法約之。桂枝湯取十二分之五。即得桂枝芍藥生薑各一兩六銖。甘草二十銖。大棗五枚。麻黄湯取九分之二。即得麻黄十六銖。桂枝十銖三分銖之二。收之得十一銖。甘草五銖三分銖之一。收之得六銖。杏仁十五個九分枚之四。收之得十六個。二湯所取相合。即共得桂枝一兩十七銖。麻黄十六銖。生薑芍藥各一兩六銖。甘草一兩二銖。大棗五枚。杏仁十六個。合方。}{\zhaoben}

服桂枝湯。大汗出。脉洪大者。与桂枝湯。若形如瘧。一日再发者。汗出便解。宜桂枝二麻黄一湯。25

\hangindent 1em
\hangafter=0
凡大汗出復後。脉洪大。形如瘧。一日再发。汗出便解。更与桂枝麻黄湯。{\yixin}

\section{桂枝二越婢一湯}

桂枝{\scriptsize 十八銖去皮} 芍藥{\scriptsize 十八銖} 麻黄{\scriptsize 十八銖} 甘草{\scriptsize 十八銖炙} 大棗{\scriptsize 四枚擘} 生薑{\scriptsize 一兩二銖切} 石膏{\scriptsize 二十四銖碎棉裹}\\
右七味。以水五升。煮麻黄一二沸。去上沫。内諸藥。煮取二升。去滓。温服一升。本云。当裁为越婢湯桂枝湯。合{\khaaitp 之}飲一升。今合为一方。桂枝湯二分。越婢湯一分。{\scriptsize 臣億等謹按。桂枝湯方。桂枝芍藥生薑各三兩。甘草二兩。大棗十二枚。越婢湯方。麻黄二兩。生薑三兩。甘草二兩。石膏半斤。大棗十五枚。今以算法約之。桂枝湯取四分之一。即得桂枝芍藥生薑各十八銖。甘草十二銖。大棗三枚。越婢湯取八分之一。即得麻黄十八銖。生薑九銖。甘草六銖。石膏二十四銖。大棗一枚八分之七。棄之。二湯所取相合。即共得桂枝芍藥甘草麻黄各十八銖。生薑一兩三銖。石膏二十四銖。大棗四枚。合方。舊云桂枝三。今取四分之一。即当云桂枝二也。越婢湯方。見仲景雜方中。外臺祕要一云起脾湯。}

太陽病。发熱。惡寒。熱多寒少。脉微弱者。此无陽也。不可{\khaaitp 復}发汗。{\khaaitp 宜桂枝二越婢一湯。}27

\section{桂枝麻黄各半湯}

桂枝{\scriptsize 一兩十六銖} 芍藥{\scriptsize 一兩} 生薑{\scriptsize 一兩切} 甘草{\scriptsize 一兩炙} 麻黄{\scriptsize 一兩去節} 大棗{\scriptsize 四枚擘} 杏仁{\scriptsize 二十四枚去皮尖兩仁者}\\
右七味。以水五升。先煮麻黄一二沸。去上沫。内諸藥。煮取一升八合。去滓。温服六合。本云桂枝湯三合。麻黄湯三合。并为六合。頓服。將息如上法。{\scriptsize 臣億等謹按。桂枝湯方。桂枝芍藥生薑各三兩。甘草二兩。大棗十二枚。麻黄湯方。麻黄三兩。桂枝二兩。甘草一兩。杏仁七十個。今以算法約之。二湯各取三分之一。即得桂枝一兩十六銖。芍藥生薑甘草各一兩。大棗四枚。杏仁二十三個零三分枚之一。收之得二十四個。合方。詳此方乃三分之一。非各半也。宜云合半湯。}

太陽病。得之八九日。如瘧狀。发熱。惡寒。熱多寒少。其人不嘔。清便續自可。一日再三发。脉微緩者。为欲愈也。脉微而惡寒者。此为陰陽俱虗。不可復{\khaaitp 吐下}发汗也。面反有熱色者。未欲解也。以其不能得汗出。身必癢。宜桂枝麻黄各半湯。23

\section{小建中湯}

桂枝{\scriptsize 三兩去皮} 甘草{\scriptsize 二兩炙} 大棗{\scriptsize 十二枚擘} 芍藥{\scriptsize 六兩} 生薑{\scriptsize 三兩切} 膠飴{\scriptsize 一升}\\
右六味。{\khaaitp 㕮咀。}以水七升。{\khaaitp 先}煮{\khaaitp 五味。}取三升。去滓。内{\khaaitp 膠}飴。令消。温服一升。日三服。嘔家不可用建中湯。以甜故也。
	\footnote{
		「令消」同吳本,宋本作「更上微火消解」。
	}

傷寒。陽脉濇。陰脉弦。法当腹中急痛。先与小建中湯。不差者。与小柴胡湯。100

傷寒二三日。心中悸而煩者。小建中湯主之。102

虗勞。裏急。悸。衄。腹中痛。夢失精。四肢痠疼。手足煩熱。咽乾口燥。小建中湯主之。

男子黄。小便自利。当与虗勞小建中湯。

婦人腹中痛。小建中湯主之。

\section{黄耆建中湯}

黄耆{\scriptsize 三兩} 桂枝{\scriptsize 三兩去皮} 生薑{\scriptsize 三兩切} 芍藥{\scriptsize 六兩} 甘草{\scriptsize 二兩炙} 大棗{\scriptsize 十二枚擘} 膠飴{\scriptsize 一升}\\
右七味。㕮咀。以水七升。先煮六味。取三升。去滓。内膠飴。令消。温服一升。日三服。

虗勞。裏急。諸不足。黄耆建中湯主之。

\section{黄耆桂枝五物湯}

黄耆{\scriptsize 三兩} 芍藥{\scriptsize 三兩} 桂枝{\scriptsize 三兩去皮} 生薑{\scriptsize 六兩切} 大棗{\scriptsize 十二枚擘}\\
右五味。㕮咀。以水六升。煮取二升。去滓。温服七合。日三服。{\scriptsize 一方有人参。}

血痹。陰陽俱微。寸口関上微。尺中小緊。外証身体不仁。如風痹狀。黄耆桂枝五物湯主之。

\section{黄耆芍藥桂枝苦酒湯}

黄耆{\scriptsize 五兩} 芍藥{\scriptsize 二兩} 桂枝{\scriptsize 三兩去皮}\\
右三味。㕮咀。以苦酒一升。水七升相和。煮取三升。去滓。温服一升。当心煩。服至六七日乃解。若心煩不止者。以苦酒阻故也。{\scriptsize 一方用美清醯代苦酒。}

問曰。黄汗之为病。身体腫。发熱。汗出而渴。狀如風水。汗沾衣。色正黄如檗汁。脉自沈。何從得之。\\
師曰。以汗出入水中浴。水從汗孔入得之。

黄汗。黄耆芍藥桂枝苦酒湯主之。

\section{桂枝甘草湯}

桂枝{\scriptsize 四兩去皮} 甘草{\scriptsize 二兩炙}\\
右二味。以水三升。煮取一升。去滓。頓服。

发汗過多。其人叉手自冒心。心下悸。欲得按者。桂枝甘草湯主之。64

\section{半夏散及湯}

半夏{\scriptsize 洗} 桂枝{\scriptsize 去皮} 甘草{\scriptsize 炙}\\
右三味。等分。各別擣篩已。合治之。白飲和服方寸匕。日三服。若不能散服者。以水一升。煎七沸。内散兩方寸匕。更煮三沸。下火。令小冷。少少咽之。半夏有毒。不当散服。

少陰病。咽中痛。半夏散及湯主之。313

\section{桂枝人参湯}

桂枝{\scriptsize 四兩別切} 甘草{\scriptsize 四兩炙} 白术{\scriptsize 三兩} 人参{\scriptsize 三兩} 乾薑{\scriptsize 三兩}\\
右五味。以水九升。先煮四味。取五升。内桂。更煮取三升。去滓。温服一升。日再夜一服。

太陽病。外証未除。而數下之。遂挾熱而利。利下不止。心下痞堅。表裏不解。桂枝人参湯主之。163

\section{理中湯(人参湯)*}

人参{\scriptsize 三兩} 乾薑{\scriptsize 三兩} 甘草{\scriptsize 三兩炙} 白术{\scriptsize 三兩}\\
右四味。擣篩。蜜和为丸。如雞子黄許大。以沸湯數合。和一丸。研碎。温服之。日三夜二服。腹中未熱。益至三四丸。然不及湯。\\
湯法。以四物。依兩數切。用水八升。煮取三升。去滓。温服一升。日三服。

{\khaaitp 若}脐上築者。腎气動也。去术。加桂四兩。\\
{\khaaitp 若}吐多者。去术。加生薑三兩。\\
{\khaaitp 若}下{\khaaitp 利}多者。復用术。\\
{\khaaitp 若}悸者。加茯苓二兩。\\
{\khaaitp 若}渴{\khaaitp 欲得水}者。加术至四兩半。\\
{\khaaitp 若}腹中痛者。加人参至四兩半。\\
{\khaaitp 若}寒者。加乾薑至四兩半。\\
{\khaaitp 若}腹滿者。去术。加附子一枚。\\

傷寒。服湯藥。下利不止。心下痞堅。服瀉心湯已。復以他藥下之。利不止。醫以理中与之。利益甚。理中者。理中焦。此利在下焦。赤石脂禹餘糧湯主之。復不止者。当利小便。159

霍亂。頭痛。发熱。身疼痛。熱多。欲飲水者。五苓散主之。寒多。不用水者。理中湯主之。386

大病差後。其人喜唾。久不了了者。胃上有寒。当温之。宜理中丸。396

胸痹。心中痞。留气結在胸。胸滿。脇下逆{\khaaitp 气}搶心。枳実薤白桂枝湯主之。人参湯亦主之。

\section{茯苓杏仁甘草湯}

茯苓{\scriptsize 三兩} 杏仁{\scriptsize 五十個去皮尖} 甘草{\scriptsize 一兩炙}\\
右三味。㕮咀。以水一斗。煮取五升。去滓。温服一升。日三服。不差。更合服。

胸痹。胸中气塞。短气。茯苓杏仁甘草湯主之。橘{\khaaitp 皮}枳{\khaaitp 実生}薑湯亦主之。

\section{茯苓戎鹽湯}

茯苓{\scriptsize 八兩} 白术{\scriptsize 二兩} 戎鹽{\scriptsize 彈丸大一枚}\\
右三味。㕮咀。以水七升。煮取三升。去滓。分温三服。

小便不利。蒲灰散主之。滑石白魚散。茯苓戎鹽湯并主之。

\section{葵子茯苓散}

葵子{\scriptsize 一升} 茯苓{\scriptsize 三兩}\\
右二味。杵为散。飲服方寸匕。日三服。小便利則愈。

{\khaaitp 婦人}妊娠。有水气。身重。小便不利。洒淅惡寒。起即頭眩。葵子茯苓散主之。

\section{甘草乾薑茯苓白术湯}

甘草{\scriptsize 二兩炙} 乾薑{\scriptsize 四兩} 茯苓{\scriptsize 四兩} 白术{\scriptsize 二兩}\\
右四味。㕮咀。以水五升。煮取三升。去滓。分温三服。腰中即温。

腎著之病。其人身体重。腰中冷。如坐水中。形如水狀。反不渴。小便自利。飲食如故。病屬下焦。身勞汗出。衣裏冷濕。久久得之。腰以下冷痛。腹重如帶五千錢。甘{\khaaitp 草乾}薑{\khaaitp 茯}苓{\khaaitp 白}术湯主之。

\section{苓桂术甘湯}

茯苓{\scriptsize 四兩} 桂枝{\scriptsize 三兩去皮} 白术{\scriptsize 二兩} 甘草{\scriptsize 二兩炙}\\
右四味。以水六升。煮取三升。去滓。分温三服。

傷寒吐下发汗後。心下逆滿。气上衝胸。起則頭眩。其脉沈緊。发汗則動經。身为振搖。苓桂术甘湯主之。67

心下有痰飲。胸脇支滿。目胘。苓桂术甘湯主之。

夫短气。有微飲。当從小便去之。苓桂术甘湯主之。腎气丸亦主之。

\section{苓桂甘棗湯}

茯苓{\scriptsize 八兩} 桂枝{\scriptsize 四兩去皮} 甘草{\scriptsize 二兩炙} 大棗{\scriptsize 十五枚擘}\\
右四味。以甘爤水一斗。先煮茯苓。減二升。内諸藥。煮取三升。去滓。温服一升。日三服。作甘爤水法。取水二斗。置大盆内。以杓揚之。水上有珠子五六千顆相逐。取用之。

发汗後。其人脐下悸。欲作奔豚。苓桂甘棗湯主之。65

\section{茯苓桂枝五味子甘草湯}

茯苓{\scriptsize 四兩} 桂枝{\scriptsize 四兩去皮} 五味子{\scriptsize 八兩碎} 甘草{\scriptsize 三兩炙}\\
右四味。㕮咀。以水八升。煮取三升。去滓。分温三服。

青龙湯下已。多唾。口燥。寸脉沈。尺脉微。手足厥逆。气從少腹上衝胸咽。手足痹。其面翕然如醉。因復下流陰股。小便難。时復冒。可与茯苓桂枝五味子甘草湯。治其气衝。{\wuben}

青龙湯下已。多唾。口燥。寸脉沈。尺脉微。手足厥逆。气從少腹上衝胸咽。手足痹。其面翕熱如醉狀。因復下流陰股。小便難。时復冒者。与茯苓桂枝五味子甘草湯。治其气衝。{\dengben}

\section{茯苓桂枝五味子甘草湯去桂加乾薑細辛}

茯苓{\scriptsize 四兩} 五味子{\scriptsize 半升碎} 甘草{\scriptsize 一兩炙} 乾薑{\scriptsize 一兩} 細辛{\scriptsize 一兩}\\
右五味。㕮咀。以水八升。煮取三升。去滓。分温三服。{\wuben}

茯苓{\scriptsize 四兩} 甘草{\scriptsize 三兩} 乾薑{\scriptsize 三兩} 細辛{\scriptsize 三兩} 五味子{\scriptsize 半升}\\
右五味。以水八升。煮取三升。去滓。温服半升。日三。{\dengben}

衝气即低。而反更欬滿者。因茯苓五味子甘草。去桂加乾薑細辛。以治其欬滿。{\wuben}

衝气即低。而反更欬。胸滿者。用桂苓五味甘草湯。去桂加乾薑細辛。以治其欬滿。{\dengben}

\section{}

欬滿則止。而復更渴。衝气復发者。以細辛乾薑为熱藥。此法不当遂渴。而渴反止者。为支飲也。支飲法当冒。冒者必嘔。嘔者復内半夏。以去其水。{\wuben}

欬滿即止。而更復渴。衝气復发者。以細辛乾薑为熱藥也。服之当遂渴。而渴反止者。为支飲也。支飲者。法当冒。冒者必嘔。嘔者復内半夏。以去其水。{\dengben}

\section{}

水去。嘔則止。其人形腫。可内麻黄。以其欲逐痹。故不内麻黄。乃内杏仁也。若逆而内麻黄者。其人必厥。所以然者。以其人血虗。麻黄发其陽故也。{\wuben}

水去。嘔止。其人形腫者。加杏仁主之。其証應内麻黄。以其人遂痹。故不内之。若逆而内之者。必厥。所以然者。以其人血虗。麻黄发其陽故也。{\dengben}

\section{}

若面熱如醉狀者。此为胃中熱。上熏其面令熱。加大黄湯和之。{\wuben}

若面熱如醉。此为胃熱上衝熏其面。加大黄以利之。{\dengben}

\section{澤瀉湯}

心下有支飲。其人苦冒眩。澤瀉湯主之。

\section{茯苓澤瀉湯}

茯苓{\scriptsize 八兩} 澤瀉{\scriptsize 四兩} 甘草{\scriptsize 二兩炙} 桂枝{\scriptsize 二兩去皮} 白术{\scriptsize 三兩} 生薑{\scriptsize 四兩切}\\
右六味。㕮咀。以水一斗。煮取三升。内澤瀉。再煮。取二升半。去滓。温服八合。日三服。

胃反。吐而渴欲飲水者。茯苓澤瀉湯主之。

%外臺載有不同條文,日後補充。

\section{茯苓甘草湯}

茯苓{\scriptsize 二兩} 桂枝{\scriptsize 二兩去皮} 甘草{\scriptsize 一兩炙} 生薑{\scriptsize 三兩切}\\
右四味。以水四升。煮取二升。去滓。分温三服。

傷寒。汗出而渴者。五苓散主之。不渴者。茯苓甘草湯主之。73

傷寒。厥而心下悸。宜先治水。当与茯苓甘草湯。卻治其厥。不尔。水漬入胃。必作利也。356

\section{五苓散}

豬苓{\scriptsize 十八銖去皮} 澤瀉{\scriptsize 一兩六銖} 白术{\scriptsize 十八銖} 茯苓{\scriptsize 十八銖} 桂枝{\scriptsize 半兩去皮}\\
右五味。擣为散。以白飲和服方寸匕。日三服。多飲暖水。汗出愈。如法將息。

太陽病发汗後。大汗出。胃中乾。煩躁不得眠。其人欲飲水。当稍飲之。令胃气和則愈。若脉浮。小便不利。微熱。消渴者。五苓散主之。{\scriptsize 即豬苓散是。}71

发汗已。脉浮數。煩渴者。五苓散主之。72

傷寒。汗出而渴者。五苓散主之。不渴者。茯苓甘草湯主之。73

中風。发熱。六七日不解而煩。有表裏証。渴欲飲水。水入則吐。此为水逆。五苓散主之。74

病在陽。当以汗解。反以水潠之或灌之。其熱被劫不得去。益煩。皮上粟起。意欲飲水。反不渴。宜服文蛤散。若不差。与五苓散。141

本以下之。故心下痞。与瀉心湯。痞不解。其人渴而口燥{\khaaitp 煩}。小便不利。五苓散主之。156

太陽病。寸{\khaaitp 口}緩。関{\khaaitp 上小}浮。尺{\khaaitp 中}弱。其人发熱。汗出。復惡寒。不嘔。但心下痞者。此为醫下之故也。若不下。其人不惡寒而渴者。此轉屬陽明。小便數者。大便必堅。不更衣十日。无所苦也。{\khaaitp 渴}欲飲水者。少少与之。但以法救之。渴者。宜五苓散。244

霍亂。頭痛。发熱。身疼痛。熱多。欲飲水者。五苓散主之。寒多。不用水者。理中湯主之。386

假令瘦人脐下悸。吐涎沫而癲眩。{\khaaitp 此}水也。五苓散主之。

脉浮。小便不利。微熱。消渴者。宜利小便。发汗。五苓散主之。

\section{茵陳五苓散}

黄疸病。茵陳五苓散主之。

\section{豬苓湯}

陽明病。脉浮緊。咽乾。口苦。腹滿而喘。发熱。汗出。不惡寒。反惡熱。身重。若发汗則躁。心憒憒。反譫語。若加温針。必怵惕。煩躁。不得眠。若下之。則胃中空虗。客气動膈。心中懊憹。舌上胎者。栀子{\khaaitp 豉}湯主之。若渴欲飲水。口乾舌燥者。白虎{\khaaitp 加人参}湯主之。若脉浮。发熱。渴欲飲水。小便不利者。豬苓湯主之。221.222.223

陽明病。汗出多而渴者。不可与豬苓湯。以汗多。胃中燥。豬苓湯復利其小便故也。224

少陰病。下利六七日。欬而嘔。渴。心煩不得眠。豬苓湯主之。319

\section{豬苓散}

豬苓{\scriptsize\khaaitp 去皮} 茯苓{ }白术{\scriptsize 各等分}\\
右三味。杵为散。飲服方寸匕。日三服。

嘔吐。而病在膈上。後思水者解。急与之。思水者。豬苓散主之。

\section{牡蛎澤瀉散}

牡蛎{\scriptsize 熬} 澤瀉 蜀漆{\scriptsize 煖水洗去腥} 葶藶子{\scriptsize 熬} 商陸根{\scriptsize 熬} 海藻{\scriptsize 洗去鹹} 栝蔞根{\scriptsize 各等分}\\
右七味。異擣。下篩为散。更於臼中治之。白飲和服方寸匕。日三服。小便利。止後服。

大病差後。從腰以下有水气者。牡蛎澤瀉散主之。395

\section{腎气丸}

乾地黄{\scriptsize 八兩} 薯蕷{\scriptsize 四兩} 山茱萸{\scriptsize 四兩} 澤瀉{\scriptsize 三兩} 茯苓{\scriptsize 三兩} 牡丹皮{\scriptsize 三兩} 桂枝{\scriptsize 一兩} 附子{\scriptsize 一兩炮}\\
右八味。末之。煉蜜和丸梧子大。酒下十五丸。日再服。{\khaaitp 加至二十五丸。}

崔氏八味丸。治腳气上入。少腹不仁。

虗勞。腰痛。少腹拘急。小便不利者。八味腎气丸主之。

夫短气。有微飲。当從小便去之。苓桂术甘湯主之。腎气丸亦主之。

男子消渴。小便反多。以飲一斗。小便一斗。腎气丸主之。

問曰。婦人病。飲食如故。煩熱不得卧。而反倚息者。何也。\\
師曰。此名轉胞。不得尿也。以胞系了戾。故致此病。但利小便則愈。宜腎气丸。

\section{栝蔞瞿麥丸}

栝蔞根{\scriptsize 二兩} 茯苓{\scriptsize 三兩} 薯蕷{\scriptsize 三兩} 附子{\scriptsize 大者一枚炮去皮} 瞿麥{\scriptsize 一兩}\\
右五味。末之。煉蜜和为丸梧子大。飲服三丸。日三服。不知。增至七八丸。以小便利。腹中温为知。

小便不利者。有水气。其人若渴。栝蔞瞿麥丸主之。

\section{麻黄湯}

麻黄{\scriptsize 三兩去節} 桂枝{\scriptsize 二兩去皮} 甘草{\scriptsize 一兩炙} 杏仁{\scriptsize 七十個去皮尖}\\
右四味。以水九升。先煮麻黄。減二升。去上沫。内諸藥。煮取二升半。去滓。温服八合。覆取微似汗。不須啜粥。餘如桂枝法將息。

太陽病。頭痛。发熱。身疼。腰痛。骨節疼痛。惡風。无汗而喘。麻黄湯主之。35

太陽与陽明合病。喘而胸滿者。不可下。宜麻黄湯。36

太陽病。十日已去。脉浮細而嗜卧者。外已解也。設胸滿脇痛者。与小柴胡湯。脉{\khaaitp 但}浮者。与麻黄湯。37

太陽病。脉浮緊。无汗。发熱。身疼痛。八九日不解。表証仍在。此当发其汗。服藥已。微除。其人发煩目暝。劇者必衄。衄乃解。所以然者。陽气重故也。麻黄湯主之。46

脉浮者。病在表。可发汗。宜麻黄湯。51

{\khaaitp 太陽病。}脉浮數者。可发汗。宜麻黄湯。52

傷寒。脉浮緊。不发汗。因致衄者。宜麻黄湯。55

{\khaaitp 寸口}脉浮而緊。浮則为風。緊則为寒。風則傷衛。寒則傷榮。榮衛俱病。骨節煩疼。当发其汗。{\khaaitp 宜麻黄湯。}

陽明中風。脉弦浮大。而短气。腹都滿。脇下及心痛。久按之。气不通。鼻乾。不得汗。嗜卧。一身及目悉黄。小便難。有潮熱。时时噦。耳前後腫。刺之小差。外不解。病過十日。脉續浮者。与{\khaaitp 小}柴胡湯。脉但浮。无餘証者。与麻黄湯。若不尿。腹滿加噦者。不治。231.232

陽明病。脉浮。无汗而喘者。发汗則愈。宜麻黄湯。235

\section{麻黄加术湯}

麻黄{\scriptsize 三兩去節} 桂枝{\scriptsize 二兩去皮} 甘草{\scriptsize 一兩炙} 杏仁{\scriptsize 七十個去皮尖} 白术{\scriptsize 四兩}\\
右五味。㕮咀。以水九升。先煮麻黄一二沸。去上沫。内諸藥。煮取二升。去滓。温服八合。覆取微似汗。

濕家。身煩疼。可与麻黄加术湯。发其汗为宜。慎不可以火攻之。

\section{甘草麻黄湯}

甘草{\scriptsize 二兩炙} 麻黄{\scriptsize 四兩去節}\\
右二味。㕮咀。以水五升。先煮麻黄{\khaaitp 再沸}。去上沫。内甘草。煮取三升。去滓。温服一升。重覆汗出。不汗再服。慎風寒。

裏水。越婢加术湯主之。甘草麻黄湯亦主之。

\section{麻黄附子甘草湯(附子麻黄湯)}

麻黄{\scriptsize 二兩去節} 甘草{\scriptsize 二兩炙} 附子{\scriptsize 一枚炮去皮破八片}\\
右三味。以水七升。先煮麻黄一兩沸。去上沫。内諸藥。煮取三升。去滓。温服一升。日三服。{\zhaoben}

附子{\scriptsize 一枚炮去皮破八片} 麻黄{\scriptsize 二兩去節} 甘草{\scriptsize 二兩炙}\\
右三味。㕮咀。以水七升。先煮麻黄再沸。去上沫。内諸藥。煮取二升半。去滓。温服八分。日三服。{\wuben}

少陰病。得之二三日。麻黄附子甘草湯微发汗。以二三日无{\khaaitp 裏}証。故微发汗。302

水之为病。其脉沈小。屬少陰。浮者为風。无水。虗胀者为气。水。发其汗即已。脉沈者。宜麻黄附子湯。浮者。宜杏子湯。

\section{麻黄細辛附子湯}

麻黄{\scriptsize 二兩去節} 細辛{\scriptsize 二兩} 附子{\scriptsize 一枚炮去皮破八片}\\
右三味。以水一斗。先煮麻黄。減二升。去上沫。内諸藥。煮取三升。去滓。温服一升。日三服。

少陰病。始得之。反发熱。脉沈者。麻黄細辛附子湯主之。301

\section{麻杏甘石湯}

麻黄{\scriptsize 四兩去節} 杏仁{\scriptsize 五十個去皮尖} 甘草{\scriptsize 二兩炙} 石膏{\scriptsize 八兩碎綿裹}\\
右四味。以水七升。煮麻黄。減二升。去上沫。内諸藥。煮取二升。去滓。温服一升。

发汗後。汗出而喘。无大熱者。可与麻杏甘石湯。63

下後。汗出而喘。无大熱者。可与麻杏甘石湯。162

\section{麻杏薏甘湯}

麻黄{\scriptsize 二兩去節} 杏仁{\scriptsize 三十個去皮尖} 薏苡仁{\scriptsize 一兩} 甘草{\scriptsize 一兩炙}\\
右四味。㕮咀。以水四升。先煮麻黄一二沸。去上沫。内諸藥。煮取二升。去滓。分温再服。

病者一身{\sungtpii 𥁞}疼。发熱。日晡所劇者。名風濕。此病傷於汗出当風。或久傷取冷所致也。可与麻杏薏甘湯。

\section{牡蛎湯}

牡蛎{\scriptsize 四兩熬} 麻黄{\scriptsize 四兩去節} 甘草{\scriptsize 二兩炙} 蜀漆{\scriptsize 三兩洗去腥}\\
右四味。㕮咀。以水八升。先煮蜀漆。麻黄。去上沫。得六升。内諸藥。煮取二升。去滓。温服一升。吐則勿更服。{\scriptsize 見外臺}

牡蛎湯。治牡瘧。

\section{麻黄淳酒湯}

麻黄{\scriptsize 三兩去節綿裹}\\
右一味。以美清酒五升。煮取二升半。去滓。頓服{\sungtpii 𥁞}。冬月用酒。春月用水煮之。

黄疸。麻黄淳酒湯主之。

\section{半夏麻黄丸}

半夏{\scriptsize 洗} 麻黄{\scriptsize 去節等分}\\
右二味。末之。煉蜜和丸如小豆大。飲服三丸。日三服。

心下悸者。半夏麻黄丸主之。

\section{小青龙湯*}

麻黄{\scriptsize 三兩去節} 芍藥{\scriptsize 三兩} 細辛{\scriptsize 三兩} 乾薑{\scriptsize 三兩} 甘草{\scriptsize 三兩炙} 桂枝{\scriptsize 三兩去皮} 五味子{\scriptsize 半升} 半夏{\scriptsize 半升洗}\\
右八味。以水一斗。先煮麻黄。減二升。去上沫。内諸藥。煮取三升。去滓。温服一升。

渴者。去半夏。加栝蔞根三兩。\\
微利者。去麻黄。加蕘花。如一雞子。熬令赤色。\\
噎者。去麻黄。加附子一枚炮。\\
小便不利。少腹滿者。去麻黄。加茯苓四兩。\\
喘者。去麻黄。加杏仁半升。去皮尖。\\
蕘花不治利。麻黄主喘。今此語反之。疑非仲景意。{\scriptsize 臣億等謹按。小青龍湯。大要治水。又按本草。蕘花下十二水。若水去。利則止也。又按千金。形腫者應内麻黄。乃内杏仁者。以麻黄发其陽故也。以此証之。豈非仲景意也。}

傷寒。表不解。心下有水气。乾嘔。发熱而欬。或渴。或利。或噎。或小便不利。少腹滿。或喘。小青龙湯主之。40

傷寒。心下有水气。欬而微喘。发熱。不渴。服湯已而渴者。此寒去。为欲解。小青龙湯主之。41

%病溢飲者。当发其汗。大青龙湯主之。小青龙湯亦主之。

欬逆倚息。小青龙湯主之。{\wuben}

欬逆倚息。不得卧。小青龙湯主之。{\dengben}

婦人吐涎沫。醫反下之。心下即痞。当先治其吐涎沫。宜小青龙湯。涎沫止。乃治痞。宜瀉心湯。

\section{小青龙加石膏湯}

麻黄{\scriptsize 三兩去節} 芍藥{\scriptsize 三兩} 桂枝{\scriptsize 三兩} 細辛{\scriptsize 三兩} 甘草{\scriptsize 三兩炙} 乾薑{\scriptsize 三兩} 五味子{\scriptsize 半升} 半夏{\scriptsize 半升洗} 石膏{\scriptsize 二兩碎}\\
右九味。㕮咀。以水一斗。先煮麻黄。減二升。去上沫。内諸藥。取三升。去滓。强人服一升。羸者減之。日三服。小兒服四合。

肺胀。欬而上气。煩躁而喘。脉浮者。心下有水。小青龙加石膏湯主之。

欬而上气。肺胀。其脉浮。心下有水气。脇下痛引缺盆。小青龙加石膏湯主之。

\section{大青龙湯}

麻黄{\scriptsize 六兩去節} 桂枝{\scriptsize 二兩去皮} 甘草{\scriptsize 二兩炙} 杏仁{\scriptsize 四十枚去皮尖} 生薑{\scriptsize 三兩切} 大棗{\scriptsize 十枚擘} 石膏{\scriptsize 如雞子大碎}\\
右七味。以水九升。先煮麻黄。減二升。去上沫。内諸藥。煮取三升。去滓。温服一升。取微似汗。汗出多者。温粉粉之。一服汗者。停後服。若復服。汗多亡陽。遂虗。惡風。煩躁。不得眠也。

太陽中風。脉浮緊。发熱。惡寒。身疼痛。不汗出而煩躁者。大青龙湯主之。若脉微弱。汗出。惡風者。不可服之。服之則厥。筋愓肉瞤。此为逆也。38

傷寒。脉浮緩。身不疼。但重。乍有輕时。无少陰証者。大青龙湯发之。39

病溢飲{\khaaitp 者}。当发其汗。宜大青龙湯。
%病溢飲者。当发其汗。大青龙湯主之。小青龙湯亦主之。

\section{文蛤湯}

文蛤{\scriptsize 五兩} 麻黄{\scriptsize 三兩去節} 甘草{\scriptsize 三兩炙} 杏仁{\scriptsize 五十枚去皮尖} 石膏{\scriptsize 五兩碎} 大棗{\scriptsize 十二枚擘} 生薑{\scriptsize 三兩切}\\
右七味。㕮咀。以水六升。煮取二升。去滓。温服一升。汗出愈。

吐後。渴欲得飲而貪水者。文蛤湯主之。兼主微風。脉緊。頭痛。

\section{文蛤散}

文蛤{\scriptsize 五兩}\\
右一味。杵为散。以沸湯五合。和服方寸匕。{\dengben}

文蛤{\scriptsize 五兩}\\
右一味。为散。以沸湯和一方寸匕服湯。用五合。{\zhaoben}

病在陽。当以汗解。反以水潠之或灌之。其熱被劫不得去。益煩。皮上粟起。意欲飲水。反不渴。宜服文蛤散。若不差。与五苓散。141

渴欲飲水不止者。文蛤散主之。

\section{越婢湯}

麻黄{\scriptsize 六兩去節} 石膏{\scriptsize 八兩碎} 生薑{\scriptsize 三兩切} 大棗{\scriptsize 十五枚擘} 甘草{\scriptsize 二兩炙}\\
右五味。㕮咀。以水六升。先煮麻黄{\khaaitp 再沸}。去上沫。内諸藥。煮取三升。去滓。分温三服。惡風者。加附子一枚炮。{\scriptsize 古今錄驗云。風水加术四兩。}

風水。惡風。一身悉腫。脉浮。不渴。續自汗出。无大熱。越婢湯主之。

\section{越婢加术湯}

麻黄{\scriptsize 六兩去節} 石膏{\scriptsize 八兩} 生薑{\scriptsize 三兩切} 甘草{\scriptsize 二兩炙} 大棗{\scriptsize 十五枚擘} 白术{\scriptsize 四兩}\\
右六味。㕮咀。以水六升。先煮麻黄再沸。去上沫。内諸藥。煮取三升。去滓。分温三服。惡風加附子一枚炮。

裏水者。一身面目自洪腫。其脉沈。小便不利。故令病水。假如小便自利。亡津液。故令渴也。{\wuben}

裏水者。一身面目黄腫。其脉沈。小便不利。故令病水。假如小便自利。此亡津液。故令渴也。越婢加术湯主之。{\dengben}

裏水。越婢加术湯主之。甘草麻黄湯亦主之。

越婢加术湯。治肉極。熱則身体津{\khaaitp 液}脱。腠理開。汗大泄。厉風气。下焦腳弱。

\section{越婢加半夏湯}

麻黄{\scriptsize 六兩去節} 石膏{\scriptsize 八兩碎} 生薑{\scriptsize 三兩切} 大棗{\scriptsize 十五枚擘} 甘草{\scriptsize 二兩炙} 半夏{\scriptsize 半升洗}\\
右六味。㕮咀。以水六升。先煮麻黄再沸。去上沫。内諸藥。煮取三升。去滓。分温三服。

欬而上气。此为肺胀。其人喘。目如脱狀。脉浮大者。越婢加半夏湯主之。

\section{葛根湯}

葛根{\scriptsize 四兩} 麻黄{\scriptsize 三兩去節} 桂枝{\scriptsize 二兩去皮} 生薑{\scriptsize 三兩切} 甘草{\scriptsize 二兩炙} 芍藥{\scriptsize 二兩} 大棗{\scriptsize 十二枚擘}\\
右七味。以水一斗。先煮麻黄。葛根。减二升。去白沫。内諸藥。煮取三升。去滓。温服一升。覆取微似汗。餘如桂枝法將息及禁忌。諸湯皆倣此。

太陽病。項背强几几。无汗。惡風。葛根湯主之。31

太陽与陽明合病。而自利{\khaaitp 者}。葛根湯主之。不下利。但嘔者。葛根加半夏湯主之。32.33

太陽病。无汗。而小便反少。气上衝胸。口噤不得語。欲作剛痙。葛根湯主之。

\section{葛根加半夏湯}

葛根{\scriptsize 四兩} 麻黄{\scriptsize 三兩去節} 甘草{\scriptsize 二兩炙} 芍藥{\scriptsize 二兩} 桂枝{\scriptsize 二兩去皮} 生薑{\scriptsize 二兩切} 半夏{\scriptsize 半升洗} 大棗{\scriptsize 十二枚擘}\\
右八味。以水一斗。先煮葛根。麻黄。减二升。去白沫。内諸藥。煮取三升。去滓。温服一升。覆取微似汗。

太陽与陽明合病。而自利{\khaaitp 者}。葛根湯主之。不下利。但嘔者。葛根加半夏湯主之。32.33

\section{葛根芩連湯}

葛根{\scriptsize 八兩} 甘草{\scriptsize 二兩炙} 黄芩{\scriptsize 三兩} 黄連{\scriptsize 三兩}\\
右四味。以水八升。先煮葛根。减二升。内諸藥。煮取二升。去滓。分温再服。

太陽病。桂枝証。醫反下之。遂利不止。脉促者。表未解也。喘而汗出者。葛根芩連湯主之。34

\section{小柴胡湯*}

柴胡{\scriptsize 八兩} 黄芩{\scriptsize 三兩} 人参{\scriptsize 三兩} 甘草{\scriptsize 三兩炙} 生薑{\scriptsize 三兩切} 大棗{\scriptsize 十二枚擘} 半夏{\scriptsize 半升洗}\\
右七味。以水一斗二升。煮取六升。去滓。再煎。取三升。温服一升。日三服。

若胸中煩而不嘔者。去半夏。人参。加栝蔞実一枚。\\
若渴者。去半夏。加人参合前成四兩半。栝蔞根四兩。\\
若腹中痛者。去黄芩,加芍藥三兩。\\
若脇下痞堅者。去大棗。加牡蛎六兩。\\
若心下悸。小便不利者。去黄芩。加茯苓四兩。\\
若不渴。外有微熱者。去人参。加桂三兩。温覆。微发其汗。\\
若欬者。去人参。大棗。生薑。加五味子半升。乾薑二兩。

太陽病。十日已去。脉浮細而嗜卧者。外已解也。設胸滿脇痛者。与小柴胡湯。脉{\khaaitp 但}浮者。与麻黄湯。37

血弱气{\sungtpii 𥁞}。腠理開。邪气因入。与正气相摶。結於脇下。正邪分爭。往來寒熱。休作有时。默默不欲飲食。臓腑相連。其痛必下。邪高痛下。故使嘔也。小柴胡湯主之。服柴胡湯已。渴者。屬陽明。以法治之。97

傷寒五六日。中風。往來寒熱。胸脇苦滿。默默不欲飲食。心煩。喜嘔。或胸中煩而不嘔。或渴。或腹中痛。或脇下痞堅。或心下悸。小便不利。或不渴。外有微熱。或欬。小柴胡湯主之。96

得病六七日。脉遲浮弱。惡風寒。手足温。醫再三下之。不能食。其人脇下滿{\khaaitp 痛}。面目及身黄。頸項强。小便難。与柴胡湯後必下重。本渴。飲水而嘔。柴胡{\khaaitp 湯}不復中与也。食穀者噦。98

傷寒四五日。身熱。惡風。頸項强。脇下滿。手足温而渴。小柴胡湯主之。99

傷寒。陽脉濇。陰脉弦。法当腹中急痛。先与小建中湯。不差者。与小柴胡湯。100

傷寒中風。有柴胡証。但見一証便是。不必悉具。101

凡柴胡湯証而下之。柴胡証不罷者。復与柴胡湯。必蒸蒸而振。卻发熱汗出而解。101

太陽病。過經十餘日。反再三下之。後四五日。柴胡証仍在。先与小柴胡湯。嘔不止。心下急。其人鬱鬱微煩者。为未解。与大柴胡湯下之則愈。103

傷寒十三日不解。胸脇滿而嘔。日晡所发潮熱{\khaaitp 。已}而微利。此本当柴胡湯下之。不得利。今反利者。知醫以丸藥下之。非其治也。潮熱者。実也。先宜服小柴胡湯以解其外。後以柴胡加芒硝湯主之。104

婦人中風七八日。續得寒熱。发作有时。經水適斷。此为熱入血室。其血必結。故使如瘧狀。发作有时。小柴胡湯主之。144

傷寒五六日。頭汗出。微惡寒。手足冷。心下滿。口不欲食。大便堅。其脉細。此为陽微結。必有表。復有裏。沈亦为病在裏。汗出为陽微。假令純陰結。不得有外証。悉入在裏。此为半在外半在裏。脉雖沈緊。不得为少陰病。所以然者。陰不得有汗。今頭汗出。故知非少陰也。可与{\khaaitp 小}柴胡湯。設不了了者。得屎而解。148

傷寒五六日。嘔而发熱。柴胡湯証具。而以他藥下之。柴胡証仍在者。復与柴胡湯。此雖已下之。不为逆。必蒸蒸而振。卻发熱汗出而解。若心下滿而堅痛者。此为結胸。宜大陷胸湯。若但滿而不痛者。此为痞。柴胡{\khaaitp 湯}不復中与也。宜半夏瀉心湯。149

陽明病。发潮熱。大便溏。小便自可。胸脇滿不去。小柴胡湯主之。229

陽明病。脇下堅滿。不大便而嘔。舌上白胎者。可与小柴胡湯。上焦得通。津液得下。胃气因和。身濈然汗出而解。230

陽明中風。脉弦浮大。而短气。腹都滿。脇下及心痛。久按之。气不通。鼻乾。不得汗。嗜卧。一身及目悉黄。小便難。有潮熱。时时噦。耳前後腫。刺之小差。外不解。病過十日。脉續浮者。与{\khaaitp 小}柴胡湯。脉但浮。无餘証者。与麻黄湯。若不尿。腹滿加噦者。不治。231.232

太陽病不解。轉入少陽。脇下堅滿。乾嘔。不能食。往來寒熱。尚未吐下。脉沈緊者。可与小柴胡湯。266

嘔而发熱者。小柴胡湯主之。379

傷寒差已後。更发熱者。小柴胡湯主之。脉浮者。以汗解之。脉沈実者。以下解之。394

諸黄。腹痛而嘔者。宜柴胡湯。

產婦鬱{\khaaitp 冒}。其脉微弱。不能食。大便反堅。但頭汗出。所以然者。血虗而厥。厥而必冒。冒家欲解。必大汗出。以血虗下厥。孤陽上出。故但頭汗出。所以產婦喜汗出者。亡陰血虗。陽气獨盛。故当汗出。陰陽乃復。所以便堅者。嘔。不能食也。小柴胡湯主之。病解。能食。七八日。而更发熱者。此为胃熱气実。大承气湯主之。{\wuben}

產婦鬱冒。其脉微弱。不能食。大便反堅。但頭汗出。所以然者。血虗而厥。厥而必冒。冒家欲解。必大汗出。以血虗下厥。孤陽上出。故頭汗出。所以產婦喜汗出者。亡陰血虗。陽气獨盛。故当汗出。陰陽乃復。大便堅。嘔。不能食。小柴胡湯主之。病解。能食。七八日。更发熱者。此为胃実。大承气湯主之。{\dengben}

婦人在草蓐得風。四肢苦煩熱。皆自发露所为。頭痛者。与小柴胡湯。頭不痛。但煩者。与三物黄芩湯。

\section{柴胡加芒硝湯}

柴胡{\scriptsize 二兩十六銖} 黄芩{\scriptsize 一兩} 人参{\scriptsize 一兩} 甘草{\scriptsize 一兩炙} 生薑{\scriptsize 一兩切} 半夏{\scriptsize 二十銖本云五枚洗} 大棗{\scriptsize 四枚擘} 芒硝{\scriptsize 二兩}\\
右八味。以水四升。煮取二升。去滓。内芒硝。更煮微沸。分温再服。不觧更作。

傷寒十三日不解。胸脇滿而嘔。日晡所发潮熱{\khaaitp 。已}而微利。此本当柴胡湯下之。不得利。今反利者。知醫以丸藥下之。非其治也。潮熱者。実也。先宜服小柴胡湯以解其外。後以柴胡加芒硝湯主之。104

\section{柴胡加大黄芒硝桑螵蛸湯}

右前七味。以水四升。煮取二升。去滓。下芒硝大黄桑螵蛸。煮取一升半。去滓。温服五合。微下即愈。本方柴胡湯。再服以解其外。餘一服。加芒硝大黄桑螵蛸。{\yuhan}

右以前七味。以水七升。下芒硝三合。大黄四分。桑螵蛸五枚。煮取一升半。去滓。温服五合。微下即愈。本云柴胡湯。再服以解其外。餘二升。加芒硝大黄桑螵蛸也。{\yifang}

\section{小柴胡去半夏加栝蔞湯}

柴胡{\scriptsize 八兩} 人参{\scriptsize 三兩} 黄芩{\scriptsize 三兩} 甘草{\scriptsize 三兩炙} 栝蔞根{\scriptsize 四兩} 生薑{\scriptsize 二兩切} 大棗{\scriptsize 十二枚擘}\\
右七味。㕮咀。以水一斗二升。煮取六升。去滓。再煎取三升。温服一升。日三。{\scriptsize 見外臺經心錄治勞瘧}

瘧病发渴者。与小柴胡去半夏加栝蔞湯。

\section{柴胡桂枝湯}

桂枝{\scriptsize 一兩半去皮} 黄芩{\scriptsize 一兩半} 人参{\scriptsize 一兩半} 甘草{\scriptsize 一兩炙} 半夏{\scriptsize 二合半洗} 芍藥{\scriptsize 一兩半} 大棗{\scriptsize 六枚擘} 生薑{\scriptsize 一兩半切} 柴胡{\scriptsize 四兩}\\
右九味。以水七升。煮取三升。去滓。温服一升。本云人参湯。作如桂枝法。加半夏柴胡黄芩。復如柴胡法。今用人参作半劑。

傷寒六七日。发熱。微惡寒。肢節煩疼。微嘔。心下支結。外証未去者。柴胡桂枝湯主之。146

发汗多。亡陽。狂語者。不可下。与柴胡桂枝湯。和其榮衛。以通津液。後自愈。

寒疝。腹中痛者。柴胡桂枝湯主之。{\wuben}

柴胡桂枝湯。治心腹卒中痛者。{\dengben}

\section{柴胡桂枝乾薑湯}

柴胡{\scriptsize 八兩} 桂枝{\scriptsize 三兩去皮} 乾薑{\scriptsize 二兩} 栝蔞根{\scriptsize 四兩} 黄芩{\scriptsize 三兩} 牡蛎{\scriptsize 二兩熬} 甘草{\scriptsize 二兩炙}\\
右七味。{\khaaitp 㕮咀。}以水一斗二升。煮取六升。去滓。再煎。取三升。温服一升。日三服。初服微煩。{\khaaitp 復服。}汗出{\khaaitp 便}愈。

傷寒五六日。已发汗而復下之。胸脇滿。微結。小便不利。渴而不嘔。但頭汗出。往來寒熱。心煩。此为未解。柴胡桂枝乾薑湯主之。147

柴胡桂薑湯。{\scriptsize 此方治寒多微有熱。或但寒不熱。服一剂如神。故錄之。}

\section{柴胡加龙骨牡蛎湯}

傷寒八九日。下之。胸滿。煩。驚。小便不利。譫語。一身{\khaaitpii 𥁞}{\khaaitp 重。}不可轉側。柴胡加龙骨牡蛎湯主之。107

\section{大柴胡湯}

柴胡{\scriptsize 八兩} 黄芩{\scriptsize 三兩} 芍藥{\scriptsize 三兩} 半夏{\scriptsize 半升洗} 生薑{\scriptsize 五兩切} 枳実{\scriptsize 四枚炙} 大棗{\scriptsize 十二枚擘} {\khaaitp 大黄{\scriptsize 二兩}}\\
右七味。以水一斗二升。煮取六升。去滓。再煎。{\khaaitp 取三升。}温服一升。日三服。一方加大黄二兩。若不加恐不名大柴胡也。

太陽病。過經十餘日。反再三下之。後四五日。柴胡証仍在。先与小柴胡湯。嘔不止。心下急。其人鬱鬱微煩者。为未解。与大柴胡湯下之則愈。103

傷寒十餘日。熱結在裏。復往來寒熱者。与大柴胡湯。但結胸。无大熱者。此为水結在胸脇。{\khaaitp 但}頭微汗出。大陷胸湯主之。136

傷寒。发熱。汗出不解。心中痞堅。嘔吐。下利。大柴胡湯主之。165

病腹中滿痛者。此为実也。当下之。宜大柴胡湯。{\wuben}

按之心下滿痛者。此为実也。当下之。宜大柴胡湯。{\dengben}

\section{白虎湯}

知母{\scriptsize 六兩} 石膏{\scriptsize 一斤碎綿裹} 甘草{\scriptsize 二兩炙} 粳米{\scriptsize 六合}\\
右四味。以水一斗。煮米熟。湯成。去滓。温服一升。日三服。

傷寒。脉浮滑。此以表有熱。裏有寒。白虎湯主之。176

三陽合病。腹滿。身重。難以轉側。口不仁。面垢。譫語。遺尿。发汗則譫語{\khaaitp 甚}。下之則額上生汗。手足厥冷。自汗。白虎湯主之。219

傷寒。脉滑而厥者。裏有熱也。白虎湯主之。350

\section{白虎加人参湯}

知母{\scriptsize 六兩} 石膏{\scriptsize 一斤碎綿裹} 甘草{\scriptsize 二兩炙} 粳米{\scriptsize 六合} 人参{\scriptsize 三兩}\\
右五味。以水一斗。煮米熟。湯成。去滓。温服一升。日三服。
	\footnote{
	「石膏一斤」吳本作「石膏一升」。
	}

服桂枝湯。大汗出{\khaaitp 後}。大煩渴不解。脉洪大者。白虎{\khaaitp 加人参}湯主之。26

傷寒或吐或下後。七八日不解。熱結在裏。表裏俱熱。时时惡風。大渴。舌上乾燥而煩。欲飲水數升。白虎{\khaaitp 加人参}湯主之。168

傷寒。无大熱。口燥渴。心煩。背微惡寒。白虎{\khaaitp 加人参}湯主之。169

傷寒。脉浮。发熱。无汗。其表不解。不可与白虎湯。渴欲飲水。无表証者。白虎{\khaaitp 加人参}湯主之。170

陽明病。脉浮緊。咽乾。口苦。腹滿而喘。发熱。汗出。不惡寒。反惡熱。身重。若发汗則躁。心憒憒。反譫語。若加温針。必怵惕。煩躁。不得眠。若下之。則胃中空虗。客气動膈。心中懊憹。舌上胎者。栀子{\khaaitp 豉}湯主之。若渴欲飲水。口乾舌燥者。白虎{\khaaitp 加人参}湯主之。若脉浮。发熱。渴欲飲水。小便不利者。豬苓湯主之。221.222.223

太陽中熱者。暍是也。其人汗出。惡寒。身熱而渴。白虎{\khaaitp 加人参}湯主之。

\section{白虎加桂枝湯}

知母{\scriptsize 六兩} 甘草{\scriptsize 二兩炙} 石膏{\scriptsize 一斤碎綿裹} 粳米{\scriptsize 六合} 桂枝{\scriptsize 三兩去皮}\\
右五味。㕮咀以水一斗二升。煮米熟。去滓。煎取三升。温服一升。日三服。汗出愈。

温瘧者。其脉如平。身无寒。但熱。骨節疼煩。时嘔。白虎加桂枝湯主之。

\section{小承气湯}

傷寒。不大便六七日。頭痛。有熱者。与承气湯。其小便清者。此为不在裏。續在表也。当发其汗。頭痛者必衄。宜桂枝湯。56

陽明病。脉遲。雖汗出。不惡寒。其身必重。短气。腹滿而喘。有潮熱。如此者。其外为解。可攻其裏。若手足濈然汗出者。此大便已堅。{\khaaitp 大}承气湯主之。若汗多。微发熱。惡寒者。为外未解。{\khaaitp 桂枝湯主之。}其熱不潮。未可与承气湯。若腹大滿。而不大便者。可与小承气湯。微和其胃气。勿令至大下。208

陽明病。潮熱。大便微堅者。可与{\khaaitp 大}承气湯。不堅者。不可与之。若不大便六七日。恐有燥屎。欲知之法。可少与小承气湯。若腹中轉失气者。此有燥屎也。乃可攻之。若不轉失气者。此但頭堅後溏。不可攻之。攻之必腹滿。不能食也。欲飲水者。与水即噦。其後发熱者。必大便復堅而少也。以小承气湯和之。若不轉失气者。慎不可攻之。209

陽明病。其人多汗。津液外出。胃中燥。大便必堅。堅則譫語。{\khaaitp 小}承气湯主之。{\khaaitp 若一服譫語止。莫復服。}213

陽明病。譫語。发潮熱。脉滑疾者。{\khaaitp 小}承气湯主之。因与承气湯一升。腹中轉失气者。復与一升。若不轉失气者。勿更与之。明日又不大便。脉反微濇者。此为裏虗。为難治。不可復与承气湯。214

太陽病吐下发汗後。微煩。小便數。大便因堅。可与小承气湯。和之則愈。250

得病二三日。脉弱。无太陽柴胡証。煩躁。心下堅。至四日。雖能食。以{\khaaitp 小}承气湯少与。微和之。令小安。至六日。与承气湯一升。若不大便六七日。小便少者。雖不大便。但頭堅後溏。未定成堅。攻之必溏。当須小便利。屎定堅。乃可攻之。宜{\khaaitp 大}承气湯。251

下利。譫語者。有燥屎也。宜{\khaaitp 小}承气湯。374

小承气湯。治大便不通。噦。數譫語。

\section{厚朴三物湯}

腹滿。脉數。厚朴三物湯主之。{\wuben}

痛而閉者。厚朴三物湯主之。{\dengben}

\section{厚朴七物湯}

病腹滿。发熱十日。脉浮而數。飲食如故。厚朴七物湯主之。

\section{大承气湯}

陽明病。脉遲。雖汗出。不惡寒。其身必重。短气。腹滿而喘。有潮熱。如此者。其外为解。可攻其裏。若手足濈然汗出者。此大便已堅。{\khaaitp 大}承气湯主之。若汗多。微发熱。惡寒者。为外未解。{\khaaitp 桂枝湯主之。}其熱不潮。未可与承气湯。若腹大滿。而不大便者。可与小承气湯。微和其胃气。勿令至大下。208

陽明病。潮熱。大便微堅者。可与{\khaaitp 大}承气湯。不堅者。不可与之。若不大便六七日。恐有燥屎。欲知之法。可少与小承气湯。若腹中轉失气者。此有燥屎也。乃可攻之。若不轉失气者。此但頭堅後溏。不可攻之。攻之必腹滿。不能食也。欲飲水者。与水即噦。其後发熱者。必大便復堅而少也。以小承气湯和之。若不轉失气者。慎不可攻之。209

傷寒。吐下後未解。不大便五六日。至十餘日。其人日晡所发潮熱。不惡寒。獨語。如見鬼{\khaaitp 神之}狀。若劇者。发則不識人。順衣妄撮。怵惕不安。微喘。直視。脉弦者生。濇者死。{\khaaitp 若}微者。但发熱。譫語。{\khaaitp 大}承气湯主之。若一服利。止後服。212

陽明病。譫語。有潮熱。反不能食者。{\khaaitp 胃中}必有燥屎五六枚。若能食者。但堅耳。{\khaaitp 大}承气湯主之。215

汗出。譫語者。以有燥屎在胃中。此風也。{\khaaitp 須下者。}過經乃可下之。下之若早。語言必亂。以表虗裏実故也。下之則愈。宜{\khaaitp 大}承气湯。217

二陽并病。太陽証罷。但发潮熱。手足漐漐汗出。大便難。而譫語者。下之則愈。宜{\khaaitp 大}承气湯。220

陽明病。下之。心中懊憹而煩。胃中有燥屎者。可攻。其人腹微滿。頭堅後溏者。不可攻之。若有燥屎者。宜{\khaaitp 大}承气湯。238

病者煩熱。汗出即解。復如瘧狀。日晡所发者。屬陽明。脉実者。当下之。脉浮虗者。当发其汗。下之宜{\khaaitp 大}承气湯。发汗宜桂枝湯。240

大下後。六七日不大便。煩不解。腹滿痛者。此有燥屎。所以然者。本有宿食故也。宜{\khaaitp 大}承气湯。241

病者小便不利。大便乍難乍易。时有微熱。怫㥜不能卧者。有燥屎故也。宜{\khaaitp 大}承气湯。242

得病二三日。脉弱。无太陽柴胡証。煩躁。心下堅。至四日。雖能食。以{\khaaitp 小}承气湯少与。微和之。令小安。至六日。与承气湯一升。若不大便六七日。小便少者。雖不大便。但頭堅後溏。未定成堅。攻之必溏。当須小便利。屎定堅。乃可攻之。宜{\khaaitp 大}承气湯。251

傷寒六七日。目中不了了。睛不和。无表{\khaaitp 裏}証。大便難。身微熱者。此为実也。急下之。宜{\khaaitp 大}承气湯。252

陽明病。发熱。汗多者。急下之。宜{\khaaitp 大}承气湯。253

发汗不解。腹滿痛者。急下之。宜{\khaaitp 大}承气湯。254

腹滿不減。減不足言。当下之。宜{\khaaitp 大}承气湯。255

脉滑而數者。有宿食也。当下之。宜{\khaaitp 大}承气湯。256

少陰病。得之二三日。口燥。咽乾者。急下之。宜{\khaaitp 大}承气湯。320

少陰病。{\khaaitp 下}利清水。色青者。心下必痛。口乾燥者。急下之。宜{\khaaitp 大}承气湯。321

少陰病六七日。腹滿。不大便者。急下之。宜{\khaaitp 大}承气湯。322

下利。三部脉皆平。按之心下堅者。急下之。宜{\khaaitp 大}承气湯。

下利。脉遲而滑者。{\khaaitp 内}実也。利未欲止。当下之。宜{\khaaitp 大}承气湯。

問曰。人病有宿食。何以別之。\\
師曰。寸口脉浮大。按之反濇。尺中亦微而濇。故知有宿食。当下之。宜{\khaaitp 大}承气湯。

下利。不欲食者。有宿食也。当下之。宜{\khaaitp 大}承气湯。

下利{\khaaitp 已}差。至其时復发者。此为病不{\sungtpii 𥁞}。当復下之。宜{\khaaitp 大}承气湯。

下利。脉反滑{\khaaitp 者}。当有所去。下乃愈。宜大承气湯。

病腹中滿痛者。为実。当下之。宜大承气湯。

脉雙弦而遲。心下堅。脉大而緊者。陽中有陰也。可下之。宜{\khaaitp 大}承气湯。

{\khaaitp 剛}痙为病。胸滿。口噤。卧不著席。腳攣急。其人必齘齒。可与大承气湯。

產婦鬱{\khaaitp 冒}。其脉微弱。不能食。大便反堅。但頭汗出。所以然者。血虗而厥。厥而必冒。冒家欲解。必大汗出。以血虗下厥。孤陽上出。故但頭汗出。所以產婦喜汗出者。亡陰血虗。陽气獨盛。故当汗出。陰陽乃復。所以便堅者。嘔。不能食也。小柴胡湯主之。病解。能食。七八日。而更发熱者。此为胃熱气実。大承气湯主之。{\wuben}

產婦鬱冒。其脉微弱。不能食。大便反堅。但頭汗出。所以然者。血虗而厥。厥而必冒。冒家欲解。必大汗出。以血虗下厥。孤陽上出。故頭汗出。所以產婦喜汗出者。亡陰血虗。陽气獨盛。故当汗出。陰陽乃復。大便堅。嘔。不能食。小柴胡湯主之。病解。能食。七八日。更发熱者。此为胃実。大承气湯主之。{\dengben}

婦人產後七八日。无太陽証。少腹堅痛。此惡露不{\sungtpii 𥁞}。不大便四五日。趺陽脉微実。再倍其人发熱。日晡所煩躁者。不食。食即譫語。利之即愈。宜大承气湯。熱在裏。結在膀胱也。{\wuben}

{\khaaitp 婦人}產後七八日。无太陽証。少腹堅痛。此惡露不{\sungtpii 𥁞}。不大便。煩躁。发熱。切脉微実。再倍发熱。日晡时煩躁者。不食。食則譫語。至夜即愈。宜大承气湯主之。熱在裏。結在膀胱也。{\dengben}

\section{大黄黄連瀉心湯
	\footnote{
		唐弘宇按:「大黄黄連瀉心湯」与「瀉心湯」,可能是一方的不同名称,也可能是兩個不同方劑。我採取保守処理,即把他們看作兩個方劑:大黄黄連瀉心湯,无黄芩,麻沸湯漬服,瀉熱力量較弱,主治心下熱痞証;瀉心湯,有黄芩,水煎服,瀉熱力量較强,主治血熱妄行吐衄。第156條、第164條、「婦人吐涎沫」條,雖云「瀉心湯」,但其主治是「心下痞」,可知此三條中的「瀉心湯」是大黄黄連瀉心湯的簡称,故將其歸在大黄黄連瀉心湯下。
	}
}

大黄{\scriptsize 二兩} 黄連{\scriptsize 一兩}\\
右二味。以麻沸湯漬之。須臾。絞去滓。分温再服。{\scriptsize 臣億等看詳。大黄黄連瀉心湯。諸本皆二味。又後附子瀉心湯用大黄黄連黄芩附子。恐是前方中亦有黄芩。後但加附子也。故後云附子瀉心湯。本云加附子是也。}

心下痞。按之濡。其脉関上浮者。大黄{\khaaitp 黄連}瀉心湯主之。154

本以下之。故心下痞。与瀉心湯。痞不解。其人渴而口燥{\khaaitp 煩}。小便不利。五苓散主之。156

傷寒。大下後。復发汗。心下痞。惡寒者。表未解也。不可攻痞。当先解表。表解乃可攻痞。解表宜桂枝湯。攻痞宜大黄黄連瀉心湯。164

婦人吐涎沫。醫反下之。心下即痞。当先治其吐涎沫。宜小青龙湯。涎沫止。乃治痞。宜瀉心湯。

\section{瀉心湯}

大黄{\scriptsize 二兩} 黄連{\scriptsize 一兩} 黄芩{\scriptsize 一兩}\\
右三味。㕮咀。以水三升。煮取一升。頓服。亦治霍亂。{\scriptsize 傷寒論以麻沸湯漬服之。見千金。}

心气不足。吐血。衄血。瀉心湯主之。

\section{附子瀉心湯}

心下痞。而復惡寒。汗出者。附子瀉心湯主之。155

\section{大黄附子湯}

脇下偏痛。{\khaaitp 发熱。}其脉緊弦。此寒也。{\khaaitp 当}以温藥下之。宜大黄附子湯。

\section{大黄甘草湯}

大黄{\scriptsize 四兩} 甘草{\scriptsize 一兩炙}\\
右二味。㕮咀。以水三升。煮取一升。去滓。分温再服。

食已即吐者。大黄甘草湯主之。

\section{調胃承气湯}

芒硝{\scriptsize 半升} 甘草{\scriptsize 二兩炙} 大黄{\scriptsize 四兩去皮清酒洗}\\
右三味。以水三升。煮取一升。去滓。内芒硝。更煮兩沸。頓服。

傷寒。脉浮。自汗出。小便數。心煩。微惡寒。腳攣急。反与桂枝湯。欲攻其表。得之便厥。咽乾。煩躁。吐{\sungtpii 𠱘}者。当作甘草乾薑湯。以復其陽。若厥愈。足温者。更作芍藥甘草湯与之。其腳即伸。若胃气不和。譫語者。少与{\khaaitp 調胃}承气湯。若重发汗。復加燒針者。四逆湯主之。29

发汗不解。反惡寒者。虗故也。芍藥甘草附子湯主之。不惡寒。但熱者。実也。当和胃气。宜調胃承气湯。68.70

太陽病未解。脉陰陽俱微。必先振汗出而解。但陽{\khaaitp 脉}微者。先汗之而解。但陰{\khaaitp 脉}微者。先下之而解。汗之宜桂枝湯。下之宜{\khaaitp 調胃}承气湯。94

傷寒十三日。過經。譫語者。内有熱也。当以湯下之。小便利者。大便当堅。而反{\khaaitp 下}利。脉調和者。知醫以丸藥下之。非其治也。自利者。脉当微厥。今反和者。此为内実也。{\khaaitp 調胃}承气湯主之。105

太陽病。過經十餘日。心下温温欲吐。而胸中痛。大便反溏。腹微滿。鬱鬱微煩。先{\khaaitp 此}时自極吐下者。与{\khaaitp 調胃}承气湯。若不尔者。不可与。但欲嘔。胸中痛。微溏者。此非柴胡湯証。以嘔。故知極吐下也。123

陽明病。不吐下而{\khaaitp 心}煩者。可与{\khaaitp 調胃}承气湯。207

太陽病三日。发汗不解。蒸蒸发熱者。{\khaaitp 屬胃也。調胃}承气湯主之。248

傷寒吐後。腹胀滿者。与{\khaaitp 調胃}承气湯。249

\section{橘皮大黄朴消湯}

\section{桃仁承气湯}

太陽病不解。熱結膀胱。其人如狂。血自下。下之即愈。其外不解者。尚未可攻。当先解其外。{\khaaitp 宜桂枝湯。}外解已。{\khaaitp 但}少腹急結者。乃可攻之。宜桃仁承气湯。106

\section{大黄牡丹湯}

大黄{\scriptsize 四兩} 牡丹{\scriptsize 一兩} 桃仁{\scriptsize 五十枚去皮尖} 瓜子{\scriptsize 半升} 芒硝{\scriptsize 三合}\\
右五味。㕮咀。以水六升。煮取一升。去滓。内芒硝。再煎一沸。頓服之。有膿当下。如无当下血。

腸癰者。少腹腫痞。按之即痛如淋。小便自調。时时发熱。自汗出。復惡寒。脉遲緊者。膿未成。可下之。当有血。脉洪數者。膿已成。不可下也。大黄牡丹湯主之。

\section{大黄甘遂湯}

大黄{\scriptsize 四兩} 甘遂{\scriptsize 二兩} 阿膠{\scriptsize 二兩}\\
右三味。㕮咀。以水三升。煮取一升。去滓。頓服。其血当下。

婦人少腹滿如敦狀。小便微難而不渴。生後者。此为水与血并結在血室也。大黄甘遂湯主之。

\section{下瘀血湯}

產婦腹痛。法当与枳実芍藥散。假令不愈者。此为腹中有乾血著脐下。宜下瘀血湯主之。

\section{抵当湯}

水蛭{\scriptsize 三十個熬} 蝱虫{\scriptsize 三十個去翅足熬} 桃仁{\scriptsize 二十個去皮尖} 大黄{\scriptsize 三兩酒洗}\\
右四味。以水五升。煮取三升。去滓。温服一升。不下更服。{\zhaoben}

水蛭{\scriptsize 三十枚熬} 蝱虫{\scriptsize 三十枚去足翅熬} 桃仁{\scriptsize 二七枚去皮尖熬} 大黄{\scriptsize 三兩}\\
右四味。㕮咀。以水五升。煮取三升。去滓。温服一升。当血下。不下再服。\\
亦治男子膀胱滿急。有瘀血者。{\wuben}

水蛭{\scriptsize 三十個熬} 蝱虫{\scriptsize 三十枚熬去翅足} 桃仁{\scriptsize 廿個去皮尖} 大黄{\scriptsize 三兩酒浸}\\
右四味。为末。以水五升。煮取三升。去滓。温服一升。{\dengben}

太陽病六七日。表証仍在。脉微而沈。反不結胸。其人发狂。此熱在下焦。少腹当堅滿。小便自利者。下血乃愈。所以然者。以太陽隨經。瘀熱在裏故也。抵当湯主之。124

太陽病。身黄。脉沈結。少腹堅。小便不利者。为无血也。小便自利。其人如狂者。血証諦也。抵当湯主之。125

陽明証。其人喜忘者。必有畜血。所以然者。本有久瘀血。故令喜忘。屎雖堅。大便反易。其色必黑。抵当湯主之。237

病人无表裏証。发熱七八日。雖脉浮數。可下之。{\khaaitp 宜大柴胡湯。}假令下已。脉數不解。合熱則消穀善飢。至六七日。不大便者。有瘀血。宜抵当湯。若脉數不解。而下不止。必挾熱。便膿血。257.258

婦人經水不利。抵当湯主之。

\section{抵当丸}

水蛭{\scriptsize 二十個熬} 蝱虫{\scriptsize 二十個去翅足熬} 桃仁{\scriptsize 二十五個去皮尖} 大黄{\scriptsize 三兩}\\
右四味。擣。分四丸。以水一升。煮一丸。取七合服之。晬时当下血。若不下者更服。

傷寒。有熱。少腹滿。應小便不利。今反利者。为有血也。当下之。宜抵当丸。126

\section{土瓜根散}

土瓜根{\scriptsize 三分} 芍藥{\scriptsize 三分} 桂枝{\scriptsize 三分去皮} 䗪虫{\scriptsize 三分熬}\\
右四味。杵为散。酒服方寸匕。日三服。{\khaaitp 陰顛腫亦主之。}

{\khaaitp 婦人}帶下。經水不利。少腹滿痛。經一月再見者。土瓜根散主之。

\section{甘草湯}

甘草{\scriptsize 二兩}\\
右一味。以水三升。煮取一升半。去滓。温服七合。日再服。

少陰病二三日。咽痛者。可与甘草湯。不差者。与桔梗湯。311

\section{桔梗湯}

桔梗{\scriptsize 一兩} 甘草{\scriptsize 二兩}\\
右二味。以水三升。煮取一升。去滓。分温再服。

少陰病二三日。咽痛者。可与甘草湯。不差者。与桔梗湯。311

欬而胸滿。振寒。脉數。咽乾。不渴。时出濁唾腥臭。久久吐膿如米粥者。为肺癰。桔梗湯主之。

\section{排膿湯}

甘草{\scriptsize 二兩炙} 桔梗{\scriptsize 三兩} 生薑{\scriptsize 一兩切} 大棗{\scriptsize 十枚擘}\\
右四味。㕮咀。以水三升。煮取一升。去滓。温服五合。日再服。

\section{芍藥甘草湯}

芍藥{\scriptsize 四兩} 甘草{\scriptsize 四兩炙}\\
右二味。以水三升。煮取一升五合。去滓。分温再服。

傷寒。脉浮。自汗出。小便數。心煩。微惡寒。腳攣急。反与桂枝湯。欲攻其表。得之便厥。咽乾。煩躁。吐{\sungtpii 𠱘}者。当作甘草乾薑湯。以復其陽。若厥愈。足温者。更作芍藥甘草湯与之。其腳即伸。若胃气不和。譫語者。少与{\khaaitp 調胃}承气湯。若重发汗。復加燒針者。四逆湯主之。29

\section{芍藥甘草附子湯}

芍藥{\scriptsize 三兩} 甘草{\scriptsize 三兩炙} 附子{\scriptsize 一枚炮去皮破八片}\\
右三味。以水五升。煮取一升五合。去滓。分温三服。疑非仲景方。

发汗不解。反惡寒者。虗故也。芍藥甘草附子湯主之。不惡寒。但熱者。実也。当和胃气。宜調胃承气湯。68.70

\section{甘遂半夏湯}

甘遂{\scriptsize 大者三枚} 半夏{\scriptsize 十二枚洗以水一升煮取半升去滓} 芍藥{\scriptsize 五枚} 甘草{\scriptsize 如指大一枚炙一本无}\\
右四味。㕮咀。以水二升。煮取半升。去滓。以蜜半升。和藥汁。煮取八合。頓服之。

病者脉伏。其人欲自利。利反快。雖利。心下續堅滿。此为留飲欲去故也。甘遂半夏湯主之。

\section{甘草小麥大棗湯}

甘草{\scriptsize 三兩炙} 小麥{\scriptsize 一升} 大棗{\scriptsize 十枚擘}\\
右三味。㕮咀。以水六升。煮取三升。去滓。分温三服。亦補脾气。

婦人臓躁。喜悲傷。欲哭。象如神靈所作。數欠伸。甘草小麥大棗湯主之。

\section{甘草粉蜜湯}

甘草{\scriptsize 二兩炙} 粉{\scriptsize 一兩} 蜜{\scriptsize 四兩}\\
右三味。㕮咀。以水三升。先煮甘草。取二升。去滓。内粉。蜜。攪令和。煎如薄粥。温服一升。差即止。

蛔虫之为病。令人吐涎。心痛。发作有时。毒藥不止。甘草粉蜜湯主之。

\section{生薑甘草湯}

生薑{\scriptsize 五兩切} 人参{\scriptsize 二兩} 甘草{\scriptsize 四兩炙} 大棗{\scriptsize 十五枚擘}\\
右四味。㕮咀。以水七升。煮取三升。去滓。分温三服。

肺痿。欬唾涎沫不止。咽燥而渴。生薑甘草湯主之。

\section{甘草乾薑湯}

甘草{\scriptsize 四兩炙} 乾薑{\scriptsize 二兩}\\
右二味。以水三升。煮取一升五合。去滓。分温再服。

傷寒。脉浮。自汗出。小便數。心煩。微惡寒。腳攣急。反与桂枝湯。欲攻其表。得之便厥。咽乾。煩躁。吐{\sungtpii 𠱘}者。当作甘草乾薑湯。以復其陽。若厥愈。足温者。更作芍藥甘草湯与之。其腳即伸。若胃气不和。譫語者。少与{\khaaitp 調胃}承气湯。若重发汗。復加燒針者。四逆湯主之。29

肺痿。吐涎沫。而不能欬者。其人不渴。必遺尿。小便數。所以然者。以上虗不能制下故也。此为肺中冷。必眩。甘草乾薑湯以温其病。{\wuben}

肺痿。吐涎沫。而不欬者。其人不渴。必遺尿。小便數。所以然者。以上虗不能制下故也。此为肺中冷。必眩。多涎唾。甘草乾薑湯以温之。若服湯已。渴者。屬消渴。{\dengben}

\section{乾薑附子湯}

乾薑{\scriptsize 一兩} 附子{\scriptsize 一枚生用去皮切八片}\\
右二味。以水三升。煮取一升。去滓。頓服。

下之後。復发汗。晝日煩躁不得眠。夜而安靜。不嘔。不渴。无表証。脉沈微。身无大熱。乾薑附子湯主之。61

\section{四逆湯}

甘草{\scriptsize 二兩炙} 乾薑{\scriptsize 一兩半} 附子{\scriptsize 一枚生用去皮破八片}\\
右三味。{\khaaitp 㕮咀。}以水三升。煮取一升二合。去滓。分温再服。强人可大附子一枚。乾薑三兩。

傷寒。脉浮。自汗出。小便數。心煩。微惡寒。腳攣急。反与桂枝湯。欲攻其表。得之便厥。咽乾。煩躁。吐{\sungtpii 𠱘}者。当作甘草乾薑湯。以復其陽。若厥愈。足温者。更作芍藥甘草湯与之。其腳即伸。若胃气不和。譫語者。少与{\khaaitp 調胃}承气湯。若重发汗。復加燒針者。四逆湯主之。29

傷寒。醫下之。續得下利。清穀不止。身体疼痛。急当救裏。後身体疼痛。清便自調。急当救表。救裏宜四逆湯。救表宜桂枝湯。91

病发熱。頭痛。脉反沈。若不差。身体疼痛。当救其裏。宜四逆湯。92

{\khaaitp 陽明病。}脉浮而遲。表熱裏寒。下利清穀者。四逆湯主之。225

自利。不渴者。屬太陰。以其臓有寒故也。当温之。宜四逆輩。277

少陰病。脉沈者。急温之。宜四逆湯。323

少陰病。其人飲食入則吐。心中温温欲吐。復不能吐。始得之。手足寒。脉弦遲。此胸中実。不可下也。当吐之。若膈上有寒飲。乾嘔者。不可吐。当温之。宜四逆湯。324

大汗出。熱不去。内拘急。四肢疼。{\khaaitp 又}下利。厥逆而惡寒。四逆湯主之。353

大汗{\khaaitp 出}或大下利。而厥冷者。四逆湯主之。354

下利。腹{\khaaitp 胀}滿。身体疼痛者。先温其裏。乃攻其表。温裏宜四逆湯。攻表宜桂枝湯。372

嘔而脉弱。小便復利。身有微熱。見厥者。難治。四逆湯主之。377

吐利。汗出。发熱。惡寒。四肢拘急。手足厥冷。四逆湯主之。388

既吐且利。小便復利。而大汗出。下利清穀。裏寒外熱。脉微欲絕。四逆湯主之。388

\section{通脉四逆湯*}

甘草{\scriptsize 二兩炙} 附子{\scriptsize 大者一枚生用去皮破八片} 乾薑{\scriptsize 三兩强人可四兩}\\
右三味。以水三升。煮取一升二合。去滓。分温再服。其脉即出者愈。

{\khaaitp 若}面赤者。加蔥{\khaaitp 白}九莖。\\
{\khaaitp 若}腹痛者。加芍藥二兩。\\
{\khaaitp 若}嘔者。加生薑二兩。\\
{\khaaitp 若}咽痛者。加桔梗一兩。\\
{\khaaitp 若}利止而脉不出者。加人参二兩。\\
{\khaaitp 病皆与方相應者乃服之。}
	\footnote{
		「加芍藥」玉函同,其餘諸本均作「去蔥加芍藥」。「加桔梗」前諸本均有「去芍藥」三字,編者刪。「加人参」前諸本均有「去桔梗」三字,編者刪。「病皆」至「服之」十字,玉函、成本无。
	}

少陰病。下利清穀。裏寒外熱。手足厥逆。脉微欲絕。身反不惡寒。其人面赤。或腹痛。或乾嘔。或咽痛。或利止而脉不出。通脉四逆湯主之。317

下利清穀。裏寒外熱。汗出而厥。通脉四逆湯主之。370

\section{四逆加人参湯}

甘草{\scriptsize 二兩炙} 附子{\scriptsize 一枚生去皮破八片} 乾薑{\scriptsize 一兩半} 人参{\scriptsize 一兩}\\
右四味。以水三升。煮取一升二合。去滓。分温再服。

惡寒。脉微。而復利。利止。亡血也。四逆加人参湯主之。385

\section{茯苓四逆湯}

茯苓{\scriptsize 四兩} 人参{\scriptsize 一兩} 附子{\scriptsize 一枚生用去皮破八片} 甘草{\scriptsize 二兩炙} 乾薑{\scriptsize 一兩半}\\
右五味。以水五升。煮取三升。去滓。温服七合。日二服。

发汗或下之。{\khaaitp 病仍}不解。煩躁。茯苓四逆湯主之。69

\section{通脉四逆加豬膽汁湯}

甘草{\scriptsize 二兩炙} 乾薑{\scriptsize 三兩强人可四兩} 附子{\scriptsize 大者一枚生去皮破八片} 豬膽汁{\scriptsize 半合}\\
右四味。以水三升。煮取一升二合。去滓。内豬膽汁。分温再服。其脉即來。无豬膽。以羊膽代之。

吐利已斷。汗出而厥。四肢拘急不解。脉微欲絕。通脉四逆加豬膽汁湯主之。390

\section{白通湯}

葱白{\scriptsize 四莖} 乾薑{\scriptsize 一兩} 附子{\scriptsize 一枚生去皮破八片}\\
右三味。以水三升。煮取一升。去滓。分温再服。

少陰病。下利。白通湯主之。314

少陰病。下利。脉微。服白通湯。利不止。厥逆。无脉。乾嘔。煩者。白通加豬膽汁湯主之。服湯脉暴出者死。微{\khaaitp 微}續{\khaaitp 出}者生。315

\section{白通加豬膽汁湯}

蔥白{\scriptsize 四莖} 乾薑{\scriptsize 一兩} 附子{\scriptsize 一枚生去皮破八片} 人尿{\scriptsize 五合} 豬膽汁{\scriptsize 一合}\\
右五味。以水三升。煮取一升。去滓。内膽汁。人尿。和令相得。分温再服。若无膽亦可用。

少陰病。下利。脉微。服白通湯。利不止。厥逆。无脉。乾嘔。煩者。白通加豬膽汁湯主之。服湯脉暴出者死。微{\khaaitp 微}續{\khaaitp 出}者生。315

\section{玄武湯*}

茯苓{\scriptsize 三兩} 芍藥{\scriptsize 三兩} 生薑{\scriptsize 三兩} 白术{\scriptsize 二兩} 附子{\scriptsize 一枚。炮。去皮。破八片}\\
右五味。以水八升。煮取三升。去滓。温服七合。日三服。

若欬者。加五味子半升。細辛一兩。乾薑一兩。\\
若小便自利者。去茯苓。\\
若不利者。去芍藥。加乾薑二兩。\\
若嘔者。去附子。加生薑至八兩。

太陽病。发汗。汗出不解。其人仍发熱。心下悸。頭眩。身瞤動。振振欲躃地。玄武湯主之。82

少陰病。二三日不已。至四五日。腹痛。小便不利。四肢沈重疼痛而利。此为有水气。其人或欬。或小便{\khaaitp 自}利。或不利。或嘔。玄武湯主之。316

\section{附子湯}

附子{\scriptsize 二枚炮去皮破八片} 茯苓{\scriptsize 三兩} 人参{\scriptsize 二兩} 白术{\scriptsize 四兩} 芍藥{\scriptsize 三兩}\\
右五味。以水八升。煮取三升。去滓。温服一升。日三服。

少陰病。得之一二日。口中和。其背惡寒者。当灸之。附子湯主之。304

少陰病。身体痛。手足寒。骨節痛。脉沈者。附子湯主之。305

婦人懷娠六七月。脉弦。发熱。其胎愈胀。腹痛。惡寒者。少腹如扇之狀。所以然者。子臓開故也。当以附子湯温其臓。{\scriptsize 方未見。}

\section{附子粳米湯}

附子{\scriptsize 一枚炮去皮破八片} 半夏{\scriptsize 半升洗} 甘草{\scriptsize 一兩炙} 大棗{\scriptsize 十枚擘} 粳米{\scriptsize 半升}\\
右五味。㕮咀。以水八升。煮米熟。湯成去滓。温服一升。日三服。

腹中寒气。雷鳴。切痛。胸脇逆滿。嘔吐。附子粳米湯主之。

\section{赤丸}

茯苓{\scriptsize 四兩} 半夏{\scriptsize 四兩洗} 細辛{\scriptsize 一兩} 烏頭{\scriptsize 二兩炮去皮} 附子{\scriptsize 二兩炮去皮} 射罔{\scriptsize 一枚如棗大}\\
右六味。末之。内真朱为色。煉蜜和丸。如麻子大。先食酒飲服一丸。日再夜一服。不知。二丸为度。

寒气厥逆。赤丸主之。

\section{大烏頭煎}

烏頭{\scriptsize 十五枚熬黑不㕮咀}\\
右一味。以水三升。煮取一升。去滓。内蜜二升。煎。令水气{\sungtpii 𥁞}。取二升。强人服七合。弱人服五合。不差。明日更服。慎不可一日再服。

腹痛。脉弦而緊。弦則衛气不行。{\khaaitp 衛气不行}即惡寒。緊則不欲食。邪正相摶。即为寒疝。寒疝遶脐痛。若发則白汗出。手足厥冷。其脉沈弦者。大烏頭煎主之。

\section{烏頭湯}

烏頭{\scriptsize 五枚㕮咀以蜜二升煎取一升即取烏頭} 甘草{\scriptsize 炙} 麻黄{\scriptsize 三兩去節} 芍藥{\scriptsize 三兩} 黄耆{\scriptsize 三兩}\\
右五味。㕮咀四味。以水三升。煮取一升。去滓。内蜜煎中。更煎之。服七合。不知。{\sungtpii 𥁞}服之。
\footnote{本方吳本甘草麻黄无劑量,鄧本甘草无劑量。}

病歷節。疼痛。不可屈伸。烏頭湯主之。

烏頭湯。治腳气。疼痛。不可屈伸。

烏頭湯。治寒疝。腹中絞痛。賊風入攻五臟。拘急不得轉側。发作有时。使人陰縮。手足厥逆。

\section{薏苡附子散}

薏苡仁{\scriptsize 十五兩} 大附子{\scriptsize 十枚炮}\\
右二味。杵为散。服方寸匕。日三服。

胸痹緩急者。薏苡附子散主之。

\section{薏苡仁附子敗醬散}

薏苡仁{\scriptsize 十分} 附子{\scriptsize 二分炮去皮} 敗醬{\scriptsize 五分}\\
右三味。杵为末。取方寸匕。以水二升。煎取一升。頓服之。小便当下。

腸癰之为病。其身甲錯。腹皮急。按之濡。如腫狀。腹无積聚。身无熱。脉數。此为腸内有{\khaaitp 癰}膿。薏苡{\khaaitp 仁}附子敗醬散主之。

\section{天雄散}

天雄{\scriptsize 三兩炮去皮} 白术{\scriptsize 八兩} 桂枝{\scriptsize 六兩} 龙骨{\scriptsize 三兩}\\
右四味。杵为散。酒服半錢匕。不知。稍增之。

夫失精家。少腹弦急。陰頭寒。目眩。髮落。脉極虗芤遲。为清穀。亡血。失精。脉得諸芤動微緊。男子失精。女子夢交。桂枝加龙骨牡蛎湯主之。天雄散亦主之。

\section{蜀漆散}

蜀漆{\scriptsize 洗去腥} 雲母{\scriptsize 燒之三日三夜} 龙骨{\scriptsize 等分}\\
右三味。杵为散。未发前以漿水服半錢。温瘧加蜀漆半分。臨发时服一錢匕。{\scriptsize 一方雲母作雲石。}

瘧。多寒者。名曰牡瘧。蜀漆散主之。

\section{栀子豉湯}

栀子{\scriptsize 十四個擘} 香豉{\scriptsize 四合綿裹}\\
右二味。以水四升。先煮栀子。得二升半。内豉。煮取一升半。去滓。分为二服。温進一服。得吐者。止後服。

发汗吐下後。虗煩。不得眠。若劇者。反覆顛倒。心中懊憹。栀子{\khaaitp 豉}湯主之。若少气者。栀子甘草{\khaaitp 豉}湯主之。若嘔者。栀子生薑{\khaaitp 豉}湯主之。76

发汗或下之。煩熱。胸中窒者。栀子{\khaaitp 豉}湯主之。77

傷寒五六日。大下之後。身熱不去。心中結痛者。未欲解也。栀子{\khaaitp 豉}湯主之。78

凡用栀子湯。其人微溏者。不可与服之。81

陽明病。脉浮緊。咽乾。口苦。腹滿而喘。发熱。汗出。不惡寒。反惡熱。身重。若发汗則躁。心憒憒。反譫語。若加温針。必怵惕。煩躁。不得眠。若下之。則胃中空虗。客气動膈。心中懊憹。舌上胎者。栀子{\khaaitp 豉}湯主之。若渴欲飲水。口乾舌燥者。白虎{\khaaitp 加人参}湯主之。若脉浮。发熱。渴欲飲水。小便不利者。豬苓湯主之。221.222.223

陽明病。下之。其外有熱。手足温。不結胸。心中懊憹。飢不能食。但頭汗出。栀子{\khaaitp 豉}湯主之。228

下利後更煩。按之心下濡者。为虗煩也。栀子{\khaaitp 豉}湯主之。375

\section{栀子甘草豉湯}

栀子{\scriptsize 十四個擘} 甘草{\scriptsize 二兩炙} 香豉{\scriptsize 四合綿裹}\\
右三味。以水四升。先煮栀子甘草。取二升半。内豉。煮取一升半。去滓。分二服。温進一服。得吐者。止後服。

发汗吐下後。虗煩。不得眠。若劇者。反覆顛倒。心中懊憹。栀子{\khaaitp 豉}湯主之。若少气者。栀子甘草{\khaaitp 豉}湯主之。若嘔者。栀子生薑{\khaaitp 豉}湯主之。76

\section{栀子生薑豉湯}

栀子{\scriptsize 十四個擘} 生薑{\scriptsize 五兩} 香豉{\scriptsize 四合綿裹}\\
右三味。以水四升。先煮栀子生薑。取二升半。内豉。煮取一升半。去滓。分二服。温進一服。得吐者。止後服。

发汗吐下後。虗煩。不得眠。若劇者。反覆顛倒。心中懊憹。栀子{\khaaitp 豉}湯主之。若少气者。栀子甘草{\khaaitp 豉}湯主之。若嘔者。栀子生薑{\khaaitp 豉}湯主之。76

\section{枳実栀子湯}

大病差後勞復者。枳実栀子湯主之。393

\section{栀子枳実豉大黄湯}

酒黄疸。心中懊憹。或熱痛。栀子{\khaaitp 枳実豉}大黄湯主之。

\section{大黄黄蘗栀子硝石湯}

黄疸。腹滿。小便不利而赤。自汗出。此为表和裏実。当下之。宜大黄{\khaaitp 黄蘗栀子}硝石湯。

\section{茵陳蒿湯}

陽明病。发熱。汗出者。此为熱越。不能发黄也。但頭汗出。身无汗。齐頸而還。小便不利。渴引水漿者。此为瘀熱在裏。身必发黄。茵陳{\khaaitp 蒿}湯主之。236

傷寒七八日。身黄如橘。小便不利。腹微滿者。茵陳{\khaaitp 蒿}湯主之。260

穀疸之为病。寒熱不食。食即頭眩。心胸不安。久久发黄。为穀疸。茵陳蒿湯主之。

\section{栀子蘗皮湯}

傷寒。身黄。发熱。栀子蘗皮湯主之。261

\section{栀子厚朴湯}

栀子{\scriptsize 十四個擘} 厚朴{\scriptsize 四兩炙去皮} 枳実{\scriptsize 四枚水浸炙令黄}\\
右三味。以水三升半。煮取一升半。去滓。分二服。温進一服。得吐者。止後服。

傷寒下後。煩而腹滿。卧起不安。栀子厚朴湯主之。79

\section{栀子乾薑湯}

栀子{\scriptsize 十四個擘} 乾薑{\scriptsize 二兩}\\
右二味。以水三升半。煮取一升半。去滓。分二服。一服得吐者。止後服。

傷寒。醫以丸藥大下之。身熱不去。微煩。栀子乾薑湯主之。80

\section{酸棗{\khaaitp 仁}湯}

酸棗仁{\scriptsize 二升} 甘草{\scriptsize 一兩炙} 知母{\scriptsize 二兩} 茯苓{\scriptsize 二兩} 芎藭{\scriptsize 二兩}\\
右五味。㕮咀。以水八升。煮酸棗仁。得六升。内諸藥。煮取三升。去滓。分温三服。{\scriptsize 深師有生薑二兩。}

虗勞。虗煩。不得眠。酸棗{\khaaitp 仁}湯主之。

\section{大陷胸湯}

大黄{\scriptsize 六兩去皮酒洗} 芒硝{\scriptsize 一升} 甘遂末{\scriptsize 一錢匕}\\
右三味。以水六升。煮大黄。取二升。去滓。内芒硝。煮兩沸。内甘遂末。温服一升。得快利。止後服。

太陽病。脉浮而動數。浮則为風。數則为熱。動則为痛。數則为虗。頭痛。发熱。微盜汗出。而反惡寒。其表未解。醫反下之。動數變遲。膈内拒痛。胃中空虗。客气動膈。短气。躁煩。心中懊憹。陽气内陷。心下因堅。則为結胸。大陷胸湯主之。若不結胸。但頭汗出。餘処无汗。齐頸而還。小便不利。身必发黄。134

傷寒六七日。結胸熱実。脉沈緊。心下痛。按之如石堅。大陷胸湯主之。135

傷寒十餘日。熱結在裏。復往來寒熱者。与大柴胡湯。但結胸。无大熱者。此为水結在胸脇。{\khaaitp 但}頭微汗出。大陷胸湯主之。136

太陽病。重发汗而復下之。不大便五六日。舌上燥而渴。日晡所小有潮熱。從心下至少腹堅滿而痛不可近。大陷胸湯主之。137

傷寒五六日。嘔而发熱。柴胡湯証具。而以他藥下之。柴胡証仍在者。復与柴胡湯。此雖已下之。不为逆。必蒸蒸而振。卻发熱汗出而解。若心下滿而堅痛者。此为結胸。宜大陷胸湯。若但滿而不痛者。此为痞。柴胡{\khaaitp 湯}不復中与也。宜半夏瀉心湯。149

\section{大陷胸丸}

結胸者。項亦强。如柔痙狀。下之則和。宜大陷胸丸。131

\section{小陷胸湯}

黄連{\scriptsize 一兩} 半夏{\scriptsize 半升洗} 栝蔞実{\scriptsize 大者一枚}\\
右三味。以水六升。先煮栝蔞。取三升。去滓。内諸藥。煮取二升。去滓。分温三服。

小結胸者。正在心下。按之則痛。其脉浮滑。小陷胸湯主之。138

\section{枳実薤白桂枝湯}

枳実{\scriptsize 四枚炙} 厚朴{\scriptsize 四兩炙} 薤白{\scriptsize 八兩切} 桂枝{\scriptsize 一兩去皮} 栝蔞実{\scriptsize 一枚擣}\\
右五味。㕮咀。以水五升。先煮枳実。厚朴。取二升。去滓。内諸藥。煮三沸。去滓。分温三服。

胸痹。心中痞。留气結在胸。胸滿。脇下逆{\khaaitp 气}搶心。枳実薤白桂枝湯主之。人参湯亦主之。

\section{栝蔞薤白白酒湯}

栝蔞実{\scriptsize 一枚擣} 薤白{\scriptsize 八兩切} 白酒{\scriptsize 七升}\\
右三味。同煮。取二升。去滓。分温再服。

胸痹之病。喘息欬唾。胸背痛。短气。寸口脉沈而遲。関上小緊數。栝蔞薤白白酒湯主之。

\section{栝蔞薤白半夏湯}

栝蔞実{\scriptsize 一枚擣} 薤白{\scriptsize 三兩切} 半夏{\scriptsize 半升洗切} 白酒{\scriptsize 一斗}\\
右四味。同煮。取四升。去滓。温服一升。日三服。

胸痹。不得卧。心痛徹背者。栝蔞薤白半夏湯主之。

\section{瓜蒂散}

瓜蒂{\scriptsize 一分熬黄} 赤小豆{\scriptsize 一分}\\
右二味。各別擣篩。为散已。合治之。取一錢匕。以香豉一合。用熱湯七合。煮作稀糜。去滓。取汁和散。温頓服之。不吐者。少少加。得快吐乃止。

病如桂枝証。頭不痛。項不强。寸{\khaaitp 口}脉微浮。胸中痞堅。气上衝咽喉。不得息。此为胸有寒。当吐之。宜瓜蒂散。166

病者手足厥冷。脉乍緊。邪結在胸中。心下滿而煩。飢不能食。病在胸中。当吐之。宜瓜蒂散。355

宿食在上脘。当吐之。宜瓜蒂散。

\section{小半夏湯}

半夏{\scriptsize 一升洗} 生薑{\scriptsize 八兩}\\
右二味。切。以水七升。煮取一升半。去滓。分温再服。

嘔家本渴。渴者为欲解。今反不渴。心下有支飲故也。小半夏湯主之。

黄疸病。小便色不變。欲自利。腹滿而喘。不可除熱。熱除必噦。噦者。小半夏湯主之。

諸嘔吐。穀不得下者。小半夏湯主之。

\section{小半夏加茯苓湯}

半夏{\scriptsize 一升洗} 生薑{\scriptsize 八兩} 茯苓{\scriptsize 三兩一方四兩}\\
右三味。切。以水七升。煮取一升五合。去滓。分温再服。

卒嘔吐。心下痞。膈間有水。眩悸者。{\khaaitp 小}半夏加茯苓湯主之。

先渴卻嘔。为水停心下。此屬飲家。小半夏加茯苓湯主之。{\wuben}

先渴後嘔。为水停心下。此屬飲家。小半夏茯苓湯主之。{\dengben}

\section{大半夏湯}

半夏{\scriptsize 三升洗完用} 人参{\scriptsize 三兩切} 白蜜{\scriptsize 一升}\\
右三味。以泉水一斗二升和蜜揚之二百四十遍煮藥。取二升半。去滓。温服一升。餘分再服。

胃反。嘔吐者。大半夏湯主之。

%千金方、外臺載有本方不同的條文,以後補充。

\section{生薑半夏湯}

生薑汁{\scriptsize 一升} 半夏{\scriptsize 半升洗切}\\
右二味。以水三升。煮半夏。取二升。内生薑汁。煮取一升半。去滓。小冷。分四服。日三夜一服。若一服止。停後服。

病人胸中似喘不喘。似嘔不嘔。似噦不噦。徹心中憒憒然无奈者。生薑半夏湯主之。

\section{苦酒湯}

半夏{\scriptsize 洗破如棗核十四枚} 雞子{\scriptsize 一枚去黄内上苦酒著雞子殼中}\\
右二味。内半夏著苦酒中。以雞子殼置刀環中。安火上。令三沸。去滓。少少含嚥之。不差。更作三劑。

少陰病。咽中傷。生瘡。不能語言。聲不出者。苦酒湯主之。312

\section{半夏厚朴湯}

半夏{\scriptsize 一升洗} 厚朴{\scriptsize 三兩炙} 茯苓{\scriptsize 四兩} 生薑{\scriptsize 五兩切} 乾蘇枼{\scriptsize 二兩}\\
右五味。㕮咀。以水七升。煮取四升。去滓。分温四服。日三夜一服。{\scriptsize 一作治胸滿。心下堅。咽中怗怗。如有炙肉。吐之不出。吞之不下。}

婦人咽中如有炙臠。半夏厚朴湯主之。

\section{半夏乾薑散}

半夏{\scriptsize 洗} 乾薑{\scriptsize 各等分}\\
右二味。杵为散。取方寸匕。漿水一升半。煎取七合。頓服之。

乾嘔。吐{\sungtpii 𠱘}。吐涎沫。半夏乾薑散主之。

\section{厚朴生薑半夏甘草人参湯}

厚朴{\scriptsize 八兩炙去皮} 生薑{\scriptsize 八兩切} 半夏{\scriptsize 半升洗} 甘草{\scriptsize 二兩} 人参{\scriptsize 一兩}\\
右五味。以水一斗。煮取三升。去滓。温服一升。日三服。

发汗後。腹胀滿者。厚朴{\khaaitp 生薑半夏甘草人参}湯主之。66

\section{乾薑人参半夏丸}

乾薑{\scriptsize 一兩} 人参{\scriptsize 一兩} 半夏{\scriptsize 半兩洗}\\
右三味。末之。以生薑汁和为丸。如梧子大。飲服一丸。日三服。

{\khaaitp 婦人}妊娠。嘔吐不止。乾薑人参半夏丸主之。

\section{乾薑黄芩黄連人参湯}

乾薑{\scriptsize 三兩} 黄芩{\scriptsize 三兩} 黄連{\scriptsize 三兩} 人参{\scriptsize 三兩}
右四味。以水六升。煮取二升。去滓。分温再服。

傷寒。本自寒下。醫復吐{\khaaitp 下}之。寒格。更逆吐{\khaaitp 下}。食入即出。乾薑黄芩黄連人参湯主之。359

\section{黄連湯}

黄連{\scriptsize 三兩} 甘草{\scriptsize 三兩炙} 乾薑{\scriptsize 三兩} 桂枝{\scriptsize 三兩} 人参{\scriptsize 二兩} 半夏{\scriptsize 半升洗} 大棗{\scriptsize 十二枚擘}
右七味。以水一斗。煮取六升。去滓。分温五服。晝三夜二。疑非仲景方。

傷寒。胸中有熱。胃中有邪气。腹中痛。欲嘔吐。黄連湯主之。173

\section{半夏瀉心湯}

半夏{\scriptsize 半升洗} 黄芩{\scriptsize 三兩} 乾薑{\scriptsize 三兩} 人参{\scriptsize 三兩} 甘草{\scriptsize 三兩炙} 黄連{\scriptsize 一兩} 大棗{\scriptsize 十二枚擘}\\
右七味。以水一斗。煮取六升。去滓。再煎。取三升。温服一升。日三服。{\zhaoben}

半夏{\scriptsize 半升洗} 黄芩{\scriptsize 二兩} 人参{\scriptsize 二兩} 甘草{\scriptsize 二兩炙} 乾薑{\scriptsize 二兩} 黄連{\scriptsize 一兩} 大棗{\scriptsize 十二枚擘}\\
右七味。㕮咀。以水一斗。煮取六升。去滓。再煎。取三升。温服一升。日三服。{\wuben}

傷寒五六日。嘔而发熱。柴胡湯証具。而以他藥下之。柴胡証仍在者。復与柴胡湯。此雖已下之。不为逆。必蒸蒸而振。卻发熱汗出而解。若心下滿而堅痛者。此为結胸。宜大陷胸湯。若但滿而不痛者。此为痞。柴胡{\khaaitp 湯}不復中与也。宜半夏瀉心湯。149

嘔而腸鳴。心下痞者。半夏瀉心湯主之。

\section{甘草瀉心湯}

甘草{\scriptsize 四兩炙} 黄芩{\scriptsize 三兩} 人参{\scriptsize 三兩} 乾薑{\scriptsize 三兩} 黄連{\scriptsize 一兩} 大棗{\scriptsize 十二枚擘} 半夏{\scriptsize 半升洗}\\
右七味。㕮咀。以水一斗。煮取六升。去滓。再煎。温服一升。日三服。{\scriptsize 臣億等謹按。上生薑瀉心湯法。本云理中人参黄芩湯。今詳瀉心以療痞。痞气因发陰而生。是半夏生薑甘草瀉心三方皆本於理中也。其方必各有人参。今甘草瀉心湯中无者。脱落之也。又按。千金并外臺祕要。治傷寒䘌食用此方。皆有人参。知脱落无疑。}
	\footnote{
		《影印孫思邈本傷寒論校注考證》:錢超塵「按:今考《千金要方》卷九甘草瀉心湯方,宋臣小註云「加人参三兩乃是」,又考《千金要方》卷十傷寒不发汗變成狐惑病第四亦載此方,今錄之如下:半夏{\scriptsize 半升}{ }黄芩{ }人参{ }乾薑{\scriptsize 各三兩}{ }黄連{\scriptsize 一兩}{ }甘草{\scriptsize 三兩}{ }大棗{\scriptsize 十二枚}。又考《外臺祕要》卷二傷寒狐惑病方亦載此方,錄之如下:半夏{\scriptsize 半升}{ }黄芩{\scriptsize 三兩}{ }人参{\scriptsize 三兩}{ }乾薑{\scriptsize 三兩}{ }黄連{\scriptsize 一兩}{ }甘草{\scriptsize 四兩炙}{ }大棗{\scriptsize 十二枚擘}。綜觀《千金要方》、《外臺祕要》所載之甘草瀉心湯方,確有人参三兩无疑,林億等所校是也。北宋校正醫書局所校之《傷寒論》每有極精闢処,此即其中一例也。」
	}

傷寒中風。醫反下之。其人下利。日數十行。穀不化。腹中雷鳴。心下痞堅而滿。乾嘔。心煩。不{\khaaitp 能}得安。醫見心下痞。谓病不{\sungtpii 𥁞}。復下之。其痞益甚。此非結熱。但以胃中虗。客气上逆。故使之堅。甘草瀉心湯主之。158

狐惑之为病。狀如傷寒。默默欲眠。目不得閉。卧起不安。蝕於喉为惑。蝕於陰为狐。不欲飲食。惡聞食臭。其面目乍赤。乍黑。乍白。蝕於上部則聲喝。甘草瀉心湯主之。蝕於下部則咽乾。苦参湯洗之。蝕於肛者。雄黄熏之。

\section{生薑瀉心湯}

生薑{\scriptsize 四兩切} 甘草{\scriptsize 三兩炙} 人参{\scriptsize 三兩} 乾薑{\scriptsize 一兩} 黄芩{\scriptsize 三兩} 半夏{\scriptsize 半升洗} 黄連{\scriptsize 一兩} 大棗{\scriptsize 十二枚擘}\\
右八味。以水一斗。煮取六升。去滓。再煎。取三升。温服一升。日三服。\\
附子瀉心湯。本云加附子。半夏瀉心湯。甘草瀉心湯。同体別名耳。生薑瀉心湯。本云理中人参黄芩湯。去桂枝术。加黄連。并瀉肝法。

傷寒。汗出。解之後。胃中不和。心下痞堅。乾噫食臭。脇下有水气。腹中雷鳴而利。生薑瀉心湯主之。157

\section{旋覆代赭湯}

旋覆花{\scriptsize 三兩} 人参{\scriptsize 二兩} 生薑{\scriptsize 五兩} 代赭{\scriptsize 一兩} 甘草{\scriptsize 三兩炙} 半夏{\scriptsize 半升洗} 大棗{\scriptsize 十二枚擘}\\
右七味。以水一斗。煮取六升。去滓。再煎。取三升。温服一升。日三服。

傷寒。发汗{\khaaitp 或}吐{\khaaitp 或}下。解後。心下痞堅。噫气不除者。旋覆代赭湯主之。161

\section{吳茱萸湯}

吳茱萸{\scriptsize 一升{\khaaitp 洗}} 人参{\scriptsize 三兩} 生薑{\scriptsize 六兩切} 大棗{\scriptsize 十二枚擘}\\
右四味。以水七升。煮取二升。去滓。温服七合。日三服。{\zhaoben}
	\footnote{
		「以水七升」、「煮取二升」同趙本,吳本分別作「以水五升」、「煮取三升」。唐弘宇按:吳本誤。漢代十合为一升,若每服七合,則三服約等於二升。
	}

食穀欲嘔者。屬陽明。{\khaaitp 吳}茱萸湯主之。得湯反劇者。屬上焦。243

少陰病。吐利。手足逆{\khaaitp 冷}。煩躁欲死者。{\khaaitp 吳}茱萸湯主之。309

乾嘔。吐涎沫。頭痛者。{\khaaitp 吳}茱萸湯主之。378

嘔而胸滿者。{\khaaitp 吳}茱萸湯主之。

\section{大建中湯}

蜀椒{\scriptsize 二合汗} 乾薑{\scriptsize 四兩} 人参{\scriptsize 二兩}\\
右三味。{\khaaitp 㕮咀。}以水四升。煮取二升。去滓。内膠飴一升。微火煎取一升半。分温在服。如一炊頃。可飲粥二升後更服。当一日食糜。温覆之。

心胸中大寒痛。嘔。不能飲食。腹中寒。上衝皮起。出見有頭足。上下痛而不可觸近。大建中湯主之。

\section{黄連阿膠湯}

黄連{\scriptsize 四兩} 黄芩{\scriptsize 二兩} 芍藥{\scriptsize 二兩} 雞子黄{\scriptsize 二枚} 阿膠{\scriptsize 三兩一云三挺}\\
右五味。以水六升。先煮三物。取二升。去滓。内膠。烊{\sungtpii 𥁞}。小冷。内雞子黄。攪令相得。温服七合。日三服。

少陰病。得之二三日以上。心中煩。不得卧。黄連阿膠湯主之。303

\section{黄芩湯}

黄芩{\scriptsize 三兩} 芍藥{\scriptsize 二兩} 甘草{\scriptsize 二兩炙} 大棗{\scriptsize 十二枚擘}\\
右四味。以水一斗。煮取三升。去滓。温服一升。日再夜一服。

太陽与少陽合病。自下利者。与黄芩湯。若嘔者。与黄芩加半夏生薑湯。172

\section{黄芩加半夏生薑湯}

黄芩{\scriptsize 三兩} 芍藥{\scriptsize 二兩} 甘草{\scriptsize 二兩炙} 大棗{\scriptsize 十二枚擘} 半夏{\scriptsize 半升洗} 生薑{\scriptsize 一兩半一方三兩切}\\
右六味。以水一斗。煮取三升。去滓。温服一升。日再夜一服。

太陽与少陽合病。自下利者。与黄芩湯。若嘔者。与黄芩加半夏生薑湯。172

乾嘔而利者。黄芩加半夏生薑湯主之。

\section{黄芩湯}

黄芩{\scriptsize 三兩} 人参{\scriptsize 三兩} 乾薑{\scriptsize 三兩} 桂枝{\scriptsize 二兩去皮} 大棗{\scriptsize 十二枚擘} 半夏{\scriptsize 半升洗}\\
右六味。㕮咀。以水七升。煮取三升。去滓。分温三服。

乾嘔。下利。黄芩湯主之。{\scriptsize 玉函經云人参黄芩湯}

\section{三物黄芩湯}

黄芩{\scriptsize 一兩} 苦参{\scriptsize 二兩} 乾地黄{\scriptsize 四兩}\\
右藥㕮咀。以水八升。煮取二升。去滓。温服一升。多吐下虫。{\scriptsize 見千金。}

婦人在草蓐得風。四肢苦煩熱。皆自发露所为。頭痛者。与小柴胡湯。頭不痛。但煩者。与三物黄芩湯。

\section{白頭翁湯}

白頭翁{\scriptsize 二兩} 黄蘗{\scriptsize 三兩} 黄連{\scriptsize 三兩} 秦皮{\scriptsize 三兩}\\
右四味。以水七升。煮取二升。去滓。温服一升。不愈。更服一升。

熱利下重者。白頭翁湯主之。371

下利。欲飲水者。为有熱也。白頭翁湯主之。373

\section{白頭翁加甘草阿膠湯}

白頭翁{\scriptsize 二兩} 黄連{\scriptsize 三兩} 蘗皮{\scriptsize 三兩} 秦皮{\scriptsize 三兩} 甘草{\scriptsize 二兩炙} 阿膠{\scriptsize 二兩}\\
右六味。㕮咀。以水七升。煮取二升半。去滓。内膠。令消{\sungtpii 𥁞}。分温三服。

{\khaaitp 婦人}產後下利。虗極。白頭翁加甘草阿膠湯主之。

\section{木防己湯}

木防己{\scriptsize 三兩} 桂枝{\scriptsize 二兩去皮} 石膏{\scriptsize 如雞子大十二枚} 人参{\scriptsize 四兩}\\
右四味。㕮咀。以水六升。煮取二升。去滓。分温再服。

膈間支飲。其人喘滿。心下痞堅。面色黎黑。其脉沈緊。得之數十日。醫吐下之不愈。木防己湯主之。虗者即愈。実者三日復发。復与不愈者。宜去石膏加茯苓芒硝湯。

\section{木防己去石膏加茯苓芒硝湯}

木防己{\scriptsize 二兩} 桂枝{\scriptsize 二兩去皮} 人参{\scriptsize 四兩} 茯苓{\scriptsize 四兩} 芒硝{\scriptsize 三合}\\
右五味。㕮咀。以水六升。煮取二升。去滓。内芒硝。再微煎。分温再服。微利則愈。

膈間支飲。其人喘滿。心下痞堅。面色黎黑。其脉沈緊。得之數十日。醫吐下之不愈。木防己湯主之。虗者即愈。実者三日復发。復与不愈者。宜去石膏加茯苓芒硝湯。

\section{防己茯苓湯}

防己{\scriptsize 五兩
	\footnote{
		「五兩」同吳本,鄧本作「三兩」。
	}
} 黄耆{\scriptsize 三兩} 桂枝{\scriptsize 三兩去皮} 茯苓{\scriptsize 六兩} 甘草{\scriptsize 二兩炙}\\
右五味。㕮咀。以水六升。煮取二升。去滓。分温再服。

皮水为病。四肢腫。水气在皮膚中。四肢聶聶動者。防己茯苓湯主之。

\section{防己黄耆湯*}

防己{\scriptsize 四兩} 黄耆{\scriptsize 五兩} 甘草{\scriptsize 二兩炙} 白术{\scriptsize 三兩} 生薑{\scriptsize 二兩切} 大棗{\scriptsize 十二枚擘}\\
右六味。㕮咀。以水七升。煮取二升。去滓。分温三服。喘者加麻黄。胃中不和者加芍藥。氣上衝者加桂。下有陳寒者加細辛。服後当如虫行皮中。從腰以上如冰。後坐被上。又以一被繞腰以温下。令微汗。差。{\scriptsize 腰以上。疑作腰以下。}

風濕。脉浮。身重。汗出。惡風者。防己黄耆湯主之。

風水。脉浮。身重。汗出。惡風者。防己黄耆湯主之。腹痛加芍藥。

夫風水。脉浮为在表。其人或頭汗出。表无他病。病者但下重。故知從腰以上为和。腰以下当腫及陰。難以屈伸。防己黄耆湯主之。

\section{枳実术湯}

枳実{\scriptsize 七枚} 白术{\scriptsize 二兩}\\
右二味。㕮咀。以水五升。煮取三升。去滓。分温三服。腹中軟。即当散也。

心下堅。大如盤。邊如旋盤。水飲所作。枳{\khaaitp 実}术湯主之。

\section{枳実芍藥散}

枳実{\scriptsize 燒令黑勿令太過} 芍藥{\scriptsize 等分}\\
右二味。杵为散。服方寸匕。日三服。并主癰膿。以麥屑粥下之。

{\khaaitp 婦人}產後。腹痛。煩滿。不得卧。枳実芍藥散主之。

師曰。產婦腹痛。法当与枳実芍藥散。假令不愈者。此为腹中有乾血著脐下。与下瘀血湯服之。{\khaaitp 亦}主經水不利。

\section{排膿散}

枳実{\scriptsize 十六枚炙} 芍藥{\scriptsize 六分} 桔梗{\scriptsize 二分}\\
右三味。杵为散。取雞子黄一枚。取散与雞黄等揉合令相得。飲和服之。日一服。

\section{桂枝生薑枳実湯}

桂枝{\scriptsize 三兩去皮} 枳実{\scriptsize 伍枚炙} 生薑{三兩 切}\\
右三味。㕮咀。以水六升。煮取三升。去滓。分温三服。

心中痞。諸逆。心懸痛。桂枝生薑枳実湯主之。

\section{橘{\khaaitp 皮}枳{\khaaitp 実生}薑湯}

橘皮{\scriptsize 一斤} 枳実{\scriptsize 二兩炙} 生薑{\scriptsize 半斤切}\\
右三味。㕮咀。以水五升。煮取二升。去滓。分温再服

胸痹。胸中气塞。短气。茯苓杏仁甘草湯主之。橘{\khaaitp 皮}枳{\khaaitp 実生}薑湯亦主之。

\section{橘皮湯}

橘皮{\scriptsize 四兩} 生薑{\scriptsize 半斤}\\
右二味。切。以水七升。煮取三升。去滓。温服一升。下咽即愈。

乾嘔。噦。若手足厥者。橘皮湯主之。

\section{橘皮竹茹湯}

橘皮{\scriptsize 二升} 竹茹{\scriptsize 三升} 大棗{\scriptsize 三十枚擘} 生薑{\scriptsize 半斤切} 甘草{\scriptsize 五兩炙} 人参{\scriptsize 一兩}\\
右六味。㕮咀。以水一斗。煮取三升。去滓。温服一升。日三服

噦{\sungtpii 𠱘}者。橘皮竹茹湯主之。

\section{茯苓飲}

茯苓{\scriptsize 三兩} 人参{\scriptsize 三兩} 白术{\scriptsize 三兩} 生薑{\scriptsize 四兩} 枳実{\scriptsize 二兩} 橘皮{\scriptsize 一兩半}\\
右六味。㕮咀。以水六升。煮取一升八合。去滓。分温三服。如人行八九里進之。

主心胸中有停痰宿水。自吐出水後。心胸間虗。气滿。不能食。消痰气。令能食。茯苓飲。

\section{桂枝茯苓丸}

桂枝{\scriptsize 去皮} 茯苓{ }牡丹{\scriptsize 去心} 桃仁{\scriptsize 去皮尖熬} 芍藥{\scriptsize 各等分}\\
右五味。末之。煉蜜为丸。如兔屎大。每日{\khaaitp 食前服}一丸。不知。加至三丸。

婦人妊娠。經斷三月。而得漏下。下血四十日不止。胎欲動。在於脐上。此为妊娠。六月動者。前三月經水利时。胎也。下血者。後斷三月。衃也。所以下血不止者。其癥不去故也。当下其癥。宜桂枝茯苓丸。{\wuben}

婦人宿有癥病。經斷未及三月。而得漏下不止。胎動在脐上者。为癥痼害。妊娠六月動者。前三月經水利时。胎下血者。後斷三月。衃也。所以血不止者。其癥不去故也。当下其癥。桂枝茯苓丸主之。{\dengben}

\section{膠艾湯}

阿膠{\scriptsize 二兩} 芎窮{\scriptsize 二兩} 甘草{\scriptsize 二兩炙} 艾枼{\scriptsize 三兩} 当歸{\scriptsize 三兩} 芍藥{\scriptsize 四兩} 乾地黄{\scriptsize 四兩}\\
右七味。㕮咀。以水五升清酒三升合煮。取三升。去滓。内膠。令消{\sungtpii 𥁞}。温服一升。日三服。不差更作。

師曰。婦人有漏下者。有半產後因續下血都不絕者。有妊娠下血者。假令妊娠腹中痛。为胞阻。膠艾湯主之。

\section{赤石脂禹餘糧湯}

赤石脂{\scriptsize 一斤碎} 太一禹餘糧{\scriptsize 一斤碎}\\
右二味。以水六升。煮取二升。去滓。分温三服。

傷寒。服湯藥。下利不止。心下痞堅。服瀉心湯已。復以他藥下之。利不止。醫以理中与之。利益甚。理中者。理中焦。此利在下焦。赤石脂禹餘糧湯主之。復不止者。当利小便。159

\section{桃花湯}

赤石脂{\scriptsize 一斤一半全用一半篩末} 乾薑{\scriptsize 一兩} 粳米{\scriptsize 一升}\\
右三味。以水七升。煮米令熟。去滓。温服七合。内赤石脂末方寸匕。日三服。若一服愈。餘勿服。

少陰病。下利。便膿血。桃花湯主之。306

少陰病二三日至四五日。腹痛。小便不利。下利不止。便膿血。桃花湯主之。307

下利。便膿血者。桃花湯主之。

\section{蜜煎}

食蜜{\scriptsize 七合}\\
右一味。於銅器内。微火煎。当需凝如飴狀。攪之勿令焦著。欲可丸。并手撚作挺。令頭鋭。大如指。長二寸許。当熱时急作。冷則鞕。以内榖道中。以手急抱。欲大便时乃去之。疑非仲景意。已試甚良。

陽明病。自汗出。若发汗。小便自利者。此为{\khaaitp 津液}内竭。雖堅。不可攻之。当須自欲大便。宜蜜煎。導而通之。若土瓜根及豬膽汁。皆可以導。233

\section{麻子仁丸}

麻子仁{\scriptsize 二升} 芍藥{\scriptsize 八兩} 枳実{\scriptsize 八兩炙} 大黄{\scriptsize 十六兩去皮} 厚朴{\scriptsize 一尺炙去皮} 杏仁{\scriptsize 一升去皮尖熬別作脂}\\
右六味。蜜和丸。如梧桐子大。飲服十丸。日三服。漸加。以知为度。{\zhaoben}

麻子仁{\scriptsize 一升} 芍藥{\scriptsize 八兩} 枳実{\scriptsize 十六兩炙} 大黄{\scriptsize 十六兩} 厚朴{\scriptsize 一尺炙} 杏仁{\scriptsize 一升去皮尖熬焦}\\
右六味。末之。鍊蜜和丸。如梧子大。飲服十丸。日三服。漸加。以知为度。{\wuben}

趺陽脉浮而濇。浮則胃气强。濇則小便數。浮濇相摶。大便則堅。其脾为約。麻子仁丸主之。247

\section{己椒藶黄丸}

防己{\scriptsize 一兩} 椒目{\scriptsize 一兩} 葶藶{\scriptsize 一兩熬} 大黄{\scriptsize 一兩}\\
右四味。末之。蜜和丸如梧桐子大。先食飲服一丸。日三服。稍增。口中有津液止。渴者。加芒硝半兩

腹滿。口舌乾燥。此腸間有水气。己椒藶黄丸主之。

\section{葶藶大棗瀉肺湯}

葶藶{\scriptsize 熬令黄色。擣丸如彈丸大} 大棗{\scriptsize 十二枚擘}\\
右先以水三升煮棗。取二升。去棗。内葶藶。煮取一升。頓服之。

肺癰。喘不得卧。葶藶大棗瀉肺湯主之。

肺癰。胸滿胀。一身面目浮腫。鼻塞。清涕出。不聞香臭酸辛。欬逆上气。喘鳴迫塞。葶藶大棗瀉肺湯主之。

支飲。不得息。葶藶大棗瀉肺湯主之。

\section{十棗湯}

芫花{\scriptsize 熬} 甘遂{ }大戟{\scriptsize 熬}\\
右三味。擣篩。以水一升五合。煮大棗十枚。煮取八合。去滓。内藥。强人一錢匕。羸人服半錢。平旦服之。不下者。明日更加半錢。下後。糜粥自養。

太陽中風。下利。嘔{\sungtpii 𠱘}。表解乃可攻之。其人漐漐汗出。发作有时。頭痛。心下痞堅滿。引脇下痛。乾嘔。短气。汗出。不惡寒。此为表解裏未和。十棗湯主之。152

病懸飲者。十棗湯主之。

欬家。其脉弦。为有水。可与十棗湯。

夫有支飲家。欬煩。胸中痛者。不卒死。至一百日{\khaaitp 或}一歲。与十棗湯。

\section{桔梗白散(三物白散)}

桔梗{\scriptsize 三分} 貝母{\scriptsize 三分} 巴豆{\scriptsize 一分去皮心熬研如脂}\\
右三味为散。强人飲服半錢匕。羸者減之。病在膈上者吐出。在膈下者瀉出。若下多不止。飲冷水一杯則定。

寒実結胸。无熱証者。与三物白散。141

欬而胸滿。振寒。脉數。咽乾。不渴。时出濁唾腥臭。久久吐膿如米粥者。为肺癰。桔梗白散主之。

\section{走馬湯}

巴豆{\scriptsize 二枚去皮心熬} 杏仁{\scriptsize 二枚{\khaaitp 去皮尖}}\\
右二味。取綿纏。槌令碎。熱湯二合。捻取白汁飲之。当下。老小量之。通治飛尸鬼{\khaaitp 疰}擊病。{\scriptsize {\khaaitp 并見外臺。}}

卒疝。走馬湯主之。{\wuben}

走馬湯。治中惡。心痛。腹胀。大便不通。{\dengben}

%\section{三物備急丸(備急散)}
%《傷寒論》与《金匱要略》中均未見備急丸,但是有三物備急丸,另外吳本有備急散,鄧本无。回去再查一下《類聚方》。

%大黄{\scriptsize 一兩} 乾薑{\scriptsize 一兩} 巴豆{\scriptsize 一兩。去皮心熬別研如脂}\\
%右藥各須精新。先擣大黄乾薑为末。研巴豆内中。合治一千杵。用为散。蜜和为丸亦佳。密器中貯之。莫令歇。主心腹諸卒暴百病。若中惡客忤。心腹脹滿。卒痛如錐刀刺痛。气急。口噤。停尸。卒死者。以煖水若酒。服大豆許三四丸。或不下。捧頭起。灌令下咽。須臾差。如未差。更与三丸。当腹中鳴。即吐下。便差。若口噤。亦須折齒灌之。{\scriptsize 見千金。云司空裴秀为散用。亦可先和成汁。乃傾口中。令從齒間得入。至食驗。}{\wuben}

\section{礬石湯}

礬石{\scriptsize 二兩} 
右一味。以漿水一斗五升。煎三五沸。浸腳良。

礬石湯。治腳气衝心。

\section{硝石礬石散}

硝石{ }礬石{\scriptsize 燒各等分} 
右二味。为散。以大麥粥汁。和服方寸匕。日三服。病隨大小便去。小便正黄。大便正黑。是{\sungtpii 𠊱}也。

黄家。日晡所发熱。而反惡寒。此为女勞得之。膀胱急。少腹滿。身{\sungtpii 𥁞}黄。額上黑。足下熱。因作黑疸。其腹胀如水狀。大便必黑。时溏。此女勞之病。非水也。腹滿者難治。硝石礬石散主之。

\section{礬石丸}

礬石{\scriptsize 三分燒} 杏仁{\scriptsize 一分去皮尖熬}\\
右二味。末之。煉蜜和丸。如棗核大。内臟中劇者。再内之。

婦人經水閉不利。臓堅癖不止。中有乾血。下白物。礬石丸主之。

\section{蛇床子散}

蛇床子仁\\
右一味。末之。以白粉少許。和令相得。如棗大。棉裹。内之。自然温矣。

温陰中坐藥。蛇床子散。

\section{竹枼石膏湯}

竹枼{\scriptsize 二把} 半夏{\scriptsize 半升洗
	\footnote{
		「半升洗」千金方作「一升洗」。
	}
} 麥門冬{\scriptsize 一升去心} 甘草{\scriptsize 二兩炙} 人参{\scriptsize 二兩
	\footnote{
		「二兩」《玉函》作「三兩」
	}
} 石膏{\scriptsize 一斤碎} 粳米{\scriptsize 半升}\\
右七味。以水一斗。煮取六升。去滓。内粳米。煮米熟。湯成。去米。温服一升。日三服。

傷寒解後。虗羸少气。气逆欲吐。竹枼石膏湯主之。397

\section{麥門冬湯}

麥門冬{\scriptsize 七升去心} 半夏{\scriptsize 一升洗} 人参{\scriptsize 二兩} 甘草{\scriptsize 二兩炙} 粳米{\scriptsize 三合} 大棗{\scriptsize 十二枚擘}\\
右六味。㕮咀。以水一斗二升。煮取六升。去滓。温服一升。日三夜一服。

火逆上气。咽喉不利。止逆下气者。麥門冬湯主之。

\section{雄黄}

雄黄一味为末。筒瓦二枚合之。燒。向肛熏之。

狐惑之为病。狀如傷寒。默默欲眠。目不得閉。卧起不安。蝕於喉为惑。蝕於陰为狐。不欲飲食。惡聞食臭。其面目乍赤。乍黑。乍白。蝕於上部則聲喝。甘草瀉心湯主之。蝕於下部則咽乾。苦参湯洗之。蝕於肛者。雄黄熏之。

\section{頭風摩散}

大附子{\scriptsize 一枚炮去皮} 鹽{\scriptsize 等分}\\
右二味。为散。沐了。以方寸匕摩疢上。令藥力行。

頭風摩散方。

\section{皂莢丸}

皂莢{\scriptsize 一挺刮去皮炙焦去子}\\
右一味。末之。蜜丸梧桐子大。以棗膏和湯服三丸。日三夜一服。{\wuben}

{\dengben}

欬逆。气上衝。唾濁。但坐不得卧。皂莢丸主之。{\wuben}

欬逆上气。时时唾濁。但坐不得卧。皂莢丸主之。{\dengben}

\section{葦莖湯}

葦莖{\scriptsize 二升切} 薏苡仁{\scriptsize 半升} 桃仁{\scriptsize 五十枚去皮尖} 瓜瓣{\scriptsize 半升}\\
右四味。以水一斗先煮葦莖。得五升。去滓。内諸藥。煮取二升。分温再服。当吐如膿。

葦莖湯。治欬。有微熱。煩滿。胸中甲錯。是为肺癰。

\section{当歸生薑羊肉湯}

当歸{\scriptsize 三兩} 生薑{\scriptsize 五兩切} 羊肉{\scriptsize 一斤}\\
右三味。㕮咀。以水八升。煮取三升。去滓。温服七合。日三服。若寒多者。加生薑成一斤。痛多而嘔者。加橘皮二兩。术一兩。加生薑者。亦加水五升。煮取三升二合。服之。

寒疝。腹中痛。及脇痛裏急者。当歸生薑羊肉湯主之。

{\khaaitp 婦人}產後。腹中㽲痛。当歸生薑羊肉湯主之。并治腹中寒疝。虗勞不足。

\section{蒲灰散}

蒲灰{\scriptsize 七分} 滑石{\scriptsize 三分}\\
右二味。杵为散。飲服方寸匕。日三服。

小便不利。蒲灰散主之。滑石白魚散。茯苓戎鹽湯并主之。

厥而皮水者。蒲灰散主之。

\section{滑石白魚散}

滑石{\scriptsize 二分} 亂髮{\scriptsize 二分燒} 白魚{\scriptsize 二分}\\
右三味。杵为散。飲服方寸匕。日三服。

小便不利。蒲灰散主之。滑石白魚散。茯苓戎鹽湯并主之。

\section{豬膏髮煎}

猪膏{\scriptsize 半斤} 亂髮{\scriptsize 如雞子大三枚}\\
右二味。和膏中煎之。髪消藥成。分再服。病從小便去。

諸黄。豬膏髮煎主之。

胃气下泄。陰吹而正喧。此穀气之実也。膏髮煎導之。

\section{柏枼湯}

柏葉{\scriptsize 三兩} 艾{\scriptsize 三把} 乾薑{\scriptsize 三兩}\\
右三味。㕮咀。以水五升。取馬通汁一升。合煮取一升。去滓。分温再服。

吐血不止者。柏枼湯主之。

\section{黄土湯}

甘草{\scriptsize 三兩} 乾地黄{\scriptsize 三兩} 白术{\scriptsize 三兩} 附子{\scriptsize 三兩炮去皮破八片} 阿膠{\scriptsize 三兩} 黄芩{\scriptsize 三兩} 竃中黄土{\scriptsize 半斤}\\
右七味。㕮咀。以水八升。煮取三升。去滓。分温二服。

下血。先見血。後見便。此近血也。赤小豆当歸散主之。先見便。後見血。此遠血也。黄土湯主之。

\section{雞屎白散}

鷄屎白\\
右一位。为散。取方寸匕。以水六合和温服。

轉筋之为病。其人臂腳直。脉上下行。微弦。轉筋入腹者。雞屎白散主之。

\section{蜘蛛散}

蜘蛛{\scriptsize 十四枚熬焦} 桂枝{\scriptsize 半兩去皮}\\
右二味。为散。取八分一匕。飲和服。日再服。蜜丸亦得。

陰狐疝气者。偏有大小。时时上下。蜘蛛散主之。

\section{当歸芍藥散}

婦人懷娠。腹中㽲痛。当歸芍藥散主之。

婦人腹中諸疾痛。当歸芍藥散主之。

\section{当歸貝母苦参丸}

{\khaaitp 婦人}妊娠。小便難。飲食如故。{\khaaitp 当}歸{\khaaitp 貝}母苦参丸主之。

\section{狼牙湯}

狼牙{\scriptsize 三兩}\\
右一味。㕮咀。以水四升。煮取半升。以棉裹筋。大如繭。浸湯。瀝陰中。日四遍。

{\khaaitp 婦人}陰中蝕瘡爛。狼牙湯洗之。

\section{小兒疳虫蝕齒方}

雄黄{ }葶藶{\scriptsize 各少許}\\
右二味。末之。取臘月豬脂合鎔。以槐枝棉裹頭四五枚。點藥烙之。{\scriptsize 疑非仲景方。}

\section{炙甘草湯}

傷寒。脉結代。心動悸。炙甘草湯主之。177

虗勞不足。汗出而悶。脉結。心悸。行動如常。不出百日。危急者。十一日死。炙甘草湯主之。

肺痿。涎唾多。心中温温液液者。炙甘草湯主之。

\section{当歸四逆湯}

当歸{\scriptsize 三兩} 桂枝{\scriptsize 三兩去皮} 芍藥{\scriptsize 三兩} 細辛{\scriptsize 三兩} 甘草{\scriptsize 二兩炙} 通草{\scriptsize 二兩} 大棗{\scriptsize 二十五枚擘一法十二枚}\\
右七味。以水八升。煮取三升。去滓。温服一升。日三服。

手足厥寒。脉細欲絕。当歸四逆湯主之。若其人内有久寒。当歸四逆加吳茱萸生薑湯主之。351.352

\section{当歸四逆加吳茱萸生薑湯}

当歸{\scriptsize 三兩} 芍藥{\scriptsize 三兩} 甘草{\scriptsize 二兩炙} 通草{\scriptsize 二兩} 桂枝{\scriptsize 三兩去皮} 細辛{\scriptsize 三兩} 生薑{\scriptsize 八兩切} 吳茱萸{\scriptsize 二升} 大棗{\scriptsize 二十五枚擘}\\
右九味。以水六升清酒六升合。煮取五升。去滓。分温五服。{\scriptsize 一方水酒各四升。}

手足厥寒。脉細欲絕。当歸四逆湯主之。若其人内有久寒。当歸四逆加吳茱萸生薑湯主之。351.352

\section{麻黄連軺赤小豆湯}

傷寒。瘀熱在裏。身必发黄。麻黄連軺赤小豆湯主之。262

\section{四逆散*}

欬者。加五味子乾薑各五分。并主下利。\\
悸者。加桂枝五分。\\
小便不利者。加茯苓五分。\\
腹中痛者。加附子一枚炮令坼。\\
泄利下重者。先以水五升。煮薤白三升。煮取三升。去滓。以散三方寸匕。内湯中。煮取一升半。分温再服。

少陰病。四逆。其人或欬。或悸。或小便不利。或腹中痛。或泄利下重。四逆散主之。318

\section{射干麻黄湯}

射干{\scriptsize 十三枚一法三枚} 麻黄{\scriptsize 四兩去節} 生薑{\scriptsize 四兩切} 細辛{\scriptsize 三兩} 紫菀{\scriptsize 三兩} 款冬花{\scriptsize 三兩} 五味子{\scriptsize 半升} 半夏{\scriptsize 大者八枚洗一法半升} 大棗{\scriptsize 七枚擘}\\
右九味。㕮咀。以水一斗二升。先煮麻黄兩沸。去上沫。内諸藥。煮取三升。去滓。分温三服。

欬而上气。喉中水雞聲。射干麻黄湯主之。

\section{桂枝芍藥知母湯}

諸肢節疼痛。身体魁羸。腳腫如脱。頭眩短气。温温欲吐。桂枝芍藥知母湯主之。

\section{内補当歸建中湯*}

若大虗。加飴糖六兩。湯成内之。於火上煖令飴消。若无生薑。以乾薑代之。\\
若其人去血過多。崩傷内衄不止。加地黄六兩。阿膠二兩。合八種。湯成。去滓。内阿膠。若无当歸。以芎藭代之。

治婦人產後。虗羸不足。腹中刺痛不止。吸吸少气。或苦少腹拘急攣痛引腰背。不能食飲。產後一月。日得服四五剂为善。令人强壯。内補当歸建中湯。{\wuben}

内補当歸建中湯。治婦人產後。虗羸不足。腹中刺痛不止。吸吸少气。或苦少腹拘急攣痛引腰背。不能食飲。產後一月。日得服四五剂为善。令人强壯。{\dengben}

\section{續命湯}

續命湯。治中風痱。身体不能自收。口不能言。冒昧不知痛処。或拘急不得轉側。

\section{烏梅丸}

烏梅{\scriptsize 三百枚} 細辛{\scriptsize 六兩} 乾薑{\scriptsize 十兩} 黄連{\scriptsize 十六兩} 当歸{\scriptsize 四兩} 附子{\scriptsize 六枚炮去皮} 蜀椒{\scriptsize 四兩去目及閉口者汗} 桂枝{\scriptsize 六兩去皮} 人参{\scriptsize 六兩} 黄蘗{\scriptsize 六倆}\\
右十一味。{\khaaitp 各}異擣篩。合治之。以苦酒漬烏梅一宿。去核。蒸之五斗米下。飯熟。擣成泥。和藥相得。内臼中。与蜜杵三千下。丸如梧桐子大。先食飲。服十丸。日三服。稍加{\khaaitp 至}二十丸。
	\footnote{
		「附子六枚」同吳本,趙本作「附子六兩」。「杵三千下」同吳本,趙本作「杵兩千下」。
	}

傷寒。脉微而厥。至七八日。膚冷。其人躁。无暫安时者。此为臓厥。非蛔厥也。蛔厥者。其人当吐蛔。今病者靜。而復时煩。此为臓寒。蛔上入膈。故煩。須臾復止。得食而嘔。又煩者。蛔聞食臭出。其人常自吐蛔。蛔厥者。烏梅丸主之。338

\section{大黄䗪虫丸}

五勞。虗極。羸瘦。腹滿。不能飲食。食傷。憂傷。飲傷。房室傷。飢傷。勞傷。經絡榮衛气傷。内有乾血。肌膚甲錯。兩目黯黑。緩中補虗。大黄䗪虫丸主之。

\section{麻黄升麻湯}

麻黄{\scriptsize 二兩半去節} 升麻{\scriptsize 一兩一分} 当歸{\scriptsize 一兩一分} 知母{\scriptsize 十八銖} 黄芩{\scriptsize 十八銖} 萎蕤{\scriptsize 十八銖一作菖蒲} 芍藥{\scriptsize 六銖} 天門冬{\scriptsize 六銖去心} 桂枝{\scriptsize 六銖去皮} 茯苓{\scriptsize 六銖} 甘草{\scriptsize 六銖炙} 石膏{\scriptsize 六銖碎綿裹} 白术{\scriptsize 六銖} 乾薑{\scriptsize 六銖}\\
右十四味。以水一斗。先煮麻黄一兩沸。去上沫。内諸藥。煮取三升。去滓。分温三服。相去如炊三斗米頃令{\sungtpii 𥁞}。汗出愈。

傷寒六七日。大下後。{\khaaitp 寸}脉沈遲。手足厥逆。下部脉不至。咽喉不利。唾膿血。泄利不止者。为難治。麻黄升麻湯主之。357

\section{豬膚湯}

豬膚{\scriptsize 一斤}\\
右一味。以水一斗。煮取五升。去滓。内白蜜一升。白粉五合。熬香。和令相得。温分六服。

少陰病。下利。咽痛。胸滿。心煩。豬膚湯主之。310

\section{燒裩散}

婦人中裩近陰処燒灰\\
右一味。水和服方寸匕。日三。小便即利。陰頭微腫。此为愈。婦人病。取男子裩燒服。

傷寒陰易之为病。其人身体重。少气。少腹裏急。或引陰中拘攣。熱上衝胸。頭重不欲舉。眼中生眵。{\khaaitp 眼胞赤。}膝脛拘急。燒裩散主之。392

\section{瓜蒂湯}

瓜蒂{\scriptsize 二七枚}\\
右一味。以水一升。煮取五合。去滓。頓服。

太陽中暍。身熱疼重。而脉微弱。此以夏月傷冷水。水行皮膚中所致也。瓜蒂湯主之。

\section{百合知母湯}

百合{\scriptsize 七枚擘} 知母{\scriptsize 三兩切}\\
右二味。先以水洗百合。漬一宿。当白沫出。去其水。更以泉水二升。煮取一升。去滓。別以泉水二升煮知母。取一升。去滓。後合和。重煎取一升五合。分温再服。

百合病发汗後。更发者。百合知母湯主之。

\section{百合滑石代赭湯}

百合{\scriptsize 七枚擘} 滑石{\scriptsize 三兩碎綿裹} 代赭{\scriptsize 如彈丸一枚碎綿裹}\\
右三味。先以水洗百合。漬一宿。当白沫出。去其水。更以泉水二升。煮取一升。去滓。別以泉水二升煮滑石。代赭。取一升。去滓。後合和。重煎取一升五合。分温再服。

百合病下之後。更发者。百合滑石代赭湯主之。

\section{百合雞子湯}

百合{\scriptsize 七枚擘} 雞子黄{\scriptsize 一枚}\\
右二味。先以水洗百合。漬一宿。当白沫出。去其水。更以泉水二升。煮取一升。去滓。内雞子黄。攪令調。分温再服。

百合病吐之後。更发者。百合雞子湯主之。

\section{百合地黄湯}

百合{\scriptsize 七枚擘} 生地黄汁{\scriptsize 一升}\\
右二味。先以水洗百合。漬一宿。当白沫出。去其水。更以泉水二升。煮取一升。去滓。内地黄汁。煮取一升五合。分温再服。中病勿更服。大便当如漆。

百合病。不經发汗吐下。病形如初者。百合地黄湯主之。

\section{百合洗方}

百合{\scriptsize 一升}\\
右一味。以水一斗。漬之一宿。以洗身。洗已。食煮餅。勿与鹽豉也。

百合病。經月不解。變成渴者。百合洗方主之。

\section{栝蔞牡蛎散}

栝蔞根{ }牡蛎{\scriptsize 熬等分}\\
右二味。杵为散。飲服方寸匕。日三服。

百合病。渴不差者。栝蔞牡蛎散主之。

\section{百合滑石散}

百合{\scriptsize 一兩炙} 滑石{\scriptsize 三兩}\\
右二味。杵为散。飲服方寸匕。日三服。当微利者止。勿服之。熱則除。

百合病。變发熱者。百合滑石散主之。

\section{}

百合病。變腹中滿痛者。但取百合根隨多少。熬令黄色。擣篩为散。飲服方寸匕。日三。滿消痛止。

\section{赤小豆当歸散}

赤小豆{\scriptsize 三升浸令芽出曝乾} 当歸{\scriptsize 三兩}\\
右二味。杵为散。漿水服方寸匕。日三服。

病者脉數。无熱。微煩。默默。但欲卧。汗出。初得之三四日。目赤如鳩眼。七八日目四眥黑。若能食者。膿已成也。赤{\khaaitp 小}豆当歸散主之。

下血。先見血。後見便。此近血也。赤小豆当歸散主之。先見便。後見血。此遠血也。黄土湯主之。

\section{升麻鱉甲湯}

升麻{\scriptsize 二兩} 当歸{\scriptsize 一兩} 蜀椒{\scriptsize 一兩汗} 鱉甲{\scriptsize 如手大一片炙} 甘草{\scriptsize 二兩炙} 雄黄{\scriptsize 半兩研}\\
右六味。㕮咀。以水四升。煮取一升。去滓。頓服之。老小再服。取汗。陰毒去雄黄。蜀椒。{\scriptsize 肘後。千金。陽毒用升麻湯。无鱉甲。有桂。陰毒用甘草湯。无雄黄。}

陽毒之为病。面赤斑斑如錦文。喉咽痛。唾膿血。五日可治。七日不可治。\\
陰毒之为病。面目青。狀如被打。喉咽痛。死生与陽毒同。升麻鱉甲湯并主之。{\wuben}

陽毒之为病。面赤斑斑如錦文。咽喉痛。唾膿血。五日可治。七日不可治。升麻鱉甲湯主之。\\
陰毒之为病。面目青。身痛如被杖。咽喉痛。五日可治。七日不可治。升麻鱉甲湯去雄黄蜀椒主之。{\dengben}

%\section{升麻鱉甲湯去雄黄蜀椒}

\section{鱉甲煎丸}

鱉甲{\scriptsize 十二分炙} 烏扇{\scriptsize 三分燒} 黄芩{\scriptsize 三分} 柴胡{\scriptsize 六分} 鼠婦{\scriptsize 三分熬} 乾薑{\scriptsize 三分} 大黄{\scriptsize 三分} 芍藥{\scriptsize 五分} 桂枝{\scriptsize 三分去皮} 葶藶{\scriptsize 一分熬} 石韋{\scriptsize 三分去毛} 厚朴{\scriptsize 三分} 牡丹{\scriptsize 五分去心} 瞿麥{\scriptsize 二分} 紫威{\scriptsize 三分} 半夏{\scriptsize 一分洗} 人参{\scriptsize 一分} 䗪虫{\scriptsize 五分熬} 阿膠{\scriptsize 三分炙} 蜂窠{\scriptsize 四分熬} 赤硝{\scriptsize 十二分} 蜣蜋{\scriptsize 六分熬} 桃仁{\scriptsize 二分去皮尖熬焦}\\
右二十三味。为末。取煆竈下灰一斗。清酒一斛五斗浸灰。{\sungtpii 𠊱}酒{\sungtpii 𥁞}一半。着鱉甲於中煮。令泛爛如膠漆。絞取汁。内諸藥煎。为丸如梧桐子大。空心服七丸。日三服。{\scriptsize 千金用鱉甲十二片。又有海藻三分。大戟一分。䗪虫五分。无鼠婦赤硝二味。以鱉甲煎。和諸藥为丸。}

問曰。瘧以月一日发。当以十五日愈。設不差。当月{\sungtpii 𥁞}解。如其不差。当云何。\\
師曰。此結为癥瘕。名曰瘧母。急治之。宜鱉甲煎丸。

\section{矦氏黑散}

菊花{\scriptsize 四十分} 白术{\scriptsize 十分} 細辛{\scriptsize 三分} 茯苓{\scriptsize 三分} 牡蛎{\scriptsize 三分熬} 桔梗{\scriptsize 八分} 防風{\scriptsize 十分} 人参{\scriptsize 三分} 礬石{\scriptsize 三分熬} 黄芩{\scriptsize 五分} 当歸{\scriptsize 三分} 乾薑{\scriptsize 三分} 芎窮{\scriptsize 三分} 桂枝{\scriptsize 三分去皮}\\
右十四味。杵为散。酒服方寸匕。日一服。初服二十日。温酒下之。禁一切魚肉大蒜。常宜冷食。六十日止。即藥積在腹中不下也。熱食即下矣。冷食自能助藥力。{\scriptsize 外臺有鐘乳礬石各三分。无桔梗。}

大風。四肢煩重。心中惡寒不足者。矦氏黑散主之。{\scriptsize 外臺治風癲。}

\section{風引湯}

大黄{\scriptsize 四兩} 乾薑{\scriptsize 四兩} 龙骨{\scriptsize 四兩} 桂枝{\scriptsize 三兩去皮} 甘草{二兩炙} 牡蛎{\scriptsize 二兩熬} 凝水石{\scriptsize 六兩} 滑石{\scriptsize 六兩} 赤石脂{\scriptsize 六兩} 白石脂{\scriptsize 六兩} 石膏{\scriptsize 六兩} 紫石英{\scriptsize 六兩}\\
右十二味。杵粗。篩。以韋囊盛之。取三指撮井華水三升。煮三沸。去滓。温服一升。{\scriptsize 深師云。治大人風引。少小驚癇。瘈瘲日數十发。醫所不療。除熱方。巢源。腳气宜風引湯。}

風引湯。除熱{\khaaitp 。主}癱癇。

\section{防己地黄湯}

防己{\scriptsize 一分} 桂枝{\scriptsize 三分去皮} 防風{\scriptsize 三分} 甘草{\scriptsize 二分炙}\\
右四味。㕮咀。以酒一杯。漬之一宿。絞取汁。取生地黄二斤。㕮咀。蒸之如斗米飯久。以銅器盛其汁。更絞地黄等汁。和分再服。

病如狂狀。妄行。獨語不休。无寒熱。其脉浮。防己地黄湯主之。

\hangindent 1em
\hangafter=0
言語狂錯。眼目茫茫。或見鬼。精神昏亂。防己地黄湯方。{\qianjin}

\section{三黄湯*}

麻黄{\scriptsize 去節五分} 獨活{\scriptsize 四分} 細辛{\scriptsize 二分} 黄耆{\scriptsize 二分} 黄芩{\scriptsize 三分}\\
右五味。㕮咀。以水六升。煮取二升。去滓。分温三服。一服小汗。兩服大汗。心熱加大黄二分。腹痛加枳実一枚。气逆加人参三分。悸加牡蛎三分。渴加栝蔞根三分。先有寒加附子一枚。{\scriptsize 見千金。}

三黄湯。治中風。手足拘急。百節疼痛。煩熱心亂。惡寒。經日不欲飲食。

\section{薯蕷丸}

薯蕷{\scriptsize 三十分} 当歸{\scriptsize 十分} 桂枝{\scriptsize 十分去皮} 麹{\scriptsize 十分} 乾地黄{\scriptsize 十分} 大豆黄卷{\scriptsize 十分} 甘草{\scriptsize 二十八分炙} 人参{\scriptsize 七分} 芎窮{\scriptsize 六分} 芍藥{\scriptsize 六分} 白术{\scriptsize 六分} 麥門冬{\scriptsize 六分去心} 杏仁{\scriptsize 六分去皮尖熬} 柴胡{\scriptsize 五分} 桔梗{\scriptsize 五分} 茯苓{\scriptsize 五分} 阿膠{\scriptsize 七分炙} 乾薑{\scriptsize 三分} 白斂{\scriptsize 二分} 防風{\scriptsize 六分} 大棗{\scriptsize 百枚为膏}\\
右二十一味。末之。煉蜜和丸。如彈子大。空腹。酒服一丸。一百丸为劑。
	\footnote{
		「阿膠七分炙」吳本作「阿膠炙各七分」,鄧本作「阿膠七分」。
	}

虗勞。諸不足。風气百疾。薯蕷丸主之。

\section{獺肝散}

獺肝一具。炙乾。末之。水服方寸匕。日三服。{\scriptsize 見肘後。恐非仲景方。}{\wuben}

%獺肝{\scriptsize 一具}\\
%陰乾。末之。水服方寸匕。日三服。{\dengben}
%手邊暫时沒有資料,此方等開學後去學校查書再校。

獺肝散。治冷勞。又主鬼疰。一門相染。

\section{厚朴麻黄湯}

厚朴{\scriptsize 五兩炙} 麻黄{\scriptsize 四兩去節} 石膏{\scriptsize 如雞子大碎} 杏仁{\scriptsize 半升去皮尖} 乾薑{\scriptsize 二兩} 細辛{\scriptsize 二兩} 小麥{\scriptsize 一升} 五味子{\scriptsize 半升} 半夏{\scriptsize 半升洗}\\
右九味。㕮咀。以水一斗二升。先煮小麥熟。去滓。内諸藥。煮取三升。去滓。温服一升。日三服。

上气。脉浮者。厚朴麻黄湯主之。脉沈者。澤漆湯主之。{\wuben}

欬而脉浮者。厚朴麻黄湯主之。脉沈者。澤漆湯主之。{\dengben}

\section{澤漆湯}

澤瀉{\scriptsize 三斤以東流水五斗煮取一斗五升} 半夏{\scriptsize 半升洗} 紫参{\scriptsize 五兩一作紫菀} 生薑{\scriptsize 五兩切} 白前{\scriptsize 五兩} 甘草{\scriptsize 三兩炙} 黄芩{\scriptsize 三兩} 人参{\scriptsize 三兩} 桂枝{\scriptsize 三兩去皮}\\
右九味。㕮咀。内澤漆汁中煮取五升。去滓。温服五合。至夜{\sungtpii 𥁞}。

上气。脉浮者。厚朴麻黄湯主之。脉沈者。澤漆湯主之。{\wuben}

欬而脉浮者。厚朴麻黄湯主之。脉沈者。澤漆湯主之。{\dengben}

\section{奔豚湯}

甘草{\scriptsize 二兩{\khaaitp 炙}} 芎藭{\scriptsize 二兩} 当歸{\scriptsize 二兩} 半夏{\scriptsize 四兩{\khaaitp 洗}} 黄芩{\scriptsize 二兩} 生葛{\scriptsize 五兩} 芍藥{\scriptsize 二兩} 生薑{\scriptsize 四兩{\khaaitp 切}} 甘李根白皮{\scriptsize 一升{\khaaitp 切}}

右九味。{\khaaitp 㕮咀。}以水二斗。煮取五升。{\khaaitp 去滓。}温服一升。日三夜一服。

奔豚。气上衝胸。腹痛。往來寒熱。奔豚湯主之。

\section{烏頭赤石脂丸}

烏頭{\scriptsize 一分炮去皮} 附子{\scriptsize 半兩炮去皮一法一分} 赤石脂{\scriptsize 一兩一法二分} 乾薑{\scriptsize 一兩一法二分} 蜀椒{\scriptsize 一兩汗一法二分}\\
右五味。末之。蜜丸如梧子大。先食服一丸。日三服。不知。稍增之。

心痛徹背。背痛徹心。烏頭赤石脂丸主之。

\section{九痛丸}

附子{\scriptsize 三兩炮去皮} 巴豆{\scriptsize 一兩去皮心熬研如脂} 生狼牙{\scriptsize 一兩炙令香} 人参{\scriptsize 一兩} 乾薑{\scriptsize 一兩} 吳茱萸{\scriptsize 一兩}\\
右六味。末之。煉蜜和丸如梧子大。酒下。强人初服三丸。日一服。弱者二丸。兼治卒中惡。腹脹痛。口不能言。又治連年積冷。流注心胸痛。并冷腫上气。落馬。墜車。血疾等。皆主之。禁口如常法。{\wuben}

{\dengben}%待錄入

九痛丸。治九種心痛。{\khaaitp 一虫心痛。二疰心痛。三風心痛。四悸心痛。五食心痛。六飲心痛。七冷心痛。八熱心痛。九去來心痛。}

\section{温經湯}

吳茱萸{\scriptsize 三兩} 当歸{\scriptsize 二兩} 芎藭{\scriptsize 二兩} 芍藥{\scriptsize 二兩} 麥門冬{\scriptsize 一升去心} 人参{\scriptsize 二兩} 桂枝{\scriptsize 二兩去皮} 阿膠{\scriptsize 二兩} 牡丹{\scriptsize 二兩去心} 生薑{\scriptsize 二兩} 甘草{\scriptsize 二兩炙} 半夏{\scriptsize 半升}\\
右十二味。㕮咀。以水一斗。煮取三升。去滓。分温三服。\\
亦主婦人少腹寒。久不作軀。兼主崩中去血。或月水來過多。及過期不來。

問曰。婦人年五十所。病下血。數十日不止。暮即发熱。少腹裏急{\khaaitp 痛}。腹滿。手掌煩熱。唇口乾燥。何也。\\
師曰。此病屬帶下。何以故。曾經半產。瘀血在少腹不去。何以知之。其証唇口乾燥。故知之。当以温經湯主之。

\section{旋覆花湯}

旋覆花{\scriptsize 三兩} 蔥{\scriptsize 十四莖} 新絳{\scriptsize 少許}\\
右三味。以水三升。煮取一升。去滓。頓服之。

肝著。其人常欲蹈其胸上。先未苦时。但欲飲熱。旋覆花湯主之。

寸口脉弦而大。弦則为減。大則为芤。減則为寒。芤則为虗。寒虗相摶。此名曰革。婦人則半產漏下。旋覆花湯主之。

\section{厚朴大黄湯}

厚朴{\scriptsize 一尺去皮炙} 大黄{\scriptsize 六兩} 枳実{\scriptsize 四枚炙}\\
右三味。㕮咀。以水五升。煮取二升。去滓。分温再服。
%厚朴一尺可能是錯誤,等開學後查。

支飲。胸滿者。厚朴大黄湯主之。

\section{瓜蒂湯}

瓜蒂{\scriptsize 二七枚}\\
右一味。以水一升。煮取五合。去滓。頓服。

諸黄。瓜蒂湯主之。

\section{紫参湯}

紫参{\scriptsize 半斤} 甘草{\scriptsize 三兩炙}\\
右二味。㕮咀。以水五升。先煮紫参。取二升。内甘草。煮取一升半。去滓。分温三服。{\scriptsize 疑非仲景方。}

下利。肺痛。紫参湯主之。

\section{訶梨勒散}

訶梨勒{\scriptsize 十枚以麵裹煻灰火中煨之令麵熟去核}\\
右一味。細为散。粥飲和。頓服之。{\scriptsize 疑非仲景方。}

气利。訶梨勒散主之。

\section{王不留行散}

王不留行{\scriptsize 十分八月八日采之} 蒴藋細枼{\scriptsize 十分七月七日采之} 桑東南根{\scriptsize 如指大白皮十分三月三日采} 甘草{\scriptsize 十八分炙} 蜀椒{\scriptsize 三分去目及閉口者汗} 黄芩{\scriptsize 二分} 乾薑{\scriptsize 二分} 芍藥{\scriptsize 二分} 厚朴{\scriptsize 二分炙}\\
右九味。桑東南根以上三味。燒为灰。存性。勿令灰過。各別杵篩。合治之为散。病者与方寸匕服之。小瘡即粉之。中大瘡但服之。產後亦可服。如風寒桑根勿取之。前三物皆陰乾百日。

病金瘡。王不留行散主之。

\section{黄連粉}

浸淫瘡。從口流向四肢者。可治。從四肢流來入口者。不可治。黄連粉主之。{\scriptsize 方未見。}

\section{藜蘆甘草湯}

病人常以手指臂脛動。此人身体瞤瞤者。藜蘆甘草湯主之。{\scriptsize 方未見。}

\section{当歸散}

当歸{\scriptsize 一斤} 黄芩{\scriptsize 一斤} 芍藥{\scriptsize 一斤} 芎窮{\scriptsize 一斤} 白术{\scriptsize 半斤}\\
右五味。杵为散。酒飲服方寸匕。日再服。妊娠常服即易產。胎无苦疾。產後百病悉主之。

婦人妊娠。宜常服当歸散。

\section{白术散}

白术{\scriptsize 四分} 芎窮{\scriptsize 四分} 蜀椒{\scriptsize 三分汗} 牡蛎{\scriptsize 二分熬} \\
右四味。杵为散。酒服一錢匕。日三夜一服。但苦痛。加芍藥。心下毒痛。倍加芎窮。心煩吐痛。不能食飲。加細辛一兩。半夏錢大二十枚。服之後。更以醋漿水服之。若嘔。亦以醋漿水服之。復不解者。小麥汁服之。已後若渴者。大麥粥服之。病雖愈。{\sungtpii 𥁞}服之勿置。{\scriptsize 見外臺出古今錄驗。}

妊娠養胎。白术散主之。

\section{竹枼湯*}

竹枼{\scriptsize 一把} 葛根{\scriptsize 三兩} 防風{\scriptsize 一兩} 桔梗{\scriptsize 一兩} 桂枝{\scriptsize 一兩去皮} 人参{\scriptsize 一兩} 甘草{\scriptsize 一兩炙} 附子{\scriptsize 一枚炮去皮破八片} 大棗{\scriptsize 十五枚擘} 生薑{\scriptsize 五兩切}\\
右十味。㕮咀。以水一斗。煮取二升半。去滓。分温三服。温覆使汗出。頸項强。用大附子一枚。破之如豆大。煎藥。揚去沫。嘔者。加半夏半升洗。

產後中風。发熱。面{\khaaitp 正}赤。喘而頭痛。竹枼湯主之。

\section{竹皮大丸*}

生竹茹{\scriptsize 二分} 石膏{\scriptsize 二分研} 桂枝{\scriptsize 一分去皮} 甘草{\scriptsize 七分炙} 白薇{\scriptsize 一分}\\
右五味。末之。棗肉和丸如彈丸大。以飲服一丸。日三夜二服。有熱者。倍白薇。煩喘者。加柏実一分。

婦人產中虗。煩亂。嘔{\sungtpii 𠱘}。安中益气。竹皮大丸主之。

\section{紅藍花酒}

紅藍花{\scriptsize 一大兩}\\
右一味。以酒一大升。煎强半。頓服。不止再服。{\scriptsize 疑非仲景方。}{\wuben}

紅藍花{\scriptsize 一兩}\\
右一味。以酒一大升。煎減半。頓服一半。未止再服。{\scriptsize 疑非仲景方。}{\dengben}

紅藍花酒。治婦人六十二種風。兼主腹中血气刺痛。

\section{禹餘糧丸{\scriptsize (方本闕)}}

汗家。重发汗。必恍惚心亂。小便已。陰疼。与禹餘糧丸。88

\section{膠薑湯{\scriptsize (諸本无膠薑湯方)}}

婦人陷經。漏下黑不解。膠薑湯主之。{\scriptsize 臣億等按。諸本无膠薑湯{\khaaitp 方}。恐是前妊娠中膠艾湯也。}

\part{傷寒論}

\chapter{辨太陽病}

太陽之为病。{\khaaitp 脉浮。}頭項强痛。而惡寒。1

太陽病。发熱。汗出。惡風。脉緩者。为中風。2

%太陽中風。发熱而惡寒。0
%
%\hangindent 1em
%\hangafter=0
%太陽中風。发熱而惡寒。宜桂枝湯。{\gaoben}

太陽病。或已发熱。或未发熱。必惡寒。体痛。嘔{\sungtpii 𠱘}。脉陰陽俱緊者。为傷寒。3

%傷寒一日。太陽脉弱。至四日。太陰脉大。0
%	\footnote{
%		此條見《千金翼》卷九第二、《玉函》卷二第三,《脉經》、趙本、成本无。
%	}

傷寒一日。太陽受之。脉若靜者。为不傳。頗欲吐。或躁煩。脉數急者。乃为傳。4

傷寒{\khaaitp 二三日}。陽明少陽証不見者。为不傳。5
	\footnote{
		「陽明少陽証」除趙本外其它版本均作「其二陽証」。
	}

傷寒三日。陽明脉大{\khaaitp 者。为欲傳}。186

傷寒三日。少陽脉小者。为欲已。271

%太陽病三四日。不吐下。見芤乃汗之。0
%	\footnote{
%	此條見《千金翼》卷九太陽病用桂枝湯法第一、《玉函》卷二辨太陽病形証第三,趙本、成本、《脉經》无。
%	}

太陽病。发熱而渴。不惡寒者。为温病。若发汗已。身灼熱者。为風温。6

風温{\khaaitp 之}为病。脉陰陽俱浮。自汗出。身重。多眠睡。鼻息必鼾。語言難出。若被下者。小便不利。直視。失溲。若被火者。微发黄{\khaaitp 色}。劇則如驚癇。时痸瘲。若火熏之。一逆尚引日。再逆促命期。6
	\footnote{
		「痸瘲」同趙本,玉函作「掣縱发作」。
	}

病有发熱而惡寒者。发於陽也。不熱而惡寒者。发於陰也。发於陽者七日愈。发於陰者六日愈。以陽數七。陰數六故也。7

太陽病。頭痛。至七日以上自愈者。其經{\sungtpii 𥁞}故也。若欲作再經者。当針足陽明。使經不傳則愈。8
	\footnote{
		「以上自愈者」同趙本、《千金翼》卷九,《玉函》卷六、《千金翼》卷十作「自当愈」,《玉函》卷二作「有当愈者」。
	}

太陽病欲解时。從巳{\sungtpii 𥁞}未。9
	\footnote{
		「從巳{\sungtpii 𥁞}未」同《千金翼》、《玉函》,趙本、成本作「從巳至未上」。
	}

風家。表解而不了了者。十二日愈。10

病人身大熱。反欲得衣者。熱在皮膚。寒在骨髓也。身大寒。反不欲近衣者。寒在皮膚。熱在骨髓也。11
	\footnote{
		此條見趙本、成本、《玉函》卷二,《千金翼》无。関於此條是否为仲景原文,歷來註家聚訟不已,考趙本卷二子目,亦无此條之提示,山田正珍云:「此條不似仲景氏辭气,疑是古語。」唐弘宇按:此條以「皮膚」、「骨髓」劃分人体内外層次,這是扁鵲醫學的特徵,我認为此條可能是作者引用古語,也可能是後人將古語沾益於此。
	}

太陽中風。{\khaaitp 脉}陽浮而陰弱。陽浮者熱自发。陰弱者汗自出。嗇嗇惡寒。淅淅惡風。翕翕发熱。鼻鳴。乾嘔。桂枝湯主之。12
	\footnote{
		「脉陽浮而陰弱」《千金翼》卷九、《脉經》卷七第二、《玉函》卷二第三作「陽浮而陰濡弱」,《玉函》卷五第十四作「脉陽浮而陰濡弱」,宋本、成本作「陽浮而陰弱」,《千金方》作「其脉陽浮而陰弱」,《聖惠方》卷八辨太陽病形証作「脉其陽浮而弱」。
%		孫世揚:「嗇嗇」今諺称「冷瑟瑟」。
	}

太陽病。发熱。汗出。此为榮弱衛强。故使汗出。欲救邪風。宜桂枝湯。95

太陽病。頭痛。发熱。汗出。惡風。桂枝湯主之。13

太陽病。項背强几几。反汗出。惡風。桂枝{\khaaitp 加葛根}湯主之。14
%	\footnote{
%		孫世揚:「几几」当作「掔掔」,説文解字段玉裁註云:「掔之言緊也。」。
%	}

太陽病。下之。其气上衝者。可与桂枝湯。不衝者。不可与之。15

太陽病三日。已发汗吐下温針而不解。此为壞病。桂枝湯不復中与也。觀其脉証。知犯何逆。隨証治之。16

桂枝湯本为解肌。若其人脉浮緊。发熱。无汗。不可与也。常須識此。勿令誤也。16

酒客不可与桂枝湯。得之則嘔。以酒客不喜甘故也。17

喘家作桂枝湯。加厚朴杏仁佳。18

服桂枝湯吐者。其後必吐膿血。19

太陽病。发汗。遂漏不止。其人惡風。小便難。四肢微急。難以屈伸。桂枝加附子湯主之。20

太陽病。下之。脉促。胸滿者。桂枝去芍藥湯主之。若微{\khaaitp 惡}寒者。桂枝去芍藥加附子湯主之。21.22
	\footnote{
		「微惡寒」同《玉函》卷二第三、成本,其它版本均无「惡」字。
	}

太陽病。得之八九日。如瘧狀。发熱。惡寒。熱多寒少。其人不嘔。清便續自可。一日再三发。脉微緩者。为欲愈也。脉微而惡寒者。此为陰陽俱虗。不可復{\khaaitp 吐下}发汗也。面反有熱色者。未欲解也。以其不能得汗出。身必癢。宜桂枝麻黄各半湯。23
	\footnote{
		「續自可」《玉函》卷二第三作「自調」。
		「面反有熱色」聖惠方作「面色赤有熱」。
	}

太陽病。初服桂枝湯。反煩不解者。当先刺風池風府。卻与桂枝湯即愈。24

服桂枝湯。大汗出。脉洪大者。与桂枝湯。若形如瘧。一日再发者。汗出便解。宜桂枝二麻黄一湯。25

%\hangindent 1em
%\hangafter=0
%凡大汗出復後。脉洪大。形如瘧。一日再发。汗出便解。更与桂枝麻黄湯。{\yixin}
%
%\hangindent 1em
%\hangafter=0
%大汗出後。脉猶洪大。形如瘧。日一发。汗出便解方。{\yixin}

服桂枝湯。大汗出{\khaaitp 後}。大煩渴不解。脉洪大者。白虎{\khaaitp 加人参}湯主之。26
	\footnote{
		「大汗出後」同宋本、成本、玉函卷二第三,脉經、玉函卷六第十九作「大汗出」,外臺作「大汗後」。
		「脉洪大者」同宋本、成本、外臺,脉經、玉函卷六第十九作「若脉洪大」,玉函卷二第三作「若脉洪大者」。
		「白虎加人参湯主之」同宋本卷二、成本卷二、玉函卷二第三,宋本卷八、外臺作「屬白虎加人参湯」,脉經作「屬白虎湯」,玉函卷六第十九作「屬白虎湯証」,千金翼作「与白虎湯」。
	}

太陽病。发熱。惡寒。熱多寒少。脉微弱者。此无陽也。不可{\khaaitp 復}发汗。{\khaaitp 宜桂枝二越婢一湯。}27

服桂枝湯。{\khaaitp 或}下之。仍頭項强痛。翕翕发熱。无汗。心下滿。微痛。小便不利。桂枝去桂加茯苓术湯主之。28
	\footnote{
		「白术」脉經无「白」字。
	}

傷寒。脉浮。自汗出。小便數。心煩。微惡寒。腳攣急。反与桂枝湯。欲攻其表。得之便厥。咽乾。煩躁。吐{\sungtpii 𠱘}者。当作甘草乾薑湯。以復其陽。若厥愈。足温者。更作芍藥甘草湯与之。其腳即伸。若胃气不和。譫語者。少与{\khaaitp 調胃}承气湯。若重发汗。復加燒針者。四逆湯主之。29

太陽病。項背强几几。无汗。惡風。葛根湯主之。31

太陽与陽明合病。而自利{\khaaitp 者}。葛根湯主之。不下利。但嘔者。葛根加半夏湯主之。32.33
	\footnote{
		「而自利者」趙本作「者必自下利」,《脉經》作「而自利不嘔者」。
		唐弘宇按:此兩條参照第172條的結構,合为一條。
	}

太陽病。桂枝証。醫反下之。遂利不止。脉促者。表未解也。喘而汗出者。葛根芩連湯主之。34

太陽病。頭痛。发熱。身疼。腰痛。骨節疼痛。惡風。无汗而喘。麻黄湯主之。35

太陽与陽明合病。喘而胸滿者。不可下。宜麻黄湯。36

太陽病。十日已去。脉浮細而嗜卧者。外已解也。設胸滿脇痛者。与小柴胡湯。脉{\khaaitp 但}浮者。与麻黄湯。37
	\footnote{
		「脉但浮」脉經、玉函作「脉浮」。
	}

太陽中風。脉浮緊。发熱。惡寒。身疼痛。不汗出而煩躁者。大青龙湯主之。若脉微弱。汗出。惡風者。不可服之。服之則厥。筋愓肉瞤。此为逆也。38
	\footnote{
		「煩躁者」《脉經》卷七第二、《玉函》卷二第三、卷五第十四作「煩躁頭痛」。
		「愓」同《千金方》、成本,《千金翼》、《脉經》、趙本作「惕」,《説文》:「惕,敬也。」此処應为「愓」,「愓」与「蕩」同,为動蕩之意,「筋愓」者,謂筋脉跳動也。
	}

傷寒。脉浮緩。身不疼。但重。乍有輕时。无少陰証者。大青龙湯发之。39
	\footnote{
		「乍」《聖惠方》卷八辨太陽病形証作「或」。
	}

傷寒。表不解。心下有水气。乾嘔。发熱而欬。或渴。或利。或噎。或小便不利。少腹滿。或喘。小青龙湯主之。40

傷寒。心下有水气。欬而微喘。发熱。不渴。服湯已而渴者。此寒去。为欲解。小青龙湯主之。41

太陽病。外証未解。其脉浮弱。当以汗解。宜桂枝湯。42

太陽病。下之。微喘者。表未解也。桂枝{\khaaitp 加厚朴杏仁}湯主之。43
	\footnote{
%		「表未解也」同趙本卷七辨可发汗篇、《千金方》卷九第五,《聖惠方》作「外未解也」,趙本第四十三條、成本、《玉函》、《脉經》作「表未解故也」。
		「桂枝加厚朴杏仁湯」一云「麻黄湯」。
	}

太陽病。外証未解者。不可下。下之为逆。欲解外者。宜桂枝湯。44
%	\footnote{
%		「欲解外者宜桂枝湯」《脉經》无。
%	}

太陽病。先发汗不解而下之。其脉浮者不愈。浮为在外。而反下之。故令不愈。今脉浮。故在外。当解其外則愈。宜桂枝湯。45

太陽病。脉浮緊。无汗。发熱。身疼痛。八九日不解。表証仍在。此当发其汗。服藥已。微除。其人发煩目暝。劇者必衄。衄乃解。所以然者。陽气重故也。麻黄湯主之。46
%	\footnote{
%		「身疼痛」《千金翼》卷九第二、《脉經》卷七第二、《玉函》卷二第三、《玉函》卷五第十四、《外臺》卷三作「其身疼痛」,《聖惠方》卷八作「身痛」。
%	}

太陽病。脉浮緊。发熱。身无汗。自衄者愈。47

二陽并病。太陽初得病时。发其汗。汗先出{\khaaitp 。復}不徹。因轉屬陽明。續自微汗出。不惡寒。若太陽病証不罷者。不可下。下之为逆。如此者可小发汗。設面色緣緣正赤者。陽气怫鬱不得越。当解之熏之。当汗不汗。其人躁煩。不知痛処。乍在腹中。乍在四肢。按之不可得。其人短气。但坐。以汗出不徹故也。更发汗則愈。何以知汗出不徹。以脉濇故知之。48

脉浮數者。法当汗出而愈。若下之。身体重。心悸者。不可发汗。当自汗出而解。所以然者。尺中脉微。此裏虗。須表裏実。津液和。即自汗出愈。49

脉浮而緊。法当身体疼痛。当以汗解之。假令尺中脉遲者。不可发汗。何以知然。以榮气不足。血少故也。50

脉浮者。病在表。可发汗。宜麻黄湯。51
	\footnote{
		「麻黄湯」一云「桂枝湯」。
	}

{\khaaitp 太陽病。}脉浮數者。可发汗。宜麻黄湯。52
	\footnote{
		「麻黄湯」一云「桂枝湯」。
	}

病常自汗出者。此为榮气和。衛气不和也。榮行脉中。衛行脉外。復发其汗。衛和則愈。宜桂枝湯。53
	\footnote{
		「此为榮气和衛气不和也」《玉函》作「此为營气与衛气不和也」,
		《脉經》作「此为榮气和榮气和而外不解此衛不和也」,
		《千金方》作「此为榮气和榮气和而外不解此为衛不和也」,
		《千金翼》作「此为榮气和衛气不和故也」,
		《聖惠方》作「此为榮气和衛气不和」,
		趙本作「此为榮气和榮气和者外不谐以衛气不共榮气谐和故尔」。
	}

病人臓无他病。时发熱。自汗出。而不愈者。此衛气不和也。先其时发汗則愈。宜桂枝湯。54

傷寒。脉浮緊。不发汗。因致衄者。宜麻黄湯。55

{\khaaitp 寸口}脉浮而緊。浮則为風。緊則为寒。風則傷衛。寒則傷榮。榮衛俱病。骨節煩疼。当发其汗。{\khaaitp 宜麻黄湯。}
	\footnote{
		此條見趙本卷一辨脉法、趙本卷七辨可发汗篇、《脉經》卷七第二、《玉函》卷二第二、《玉函》卷五第十四、《聖惠方》卷八辨傷寒脉{\sungtpii 𠊱}、《聖惠方》卷八辨可发汗形証。
	}

傷寒。不大便六七日。頭痛。有熱者。与承气湯。其小便清者。此为不在裏。續在表也。当发其汗。頭痛者必衄。宜桂枝湯。56
	\footnote{
		「与承气湯」《玉函》卷二第三作「未可与承气湯」、《玉函》卷五第十四作「不可与承气湯」。
		「其小便清者」《脉經》、《千金翼》卷九作「其大便反青」,玉函作「其小便反清」,外臺作「其人小便反清者」。
	}

傷寒。发汗已解。半日許復煩。脉浮數者。可復发汗。宜桂枝湯。57
	\footnote{
		「半日許復煩」《聖惠方》卷八辨太陽病形証作「半日後復煩躁」、《千金方》卷九发汗吐下後作「半日許復心煩熱」。
	}

凡病。或发汗。或吐。或下。或亡血。{\khaaitp 内}无津液而陰陽自和者。必自愈。58

大下後。復发汗。其人小便不利。此亡津液。勿治之。得小便利。必自愈。59

下之後。復发汗。必振寒。脉微細。所以然者。内外俱虗故也。60

下之後。復发汗。晝日煩躁不得眠。夜而安靜。不嘔。不渴。无表証。脉沈微。身无大熱。乾薑附子湯主之。61

发汗後。身体疼痛。其脉沈遲。桂枝加芍藥生薑人参湯主之。62

发汗後。汗出而喘。无大熱者。可与麻杏甘石湯。63
	\footnote{
		唐弘宇按:「发汗後」下諸本均有「不可更行桂枝湯」七字,唯獨《醫心方》此條作「治傷寒发汗出而喘无大熱」,顯然《醫心方》文意更通順,故從。第162條亦從此例。
	}

发汗過多。其人叉手自冒心。心下悸。欲得按者。桂枝甘草湯主之。64

发汗後。其人脐下悸。欲作奔豚。苓桂甘棗湯主之。65

发汗後。腹胀滿者。厚朴{\khaaitp 生薑半夏甘草人参}湯主之。66

傷寒吐下发汗後。心下逆滿。气上衝胸。起則頭眩。其脉沈緊。发汗則動經。身为振搖。苓桂术甘湯主之。67
	\footnote{
		「吐下发汗後」玉函作「若吐若下若发汗後」,宋本作「若吐若下後」。「白术」脉經作「术」。
	}

发汗不解。反惡寒者。虗故也。芍藥甘草附子湯主之。不惡寒。但熱者。実也。当和胃气。宜調胃承气湯。68.70
	\footnote{
		「調胃承气湯」除宋本外其它版本均作「小承气湯」。
	}

发汗或下之。{\khaaitp 病仍}不解。煩躁。茯苓四逆湯主之。69

太陽病发汗後。大汗出。胃中乾。煩躁不得眠。其人欲飲水。当稍飲之。令胃气和則愈。若脉浮。小便不利。微熱。消渴者。五苓散主之。71
	\footnote{
		「其人欲飲水当稍飲之」宋本作「欲得飲水者少少与飲之」。
	}

发汗已。脉浮數。煩渴者。五苓散主之。72

傷寒。汗出而渴者。五苓散主之。不渴者。茯苓甘草湯主之。73

中風。发熱。六七日不解而煩。有表裏証。渴欲飲水。水入則吐。此为水逆。五苓散主之。74
	\footnote{
		根據宋本子目小註,第74條「下別有三病証」。
		趙本第75條无方,是为証,第76條有方,是为法,則趙本74條後僅有一証。
		《玉函》、成本第75條分为兩條,第76條亦分为兩條,則74條後有三証,与子目之説相符,故從。
		《玉函》部分條文排序優於趙本,於此可見一斑。
	}

未持脉时。病人叉手自冒心。師因教试令欬。而不即欬者。此必兩耳聋无聞也。所以然者。以重发汗。虗故也。75

发汗後。飲水多者必喘。以水灌之亦喘。75

发汗後。水藥不得入口。为逆。{\khaaitp 若更发汗。必吐下不止。}76

发汗吐下後。虗煩。不得眠。若劇者。反覆顛倒。心中懊憹。栀子{\khaaitp 豉}湯主之。若少气者。栀子甘草{\khaaitp 豉}湯主之。若嘔者。栀子生薑{\khaaitp 豉}湯主之。76

发汗或下之。煩熱。胸中窒者。栀子{\khaaitp 豉}湯主之。77
	\footnote{
		「胸中窒者」《脉經》「窒」作「塞」,《千金方》作「胸中窒气逆搶心者」。
	}

傷寒五六日。大下之後。身熱不去。心中結痛者。未欲解也。栀子{\khaaitp 豉}湯主之。78

傷寒下後。煩而腹滿。卧起不安。栀子厚朴湯主之。79
	\footnote{
		「煩而」宋本作「心煩」。
	}

傷寒。醫以丸藥大下之。身熱不去。微煩。栀子乾薑湯主之。80

凡用栀子湯。其人微溏者。不可与服之。81
	\footnote{
		「其人」趙本作「病人舊」。
	}

太陽病。发汗。汗出不解。其人仍发熱。心下悸。頭眩。身瞤動。振振欲躃地。玄武湯主之。82
	\footnote{
		「躃」宋本作「擗」,編者改。《集韻》:「躃,仆也」;《龍龕手鑑》:「躃,倒也」。
	}

咽喉乾燥者。不可发汗。83

淋家不可发汗。发汗必便血。84

瘡家。雖身疼痛。不可发汗。汗出則痙。85

衄家不可发汗。汗出必額上促急{\khaaitp 緊}。直視不能眴。不得眠。86
	\footnote{
		「額上促急緊」金匱吳本同,宋本作「額上陷脉急緊」,脉經、玉函作「額陷脉上促急而緊」,千金翼作「額上促急」。唐弘宇按:宋本、脉經、玉函此條文意不通,從金匱。
	}

亡血家不可发汗。汗出則寒慄而振。87

汗家。重发汗。必恍惚心亂。小便已。陰疼。与禹餘糧丸。88

\hangindent 1em
\hangafter=0
凡失血者。不可发汗。发汗必恍惚心亂。{\gaoben}88

病者有寒。復发汗。胃中冷。必吐蛔。89
	\footnote{
		「吐蛔」一作「吐{\sungtpii 𠱘}」。
	}

本发汗。而復下之。为逆。若先发汗。治不为逆。本先下之。而反汗之。为逆。若先下之。治不为逆。90

傷寒。醫下之。續得下利。清穀不止。身体疼痛。急当救裏。後身体疼痛。清便自調。急当救表。救裏宜四逆湯。救表宜桂枝湯。91
%	\footnote{
%		《脉經》卷七第二病可发汗証、宋本卷七辨可发汗作「下利後身体疼痛清便自調急当救表宜桂枝湯」,
%		《脉經》卷七第八病发汗吐下以後証作「傷寒醫下之續得下利清穀不止身体疼痛急当救裏身体疼痛清便自調急当救表救裏宜四逆湯救表宜桂枝湯」,
%		《脉經》卷七病可温証第九作「傷寒醫下之續得下利清穀不止身体疼痛急当救裏宜温之以四逆湯」,
%		《玉函》卷二辨太陽病形証第三、趙本第九十一條、趙本卷十辨发汗吐下後、成本第九十一條、成本卷十辨发汗吐下後作「傷寒醫下之續得下利清穀不止身体疼痛急当救裏後身疼痛清便自調急当救表救裏宜四逆湯救表宜桂枝湯」,
%		《玉函》卷六辨可温病形証第二十作「傷寒醫下之而續得下利清穀不止身体疼痛急当救裏宜温之以四逆湯」,
%		《玉函》卷六辨发汗吐下後病形証治第十九作「傷寒醫下之續得下利清穀不止身体疼痛急当救裏後身体疼痛清便自調急当救表救裏宜四逆湯救表宜桂枝湯」。
%	}

病发熱。頭痛。脉反沈。若不差。身体疼痛。当救其裏。宜四逆湯。92

太陽病。先下而不愈。因復发汗。表裏俱虗。其人因冒。冒家当汗出自愈。所以然者。汗出表和故也。裏未和。然後{\khaaitp 復}下之。93
	\footnote{
		「裏未和」同趙本卷二辨太陽病脉証并治第五、《玉函》卷三,趙本卷十第二十二作「得表和」,《玉函》卷六、《脉經》卷七第八、《千金翼》卷十作「表和」。
	}

太陽病未解。脉陰陽俱微。必先振汗出而解。但陽{\khaaitp 脉}微者。先汗之而解。但陰{\khaaitp 脉}微者。先下之而解。汗之宜桂枝湯。下之宜{\khaaitp 調胃}承气湯。94
	\footnote{
		「陰陽俱微」《千金翼》卷九第一、《脉經》卷七第七、《玉函》卷二第三及卷五第十八、趙本、成本作「陰陽俱停」,《脉經》卷七第七作「陰陽俱沈」。
		「停」字下林億孫奇等註云「一作微」。唐弘宇按:「脉停」意義不明。程應旄《傷寒後條辨》、錢潢《傷寒溯源集》、《醫宗金鑑》訓为「停止」,成无己訓为「勻停」、「調和」。查《内經》、《難經》、《傷寒論》均无停脉,且後句的「但陽脉微」、「但陰脉微」当是承接前句而來,故依宋臣所校,將此処改为「微」。
		「調胃承气湯」一云「大柴胡湯」。
	}

血弱气{\sungtpii 𥁞}。腠理開。邪气因入。与正气相摶。結於脇下。正邪分爭。往來寒熱。休作有时。默默不欲飲食。臓腑相連。其痛必下。邪高痛下。故使嘔也。小柴胡湯主之。服柴胡湯已。渴者。屬陽明。以法治之。97

得病六七日。脉遲浮弱。惡風寒。手足温。醫再三下之。不能食。其人脇下滿{\khaaitp 痛}。面目及身黄。頸項强。小便難。与柴胡湯後必下重。本渴。飲水而嘔。柴胡{\khaaitp 湯}不復中与也。食穀者噦。98

傷寒五六日。中風。往來寒熱。胸脇苦滿。默默不欲飲食。心煩。喜嘔。或胸中煩而不嘔。或渴。或腹中痛。或脇下痞堅。或心下悸。小便不利。或不渴。外有微熱。或欬。小柴胡湯主之。96
	\footnote{
		「傷寒五六日中風往來寒熱」同《千金翼》卷九第四、趙本、成本,《脉經》、《玉函》卷五第十三作「中風往來寒熱傷寒五六日已後」,《玉函》卷二第三作「中風五六日傷寒往來寒熱」。
		「外有微熱」同《千金翼》、《脉經》、《玉函》,趙本作「身有微热」。
	}

傷寒四五日。身熱。惡風。頸項强。脇下滿。手足温而渴。小柴胡湯主之。99

傷寒。陽脉濇。陰脉弦。法当腹中急痛。先与小建中湯。不差者。与小柴胡湯。100
	\footnote{
	「小柴胡湯」聖惠方作「大柴胡湯」。
	}

傷寒中風。有柴胡証。但見一証便是。不必悉具。101
	\footnote{
	「有柴胡証」玉函作「有小柴胡証」。
	}

凡柴胡湯証而下之。柴胡証不罷者。復与柴胡湯。必蒸蒸而振。卻发熱汗出而解。101

傷寒二三日。心中悸而煩者。小建中湯主之。102

太陽病。過經十餘日。反再三下之。後四五日。柴胡証仍在。先与小柴胡湯。嘔不止。心下急。其人鬱鬱微煩者。为未解。与大柴胡湯下之則愈。103
	\footnote{
		「嘔不止心下急」同趙本、《外臺》卷一,《千金翼》卷九第四、《脉經》卷七第八、《玉函》卷二第三、卷六第十九作「嘔止小安」。
	}

傷寒十三日不解。胸脇滿而嘔。日晡所发潮熱{\khaaitp 。已}而微利。此本当柴胡湯下之。不得利。今反利者。知醫以丸藥下之。非其治也。潮熱者。実也。先宜服小柴胡湯以解其外。後以柴胡加芒硝湯主之。104
	\footnote{
	「已而微利」《外臺》卷一論傷寒日數病源作「畢而微利」。
	「此本当柴胡湯下之」同《脉經》卷七第八,《千金翼》卷九第四作「此本当柴胡下之」,《玉函》卷六第十九作「此証当柴胡湯下之」,趙本、成本、《外臺》卷一、《玉函》卷二第三作「此本柴胡証」。
	}

柴胡加大黄芒硝桑螵蛸湯。0
	\footnote{
		趙本无此條,《千金翼》、《玉函》有方无証。
	}

傷寒十三日。過經。譫語者。内有熱也。当以湯下之。小便利者。大便当堅。而反{\khaaitp 下}利。脉調和者。知醫以丸藥下之。非其治也。自利者。脉当微厥。今反和者。此为内実也。{\khaaitp 調胃}承气湯主之。105

太陽病不解。熱結膀胱。其人如狂。血自下。下之即愈。其外不解者。尚未可攻。当先解其外。{\khaaitp 宜桂枝湯。}外解已。{\khaaitp 但}少腹急結者。乃可攻之。宜桃仁承气湯。106
	\footnote{
		「下之即愈」同《脉經》卷七第七,《千金翼》卷九第五、《脉經》卷七第二、《玉函》卷二第三、卷五第十四、卷五第八作「下者即愈」,趙本、成本作「下者愈」。
	}

傷寒八九日。下之。胸滿。煩。驚。小便不利。譫語。一身{\khaaitpii 𥁞}{\khaaitp 重。}不可轉側。柴胡加龙骨牡蛎湯主之。107
	\footnote{
		「一身{\sungtpii 𥁞}重不可轉側」同《玉函》卷二第三、趙本、成本、《外臺》卷一,《千金翼》卷九第四、《脉經》卷七第八、《玉函》卷六第十九、《聖惠方》卷八作「一身不可轉側」。
	}

傷寒。腹滿。譫語。寸口脉浮而緊。此为肝乘脾。名曰縱。当刺期門。108

傷寒。发熱。嗇嗇惡寒。其人大渴。欲飲水者。其腹必滿。自汗出。小便利。其病欲解。此为肝乘肺。名曰横。当刺期門。109
	\footnote{
		唐弘宇按:本條後,第110、111條为衍文,第112至119條,其内容不是太陽病,今移至《奔豚气吐膿驚怖火邪》篇。
	}

%傷寒。脉浮。醫以火迫劫之。亡陽。{\khaaitp 必}驚狂。卧起不安。桂枝去芍藥加蜀漆牡蛎龙骨救逆湯主之。112
%
%傷寒。其脉不弦緊而弱{\khaaitp 。弱}者必渴。被火必譫語。{\khaaitp 弱者。发熱。脉浮。解之当汗出愈。}113
%
%太陽病。以火熏之。不得汗。其人必躁。到經不解。必清血。114
%	\footnote{
%		宋本「必清血」下有「名为火邪」四字。
%	}
%	
%\hangindent 1em
%\hangafter=0
%太陽病。以火蒸之。不得汗者。其人必燥結。若不結。必下清血。其脉躁者。必发黄也。{\gaoben}114
%
%脉浮。熱甚。而反灸之。此为実。実以虗治。因火而動。咽燥。必吐血。115
%
%微數之脉。慎不可灸。因火为邪。則为煩逆。追虗逐実。血散脉中。火气雖微。内攻有力。焦骨傷筋。血難復也。116
%
%\hangindent 1em
%\hangafter=0
%凡微數之脉。不可灸。因熱为邪。必致煩逆。内有損骨傷筋血枯之患。{\gaoben}116
%
%脉浮。当以汗解。而反灸之。邪无從出。因火而盛。病從腰以下必重而痹。此为火逆。若欲自解。当先煩。煩乃有汗。隨汗而解。何以知之。脉浮。故知汗出当解。116
%
%\hangindent 1em
%\hangafter=0
%脉当以汗解。反以灸之。邪无所去。因火而盛。病当必重。此为逆治。若欲解者。当发其汗而解也。{\gaoben}116
%
%燒針令其汗。針処被寒。核起而赤者。必发奔豚。气從少腹上衝心者。灸其核上各一壯。与桂枝加桂湯。117
%
%火逆。下之。因燒針。煩躁者。桂枝甘草龙骨牡蛎湯主之。118
%
%傷寒。加温針必驚。119

太陽病。当惡寒。发熱。今自汗出。反不惡寒。发熱。関上脉細數。此醫吐之過也。一二日吐之者。腹中飢。口不能食。三四日吐之者。不喜糜粥。欲食冷食。朝食暮吐。此醫吐之所致也。此为小逆。120

太陽病吐之。但太陽病当惡寒。今反不惡寒。不欲近衣。此为吐之内煩也。121

病人脉數。數为熱。当消穀引食。而反吐者。以醫发其汗。令陽气微。膈气虗。脉則为數。數为客熱。不能消穀。胃中虗冷。故吐也。122

太陽病。過經十餘日。心下温温欲吐。而胸中痛。大便反溏。腹微滿。鬱鬱微煩。先{\khaaitp 此}时自極吐下者。与{\khaaitp 調胃}承气湯。若不尔者。不可与。但欲嘔。胸中痛。微溏者。此非柴胡湯証。以嘔。故知極吐下也。123
	\footnote{
		「心下温温欲吐」《玉函》「温温」作「嗢嗢」。「嗢」讀作[wà]。《説文》:「嗢,咽也」,吞咽之意。《漢語大詞典》:「嗢嗢,象聲詞,反胃欲嘔的聲音。」錢超塵説:「欲吐而吐不暢快的樣子曰嗢嗢欲吐」,他認为《玉函》用「嗢」是本字,趙本作「温」是假借字。
		唐弘宇按:錢老對「嗢嗢欲吐」的解釋沒有證據支持,我比較認同《漢語大詞典》的解釋。「心下温温欲吐」有兩種斷句方式:第一種將「嗢嗢欲吐」作一句讀,模擬欲嘔吐时发出的聲音;第二種將「心下温温」四字作一句讀。若將「温温欲吐」作一句讀,則剩下一個孤立的「心下」,這顯然不合理,所以我認为「心下温温」四字当作一句讀,是形容一種主觀感覺。我推測「温温」應該是一種方言,即使在今天,方言也經常有无法用正字書寫的情況,只能用借字代替,此処的「温温」應該就是這樣一種情況。
		《金匱》肺痿篇:「肺痿。涎唾多。心中温温液液者。炙甘草湯主之。」很明顯此條中的「心中温温液液」就是一種主觀感覺,「液液」的部首也是水部,可証明我的推測。
		《金匱》歷節篇:「諸肢節疼痛。身体魁羸。腳腫如脱。頭眩短气。温温欲吐。桂枝芍藥知母湯主之。」此條的「温温」或可作「嗢嗢」。
	}

太陽病六七日。表証仍在。脉微而沈。反不結胸。其人发狂。此熱在下焦。少腹当堅滿。小便自利者。下血乃愈。所以然者。以太陽隨經。瘀熱在裏故也。抵当湯主之。124
	\footnote{
		「瘀熱在裏」《千金翼》保元堂本、《千金翼》世補齋本作「瘀血在裏」。
		%「抵当湯主之」《千金翼》卷九太陽病雜療法第七作「宜下之以抵当湯」。
	}

太陽病。身黄。脉沈結。少腹堅。小便不利者。为无血也。小便自利。其人如狂者。血証諦也。抵当湯主之。125

傷寒。有熱。少腹滿。應小便不利。今反利者。为有血也。当下之。宜抵当丸。126
	\footnote{
		「当下之」三字下宋本、玉函、千金翼有「不可餘藥」四字。
	}

太陽病。小便利者。以飲水多。必心下悸。小便少者。必苦裏急也。127

問曰。病有結胸。有臓結。其狀何如。\\
答曰。按之痛。寸口脉浮。関上自沈。为結胸。\\
問曰。何谓臓結。\\
答曰。如結胸狀。飲食如故。时时下利。寸口脉浮。関上細沈而緊。为臓結。舌上白胎滑者。難治。128.129

臓結无陽証。不往來寒熱。其人反靜。舌上胎滑者。不可攻也。130

病发於陽。而反下之。熱入因作結胸。病发於陰。而反下之。因作痞。所以成結胸者。以下之太早故也。131

結胸者。項亦强。如柔痙狀。下之則和。宜大陷胸丸。131
	\footnote{
		唐弘宇按:根據宋本子目小註,第131條「前後有結胸臓結病六証」。趙本第131條之前的第128、129、130條为結胸臓結証,之後的第132、133條为結胸証,前後共五証,非六証。成本第128、129條合为一條,其餘同趙本,共四証,非六証。《玉函》卷三第四第131條分为兩條,則131條前有四証,後有二証,共六証,与子目之説相合。
		又,根據第128、129條文義,將其合为一條。
	}

結胸証。脉浮大者。不可下。下之則死。132

結胸証悉具。而煩躁者死。133

太陽病。脉浮而動數。浮則为風。數則为熱。動則为痛。數則为虗。頭痛。发熱。微盜汗出。而反惡寒。其表未解。醫反下之。動數變遲。膈内拒痛。胃中空虗。客气動膈。短气。躁煩。心中懊憹。陽气内陷。心下因堅。則为結胸。大陷胸湯主之。若不結胸。但頭汗出。餘処无汗。齐頸而還。小便不利。身必发黄。134
	\footnote{
		「膈内拒痛」《脉經》卷七第八、《千金翼》卷九第六作「頭痛即眩」,《玉函》卷三第四、卷六第十九作「頭痛則眩」。
	}

傷寒六七日。結胸熱実。脉沈緊。心下痛。按之如石堅。大陷胸湯主之。135

傷寒十餘日。熱結在裏。復往來寒熱者。与大柴胡湯。但結胸。无大熱者。此为水結在胸脇。{\khaaitp 但}頭微汗出。大陷胸湯主之。136
	\footnote{
		「熱結在裏」《千金翼》卷九第四作「邪气結在裏」。
		此條趙本卷九辨可下篇、《外臺》卷二分作兩條。
	}

太陽病。重发汗而復下之。不大便五六日。舌上燥而渴。日晡所小有潮熱。從心下至少腹堅滿而痛不可近。大陷胸湯主之。137

小結胸者。正在心下。按之則痛。其脉浮滑。小陷胸湯主之。138

太陽病二三日。不能卧。但欲起。心下必結。脉微弱者。此本寒也。而反下之。利止者。必結胸。未止者。四五日復下之。此挾熱利也。139
	\footnote{
		「此本寒也」同《千金翼》卷九第六、《脉經》卷七第八、《玉函》卷三第四、卷六第十九,趙本、成本作「此本有寒分也」,《外臺》卷二作「本有久寒也」。
	}

太陽病。下之。其脉促。不結胸者。此为欲解。脉浮者。必結胸。脉緊者。必咽痛。脉弦者。必兩脇拘急。脉細數者。頭痛未止。脉沈緊者。必欲嘔。脉沈滑者。挾熱利。脉浮滑者。必下血。140

病在陽。当以汗解。反以水潠之或灌之。其熱被劫不得去。益煩。皮上粟起。意欲飲水。反不渴。宜服文蛤散。若不差。与五苓散。141
	\footnote{
		「反以水潠之或灌之」《千金翼》卷九第六、《脉經》卷七第十四作「而反以水噀之若灌之」,《玉函》卷三第四、卷六第二十七作「而反以水潠之若灌之」,趙本、成本、《外臺》卷二作「反以冷水潠之」。
		《説文》:「潠,含水噴也。」《古今韻會舉要》:「潠,噴水也,亦作噀。」《龍龕手鑑》:「噀,俗;{\sungtpii 𠹀},正也,与潠同。」
		唐弘宇按:此句「或」諸本均作「若」,錢超塵説:「若,選擇連詞,義为或、或者。」據改。
		「其熱被劫不得去」同趙本、成本、《玉函》卷三第四,《脉經》卷七第十四、《玉函》卷六第二十七、《千金翼》卷九第六、《外臺》卷二作「其熱卻不得去」。
		「益煩」同《千金翼》卷九第六、《脉經》卷七第十四、《玉函》卷三第四、卷六第二十七,趙本、成本、《外臺》卷二作「弥更益煩」。唐弘宇按:「彌」、「更」、「益」三字意義重復。
	}

寒実結胸。无熱証者。与三物白散。141
	\footnote{
		此條趙本、成本、《脉經》、《玉函》均与上條連寫为一條,《千金翼》卷九第六、《外臺》卷二單獨列为一條。
	}

太陽与少陽并病。頭項强痛。或眩冒。时如結胸。心下痞堅。当刺大椎第一間。肺腧。肝腧。慎不可发汗。发汗則譫語。脉弦。譫語五日不止者。当刺期門。142

婦人中風。发熱。惡寒。經水適來。得之七八日。熱除。脉遲。身涼。胸脇下滿。如結胸狀。譫語。此为熱入血室。当刺期門。隨其{\khaaitp 虗}実而取之。143
	\footnote{「隨其虗実而取之」脉經、玉函、千金翼同,宋本、玉函作「隨其実而取之」,成本作「隨其実而瀉之」。}

婦人中風七八日。續得寒熱。发作有时。經水適斷。此为熱入血室。其血必結。故使如瘧狀。发作有时。小柴胡湯主之。144

婦人傷寒。发熱。經水適來。晝日明了。暮則譫語。如見鬼狀。此为熱入血室。无犯胃气及上二焦。必自愈。145
	\footnote{「明了」宋本、玉函作「明瞭」。「二焦」脉經作「三焦」。}

傷寒六七日。发熱。微惡寒。肢節煩疼。微嘔。心下支結。外証未去者。柴胡桂枝湯主之。146

\hangindent 1em
\hangafter=0
发汗多。亡陽。狂語者。不可下。与柴胡桂枝湯。和其榮衛。以通津液。後自愈。
	\footnote{
		此條不見於趙本六經篇,而見於趙本辨发汗後病脉証并治第十七、《脉經》卷七病发汗以後証第三、《千金翼》卷九太陽病用柴胡湯法第四。「狂語」趙本、成本作「譫語」。
	}

傷寒五六日。已发汗而復下之。胸脇滿。微結。小便不利。渴而不嘔。但頭汗出。往來寒熱。心煩。此为未解。柴胡桂枝乾薑湯主之。147

\hangindent 1em
\hangafter=0
傷寒六日。已发汗及下之。其人胸脇滿。大腸微結。小腸不利而不嘔。但頭汗出。往來寒熱而煩。此为未解。宜小柴胡桂枝湯。{\gaoben}147

傷寒五六日。頭汗出。微惡寒。手足冷。心下滿。口不欲食。大便堅。其脉細。此为陽微結。必有表。復有裏。沈亦为病在裏。汗出为陽微。假令純陰結。不得有外証。悉入在裏。此为半在外半在裏。脉雖沈緊。不得为少陰病。所以然者。陰不得有汗。今頭汗出。故知非少陰也。可与{\khaaitp 小}柴胡湯。設不了了者。得屎而解。148

傷寒五六日。嘔而发熱。柴胡湯証具。而以他藥下之。柴胡証仍在者。復与柴胡湯。此雖已下之。不为逆。必蒸蒸而振。卻发熱汗出而解。若心下滿而堅痛者。此为結胸。宜大陷胸湯。若但滿而不痛者。此为痞。柴胡{\khaaitp 湯}不復中与也。宜半夏瀉心湯。149

太陽与少陽并病。而反下之。{\khaaitp 成}結胸。心下堅。下利不止。水漿不下。其人心煩。150

脉浮緊。而反下之。緊反入裏。則作痞。按之自濡。但气痞耳。151

太陽中風。下利。嘔{\sungtpii 𠱘}。表解乃可攻之。其人漐漐汗出。发作有时。頭痛。心下痞堅滿。引脇下痛。乾嘔。短气。汗出。不惡寒。此为表解裏未和。十棗湯主之。152

太陽病。醫发其汗。遂发熱。惡寒。復下之。則心下痞。此表裏俱虗。陰陽气并竭。无陽則陰獨。復加燒針。因胸煩。面色青黄。膚瞤者。難治。今色微黄。手足温者。易愈。153

心下痞。按之濡。其脉関上浮者。大黄{\khaaitp 黄連}瀉心湯主之。154

心下痞。而復惡寒。汗出者。附子瀉心湯主之。155

本以下之。故心下痞。与瀉心湯。痞不解。其人渴而口燥{\khaaitp 煩}。小便不利。五苓散主之。156
	\footnote{
		「口燥煩」《脉經》卷七第八无「煩」字。
	}

傷寒。汗出。解之後。胃中不和。心下痞堅。乾噫食臭。脇下有水气。腹中雷鳴而利。生薑瀉心湯主之。157
%	\footnote{
%		「傷寒汗出解之後」《千金方》卷九作「傷寒发汗後」,《聖惠方》卷八作「太陽病汗出後」。
%	}

傷寒中風。醫反下之。其人下利。日數十行。穀不化。腹中雷鳴。心下痞堅而滿。乾嘔。心煩。不{\khaaitp 能}得安。醫見心下痞。谓病不{\sungtpii 𥁞}。復下之。其痞益甚。此非結熱。但以胃中虗。客气上逆。故使之堅。甘草瀉心湯主之。158
%	\footnote{
%		「故使之堅」《千金方》作「使之然也」。
%	}

傷寒。服湯藥。下利不止。心下痞堅。服瀉心湯已。復以他藥下之。利不止。醫以理中与之。利益甚。理中者。理中焦。此利在下焦。赤石脂禹餘糧湯主之。復不止者。当利小便。159
	\footnote{
		「理中者理中焦」《千金翼》卷九第六作「理中治中焦」。
	}

傷寒。吐下{\khaaitp 後}发汗。虗煩。脉甚微。八九日。心下痞堅。脇下痛。气上衝咽喉。眩冒。經脉動愓者。久而成痿。160

傷寒。发汗{\khaaitp 或}吐{\khaaitp 或}下。解後。心下痞堅。噫气不除者。旋覆代赭湯主之。161

下後。汗出而喘。无大熱者。可与麻杏甘石湯。162

太陽病。外証未除。而數下之。遂挾熱而利。利下不止。心下痞堅。表裏不解。桂枝人参湯主之。163

傷寒。大下後。復发汗。心下痞。惡寒者。表未解也。不可攻痞。当先解表。表解乃可攻痞。解表宜桂枝湯。攻痞宜大黄黄連瀉心湯。164

傷寒。发熱。汗出不解。心中痞堅。嘔吐。下利。大柴胡湯主之。165

病如桂枝証。頭不痛。項不强。寸{\khaaitp 口}脉微浮。胸中痞堅。气上衝咽喉。不得息。此为胸有寒。当吐之。宜瓜蒂散。166
	\footnote{「脉微浮」脉經作「脉微細」。「气上衝」脉經、玉函作「气上撞」。}

諸亡血。虗家。不可与瓜蒂散。
	\footnote{
		唐弘宇按:此條諸本均在瓜蒂散煎服法中,未入正文。根據宋本獨有的子目,第166條瓜蒂散証後註云:「下有不可与瓜蒂散証」,也就是説第166條後当有一條不可与瓜蒂散的條文,故編者將此條列为正文。
	}

病者脇下素有痞。連在脐傍。痛引少腹。入陰筋者。此为臓結。死。167
	\footnote{「入陰筋者」玉函作「入陰俠陰筋者」。}

傷寒或吐或下後。七八日不解。熱結在裏。表裏俱熱。时时惡風。大渴。舌上乾燥而煩。欲飲水數升。白虎{\khaaitp 加人参}湯主之。168

\hangindent 1em
\hangafter=0
傷寒六日不解。熱結在裏。但熱。时时惡風。大渴。舌乾。煩躁。宜白虎湯。{\gaoben}168

凡用白虎湯。立夏後至立秋前得用之。立秋後不可服。

春三月。病常苦裏冷。不可与白虎湯。与之則嘔利而腹痛。
	\footnote{
		「春三月」宋本作「正月二月三月」。
		「病常苦裏冷」宋本、千金作「尚凜冷」。
	}

諸亡血。虗家。不可与白虎湯。得之則腹痛而利。但当温之。
	\footnote{
		此條千金无。「得之則腹痛而利」玉函作「得之腹痛而利者」,宋本作「得之則腹痛利者」。「但当温之」玉函作「急当温之」,宋本作「但可温之可愈」。
		唐弘宇按:以上三條諸本均合为一舉,位於《宋本》168條後的白虎加人参湯煎服法中,未入正文;位於《千金翼》176條後的白虎湯煎服法中,未入正文;位於《玉函》170條後,为正文。《千金方》我手邊沒有可参考的善本。根據《宋本》的子目,168條後註云:「下有不可与白虎湯証」,也就是説第168條後当有一條不可与白虎湯的條文,故編者將此三條合为一條,并列为正文。
	}

傷寒。无大熱。口燥渴。心煩。背微惡寒。白虎{\khaaitp 加人参}湯主之。169

傷寒。脉浮。发熱。无汗。其表不解。不可与白虎湯。渴欲飲水。无表証者。白虎{\khaaitp 加人参}湯主之。170

太陽与少陽合病。自下利者。与黄芩湯。若嘔者。与黄芩加半夏生薑湯。172

傷寒。胸中有熱。胃中有邪气。腹中痛。欲嘔吐。黄連湯主之。173

傷寒八九日。風濕相摶。身体疼煩。不能自轉側。不嘔。不渴。脉浮虗而濇者。桂枝附子湯主之。若其人大便堅。小便自利者。术附子湯主之。174

風濕相摶。骨節疼煩。掣痛。不得屈伸。近之則痛劇。汗出。短气。小便不利。惡風。不欲去衣。或身微腫。甘草附子湯主之。175

傷寒。脉浮滑。此以表有熱。裏有寒。白虎湯主之。176

傷寒。脉結代。心動悸。炙甘草湯主之。177
	\footnote{
		「心動悸」《玉函》作「心中驚悸」。
		唐弘宇按:趙本此條後有第178條。按照北宋校正醫書局子目的撰寫体例,則本卷子目177條下当有「下有結代脉一証」小註,而事実是沒有。又查《千金翼》、《玉函》皆无第178條。「脉按之來緩时一止復來者名曰結」見於辨脉篇,其後的結代脉語句見於《脉經》、《千金方》。本條可能是彙集相関内容後,沾益与177條之後。
		}

\chapter{辨陽明病}

陽明之为病。胃家実是也。180

\hangindent 1em
\hangafter=0
傷寒二日。陽明受病。陽明者。胃中寒是也。宜桂枝湯。{\gaoben}180

問曰。病有太陽陽明。有正陽陽明。有微陽陽明。何谓也。\\
答曰。太陽陽明者。脾約是也。正陽陽明者。胃家実是也。微陽陽明者。发汗或利小便。胃中燥。大便難是也。179
	\footnote{「微陽」宋本作「少陽」。宋本「胃中燥」下有「煩実」二字。}

問曰。何緣得陽明病。\\
答曰。太陽病。发汗或下之。亡其津液。胃中乾燥。因轉屬陽明。不更衣。内実。大便難者。为陽明病也。181

問曰。陽明病外証云何。\\
答曰。身熱。汗出。而不惡寒。{\khaaitp 但}反惡熱。182

問曰。病有得之一日。不发熱而惡寒者。何。\\
答曰。然雖一日。惡寒自罷。即汗出。惡熱也。183\\
問曰。惡寒何故自罷。\\
答曰。陽明居中。主土。万物所歸。无所復傳。始雖惡寒。二日自止。此为陽明病也。184

本太陽。初得病时。发其汗。汗先出{\khaaitp 。復}不徹。因轉屬陽明。185

\hangindent 1em
\hangafter=0
太陽病而发汗。汗雖出。復不解。不解者。轉屬陽明也。宜麻黄湯。{\gaoben}185

傷寒。发熱。无汗。嘔不能食。而反汗出濈濈然。是为轉屬陽明。185

傷寒。脉浮而緩。手足自温。是为系在太陰。太陰{\khaaitp 身}当发黄。若小便自利者。不能发黄。至七八日。大便堅者。为屬陽明。187

傷寒轉系陽明者。其人濈濈然微汗出也。188

陽明中風。口苦。咽乾。腹滿。微喘。发熱。惡寒。脉浮而緊。若下之。則腹滿。小便難也。189

陽明病。能食为中風。不能食为中寒。190

陽明{\khaaitp 病。}中寒。不能食。小便不利。手足濈然汗出。此欲作堅瘕。必大便頭堅後溏。所以然者。胃中冷。水穀不別故也。191

\hangindent 1em
\hangafter=0
陽明中寒。不能食。小便不利。手足濈然汗出。欲作堅癥也。所以然者。胃中水穀不化故也。宜桃仁承气湯。{\gaoben}191

陽明病。初欲食。小便反不利。大便自調。其人骨節疼。翕翕如有熱狀。奄然发狂。濈然汗出而解。此为水不勝穀气。与汗共并。脉緊則愈。192

陽明病欲解时。從申{\sungtpii 𥁞}戍。193

陽明病。不能食。下之不解。攻其熱必噦。所以然者。胃中虗冷故也。{\khaaitp 以其人本虗。故攻其熱必噦。}194

\hangindent 1em
\hangafter=0
陽明病。能食。下之不解。其人不能食。攻其熱必噦者。胃中虗冷也。宜半夏湯。{\gaoben}194

陽明病。脉遲。食難用飽。飽則发煩。頭眩。必小便難。此欲作穀疸。雖下之。腹滿如故。所以然者。脉遲故也。195

\hangindent 1em
\hangafter=0
陽明病。脉遲。发熱。頭眩。小便難。此欲作榖疸。下之必腹滿。宜柴胡湯。{\gaoben}195

陽明病当多汗。而反无汗。其身如虫行皮中狀。此以久虗故也。196

陽明病。反无汗。但小便利。二三日。嘔而欬。手足厥者。其人頭必痛。若不嘔。不欬。手足不厥者。頭不痛。197

陽明病。但頭眩。不惡寒。故能食而欬。其人咽必痛。若不欬者。咽不痛。198

陽明病。脉浮緊者。必潮熱。发作有时。{\khaaitp 脉}但浮者。必盜汗出。201

陽明病。无汗。小便不利。心中懊憹者。必发黄。199

\hangindent 1em
\hangafter=0
陽明病。无汗。小便不利。心中熱壅。必发黄也。宜茵陳湯。{\gaoben}199

陽明病。被火。額上微汗出。小便不利者。必发黄。200

\hangindent 1em
\hangafter=0
陽明病。被火灸。其額上微有汗出。小便不利者。必发黄也。宜茵陳湯。{\gaoben}200

陽明病。口燥。但欲漱水。不欲咽者。必衄。202

陽明病。本自汗出。醫復重发汗。病已差。其人微煩不了了者。此必大便堅故也。以亡津液。胃中乾燥。故令其堅。当問其小便日幾行。若本日三四行。今日再行者。知必大便不久出。今为小便數少。津液当還入胃中。故知不久必大便也。203

傷寒。嘔多。雖有陽明証。不可攻之。204

陽明病。心下堅滿者。不可攻之。攻之遂利不止者死。利止者愈。205

陽明病。面合色赤者。不可攻之。{\khaaitp 攻之}必发熱。色黄。小便不利也。206
%	\footnote{
%		「面合色赤」錢超塵在出版时間較早的《影印孫思邈本校注考證》中説:「合,当也,應也。」又在後來出版的《宋本傷寒論文獻史論》中説:「合字訛,当作垢。」
%	}

陽明病。不吐下而{\khaaitp 心}煩者。可与{\khaaitp 調胃}承气湯。207

陽明病。脉遲。雖汗出。不惡寒。其身必重。短气。腹滿而喘。有潮熱。如此者。其外为解。可攻其裏。若手足濈然汗出者。此大便已堅。{\khaaitp 大}承气湯主之。若汗多。微发熱。惡寒者。为外未解。{\khaaitp 桂枝湯主之。}其熱不潮。未可与承气湯。若腹大滿。而不大便者。可与小承气湯。微和其胃气。勿令至大下。208
	\footnote{
		%「如此者其外为解可攻其裏」趙本、《千金方》、《外臺》作「者此外欲解可攻裏也」。
		「桂枝湯主之」趙本无。趙本「外未解也」下有「一法与桂枝湯」小字註釋,南宋李檉以北宋校訂本《傷寒論》为底本撰《傷寒要旨藥方》卷二大承气湯第七十三條有「桂枝湯主之」,趙本卷九辨可下篇、《千金方》卷九宜下亦有「桂枝湯主之」。
	}

陽明病。潮熱。大便微堅者。可与{\khaaitp 大}承气湯。不堅者。不可与之。若不大便六七日。恐有燥屎。欲知之法。可少与小承气湯。若腹中轉失气者。此有燥屎也。乃可攻之。若不轉失气者。此但頭堅後溏。不可攻之。攻之必腹滿。不能食也。欲飲水者。与水即噦。其後发熱者。必大便復堅而少也。以小承气湯和之。若不轉失气者。慎不可攻之。209
	\footnote{「必大便復堅而少也」脉經、玉函卷五第十七、翼方作「必復堅」,玉函卷三第五作「必復堅而少也」,聖惠方作「必腹堅脹」,宋本作「必大便復硬而少也」。}

\hangindent 1em
\hangafter=0
陽明病。有潮熱。大便堅。可与承气湯。若有結燥。乃可徐徐攻之。若无壅滯。不可攻之。攻之者。必腹滿。不能食。欲飲水者即噦。其{\sungtpii 𠊱}发熱。必腹堅胀。宜与小承气湯。{\gaoben}209

夫実則譫語。虗則鄭聲。鄭聲者。重語也。直視。譫語。喘滿者死。下利者亦死。210
	\footnote{「鄭聲者重語也」外臺作「鄭聲重語也」五字寫作雙行小字註釋。}

发汗多。重发汗。{\khaaitp 此为}亡陽。{\khaaitp 若}譫語。脉短者死。脉自和者不死。211

傷寒。吐下後未解。不大便五六日。至十餘日。其人日晡所发潮熱。不惡寒。獨語。如見鬼{\khaaitp 神之}狀。若劇者。发則不識人。順衣妄撮。怵惕不安。微喘。直視。脉弦者生。濇者死。{\khaaitp 若}微者。但发熱。譫語。{\khaaitp 大}承气湯主之。若一服利。止後服。212
	\footnote{
		「若微者」各本均无「若」字。唐弘宇按:此條「若劇者」与「若微者」为并列的兩種情況,故加「若」字。
	}

陽明病。其人多汗。津液外出。胃中燥。大便必堅。堅則譫語。{\khaaitp 小}承气湯主之。{\khaaitp 若一服譫語止。莫復服。}213

陽明病。譫語。发潮熱。脉滑疾者。{\khaaitp 小}承气湯主之。因与承气湯一升。腹中轉失气者。復与一升。若不轉失气者。勿更与之。明日又不大便。脉反微濇者。此为裏虗。为難治。不可復与承气湯。214
	\footnote{「譫語」二字下玉函、千金翼、聖惠方有「妄言」二字。}

陽明病。譫語。有潮熱。反不能食者。{\khaaitp 胃中}必有燥屎五六枚。若能食者。但堅耳。{\khaaitp 大}承气湯主之。215
	\footnote{「胃中」二字除宋本外其它版本均无。\\「大承气湯主之」宋本作「宜大承气湯下之」。}

陽明病。下血。譫語者。此为熱入血室。但頭汗出者。当刺期門。隨其実而瀉之。濈然汗出則愈。216

汗出。譫語者。以有燥屎在胃中。此風也。{\khaaitp 須下者。}過經乃可下之。下之若早。語言必亂。以表虗裏実故也。下之則愈。宜{\khaaitp 大}承气湯。217
	\footnote{「大承气湯」一云「大柴胡湯」。}

傷寒四五日。脉沈而喘滿。沈为在裏。反发其汗。津液越出。大便为難。表虗裏実。久則譫語。218

三陽合病。腹滿。身重。難以轉側。口不仁。面垢。譫語。遺尿。发汗則譫語{\khaaitp 甚}。下之則額上生汗。手足厥冷。自汗。白虎湯主之。219
	\footnote{「发汗則譫語甚」除脉經外均无「甚」字\\「自汗」宋本作「若自汗出者」。}

二陽并病。太陽証罷。但发潮熱。手足漐漐汗出。大便難。而譫語者。下之則愈。宜{\khaaitp 大}承气湯。220

陽明病。脉浮緊。咽乾。口苦。腹滿而喘。发熱。汗出。不惡寒。反惡熱。身重。若发汗則躁。心憒憒。反譫語。若加温針。必怵惕。煩躁。不得眠。若下之。則胃中空虗。客气動膈。心中懊憹。舌上胎者。栀子{\khaaitp 豉}湯主之。若渴欲飲水。口乾舌燥者。白虎{\khaaitp 加人参}湯主之。若脉浮。发熱。渴欲飲水。小便不利者。豬苓湯主之。221.222.223
	\footnote{
		據孫世揚考証,「舌上胎者」之「胎」当作「菭」,《説文解字》:「菭,水衣也。」
	}

\hangindent 1em
\hangafter=0
陽明病。脉浮。咽乾。口苦。腹滿。汗出而喘。不惡寒。反惡熱。心躁。譫語。不得眠。胃虗。客熱。舌燥。宜栀子湯。{\gaoben}221

陽明病。汗出多而渴者。不可与豬苓湯。以汗多。胃中燥。豬苓湯復利其小便故也。224

\hangindent 1em
\hangafter=0
陽明病。汗出多而渴者。不可与豬苓湯。汗多者。胃中燥也。汗少者。宜与豬苓湯利其小便。{\gaoben}224

{\khaaitp 陽明病。}脉浮而遲。表熱裏寒。下利清穀者。四逆湯主之。225

{\khaaitp 陽明病。}若胃中虗冷。不能食者。飲水即噦。226

脉浮。发熱。口乾。鼻燥。能食者。則衄。227

\hangindent 1em
\hangafter=0
脉浮。发熱。口鼻中燥。能食者。必衄。宜黄芩湯。{\gaoben}227

陽明病。下之。其外有熱。手足温。不結胸。心中懊憹。飢不能食。但頭汗出。栀子{\khaaitp 豉}湯主之。228

陽明病。发潮熱。大便溏。小便自可。胸脇滿不去。小柴胡湯主之。229

\hangindent 1em
\hangafter=0
陽明病。发潮熱。大便溏。小便自利。胸脇煩滿不止。宜小柴胡湯。{\gaoben}229

陽明病。脇下堅滿。不大便而嘔。舌上白胎者。可与小柴胡湯。上焦得通。津液得下。胃气因和。身濈然汗出而解。230

\hangindent 1em
\hangafter=0
陽明病。脇下堅滿。大便祕而嘔。口燥。宜柴胡湯。{\gaoben}230

陽明中風。脉弦浮大。而短气。腹都滿。脇下及心痛。久按之。气不通。鼻乾。不得汗。嗜卧。一身及目悉黄。小便難。有潮熱。时时噦。耳前後腫。刺之小差。外不解。病過十日。脉續浮者。与{\khaaitp 小}柴胡湯。脉但浮。无餘証者。与麻黄湯。若不尿。腹滿加噦者。不治。231.232

\hangindent 1em
\hangafter=0
陽明病。中風。其脉浮大。短气。心痛。鼻乾。嗜卧。不得汗。一身悉黄。小便難。有潮熱而噦。耳前後腫。刺之雖小差。外若不解。宜柴胡湯。{\gaoben}231.232

陽明病。自汗出。若发汗。小便自利者。此为{\khaaitp 津液}内竭。雖堅。不可攻之。当須自欲大便。宜蜜煎。導而通之。若土瓜根及豬膽汁。皆可以導。233

陽明病。脉遲。汗出多。微惡寒者。表未解也。可发汗。宜桂枝湯。234

陽明病。脉浮。无汗而喘者。发汗則愈。宜麻黄湯。235
	\footnote{「而喘者」除宋本外其它版本均作「其人必喘」。}

陽明病。发熱。汗出者。此为熱越。不能发黄也。但頭汗出。身无汗。齐頸而還。小便不利。渴引水漿者。此为瘀熱在裏。身必发黄。茵陳{\khaaitp 蒿}湯主之。236
	\footnote{「熱越」聖惠方作「熱退」。}

陽明証。其人喜忘者。必有畜血。所以然者。本有久瘀血。故令喜忘。屎雖堅。大便反易。其色必黑。抵当湯主之。237

陽明病。下之。心中懊憹而煩。胃中有燥屎者。可攻。其人腹微滿。頭堅後溏者。不可攻之。若有燥屎者。宜{\khaaitp 大}承气湯。238

病者五六日不大便。繞脐痛。煩躁。发作有时。此有燥屎。故使不大便也。239

病者煩熱。汗出則解。復如瘧狀。日晡所发者。屬陽明。脉実者。当下之。脉浮虗者。当发其汗。下之宜{\khaaitp 大}承气湯。发汗宜桂枝湯。240
	\footnote{
		「日晡所发」《脉經》卷七第七、《千金翼》同,《脉經》卷七第二、《玉函》、《宋本》作「日晡所发熱」。
		「大承气湯」一云「大柴胡湯」。
	}

\hangindent 1em
\hangafter=0
陽明病。脉実者当下。脉浮虗者当汗。下者宜承气湯。汗者宜桂枝湯。{\gaoben}240

\hangindent 1em
\hangafter=0
傷寒病。五六日不大便。繞脐痛。煩躁。汗出者。此为有結。汗出後則暫解。日晡則復发。脉実者。当宜下之。{\gaoben}239.240

大下後。六七日不大便。煩不解。腹滿痛者。此有燥屎。所以然者。本有宿食故也。宜{\khaaitp 大}承气湯。241

病者小便不利。大便乍難乍易。时有微熱。怫㥜不能卧者。有燥屎故也。宜{\khaaitp 大}承气湯。242
	\footnote{
		「怫㥜」《玉函》卷五第十八、《千金翼》作「怫鬱」,《脉經》、《玉函》卷三第五、《宋本》作「喘冒」,《醫心方》作「沸胃」。《集韻》:怫㥜,心不安也。
	}

食穀欲嘔者。屬陽明。{\khaaitp 吳}茱萸湯主之。得湯反劇者。屬上焦。243
	\footnote{
		「得湯反劇者屬上焦」八字千金翼在煎服法中。
	}

太陽病。寸{\khaaitp 口}緩。関{\khaaitp 上小}浮。尺{\khaaitp 中}弱。其人发熱。汗出。復惡寒。不嘔。但心下痞者。此为醫下之故也。若不下。其人不惡寒而渴者。此轉屬陽明。小便數者。大便必堅。不更衣十日。无所苦也。{\khaaitp 渴}欲飲水者。少少与之。但以法救之。渴者。宜五苓散。244
	\footnote{
		「太陽病」千金翼作「陽明病」。\\
		「但以法救之渴者」千金翼、脉經作「当以法救渴」。
	}

脉陽微而汗出少者。为自和。汗出多者。为太過。陽脉実。因发其汗。出多者。亦为太過。太過者。陽絕於内。亡津液。大便因堅。245

脉浮而芤。浮則为陽。芤則为陰。浮芤相摶。胃气生熱。其陽則絕。246
	\footnote{
		唐弘宇按:此條为千金方中一长段中的一句。
	}

趺陽脉浮而濇。浮則胃气强。濇則小便數。浮濇相摶。大便則堅。其脾为約。麻子仁丸主之。247

太陽病三日。发汗不解。蒸蒸发熱者。{\khaaitp 屬胃也。調胃}承气湯主之。248

傷寒吐後。腹{\khaaitp 胀}滿者。与{\khaaitp 調胃}承气湯。249

太陽病吐下发汗後。微煩。小便數。大便因堅。可与小承气湯。和之則愈。250

得病二三日。脉弱。无太陽柴胡証。煩躁。心下堅。至四日。雖能食。以{\khaaitp 小}承气湯少与。微和之。令小安。至六日。与承气湯一升。若不大便六七日。小便少者。雖不大便。但頭堅後溏。未定成堅。攻之必溏。当須小便利。屎定堅。乃可攻之。宜{\khaaitp 大}承气湯。251
	\footnote{「雖不大便」宋本作「雖不受食」,玉函作「雖不能食」。}

傷寒六七日。目中不了了。睛不和。无表{\khaaitp 裏}証。大便難。身微熱者。此为実也。急下之。宜{\khaaitp 大}承气湯。252
	\footnote{
		「无表裏証」聖惠方作「无外証」,其餘諸本均同。\\
		「大承气湯」一云「大柴胡湯」。
	}

\hangindent 1em
\hangafter=0
傷寒六七日。目中瞳子不明。无外証。大便難。微熱者。此为実。宜急下之。{\gaoben}252

陽明病。发熱。汗多者。急下之。宜{\khaaitp 大}承气湯。253
	\footnote{「大承气湯」一云「大柴胡湯」。}

发汗不解。腹滿痛者。急下之。宜{\khaaitp 大}承气湯。254
	\footnote{「大承气湯」一云「大柴胡湯」。}

腹滿不減。減不足言。当下之。宜{\khaaitp 大}承气湯。255
	\footnote{「大承气湯」一云「大柴胡湯」。}

陽明与少陽合病而利。脉不負者为順。負者。失也。互相克賊。名为負。256

脉滑而數者。有宿食也。当下之。宜{\khaaitp 大}承气湯。256
	\footnote{「大承气湯」一云「大柴胡湯」。}

病人无表裏証。发熱七八日。雖脉浮數。可下之。{\khaaitp 宜大柴胡湯。}假令下已。脉數不解。合熱則消穀善飢。至六七日。不大便者。有瘀血。宜抵当湯。若脉數不解。而下不止。必挾熱。便膿血。257.258
	\footnote{「宜大柴胡湯」玉函卷五第十八、宋本卷九、脉經卷七第七同,其它版本无。}

傷寒发汗已。身目为黄。所以然者。寒濕相摶。在裏不解故也。以为非瘀熱。而不可下。当於寒濕中求之。259

傷寒七八日。身黄如橘。小便不利。腹微滿者。茵陳{\khaaitp 蒿}湯主之。260
	\footnote{「腹微滿者」脉經、玉函作「少腹微滿」。}

傷寒。身黄。发熱。栀子蘗皮湯主之。261
	\footnote{「身黄发熱」千金翼作「其人发黄」。}

傷寒。瘀熱在裏。身必发黄。麻黄連軺赤小豆湯主之。262
	\footnote{「連軺」千金翼作「連翹」。}

\chapter{辨少陽病}

少陽之为病。口苦。咽乾。目眩。263

\hangindent 1em
\hangafter=0
傷寒三日。少陽受病。口苦乾燥。目眩。宜柴胡湯。{\gaoben}263

少陽中風。兩耳无所聞。目赤。胸中滿而煩。不可吐下。吐下則悸而驚。264

傷寒。脉弦細。頭痛。发熱者。屬少陽。少陽不可发汗。发汗則譫語。此屬胃。胃和則愈。胃不和。煩而悸。265

太陽病不解。轉入少陽。脇下堅滿。乾嘔。不能食。往來寒熱。尚未吐下。脉沈緊者。可与小柴胡湯。266

{\khaaitp 少陽病。}若已吐下发汗温針。{\khaaitp 譫語。}柴胡湯証罷。此为壞病。知犯何逆。以法治之。267

三陽合病。脉浮大。上関上。但欲寐。目合則汗。268

傷寒六七日。无大熱。其人躁煩。此为陽去入陰故也。269

傷寒三日。三陽为{\sungtpii 𥁞}。三陰当受邪。其人反能食。而不嘔。此为三陰不受邪也。270

少陽病欲解时。從寅{\sungtpii 𥁞}辰。272

\chapter{辨太陰病}

太陰之为病。腹滿而吐。食不下。下之益甚。时腹自痛。胸下堅結。273

\hangindent 1em
\hangafter=0
太陰之为病。腹滿而吐。食不下。自利益甚。时腹自痛。若下之。必胸下結硬。{\zhaoben}273

\hangindent 1em
\hangafter=0
傷寒四日。太陰受病。腹滿。吐食。下之益甚。时时腹痛。心胸堅滿。若脉浮者。可发其汗。沈者宜攻其裏也。发汗者宜桂枝湯。攻裏者宜承气湯。{\gaoben}273.276

\hangindent 1em
\hangafter=0
凡病。腹滿。吐食。下之益甚。{\yixin}273
	\footnote{
		唐弘宇按:《醫心方》与《聖惠方》條文的一部分完全相同。《醫心方》与《聖惠方》都是非常古老的版本,且都未經過宋代校正,這兩個版本的文字一致,很可能不是巧合。
		我推測本條初文可能是为「太陰之为病。腹滿。吐食。下之益甚。时腹自痛。胸下堅結。」
	}

太陰病。脉浮者。可发汗。宜桂枝湯。276

太陰中風。四肢煩疼。{\khaaitp 脉}陽微陰濇而长者。为欲愈。274

太陰病欲解时。從亥{\sungtpii 𥁞}丑。275

自利。不渴者。屬太陰。以其臓有寒故也。当温之。宜四逆輩。277

傷寒。脉浮而緩。手足自温者。系在太陰。太陰{\khaaitp 身}当发黄。若小便自利者。不能发黄。至七八日。雖暴煩。下利日十餘行。必自止。所以自止者。脾家実。腐穢已去故也。278

\hangindent 1em
\hangafter=0
傷寒。脉浮而緩。手足自温。是为系在太陰。小便不利。其人当发黄。宜茵陳湯。太陰病不解。雖暴煩。下利十餘行而自止。所以自止者。脾家実。腐穢已去故也。宜橘皮湯。278{\gaoben}

{\khaaitp 本}太陽病。醫反下之。因尔腹滿时痛者。屬太陰。桂枝加芍藥湯主之。大実痛者。桂枝加大黄湯主之。279

太陰为病。脉弱。其人續自便利。設当行大黄芍藥者。宜減之。以其人胃气弱。易動故也。280

\chapter{辨少陰病}

少陰之为病。脉微細。但欲寐。281

少陰病。欲吐不吐。心煩。但欲寐。五六日。自利而渴者。屬少陰。虗故引水自救。若小便色白者。少陰病形悉具。所以然者。以下焦虗寒。不能制水。故白也。282
	\footnote{「欲吐不吐心煩」千金翼作「欲吐而不煩」,聖惠方作「其人欲吐而不煩」。}

\hangindent 1em
\hangafter=0
傷寒五日。少陰受病。其脉微細。但欲寐。其人欲吐而不煩。五日自利而渴者。屬陰虗。故引水以自救。小便白而利者。下焦有虗寒。故不能制水。而小便白也。宜龙骨牡蛎湯。{\gaoben}281.282

病人脉陰陽俱緊。反汗出者。为亡陽。此屬少陰。法当咽痛。而復吐利。283

少陰病。欬而下利。譫語者。被火气劫故也。小便必難。以强責少陰汗也。284

\hangindent 1em
\hangafter=0
少陰病。欬而下利。譫語。是为心臓有積熱故也。小便必難。宜服豬苓湯。{\gaoben}284

少陰病。脉細沈數。病为在裏。不可发汗。285

少陰病。脉微。不可发汗。亡陽故也。陽已虗。尺中弱濇者。復不可下之。286

少陰病。脉緊。至七八日。下利。脉暴微。手足反温。脉緊反去。此为欲解。雖煩。下利。必自愈。287

少陰病。下利。若利自止。惡寒而踡。手足温者。可治。288

少陰病。惡寒而踡。时自煩。欲去衣被者。可治。289
	\footnote{「可治」千金翼作「不可治」。}

少陰中風。脉陽微陰浮者。为欲愈。290

少陰病欲解时。從子{\sungtpii 𥁞}寅。291

少陰病。吐利。手足不逆{\khaaitp 冷}。反发熱者。不死。脉不至者。灸少陰七壯。292

\hangindent 1em
\hangafter=0
少陰病。吐利。手足逆而发熱。脉不足者。灸其少陰。{\gaoben}

少陰病八九日。一身手足{\sungtpii 𥁞}熱者。以熱在膀胱。必便血。293

少陰病。但厥。无汗。而强发之。必動血。未知從何道出。或從口鼻。或從{\khaaitp 耳}目出。是为下厥上竭。为難治。294

少陰病。惡寒。身踡而利。手足逆{\khaaitp 冷}者。不治。295

少陰病。下利止而頭眩。时时自冒者死。297

少陰病。吐利。躁煩。四逆者死。296

少陰病。四逆。惡寒而踡。脉不至。不煩而躁者死。298

少陰病六七日。息高者死。299

少陰病。脉微細沈。但欲卧。汗出。不煩。自欲吐。{\khaaitp 至}五六日。自利。復煩燥。不得卧寐者死。300

少陰病。始得之。反发熱。脉沈者。麻黄細辛附子湯主之。301

少陰病。得之二三日。麻黄附子甘草湯微发汗。以二三日无{\khaaitp 裏}証。故微发汗。302
	\footnote{「无裏証」玉函、成本同,脉經、千金翼、宋本作「无証」。}

少陰病。得之二三日以上。心中煩。不得卧。黄連阿膠湯主之。303

少陰病。得之一二日。口中和。其背惡寒者。当灸之。附子湯主之。304

少陰病。身体痛。手足寒。骨節痛。脉沈者。附子湯主之。305

少陰病。下利。便膿血。桃花湯主之。306

少陰病二三日至四五日。腹痛。小便不利。下利不止。便膿血。桃花湯主之。307

少陰病。下利。便膿血者。可刺。308

少陰病。吐利。手足逆{\khaaitp 冷}。煩躁欲死者。{\khaaitp 吳}茱萸湯主之。309

少陰病。下利。咽痛。胸滿。心煩。豬膚湯主之。310

少陰病二三日。咽痛者。可与甘草湯。不差者。与桔梗湯。311

少陰病。咽中傷。生瘡。不能語言。聲不出者。苦酒湯主之。312

少陰病。咽中痛。半夏散及湯主之。313

少陰病。下利。白通湯主之。314

少陰病。下利。脉微。服白通湯。利不止。厥逆。无脉。乾嘔。煩者。白通加豬膽汁湯主之。服湯脉暴出者死。微{\khaaitp 微}續{\khaaitp 出}者生。315
	\footnote{「利不止」脉經作「下利止」,聖惠方作「止後」。\\「服湯脉暴出者死微微續出者生」脉經作「服湯藥其脉暴出者死微細者生」。}

\hangindent 1em
\hangafter=0
少陰病。下利。服白通湯。止後。厥逆。无脉。煩躁者。宜白通豬苓湯。其脉暴出者死。微微續出者生。{\gaoben}315

少陰病。二三日不已。至四五日。腹痛。小便不利。四肢沈重疼痛而利。此为有水气。其人或欬。或小便{\khaaitp 自}利。或不利。或嘔。玄武湯主之。316
	\footnote{「不利」各本均作「下利」,为編者所改。}

少陰病。下利清穀。裏寒外熱。手足厥逆。脉微欲絕。身反不惡寒。其人面赤。或腹痛。或乾嘔。或咽痛。或利止{\khaaitp 而}脉不出。通脉四逆湯主之。317
	\footnote{「身反不惡寒」千金翼、聖惠方作「身反惡寒」。\\「或利止而脉不出」聖惠方作「或时利止而脉不出者」。}

少陰病。四逆。其人或欬。或悸。或小便不利。或腹中痛。或泄利下重。四逆散主之。318

少陰病。下利六七日。欬而嘔。渴。心煩不得眠。豬苓湯主之。319

\hangindent 1em
\hangafter=0
少陰病。下利。欬而嘔。煩渴。不得眠卧。宜豬苓湯。{\gaoben}319

少陰病。得之二三日。口燥。咽乾者。急下之。宜{\khaaitp 大}承气湯。320

少陰病。{\khaaitp 下}利清水。色青者。心下必痛。口乾燥者。急下之。宜{\khaaitp 大}承气湯。321
	\footnote{
		「色青者」宋本、玉函卷四第八作「色純青」。「急下之」玉函卷四第八、聖惠方、外臺同,脉經、宋本、玉函、千金翼作「可下之」,千金翼卷十作「宜下之」。「大承气湯」一云「大柴胡湯」。
	}

少陰病六七日。腹滿。不大便者。急下之。宜{\khaaitp 大}承气湯。322

少陰病。脉沈者。急温之。宜四逆湯。323

少陰病。其人飲食入則吐。心中温温欲吐。復不能吐。始得之。手足寒。脉弦遲。此胸中実。不可下也。当吐之。若膈上有寒飲。乾嘔者。不可吐。当温之。宜四逆湯。324

少陰病。下利。脉微濇。嘔而汗出。必數更衣。反少者。当温其上。灸之。325
	\footnote{「灸之」二字下宋本、脉經有「一云灸厥陰可五十狀」小註。}

\chapter{辨厥陰病
	\footnote{
		趙本此篇篇名下有「厥利嘔噦附合一十九法方一十六首」小字註釋。趙本第326至381條都歸於此篇之内。
		章太炎説:「近世間流行之《傷寒論》誤將厥利嘔噦列入厥陰篇中,殊失仲景立論之本旨。其実厥陰篇中,僅首條提綱,及條上著有厥陰病三字者,乃为厥陰病之本病,其餘厥利嘔噦諸條,当照《金匱玉函經》,与霍亂、勞復、陰陽易等另列一篇,庶幾无誤。」從。
	}
}

厥陰之为病。消渴。气上撞{\khaaitp 心}。心中疼熱。飢而不欲食。{\khaaitp 甚者}食則吐。下之利不止。326
	\footnote{「食則吐」下,千金翼、玉函、宋本皆有「蛔」字,脉經无。}

厥陰中風。其脉微浮为欲愈。不浮为未愈。327

厥陰病欲解时。從丑{\sungtpii 𥁞}卯。328

厥陰病。渴欲飲水者。少少与之即愈。329

\hangindent 1em
\hangafter=0
傷寒六日。渴欲飲水者。宜豬苓湯。{\gaoben}329

\chapter{辨厥利嘔噦}

諸四逆。厥者。不可下之。虗家亦然。330

\hangindent 1em
\hangafter=0
諸四逆。厥者。不可吐之。虗家亦然。330

傷寒。先厥後发熱而利者。必自止。見厥復利。331

傷寒。始发熱六日。厥反九日而利。凡厥利者。当不能食。今反能食。恐为除中。食以黍餅。不发熱者。知胃气尚在。必愈。恐暴熱來出而復去也。後日脉之。其熱續在者。期之旦日夜半愈。所以然者。本发熱六日。厥反九日。復发熱三日。并前六日。亦为九日。与厥相應。故期之旦日夜半愈。後三日脉之而脉數。其熱不罷者。此为熱气有餘。必发癰膿。332

傷寒。脉遲六七日。而反与黄芩湯徹其熱。脉遲为寒。而与黄芩湯復除其熱。腹中應冷。当不能食。今反能食。此为除中。必死。333

傷寒。先厥後发熱。下利必自止。而反汗出。咽中痛者。其喉为痹。发熱。无汗。而利必自止。若不止。必便膿血。便膿血者。其喉不痹。334

傷寒一二日至四五日。厥者。必发熱。前熱者後必厥。厥深者熱亦深。厥微者熱亦微。厥應下之。而反发汗者。必口傷爛赤。335
	\footnote{「前熱者後必厥」脉經、玉函、千金翼作「前厥者後必熱」。}

凡厥者。陰陽气不相順接。便为厥。厥者。手足逆冷是也。337

傷寒。病厥五日。熱亦五日。設六日当復厥。不厥者自愈。厥終不過五日。以熱五日。故知自愈。336

傷寒。脉微而厥。至七八日。膚冷。其人躁。无暫安时者。此为臓厥。非蛔厥也。蛔厥者。其人当吐蛔。今病者靜。而復时煩。此为臓寒。蛔上入膈。故煩。須臾復止。得食而嘔。又煩者。蛔聞食臭出。其人常自吐蛔。蛔厥者。烏梅丸主之。338

傷寒。熱少。厥微。指頭寒。默默不欲食。煩躁。數日。小便利。色白者。此熱除也。欲得食。其病为愈。若厥而嘔。胸脇煩滿者。其後必便血。339

病者手足厥冷。言我不結胸。小腹滿。按之痛。此冷結在膀胱関元也。340
	\footnote{「小腹」千金翼作「少腹」。}

傷寒。发熱四日。厥反三日。復{\khaaitp 发}熱四日。厥少熱多。其病当愈。四日至七日。熱不除者。必便膿血。341

傷寒。厥四日。熱反三日。復厥五日。其病为進。寒多熱少。陽气退。故为進。342

傷寒六七日。脉微。手足厥{\khaaitp 冷}。煩躁。灸其厥陰。厥不還者死。343
	\footnote{「脉微」千金翼、聖惠方作「其脉數」。}

傷寒。{\khaaitp 发熱。}下利。厥逆。躁不得卧者死。344
	\footnote{「发熱」二字宋本、玉函有,脉經、千金翼无。}

傷寒。发熱。下利至{\khaaitp 甚。}厥不止者死。345

傷寒。六七日不利。忽发熱而利。其人汗出不止者死。有陰无陽故也。346

\hangindent 1em
\hangafter=0
傷寒。厥逆。六七日不利。便发熱而利者生。其人汗出。利不止者死。但有陰无陽故也。{\maijing}346

傷寒五六日。不結胸。腹濡。脉虗。復厥者。不可下。此为亡血。{\khaaitp 下之}死。347

傷寒。发熱而厥七日。下利者。为難治。348

傷寒。脉促。手足厥逆。可灸之。349

傷寒。脉滑而厥者。裏有熱也。白虎湯主之。350

手足厥寒。脉細欲絕。当歸四逆湯主之。若其人内有久寒。当歸四逆加吳茱萸生薑湯主之。351.352

大汗出。熱不去。内拘急。四肢疼。{\khaaitp 又}下利。厥逆而惡寒。四逆湯主之。353
	\footnote{「又下利」脉經作「下利」,千金翼作「若下利」。}

大汗{\khaaitp 出}或大下利。而厥冷者。四逆湯主之。354

病者手足厥冷。脉乍緊。邪結在胸中。心下滿而煩。飢不能食。病在胸中。当吐之。宜瓜蒂散。355

傷寒。厥而心下悸。宜先治水。当与茯苓甘草湯。卻治其厥。不尔。水漬入胃。必作利也。356

傷寒六七日。大下後。{\khaaitp 寸}脉沈遲。手足厥逆。下部脉不至。咽喉不利。唾膿血。泄利不止者。为難治。麻黄升麻湯主之。357

傷寒四五日。腹中痛。若轉气下趨少腹者。为欲自利也。358

傷寒。本自寒下。醫復吐{\khaaitp 下}之。寒格。更逆吐{\khaaitp 下}。食入即出。乾薑黄芩黄連人参湯主之。359

下利。有微熱而渴。脉弱者自愈。360

下利。脉數。有微熱。汗出者。自愈。設{\khaaitp 脉}復緊。为未解。361
	\footnote{「有微熱汗出者」千金翼作「若微发熱汗出者」。\\「設脉復緊」除千金翼外其它版本均无「脉」字。}

下利。手足厥{\khaaitp 冷}。无脉。{\khaaitp 当灸其厥陰。}灸之不温{\khaaitp 而脉不還}。反微喘者死。少陰負趺陽者为順。362

下利。寸脉反浮數。尺中自濇者。必清膿血。363

下利清穀。不可攻表。汗出必胀滿。364

下利。脉沈弦者下重。脉大者为未止。脉微弱數者为欲自止。雖发熱。不死。365

下利。脉沈而遲。其人面少赤。身有微熱。下利清穀者。必鬱冒。汗出而解。其人必微厥。所以然者。其面戴陽。下虗故也。366

下利。脉反數而渴者。今自愈。設不差。必清膿血。以有熱故也。367

下利後。脉絕。手足厥{\khaaitp 冷}。晬时脉還。手足温者生。不還{\khaaitp 不温}者死。368

傷寒。下利日十餘行。脉反実者死。369

下利清穀。裏寒外熱。汗出而厥。通脉四逆湯主之。370

熱利下重者。白頭翁湯主之。371

下利。欲飲水者。为有熱也。白頭翁湯主之。373

下利。腹{\khaaitp 胀}滿。身体疼痛者。先温其裏。乃攻其表。温裏宜四逆湯。攻表宜桂枝湯。372

下利。譫語者。有燥屎也。宜{\khaaitp 小}承气湯。374

下利後更煩。按之心下濡者。为虗煩也。栀子{\khaaitp 豉}湯主之。375

嘔家有癰膿。不可治嘔。膿{\sungtpii 𥁞}自愈。376

嘔而发熱者。小柴胡湯主之。379

嘔而脉弱。小便復利。身有微熱。見厥者。難治。四逆湯主之。377

乾嘔。吐涎沫。頭痛者。{\khaaitp 吳}茱萸湯主之。378
	\footnote{「頭痛者」玉函、千金翼作「而復頭痛」。}

傷寒。大吐大下之。極虗。復極汗者。其人外气怫鬱。復与之水。以发其汗。因得噦。所以然者。胃中寒冷故也。380

傷寒。噦而腹滿。視其前後。知何部不利。利之即愈。381

\chapter{辨霍亂}

問曰。病有霍亂者何。\\
答曰。嘔吐而利。此为霍亂。382

問曰。病发熱。頭痛。身疼。惡寒。吐利者。当屬何病。\\
答曰。当为霍亂。霍亂吐利止。復更发熱也。383
	\footnote{
	「霍亂吐利止」千金翼作「霍亂吐下利止」,玉函作「吐下利止」,宋本作「霍亂自吐下又利止」。
	}

傷寒。其脉微濇。本是霍亂。今是傷寒。卻四五日。至陰經上。轉入陰。必利。本素嘔。下利者。不治。若其人似欲大便。但反失气。而不利者。此屬陽明。便必堅。十三日愈。所以然者。經{\sungtpii 𥁞}故也。384
	\footnote{
	「必利」玉函、千金翼作「当利」,脉經作「必吐利」。
	}

下利後当便堅。堅則能食者愈。今反不能食。到後經中。頗能食。復過一經能食。過之一日当愈。若不愈。不屬陽明也。384

惡寒。脉微。而復利。利止。亡血也。四逆加人参湯主之。385
	\footnote{
	「利止必亡血」宋本、玉函作「利止亡血也」。
	}

霍亂。頭痛。发熱。身疼痛。熱多。欲飲水者。五苓散主之。寒多。不用水者。理中湯主之。386
	\footnote{
	「理中湯」同《千金翼》、《玉函》,趙本作「理中丸」。
	唐弘宇按:霍亂病勢急劇,当用湯剂,故從《千金翼》、《玉函》。
	}

吐利止。而身痛不休者。当消息和解其外。宜桂枝湯小和之。387

吐利。汗出。发熱。惡寒。四肢拘急。手足厥冷。四逆湯主之。388

既吐且利。小便復利。而大汗出。下利清穀。裏寒外熱。脉微欲絕。四逆湯主之。388

吐利已斷。汗出而厥。四肢拘急不解。脉微欲絕。通脉四逆加豬膽汁湯主之。390
	\footnote{
	「吐利已斷」同千金方、趙本,吳本、鄧本均作「吐已下斷」
	}

吐利发汗後。其人脉平。小煩者。以新虗不勝穀气故也。391

\chapter{辨陰易病已後勞復
	\footnote{
	「陰易」除千金翼外其它版本均作「陰陽易」。
	}
}

傷寒陰易之为病。其人身体重。少气。少腹裏急。或引陰中拘攣。熱上衝胸。頭重不欲舉。眼中生眵。{\khaaitp 眼胞赤。}膝脛拘急。燒裩散主之。392
	\footnote{
		「眼中生眵」同《千金方》卷十第三,趙本、成本、《玉函》卷四第十二作「眼中生花」,《外臺》卷二作「眼中生䁾」。「眼胞赤」同《玉函》卷四第十二,趙本、成本、《千金方》卷十第三、《外臺》卷二均无,《千金翼》卷十第七作「痂胞赤」。
	}

大病差後勞復者。枳実栀子湯主之。393

傷寒差已後。更发熱者。小柴胡湯主之。脉浮者。以汗解之。脉沈実者。以下解之。394

大病差後。從腰以下有水气者。牡蛎澤瀉散主之。395

傷寒解後。虗羸少气。气逆欲吐。竹枼石膏湯主之。397

大病差後。其人喜唾。久不了了者。胃上有寒。当温之。宜理中丸。396
	\footnote{
	「胃上有寒」千金翼作「胸上有寒」。
	}

病人脉已解。而日暮微煩者。以病新差。人强与穀。脾胃气尚弱。不能消穀。故令微煩。損穀即愈。398

\chapter{发汗吐下後}

发汗後。身熱。又重发汗。胃中虗冷。必反吐也。0

大下後。口燥者。裏虗故也。0

发汗多。亡陽。狂語者。不可下。与柴胡桂枝湯。和其榮衛。以通津液。後自愈。

\chapter{可与不可}

夫以为疾病至急。倉卒尋按。要者難得。故重集諸可与不可方治。比之三陰三陽篇中。此易見也。又时有不止是三陽三陰。出在諸可与不可中也。
	\footnote{
		錢超塵:『此五十六字叔和語,又見《金匱玉函經》卷五。王叔和《脉經》卷七第一節至第十八節为《傷寒論》條文,皆按可與不可排列,叔和謂之「諸可與不可」,是为叔和第一次整理者。然後又按三陽三陰整理,是为第二次整理。王叔和第三次整理者,見宋本《傷寒論》之辨不可发汗病脉証并治第十五至辨发汗吐下後病脉証并治第二十二。「重集」的原因是這些條文比放在三陰三陽中便於尋按。有些條文为三陰三陽所无,而是「出在諸可與不可中」,即出於《脉經》卷七諸可與不可。這類條文为數不多,故云「時有」。因可見《脉經》卷七諸可與不可基本为仲景《傷寒論》原始結構。換言之,《傷寒論》原始結構非按三陰三陽排列,而是按照「諸可與不可」結構排列,從兩漢文史及醫學文獻考察,辨証治療均按可與不可理法辨治,如《蘇武傳》、《華佗傳》等。王叔和拘於《素問熱論》一日傳一經之説,以为傷寒病亦如此除在《傷寒例》撰文解説外,又按照三陰三陽排列《傷寒論》經文。王叔和理解三陰三陽的意義與後世不同,王氏理解的三陰三陽是日傳一經的概念,這種痕跡在淳化本《傷寒論》和《病源》裏有明顯的反應。叔和又依当時辨証施治習慣,又進行第三次整理,形成諸可與不可諸條。』
	}

\section{不可发汗}

咽中閉塞。不可发汗。发汗即吐血。气微絕。厥冷。

\hangindent 1em
\hangafter=0
凡咽中閉塞。不可发汗。{\gaoben}

厥{\khaaitp 而脉緊}。不可发汗。发汗即聲亂。咽嘶。舌萎。聲不能出。

冬时。不可发汗。发汗必吐利。口中爛。生瘡。0

\hangindent 1em
\hangafter=0
凡積熱在臓。不宜发汗。汗則必吐。口中爛。生瘡。{\gaoben}

欬而小便利。若失小便者。不可攻表。汗出則厥。逆冷。

\hangindent 1em
\hangafter=0
欬嗽小便利者。不可攻表。汗出即逆。{\gaoben}

太陽病。发其汗。因致痙。

\hangindent 1em
\hangafter=0
太陽病。发汗太多。因致痙。(宋本。金匱)

\section{可发汗}

大法。春夏宜发汗。

凡发汗。欲令手足皆周至。汗出漐漐然。一时間許益佳。不可令如水流離。若病不解。当重发汗。汗多則亡陽。陽虗不得重发汗也。

凡服湯发汗。中病便止。不必{\sungtpii 𥁞}剂也。

凡云可发汗而无湯者。丸散亦可用。然不如湯隨証良驗。

凡脉浮者。病在外。可发汗。0

陽明病。脉浮虗者。可发汗。{\yifang}0

陽明病。脉浮數者。可发汗。{\gaoben}0

\section{可吐}

大法。春宜吐。

凡服湯吐。中病便止。不必{\sungtpii 𥁞}剂也。

病胸上諸実。胸中鬱鬱而痛。不能食。欲使人按之。而反有涎唾。下利日十餘行。其脉反遲。寸口{\khaaitp 脉}微滑。此可吐之。吐之利即止。

\hangindent 1em
\hangafter=0
夫胸心滿実。胸中鬱鬱而痛。不能食。多涎唾。下利。其脉遲反逆。寸口脉數。此可吐也。{\gaoben}

宿食在上脘。宜吐之。
	\footnote{
		「上脘」脉經、宋本作「上管」,聖惠方作「胃管」。「管」通「脘」。
	}

\section{不可下}

咽中閉塞。不可下。下之則上輕下重。水漿不下。卧則欲踡。身体急痛。下利日數十行。

諸外実。不可下。下之則发微熱。亡脉則厥。当脐握熱。

諸虗。不可下。下之則渴。引水者易愈。惡水者劇。

脉數者。久數不止。止則邪結。正气不能復。正气卻結於臓。故邪气浮之。与皮毛相得。脉數者。不可下。下之必煩。利不止。

脉浮大。應发汗。醫反下之。此为大逆。

太陽与少陽并病。心下痞堅。頸項强而眩。{\khaaitp 当刺大椎第一間。肺腧。肝腧。}不可下。171

\hangindent 1em
\hangafter=0
太陽与少陽并病。頭項强痛。或眩冒。时如結胸。心下痞堅者。当刺大椎第一間。肺腧。肝腧。慎不可发汗。发汗則譫語。脉弦。譫語五日不止者。当刺期門。142

病欲吐者。不可下。

夫病陽多者熱。下之則堅。汗出多極。发其汗亦堅。

夫病陽多熱。下之則堅。汗出多極。发其汗亦堅。{\yuhan}

欬而小便利。若失小便。不可攻其表。汗出則厥逆冷。汗出多極。发其汗亦堅。{\maijing}

\section{可下}

大法。秋宜下。

凡可下者。用湯勝丸散。

凡服湯下。中病則止。不必{\sungtpii 𥁞}剂也。

下利。三部脉皆平。按之心下堅者。急下之。宜{\khaaitp 大}承气湯。

\hangindent 1em
\hangafter=0
傷寒下痢。三部脉皆和。按其心下堅。宜急下之。{\gaoben}

下利。脉遲而滑者。{\khaaitp 内}実也。利未欲止。当下之。宜{\khaaitp 大}承气湯。

問曰。人病有宿食。何以別之。\\
師曰。寸口脉浮大。按之反濇。尺中亦微而濇。故知有宿食。当下之。宜{\khaaitp 大}承气湯。

下利。不欲食者。有宿食也。当下之。宜{\khaaitp 大}承气湯。

下利{\khaaitp 已}差。至其时復发者。此为病不{\sungtpii 𥁞}。当復下之。宜{\khaaitp 大}承气湯。

病腹中滿痛者。为実。当下之。宜大承气湯。

\hangindent 1em
\hangafter=0
病腹中滿痛者。为実。当下之。宜大柴胡湯。

脉雙弦而遲。心下堅。脉大而緊者。陽中有陰也。可下之。宜{\khaaitp 大}承气湯。

\section{可温}

大法。冬宜服温熱藥及灸。

\hangindent 1em
\hangafter=0
大法。冬宜熱藥。{\gaoben}

下利。脉遲緊。为痛未欲止。当温之。得冷者。滿而便腸垢。0

\hangindent 1em
\hangafter=0
下利。脉遲緊。为痛未止。{\gaoben}

下利。脉浮大者。此为虗。以强下之故也。当温之。与水必噦。{\khaaitp 宜当歸四逆湯。}0

\hangindent 1em
\hangafter=0
下利。脉浮大者。此皆为虗。宜温之。{\gaoben}

下利。欲食者。当温之。0

\section{可火}

下利。穀道中痛。当温之。宜灸枳実。或熬鹽等熨之。0

\hangindent 1em
\hangafter=0
凡下利後。下部中痛。当温之。宜炒枳実。若熬鹽等熨之。{\gaoben}

\section{不可刺}

大怒勿刺。{\khaaitp 已刺勿怒。}新内勿刺。{\khaaitp 已刺勿内。}大勞勿刺。{\khaaitp 已刺勿勞。}大醉勿刺。{\khaaitp 已刺勿醉。}大飽勿刺。{\khaaitp 已刺勿飽。大飢勿刺。已刺勿飢。}大渴勿刺。{\khaaitp 已刺勿渴。}大驚勿刺。

勿刺熇熇之熱。勿刺漉漉之汗。勿刺渾渾之脉。

身熱甚。陰陽皆爭者。勿刺也。其可刺者。急取之。不汗則泄。所谓勿刺者。有死徵也。

勿刺病与脉相逆者。上工刺未生。其次刺未盛。其次刺已衰。工逆此者。是谓伐形。0

\section{可刺}

婦人傷寒。懷娠。腹滿。不得小便。從腰以下重。如有水气狀。懷娠七月。太陰当養不養。此心气実。当刺。瀉勞宮及関元。小便利則愈。0

傷寒。喉痹。刺手少陰。少陰在腕当小指後動脉是也。針入三分補之。0

\section{不可水}

下利。其脉浮大。此为虗。以强下之故也。設脉浮革。因尔腸鳴。当温之。与水即噦。0

太陽病。小便利者。为水多。心下必悸。0

\section{可水}

嘔吐。而病在膈上。急思水者。与五苓散飲之。即可飲水也。0

\hangindent 1em
\hangafter=0
嘔吐。而病在膈上。後必思水者。与五苓散飲之。水亦得也。(脉經。玉函)0

\hangindent 1em
\hangafter=0
若嘔吐。熱在膈上。思水者。与五苓散。即可飲水也。{\gaoben}0

\chapter{辨脉法}

問曰。脉有陰陽。何谓也。\\
答曰。凡脉大。浮。數。動。滑。此名陽也。脉沈。濇。弱。弦。微。此名陰也。凡陰病見陽脉者生。陽病見陰脉者死。

問曰。脉有陽結陰結者。何以別之。\\
答曰。其脉{\khaaitp 自}浮而數。能食。不大便。名曰陽結。期十七日当劇。其脉{\khaaitp 自}沈而遲。不能食。身体重。大便反堅。名曰陰結。期十四日当劇。
	\footnote{宋本「不大便」下有「此为実」三字。}

問曰。病有洒淅惡寒。而復发熱者。何。\\
答曰。陰脉不足。陽往從之。陽脉不足。陰往乘之。\\
問曰。何谓陽不足。\\
答曰。假令寸口脉微。为陽不足。陰气上入陽中。則洒淅惡寒。\\
問曰。何谓陰不足。\\
答曰。尺脉弱为陰不足。陽气下陷入陰中。則发熱。\\

陽脉浮。陰脉弱。則血虗。血虗則筋急也。
	\footnote{「筋急」敦煌甲本作「傷筋」,脉經作「筋傷」。}

\hangindent 1em
\hangafter=0
脉陽浮陰濡而弱。弱則血虗。血虗則傷筋。(敦煌甲)

其脉沈者。榮气微也。

其脉浮。而汗出如流珠者。衛气衰也。

\hangindent 1em
\hangafter=0
其脉浮。則汗出如流珠。衛气衰。(敦煌甲)

榮气微者。加燒針則血留不行。更发熱而躁煩也。

脉靄靄如车盖者。名曰陽結。

脉累累如順长竿者。名曰陰結。

脉聶聶如吹榆莢者。名曰散。
	\footnote{「散」敦煌甲本作「數」。}

脉潎潎如羹上肥者。陽气微。
	\footnote{「微」玉函作「脱」。}

脉縈縈如蜘蛛絲者。陽气衰。

脉綿綿如{\khaaitp 瀉}漆之絕者。亡其血。
	\footnote{「瀉」字敦煌甲本无。}

脉來緩。时一止復來者。名曰結。脉來數。时一止復來者。名曰促。脉陽盛則促。陰盛則結。此皆病脉。
	\footnote{「陰盛則結」敦煌甲本「結」作「緩」。}

陰陽相摶。名曰動。陽動則汗出。陰動則发熱。形冷惡寒者。三焦傷也。
	\footnote{「三焦傷也」敦煌甲本作「此为進」。}

數脉見于関上。{\khaaitp 上下}无頭尾。如豆大。厥厥動搖者。名曰動。
	\footnote{敦煌甲本无「上下」二字,「如豆大」作「大如大豆」。}

陽脉浮大而濡。陰脉浮大而濡。陰脉与陽脉同等者。名曰緩。

脉浮而緊者。名曰弦。脉緊者。如轉索无常。弦者。状如弓弦。按之不移。

脉弦而大。弦則为減。大則为芤。減則为寒。芤則为虗。寒虗相摶。此名为革。婦人則半產漏下。男子則亡血失精。

問曰。病有戰而汗出。因得解者。何。\\
答曰。脉浮而緊。按之反芤。此为本虗。故当戰而汗出。其人本虗。是以发戰。以脉浮。故当汗出而解。若脉浮而數。按之不芤。此本不虗。若欲自解。但汗出耳。不发戰也。

問曰。病有不戰而汗出解者。何。\\
答曰。脉大而浮數。故不戰汗出而解。

問曰。病有不戰不汗出而解者。何。
答曰。其脉自微。此以曾发汗。或吐。或下。或亡血。内无津液。陰陽自和。必自愈。故不戰不汗出而解。

問曰。傷寒三日。脉浮數而微。病人身涼和者。何。\\
答曰。此为欲解。解以夜半。脉浮而解者。濈然汗出。脉數而解者。必能食。脉微而解者。必大汗出。

問曰。脉病欲知愈未愈者。何以別之。\\
答曰。寸口。関上。尺中三処。大小浮沈遲數同等。雖有寒熱不解者。此脉陰陽为和平。雖劇当愈。
	\footnote{「數」敦煌甲本作「疾」。}

師曰。立夏得洪大脉。是其本位。其人病身体苦疼重者。須发其汗。若明日身不疼不重者。不須发汗。若汗濈濈自出者。明日便解矣。何以言之。立夏脉洪大。是其时脉。故使然也。四时仿此。

問曰。凡病欲知何时得。何时愈。\\
答曰。假令夜半得病者。明日日中愈。日中得病者。夜半愈。何以言之。日中得病。夜半愈者。以陽得陰則解也。夜半得病。明日日中愈者。以陰得陽則解也。

寸口脉浮为在表。沈为在裏。數为在腑。遲为在臓。假令脉遲。此为在臓。

趺陽脉浮而濇。少陰脉如經者。其病在脾。法当下利。何以知之。若脉浮大者。气実血虗也。今趺陽脉浮而濇。故知脾气不足。胃气虗也。以少陰脉弦而浮纔見。此为調脉。故称如經。若反滑而數者。故知当尿膿也。
	\footnote{「以少陰脉弦而浮纔見」敦煌甲本作「少陰脉弦沈纔見为調」,聖惠方作「少陰脉弦而沈此为調脉」。}

寸口脉浮而緊。浮則为風。緊則为寒。風則傷衛。寒則傷榮。榮衛俱病。骨節煩疼。当发其汗。

趺陽脉遲而緩。胃气如經也。趺陽脉浮而數。浮則傷胃。數則動脾。此非本病。醫特下之所为也。榮衛内陷。其數先微。脉反但浮。其人必大便堅。气噫而除。何以言之。本數脉動脾。其數先微。故知脾气不治。大便堅。气噫而除。今脉反浮。其數改微。邪气獨留。心中則飢。邪熱不殺穀。潮熱发渴。數脉当遲緩。脉因前後度數如法。病者則飢。數脉不时。則生惡瘡。

師曰。病人脉微而濇者。此为醫所病也。大发其汗。又數大下之。其人亡血。病当惡寒。後乃发熱。无休止时。夏月盛熱。欲著複衣。冬月盛寒。欲裸其身。所以然者。陽微則惡寒。陰弱則发熱。此醫发其汗。令陽气微。又大下之。令陰气弱。五月之时。陽气在表。胃中虗冷。以陽气内微。不能勝冷。故欲著複衣。十一月之时。陽气在裏。胃中煩熱。以陰气内弱。不能勝熱。故欲裸其身。又陰脉遲濇。故知血亡血。
	\footnote{「夏月」敦煌甲本、圣惠方作「五月」。}

脉浮而大。心下反堅。有熱。屬臓者。攻之。不令发汗。屬腑者。不令溲數。溲數則便堅。汗多則熱愈。汗少則便難。脉遲尚未可攻。

趺陽脉微濇。少陰反堅。微則下逆。濇則躁煩。少陰堅者。便則为難。汗出在頭。穀气为下。便難者。令微溏。不令汗出。甚者遂不得便。煩逆。鼻鳴。上竭下虗。不得復通。
	\footnote{此條僅見於玉函、敦煌甲,「令微溏」敦煌甲本作「愈微溏」。}

脉浮而洪。身汗如油。喘而不休。水漿不下。形体不仁。乍靜乍亂。此为命絕。\\
問曰。上脉狀如此。未知何臓先受其災。\\
答曰。若汗出髮潤。喘而不休者。肺先絕也。身如煙熏。直視搖頭者。心先絕也。唇吻反青。四肢漐習者。肝先絕也。環口黧黑。柔汗发黄者。脾先絕也。溲便遺矢。狂言。目反直視者。腎先絕也。\\
又問。未知何臓陰陽先絕。\\
答曰。若陽气先絕。陰气後竭者。其人死。身色必青。若陰气先絕。陽气後竭者。其人死。身色必赤。腋下温。心下熱。
	\footnote{「身汗如油」玉函作「軀汗如油」,聖惠方作「身汗如沾」,敦煌甲本作「軀反如沾」。\\「喘而不休」敦煌甲本作「濡而不休」。\\「乍靜乍亂」敦煌甲本作「乍理乍亂」。}

寸口脉浮大。醫反下之。此为大逆。浮則无血。大則为寒。寒气相摶。則为腸鳴。醫乃不知。而反飲冷水。令汗大出。水得寒气。冷必相摶。其人即饐。

趺陽脉浮。浮則为虗。浮虗相摶。故令气饐。言胃气虗竭也。脉滑則噦。此为醫咎。責虗取実。守空迫血。脉浮。鼻口燥者。必衄。
	\footnote{「鼻口燥」宋本作「鼻中燥」。}

諸脉浮數。当发熱而洒淅惡寒。若有痛処。食飲如常者。畜積有膿。

脉浮而遲。面熱赤而戰愓者。六七日。当汗出而解。反发熱者。差遲。遲为无陽。不能作汗。其身必癢。
	\footnote{「面熱赤而戰愓者」敦煌甲本作「面熱而赤戴陽」。}

脉虗者。不可吐下发汗。其面反有熱色者。为欲解。不能汗出。其身必癢。
	\footnote{此條僅玉函、敦煌甲本有。}

寸口脉陰陽俱緊者。法当清邪中上。濁邪中下。清邪中上。名曰潔也。濁邪中下。名曰渾也。陰中於邪。必内慄也。表气微虗。裏气不守。故使邪中於陰。陽中於邪。必发熱。頭痛。項强。頸攣。腰痛。脛痠。所谓陽中霧露之气。故曰清邪中上。濁邪中下。陰气为慄。足膝逆冷。便尿妄出。表气微虗。裏气微急。三焦相溷。内外不通。上焦怫鬱。臓气相熏。口爛食齦。中焦不治。胃气上衝。脾气不轉。胃中为濁。榮衛不通。血凝不流。若衛气前通者。小便赤黄。与熱相摶。因熱作使。遊於經絡。出入臓腑。熱气所過。則为癰膿。若陰气前通者。陽气厥微。陰无所使。客气内入。嚏而出之。聲嗢咽塞。寒厥相追。为熱所擁。血凝自下。状如豚肝。陰陽俱厥。脾气孤弱。五液注下。下焦不闔。清便下重。令便數難。脐築湫痛。命將難全。
	\footnote{「臓气相熏」敦煌甲本、聖惠方作「臓气相動」。\\「令便數難」敦煌甲本、聖惠方作「大便數難」。}

脉陰陽俱緊。口中气出。唇口乾燥。踡卧足冷。鼻中涕出。舌上胎滑。勿妄治也。到七日以上。其人微发熱。手足温者。此为欲解。或到八日已上。反大发熱者。此为難治。設惡寒者。必欲嘔也。腹内痛者。必欲利也。
	\footnote{「七日以上」宋本、玉函作「七日以來」。}

脉陰陽俱緊。至於吐利。其脉獨不解。緊去人安。此为欲解。若脉遲。至六七日不欲食。此为晚发。水停故也。为未解。食自可者为欲解。病六七日。手足三部脉皆至。大煩。口噤不能言。其人躁擾者。必欲解也。若脉和。其人大煩。目重瞼内際黄者。此欲解也。
	\footnote{「瞼」各本均作「臉」,錢超塵認为應作「瞼」,據改。}

脉浮而數。浮为風。數为虗。風为熱。虗为寒。風虗相摶。則洒淅惡寒。
	\footnote{玉函「洒淅惡寒」後有「而发熱也」。}

趺陽脉浮而微。浮即为虗。微即汗出。
	\footnote{此條僅見於敦煌甲、玉函。}

脉浮而滑。浮{\khaaitp 則}为陽。滑{\khaaitp 則}为実。陽実相摶。其脉數疾。衛气失度。浮滑之脉數疾。发熱汗出者。此为不治。

傷寒。欬逆。上气。其脉散者死。谓其形損故也。

\hangindent 1em
\hangafter=0
脉散。其人形損。傷寒而欬。上气者死。

\chapter{平脉法}

問曰。脉有三部。陰陽相乘。榮衛血气。在人体躬。呼吸出入。上下於中。因息遊布。津液流通。隨时動作。效象形容。春弦秋浮。冬沈夏洪。察色觀脉。大小不同。一时之間。變无經常。尺寸参差。或短或长。上下乖錯。或存或亡。病輒改易。進退低昂。心迷意惑。動失纪綱。願为具陳。令得分明。\\
師曰。子之所問。道之根源。脉有三部。尺寸及関。榮衛流行。不失衡銓。腎沈心洪。肺浮肝弦。此自經常。不失銖分。出入升降。漏刻周旋。水下二刻。一周循環。当復寸口。虗実見焉。變化相乘。陰陽相干。風則浮虗。寒則牢堅。沈潛水畜。支飲急弦。動則为痛。數則熱煩。設有不應。知變所緣。三部不同。病各異端。太過可怪。不及亦然。邪不空見。中必有姦。審察表裏。三焦別焉。知其所舍。消息診看。料度腑臓。獨見若神。为子條記。傳与賢人。

師曰。呼吸者。脉之頭也。初持脉。來疾去遲。此出疾入遲。名曰内虗外実也。初持脉。來遲去疾。此出遲入疾。名曰内実外虗也。

問曰。上工望而知之。中工問而知之。下工脉而知之。願聞其説。\\
師曰。病家人请云。病人若发熱。身体疼。病人自卧。師到。診其脉。沈而遲者。知其差也。何以知之。表有病者。脉当浮大。今脉反沈遲。故知愈也。

假令病人云。腹内卒痛。病人自坐。師到。脉之。浮而大者。知其差也。何以知之。若裏有病者。脉当沈而細。今脉浮大。故知愈也。

師曰。病家人來请云。病人发熱。煩極。明日師到。病人向壁卧。此熱已去也。設令脉不和。処言已愈。設令向壁卧。聞師到。不驚起而眄視。若三言三止。脉之。咽唾者。此詐病也。設令脉自和。処言汝病大重。当須服吐下藥。鍼灸數十百処乃愈。

師持脉。病人欠者。无病也。脉之呻者。病也。言遲者。風也。搖頭言者。裏痛也。行遲者。表强也。坐而伏者。短气也。坐而下一腳者。腰痛也。裏実護腹如懷卵物者。心痛也。

師曰。伏气之病。以意{\sungtpii 𠊱}之。今月之内。欲有伏气。假令舊有伏气。当須脉之。若脉微弱者。当喉中痛似傷。非喉痹也。病人云。実咽中痛。雖尔。今復欲下利。

問曰。人病恐怖者。其脉何状。\\
師曰。脉形如循絲累累然。其面白脱色也。

問曰。人不飲。其脉何類。\\
師曰。其脉自濇。唇口乾燥也。

問曰。人愧者。其脉何類。\\
師曰。脉浮而面色乍白乍赤。

問曰。經説。脉有三菽。六菽重者。何谓也。\\
師曰。脉。人以指按之。如三菽之重者。肺气也。如六菽之重者。心气也。如九菽之重者。脾气也。如十二菽之重者。肝气也。按之至骨者。腎气也。假令下利。寸口。関上。尺中悉不見脉。然尺中时一小見。脉再舉頭者。腎气也。若見損脉來至。为難治。

問曰。脉有相乘。有縱有横。有逆有順。何也。
師曰。水行乘火。金行乘木。名曰縱。火行乘水。木行乘金。名曰横。水行乘金。火行乘木。名曰逆。金行乘水。木行乘火。名曰順也。

問曰。脉有殘賊。何谓也。\\
師曰。脉有弦。緊。浮。滑。沈。濇。此六者名曰殘賊。能为諸脉作病也。

問曰。脉有災怪。何谓也。\\
師曰。假令人病。脉得太陽。与形証相應。因为作湯。比還送湯如食頃。病人乃大吐。若下利。腹中痛。\\
師曰。我前來不見此証。今乃變異。是名災怪。\\
又問曰。何緣作此吐利。\\
答曰。或有舊时服藥。今乃发作。故名災怪耳。

問曰。東方肝脉。其形何似。\\
師曰。肝者。木也。名厥陰。其脉微弦濡弱而长。是肝脉也。肝病自得濡弱者。愈也。假令得純弦脉者。死。何以知之。以其脉如弦直。是肝臓傷。故知死也。

問曰。南方心脉。其形何似。\\
師曰。心者。火也。名少陰。其脉洪大而长。是心脉也。心病自得洪大者。愈也。假令脉來微去大。故名反。病在裏也。脉來頭小本大者。故名覆。病在表也。上微頭小者。則汗出。下微本大者。則为関格不通。不得尿。頭无汗者可治。有汗者死。

問曰。西方肺脉。其形何似。\\
師曰。肺者。金也。名大陰。其脉毛浮也。肺病自得此脉。若得緩遲者。皆愈。若得數者。則劇。何以知之。數者南方火。火剋西方金。法当癰腫。为難治也。

問曰。二月得毛浮脉。何以処言至秋当死。\\
師曰。二月之时。脉当濡弱。反得毛浮者。故知至秋死。二月肝用事。肝脉屬木。應濡弱。反得毛浮者。是肺脉也。肺屬金。金來剋木。故知至秋死。他皆仿此。

師曰。脉肥人責浮。瘦人責沈。肥人当沈。今反浮。瘦人当浮。今反沈。故責之。

師曰。寸脉下不至関。为陽絕。尺脉上不至関。为陰絕。此皆不治。決死也。若計其餘命死生之期。期以月節剋之也。

師曰。脉病人不病。名曰行尸。以无王气。卒眩仆不識人者。短命則死。人病脉不病。名曰内虗。以无穀神。雖困无苦。

問曰。翕奄沈。名曰滑。何谓也。沈为純陰。翕为正陽。陰陽和合。故令脉滑。関尺自平。陽明脉微沈。食飲自可。少陰脉微滑。滑者緊之浮名也。此为陰実。其人必股内汗出。陰下濕也。

問曰。曾为人所難。緊脉從何而來。\\
師曰。假令亡汗。若吐。以肺裏寒。故令脉緊也。假令欬者。坐飲冷水。故令脉緊也。假令下利。以胃中虗冷。故令脉緊也。

寸口衛气盛。名曰高。榮气盛。名曰章。高章相摶。名曰綱。衛气弱。名曰惵。榮气弱。名曰卑。惵卑相摶。名曰損。衛气和。名曰緩。榮气和。名曰遲。遲緩相摶。名曰沈。

寸口脉緩而遲。緩則陽气长。其色鲜。其顏光。其聲商。毛髮长。遲則陰气盛。骨髓生。血滿。肌肉緊薄鲜堅。陰陽相抱。榮衛俱行。剛柔相摶。名曰强也。

趺陽脉滑而緊。滑者胃气実。緊者脾气强。持実擊强。痛還自傷。以手把刃。坐作瘡也。

寸口脉浮而大。浮为虗。大为実。在尺为関。在寸为格。関則不得小便。格則吐{\sungtpii 𠱘}。

趺陽脉伏而濇。伏則吐{\sungtpii 𠱘}。水穀不化。濇則食不得入。名曰関格。

脉浮而大。浮为風虗。大为气强。風气相摶。必成癮疹。身体为癢。癢者名泄風。久久为痂癩。

寸口脉弱而遲。弱者衛气微。遲者榮中寒。榮为血。血寒則发熱。衛为气。气微者心内飢。飢而虗滿不能食也。

趺陽脉大而緊者。当即下利。为難治。

寸口脉弱而緩。弱者陽气不足。緩者胃气有餘。噫而吞酸。食卒不下。气填於膈上也。

趺陽脉緊而浮。浮为气。緊为寒。浮为腹滿。緊为絞痛。浮緊相摶。腸鳴而轉。轉即气動。膈气乃下。少陰脉不出。其陰腫大而虗也。

寸口脉微而濇。微者衛气不行。濇者榮气不足。榮衛不能相將。三焦无所仰。身体痹不仁。榮气不足。則煩疼。口難言。衛气虗。則惡寒數欠。三焦不歸其部。上焦不歸者。噫而酢吞。中焦不歸者。不能消穀引食。下焦不歸者。則遺溲。

趺陽脉沈而數。沈为実。數消穀。緊者病難治。

寸口脉微而濇。微者衛气衰。濇者榮气不足。衛气衰。面色黄。榮气不足。面色青。榮为根。衛为枼。榮衛俱微。則根枼枯槁。而寒慄欬逆。唾腥吐涎沫也。

趺陽脉浮而芤。浮者衛气衰。芤者榮气傷。其身体瘦。肌肉甲錯。浮芤相摶。宗气衰微。四屬斷絕。

寸口脉微而緩。微者衛气疏。疏則其膚空。緩者胃气実。実則穀消而水化也。穀入於胃。脉道乃行。而入於經。其血乃成。榮盛則其膚必疏。三焦絕經。名曰血崩。

趺陽脉微而緊。緊为寒。微則为虗。微緊相摶。則为短气。

少陰脉弱而濇。弱者微煩。濇者厥逆。

趺陽脉不出。脾不上下。身冷膚堅。

少陰脉不至。腎气微。少精血。奔气促迫。上入胸膈。宗气反聚。血結心下。陽气退下。熱歸陰股。与陰相動。令身不仁。此为尸厥。当刺期門。巨闕。

寸口脉微。尺脉緊。其人虗損多汗。知陰常在。絕不見陽也。

寸口諸微亡陽。諸濡亡血。諸弱发熱。諸緊为寒。諸乘寒者。則为厥。鬱冒不仁。以胃无穀气。脾濇不通。口急不能言。戰而慄也。

問曰。濡弱何以反適十一頭。\\
師曰。五臓六腑相乘。故令十一。

問曰。何以知乘腑。何以知乘臓。\\
師曰。諸陽浮數为乘腑。諸陰遲濇为乘臓也。

\chapter{傷寒例}

陰陽大論云。春气温和。夏气暑熱。秋气清涼。冬气冷冽。此則四时正气之序也。冬时嚴寒。万類深藏。君子固密。則不傷於寒。觸冒之者。乃名傷寒耳。其傷於四时之气。皆能为病。以傷寒为毒者。以其最成殺厉之气也。

中而即病者。名曰傷寒。不即病者。寒毒藏於肌膚。至春變为温病。至夏變为暑病。暑病者。熱極重於温也。是以辛苦之人。春夏多温熱病。皆由冬时觸寒所致。非时行之气也。

凡时行者。春时應暖而復大寒。夏时應大熱而反大涼。秋时應涼而反大熱。冬时應寒而反大温。此非其时而有其气。是以一歲之中。长幼之病多相似者。此則时行之气也。

夫欲{\sungtpii 𠊱}知四时正气为病。及时行疫气之法。皆当按斗曆占之。九月霜降節後。宜漸寒。向冬大寒。至正月雨水節後。宜解也。所以谓之雨水者。以冰雪解而为雨水故也。至驚蟄二月節後。气漸和暖。向夏大熱。至秋便涼。

從霜降以後。至春分以前。凡有觸冒霜露。体中寒即病者。谓之傷寒也。九月十月。寒气尚微。为病則輕。十一月十二月。寒冽已嚴。为病則重。正月二月。寒漸將解。为病亦輕。此以冬时不調。適有傷寒之人。即为病也。其冬有非節之暖者。名曰冬温。冬温之毒。与傷寒大異。冬温復有先後。更相重沓。亦有輕重。为治不同。証如後章。

從立春節後。其中无暴大寒。又不冰雪。而有人壯熱为病者。此屬春时陽气。发於冬时伏寒。變为温病。

從春分以後至秋分節前。天有暴寒者。皆为时行寒疫也。三月四月。或有暴寒。其时陽气尚弱。为寒所折。病熱猶輕。五月六月。陽气已盛。为寒所折。病熱則重。七月八月。陽气已衰。为寒所折。病熱亦微。其病与温及暑病相似。但治有殊耳。

十五日得一气。於四时之中。一时有六气。四六名为二十四气也。然气{\sungtpii 𠊱}亦有應至而不至。或有未應至而至者。或有至而太過者。皆成病气也。但天地動靜。陰陽鼓擊者。各正一气耳。是以彼春之暖。为夏之暑。彼秋之忿。为冬之怒。是故冬至之後。一陽爻升。一陰爻降也。夏至之後。一陽气下。一陰气上也。斯則冬夏二至。陰陽合也。春秋二分。陰陽離也。

陰陽交易。人變病焉。此君子春夏養陽。秋冬養陰。順天地之剛柔也。小人觸冒。必嬰暴疹。須知毒烈之气留在何經。而发何病。詳而取之。是以春傷於風。夏必飱泄。夏傷於暑。秋必病瘧。秋傷於濕。冬必欬嗽。冬傷於寒。春必病温。此必然之道。可不審明之。

傷寒之病。逐日淺深。以施方治。今世人傷寒。或始不早治。或治不對病。或日數久淹。困乃告醫。醫人又不依次第而治之。則不中病。皆宜臨时消息制方。无不效也。今搜采仲景舊論。錄其証{\sungtpii 𠊱}。診脉。聲色。對病真方有神驗者。擬防世急也。

又土地温涼。高下不同。物性剛柔。飡居亦異。是黄帝興四方之問。岐伯舉四治之能。以訓後賢。開其未悟者。臨病之工。宜須兩審也。

凡傷於寒則为病熱。熱雖甚不死。若兩感於寒而病者必死。

尺寸俱浮者。太陽受病也。当一二日发。以其脉上連風府。故頭項痛。腰脊强。

尺寸俱长者。陽明受病也。当二三日发。以其脉俠鼻。絡於目。故身熱。目疼。鼻乾。不得卧。

尺寸俱弦者。少陽受病也。当三四日发。以其脉循脇絡於耳。故胸脇痛而耳聋。此三經皆受病。未入於府者。可汗而已。

尺寸俱沈細者。太陰受病也。当四五日发。以其脉布胃中絡於嗌。故腹滿而嗌乾。

尺寸俱沈者。少陰受病也。当五六日发。以其脉貫腎絡於肺。系舌本。故口燥舌乾而渴。

尺寸俱微緩者。厥陰受病也。当六七日发。以其脉循陰器絡於肝。故煩滿而囊縮。此三經皆受病。已入於府。可下而已。

若兩感於寒者。一日太陽受之。即与少陰俱病。則頭痛。口乾。煩滿而渴。二日陽明受之。即与太陰俱病。則腹滿。身熱。不欲食。譫語。三日少陽受之。即与厥陰俱病。則耳聋。囊縮而厥。水漿不入。不知人者。六日死。若三陰三陽。五臓六腑皆受病。則榮衛不行。腑臓不通。則死矣。其不兩感於寒。更不傳經。不加異气者。至七日太陽病衰。頭痛少愈也。八日陽明病衰。身熱少歇也。九日少陽病衰。耳聋微聞也。十日太陰病衰。腹減如故。則思飲食。十一日少陰病衰。渴止舌乾。已而嚏也。十二日厥陰病衰。囊縱。少腹微下。大气皆去。病人精神爽慧也。若過十三日以上不間。尺寸陷者。大危。若更感異气變为他病者。当依舊壞証病而治之。

若脉陰陽俱盛。重感於寒者。變为温瘧。陽脉浮滑。陰脉濡弱者。更遇於風。變为風温。陽脉洪數。陰脉実大者。遇温熱。變为温毒。温毒为病最重也。陽脉濡弱。陰脉弦緊者。更遇温气。變为温疫。以此冬傷於寒。发为温病。脉之變証。方治如説。

凡人有疾。不时即治。隱忍冀差。以成痼疾。小兒女子。益以滋甚。时气不和。便当早言。尋其邪由。及在腠理。以时治之。罕有不愈者。患人忍之。數日乃説。邪气入藏。則難可制。此为家有患。備慮之要。

凡作湯藥。不可避晨夜。覺病須臾。即宜便治。不等早晚。則易愈矣。若或差遲。病即傳變。雖欲除治。必難为力。服藥正如方法。縱意違師。不須治之。

凡傷寒之病。多從風寒得之。始表中風寒。入裏則不消矣。未有温覆而当不消散者。若病不察証。擬欲攻之。猶当先解表。乃可下之。若表已解而内不消。大滿。大実。腹堅者。必内有燥屎。自可徐徐下之。雖經四五日。不能为害也。若病不宜下而强攻之。内虗熱入。{\khaaitp 則为}挾熱遂利。煩燥諸變。不可勝數。輕者困篤。重者必死。
	\footnote{宋本「乃可下之」後有「若表已解而内不消非大滿猶生寒熱則病不除」。}

世上之士。但務彼翕習之榮。而莫見此傾危之敗。惟明者居然能護其本。近取諸身。夫何遠之有焉。

凡发汗。温服湯藥。其方雖言日三服。若病劇不解。当促其間。可半日中{\sungtpii 𥁞}三服。若与病相阻。即便有所覺。重病者。一日一夜当晬时觀之。如服一剂。病証猶在。故当復作本湯服之。至有不肯汗出。服三剂乃解。若汗不出者。死病也。

凡得时气病。至五六日。而渴欲飲水。飲不能多。不当与也。何者?以腹中熱尚少。不能消之。便更与人作病也。至七八日。大渴欲飲水者。猶当依証与之。与之常令不足。勿極意也。言能飲一斗。与五升。若飲而腹滿。小便不利。若喘若噦。不可与之。忽然大汗出。是为自愈也。

凡得病。反能飲水。此为欲愈之病。其不曉病者。但聞病飲水自愈。小渴者乃强与飲之。因成其禍。不可復數。

凡得病。厥脉動數。服湯藥更遲。脉浮大減小。初躁後靜。此皆愈証也。

凡治温病。可刺五十九穴。又身之穴三百六十有五。其三十九穴灸之有害。七十九穴刺之为災。并中髓也。

凡脉四損。三日死。平人四息。病人脉一至。名曰四損。脉五損。一日死。平人五息。病人脉一至。名曰五損。脉六損。一时死。平人六息。病人脉一至。名曰六損。

脉盛身寒。得之傷寒。脉虗身熱。得之傷暑。脉陰陽俱盛。大汗出不解者死。脉陰陽俱虗。熱不止者死。脉至乍疏乍數者死。脉至如轉索者。其日死。譫言妄語。身微熱。脉浮大。手足温者生。逆冷。脉沈細者。不過一日死矣。此以前是傷寒熱病証{\sungtpii 𠊱}也。

\part{雜病論}

\chapter{臓腑經絡先後}

問曰。病人有气色見於面部。願聞其説。\\
師曰。鼻頭色青。腹中痛。苦冷者死。鼻頭色微黑者。有水气。色黄者。胸上有寒。色白者。亡血也。設微赤。非时者死。其目正圓者。痙。不治。又色青为痛。色黑为勞。色赤为風。色黄者便難。色鲜明者有留飲。
	\footnote{
		本段前鄧本有「問曰上工治未病」一段,吳本在臓腑經絡先後篇之前,未入正文,且此段文字都是无用不実之词,故移至衍文。
	}

師曰。病人語聲寂寂然。喜驚呼者。骨節間病。語聲喑喑然不徹者。心膈間病。語聲啾啾然細而长者。頭中病。

師曰。息搖肩者。心中堅。息引胸中上气者。欬。息张口短气者。肺痿唾沫。

師曰。吸而微數。其病在中焦。実也。当下之即愈。虗者不治。在上焦者。其吸促。在下焦者。其吸遠。此皆難治。呼吸動搖振振者。不治。

師曰。寸口脉動者。因其王时而動。假令肝王色青。四时各隨其色。肝色青而反色白。非其时色脉。皆当病。

問曰。有未至而至。有至而不至。有至而不去。有至而太過。何谓也。\\
師曰。冬至之後。甲子夜半少陽起。少陽之时陽始生。天得温和。以未得甲子。天因温和。此为未至而至也。以得甲子而天未温和。此为至而不至也。以得甲子而天大寒不解。此为至而不去也。以得甲子而天温如盛夏五六月时。此为至而太過也。

師曰。病人脉浮者在前。其病在表。浮者在後。其病在裏。腰痛背强不能行。必短气而極也。

問曰。經云。厥陽獨行。何谓也。\\
師曰。此为有陽无陰。故称厥陽。

問曰。寸脉沈大而滑。沈則为実。滑則为气。実气相摶。血气入臓即死。入腑即愈。此为卒厥。何谓也。\\
師曰。唇口青。身冷。为入臓即死。如身和。汗自出。为入腑即愈。

問曰。脉脱。入臓即死。入腑即愈。何谓也。\\
師曰。非为一病。百病皆然。譬如浸淫瘡。從口起流向四肢者。可治。從四肢流來入口者。不可治。{\khaaitp 諸}病在外者可治。入裏者即死。

問曰。陽病十八。何谓也。\\
師曰。頭痛。項。腰。脊。臂。腳掣痛。\\
問曰。陰病十八。何谓也。\\
師曰。欬。上气。喘。噦。咽。腸鳴。胀滿。心痛。拘急。五臓病各有十八。合为九十病。人又有六微。微有十八病。合为一百八病。五勞。七傷。六極。婦人三十六病。不在其中。清邪居上。濁邪居下。大邪中表。小邪中裏。穀飪之邪。從口入者。宿食也。五邪中人。各有法度。風中於前。寒中於暮。濕傷於下。霧傷於上。風令脉浮。寒令脉急。霧傷皮腠。濕流関節。食傷脾胃。極寒傷經。極熱傷絡。

問曰。病有急当救裏救表者。何谓也。\\
師曰。病。醫下之。續得下利。清穀不止。身体疼痛者。急当救裏。後身体疼痛。清便自調者。急当救表也。

夫病痼疾。加以卒病。当先治其卒病。後乃治其痼疾也。

師曰。五臓病各有所得者愈。五臓病各有所惡。各隨其所不喜者为病。病者素不應食。而反暴思之。必发熱也。

夫諸病在臓。欲攻之。当隨其所得而攻之。如渴者。与豬苓湯。餘皆仿此。

\chapter{痙濕暍}

太陽病。发熱。无汗。反惡寒者。为剛痙。

太陽病。发熱。汗出。不惡寒者。为柔痙。

太陽病。发熱。脉沈細者。为痙。
	\footnote{「为痙」金匱要略作「名曰痙为難治」。}

太陽病。发汗太多。因致痙。
	\footnote{脉經、玉函、千金翼「发汗太多」作「发其汗」。}

病者身熱足寒。頸項强急。惡寒。时頭熱。面赤。目赤。獨頭動搖。卒口噤。背反张者。痙病也。
	\footnote{「目赤」鄧本同,脉經、玉函、千金翼、吳本均作「目脉赤」。}

痙病。发其汗者。寒濕相摶。其表益虗。即惡寒甚。发其汗已。其脉如蛇。暴腹胀大者。为欲解。脉如故。反伏弦者。痙。
	\footnote{「相摶」各本均作「相得」,为編者所改。}

夫風病。下之則痙。復发汗。必拘急。

夫痙脉。按之緊如弦。直上下行。

痙病有灸瘡。難治。

瘡家。雖身疼痛。不可发汗。汗出則痙。85

太陽病。其証備。身体强。几几然。脉反沈遲。此为痙。栝蔞桂枝湯主之。

太陽病。无汗。而小便反少。气上衝胸。口噤不得語。欲作剛痙。葛根湯主之。

{\khaaitp 剛}痙为病。胸滿。口噤。卧不著席。腳攣急。其人必齘齒。可与大承气湯。

太陽病。関節疼煩。脉沈緩者。此名濕痹。濕痹之{\sungtpii 𠊱}。其人小便不利。大便反快。但当利其小便。
	\footnote{
		「関節疼煩」宋本、金匮作「関節疼痛而煩」,脉經作「関節疼痛」。「脉沈緩者」宋本、金匱作「脉沈而細者」。「此名濕痹」脉經、玉函、千金翼作「为中濕」。
	}

濕家之为病。一身{\sungtpii 𥁞}疼。发熱。身色如熏黄。

濕家。其人但頭汗出。背强。欲得被覆向火。若下之早則噦。{\khaaitp 或}胸滿。小便{\khaaitp 不}利。舌上如胎。以丹田有熱。胸上有寒。渴欲得飲而不能飲。則口燥{\khaaitp 煩}也。

濕家下之。額上汗出。微喘。小便利者死。下利不止者亦死。

問曰。風濕相摶。一身{\sungtpii 𥁞}疼痛。法当汗出而解。值天陰雨不止。師云此可发汗。汗之病不愈者。何也。\\
答曰。发其汗。汗大出者。但風气去。濕气續在。是故不愈。若治風濕者。发其汗。但微微似欲出汗者。則風濕俱去也。

濕家。病身疼。发熱。面黄而喘。頭痛。鼻塞而煩。其脉大。自能飲食。腹中和。无病。病在頭。中寒濕。故鼻塞。内藥鼻中即愈。

濕家。身煩疼。可与麻黄加术湯。发其汗为宜。慎不可以火攻之。

病者一身{\sungtpii 𥁞}疼。发熱。日晡所劇者。名風濕。此病傷於汗出当風。或久傷取冷所致也。可与麻杏薏甘湯。

\hangindent 1em
\hangafter=0
濕家。始得病时。可与薏苡麻黄湯。(外臺)

風濕。脉浮。身重。汗出。惡風者。防己黄耆湯主之。

傷寒八九日。風濕相摶。身体疼煩。不能自轉側。不嘔。不渴。脉浮虗而濇者。桂枝附子湯主之。若其人大便堅。小便自利者。术附子湯主之。174

風濕相摶。骨節疼煩。掣痛。不得屈伸。近之則痛劇。汗出短气。小便不利。惡風。不欲去衣。或身微腫者。甘草附子湯主之。

太陽中熱者。暍是也。其人汗出。惡寒。身熱而渴。白虎{\khaaitp 加人参}湯主之。

太陽中暍。身熱疼重。而脉微弱。此以夏月傷冷水。水行皮膚中所致也。瓜蒂湯主之。

太陽中暍。发熱。惡寒。身重而疼痛。其脉弦細芤遲。小便已。洒洒然毛聳。手足逆冷。小有勞。身即熱。口開。前板齒燥。若发其汗。則惡寒甚。加温針。則发熱甚。數下之。則淋甚。

\chapter{百合狐惑陰陽毒
	{
		\footnote{
			唐弘宇按:這一章的校訂主要依據《醫心方》和《千金方》。因为我缺乏資料,所以這一章的校訂很粗疏。
		}
	}
}

論曰。百合病者。百脉一宗。悉致其病也。意欲食復不能食。常默默。欲卧{\khaaitp 復}不得眠。欲行{\khaaitp 復}不能行。飲食或有美时。或有不用聞食臭时。如寒无寒。如熱无熱。口苦。小便赤。諸藥不能治。得藥則劇吐利。如有神靈者。\\
身形如和。其脉微數。每尿时頭痛者。六十日乃愈。\\
若尿时頭不痛。淅{\khaaitp 淅}然者。四十日愈。\\
若尿快然。但頭眩者。二十日愈。\\
其証或未病而預見。或病四五日而出。或病二十日或一月微見者。各隨証治之。

百合病。发汗後者。百合知母湯主之。

百合病。下之後者。百合滑石代赭湯主之。

百合病。吐之後者。百合雞子湯主之。

百合病。不經发汗吐下。病形如初者。百合地黄湯主之。

百合病。經月不解。變成渴者。百合洗方主之。渴不差者。栝蔞牡蛎散主之。
	\footnote{
		《千金方》「渴不差者栝蔞牡蛎散主之」为註文。
	}

百合病。變发熱者。百合滑石散主之。

百合病。變腹中滿痛者。但取百合根隨多少。熬令黄色。擣篩为散。飲服方寸匕。日三。滿消痛止。
	\footnote{
		此條《金匱要略》中无,從《千金方》補入。
	}

百合病。見於陰者。以陽法救之。見於陽者。以陰法救之。見陽攻陰。復发其汗。此为逆。見陰攻陽。乃復下之。此亦为逆。

狐惑之为病。狀如傷寒。默默欲眠。目不得閉。卧起不安。蝕於喉为惑。蝕於陰为狐。不欲飲食。惡聞食臭。其面目乍赤。乍黑。乍白。蝕於上部則聲喝。甘草瀉心湯主之。蝕於下部則咽乾。苦参湯洗之。蝕於肛者。雄黄熏之。

病者脉數。无熱。微煩。默默。但欲卧。汗出。初得之三四日。目赤如鳩眼。七八日目四眥黑。若能食者。膿已成也。赤{\khaaitp 小}豆当歸散主之。

陽毒之为病。面赤斑斑如錦文。喉咽痛。唾膿血。五日可治。七日不可治。\\
陰毒之为病。面目青。狀如被打。喉咽痛。死生与陽毒同。升麻鱉甲湯并主之。{\wuben}

陽毒之为病。面赤斑斑如錦文。咽喉痛。唾膿血。五日可治。七日不可治。升麻鱉甲湯主之。\\
陰毒之为病。面目青。身痛如被杖。咽喉痛。五日可治。七日不可治。升麻鱉甲湯去雄黄蜀椒主之。{\dengben}

\chapter{瘧}

師曰。瘧脉自弦。弦數者多熱。弦遲者多寒。弦小緊者下之差。弦遲者可温之。弦緊者可发汗針灸也。浮大者可吐之。弦數者風疾也。以飲食消息止之。
	\footnote{「風疾」鄧本作「風发」。}

問曰。瘧以月一日发。当以十五日愈。設不差。当月{\sungtpii 𥁞}解。如其不差。当云何。\\
師曰。此結为癥瘕。名曰瘧母。急治之。宜鱉甲煎丸。

師曰。陰气孤絕。陽气獨发。則熱而少气。煩滿。手足熱而欲嘔。名曰癉瘧。若但熱不寒者。邪气内藏於心。外舍分肉之間。令人消鑠脱肉。
	\footnote{「煩滿」鄧本作「煩寃」。}

温瘧者。其脉如平。身无寒。但熱。骨節疼煩。时嘔。白虎加桂枝湯主之。

瘧。多寒者。名曰牡瘧。蜀漆散主之。

附方

牡蛎湯。治牡瘧。

瘧病发渴者。与小柴胡去半夏加栝蔞湯。

柴胡桂薑湯。{\scriptsize 此方治寒多微有熱。或但寒不熱。服一剂如神。故錄之。}

\chapter{中風歷節}

夫風之为病。当半身不遂。或但臂不遂。此为痹。脉微而數。中風使然。

寸口脉浮而緊。緊則为寒。浮則为虗。寒虗相摶。邪在皮膚。浮者血虗。絡脉空虗。賊邪不瀉。或左或右。邪气反緩。正气即急。正气引邪。喎僻不遂。邪在於絡。肌膚不仁。邪在於經。即重不勝。邪入於腑。即不識人。邪入於臓。舌即難言。口吐涎。

大風。四肢煩重。心中惡寒不足者。矦氏黑散主之。{\scriptsize 外臺治風癲}

寸口脉遲而緩。遲則为寒。緩則为虗。榮緩則为亡血。衛緩則为中風。邪气中經。則身癢而癮疹。心气不足。邪气入中。則胸滿而短气。
	\footnote{此條吳本无。}

風引湯。除熱{\khaaitp 。主}癱癇。

病如狂狀。妄行。獨語不休。无寒熱。其脉浮。防己地黄湯主之。

\hangindent 1em
\hangafter=0
言語狂錯。眼目茫茫。或見鬼。精神昏亂。防己地黄湯方。{\qianjin}

頭風摩散。

寸口脉沈而弱。沈即主骨。弱即主筋。沈即为腎。弱即为肝。汗出入水中。如水傷心。歷節黄汗出。故曰歷節。

趺陽脉浮而滑。滑則穀气実。浮則汗自出。

少陰脉浮而弱。浮則为風。弱則血不足。風血相摶。即疼痛如掣。

盛人脉濇小。短气。自汗出。歷節疼。不可屈伸。此皆飲酒。汗出当風所致。

諸肢節疼痛。身体魁羸。腳腫如脱。頭眩短气。温温欲吐。桂枝芍藥知母湯主之。
	\footnote{
		陸淵雷:「尪羸是短小瘦弱之意,此非歷節之主証。趙本作魁羸,不誤,狀関節之腫大也。」
	}

味酸則傷筋。筋傷則緩。名曰泄。鹹則傷骨。骨傷則痿。名曰枯。枯泄相摶。名曰斷泄。榮气不通。衛不獨行。榮衛俱微。三焦无所御。四屬斷絕。身体羸瘦。獨足腫大。黄汗出。脛冷。假令发熱。便为歷節也。
	\footnote{此條吳本无。}

病歷節。疼痛。不可屈伸。烏頭湯主之。

烏頭湯。治腳气。疼痛。不可屈伸。
	\footnote{此條吳本无。}

礬石湯。治腳气衝心。

附方

續命湯。治中風痱。身体不能自收。口不能言。冒昧不知痛処。或拘急不得轉側。

三黄湯。治中風。手足拘急。百節疼痛。煩熱心亂。惡寒。經日不欲飲食。

术附子湯。治風虗。頭重眩。苦極。不知食味。暖肌補中。益精气。
	\footnote{
		《外臺祕要風頭眩方》載:《近效》术附子湯,有桂心,无生薑、大棗,并云:「此本仲景《傷寒論》方。」
	}

崔氏八味丸。治腳气上入。少腹不仁。
	\footnote{
		《外臺祕要腳气不隨方》載崔氏方五首,其中第四首云:「若腳气上入少腹,少腹不仁,即服張仲景八味丸。」此方中用山茱萸五兩,澤瀉四兩,桂心三兩,附子二兩,餘与《金匱》同。推測本方当是仲景方,是崔氏將此方用於腳气病,後人便命名此方为崔氏八味丸。\\
		《金匱玉函要略方論輯義》:「《舊唐書經籍志》云:《崔氏纂要方》十卷,崔知悌撰。《新唐書藝文志》作崔行功撰。所謂崔氏其人也,不知者或以为仲景收錄崔氏之方,故詳及之。」
	}

越婢加术湯。治肉極。熱則身体津{\khaaitp 液}脱。腠理開。汗大泄。厉風气。下焦腳弱。
	\footnote{
		「肉」俞本作「内」。\\唐弘宇按:《千金方》有一很长的段落解釋肉極,此條僅由其中不相連的幾句拼湊而成。
	}

\chapter{血痹虗勞}

問曰。血痹病從何得之。\\
師曰。夫尊樂人。骨弱肌膚盛。重因疲勞汗出。卧不时動搖。加被微風。遂得之。{\khaaitp 形如風狀。}但以脉自微濇在寸口。関上小緊。宜針引陽气。令脉和緊去則愈。

血痹。陰陽俱微。寸口関上微。尺中小緊。外証身体不仁。如風狀。黄耆桂枝五物湯主之。

夫男子平人。脉大为勞。極虗亦为勞。

男子面色薄者。主渴及亡血。卒喘悸。脉浮者。裏虗也。

男子脉虗沈弦。无寒熱。短气。裏急。小便不利。面色白。时目瞑。兼衄。少腹滿。此为勞使之然。

勞之为病。其脉浮大。手足煩。春夏劇。秋冬差。陰寒精自出。痠削不能行。

男子脉浮弱而濇。为无子。精清泠。
	\footnote{
		「精清泠」同吳本,鄧本作「精气清冷」。
	}

夫失精家。少腹弦急。陰頭寒。目眩。髮落。脉極虗芤遲。为清穀。亡血。失精。脉得諸芤動微緊。男子失精。女子夢交。桂枝加龙骨牡蛎湯主之。天雄散亦主之。
	\footnote{
		「天雄散亦主之」鄧本作「天雄散」,獨立成一條。
	}

男子平人。脉虗弱細微者。善盜汗也。

人年五六十。其病脉大者。痹俠背行。苦腸鳴。馬刀俠癭者。皆为勞得之。

脉沈小遲。名脱气。其人疾行則喘喝。手足逆寒。腹滿。甚則溏泄。食不消化也。

脉弦而大。弦則为減。大則为芤。減則为寒。芤則为虗。虗寒相摶。此名为革。婦人則半產。漏下。男子則亡血。失精。

虗勞。裏急。悸。衄。腹中痛。夢失精。四肢痠疼。手足煩熱。咽乾口燥。小建中湯主之。

虗勞。裏急。諸不足。黄耆建中湯主之。

虗勞。腰痛。少腹拘急。小便不利者。八味腎气丸主之。

虗勞。諸不足。風气百疾。薯蕷丸主之。

虗勞。虗煩。不得眠。酸棗{\khaaitp 仁}湯主之。

\hangindent 1em
\hangafter=0
虗勞。煩。悸。不得眠。酸棗湯主之。{\qianjin}

五勞。虗極。羸瘦。腹滿。不能飲食。食傷。憂傷。飲傷。房室傷。飢傷。勞傷。經絡榮衛气傷。内有乾血。肌膚甲錯。兩目黯黑。緩中補虗。大黄䗪虫丸主之。

附方

虗勞不足。汗出而悶。脉結。心悸。行動如常。不出百日。危急者。十一日死。炙甘草湯主之。

獺肝散。治冷勞。又主鬼疰。一門相染。

\chapter{肺痿肺癰欬嗽上气}

問曰。熱在上焦者。因欬为肺痿。肺痿之病。何從得之。\\
師曰。或從汗出。或從嘔吐。或從消渴。小便利數。{\khaaitp 或從便難。}又被快藥下利。重亡津液。故得之。
	\footnote{
		「或從便難」吳本无。
	}

問曰。寸口脉數。其人欬。口中反有濁唾涎沫者何。\\
師曰。{\khaaitp 此}为肺痿之病。若口中辟辟燥。欬即胸中隱隱痛。脉反滑數。此为肺癰。欬唾濃血。脉數虗者为肺痿。數実者为肺癰。

問曰。病欬逆。脉之何以知此为肺癰。当有膿血。吐之則死。其脉何類。\\
師曰。寸口脉微而數。微則为風。數則为熱。微則汗出。數則惡寒。風中於衛。呼气不入。熱過於榮。吸而不出。風傷皮毛。熱傷血脉。風舍於肺。其人則欬。口乾。喘滿。咽燥。不渴。时唾濁沫。时时振寒。熱之所過。血为凝滯。畜結癰膿。吐如米粥。始萌可救。膿成則死。

上气。面浮腫。肩息。其脉浮大。不治。又加利尤甚。

上气。喘而躁者。屬肺胀。欲作風水。发汗則愈。
%	\footnote{「喘而躁」吳本作「躁而喘」。}

肺痿。吐涎沫。而不能欬者。其人不渴。必遺尿。小便數。所以然者。以上虗不能制下故也。此为肺中冷。必眩。甘草乾薑湯以温其病。{\wuben}

肺痿。吐涎沫。而不欬者。其人不渴。必遺尿。小便數。所以然者。以上虗不能制下故也。此为肺中冷。必眩。多涎唾。甘草乾薑湯以温之。若服湯已。渴者。屬消渴。{\dengben}
	\footnote{
		「若服湯」至「消渴」九字,吳本作「服湯已小温覆之若渴者屬消渴」十三字,且为方後註釋,未入正文,脉經无。
		}

欬而上气。喉中水雞聲。射干麻黄湯主之。

欬逆。气上衝。唾濁。但坐不得卧。皂莢丸主之。{\wuben}

欬逆上气。时时唾濁。但坐不得卧。皂莢丸主之。{\dengben}

上气。脉浮者。厚朴麻黄湯主之。脉沈者。澤漆湯主之。{\wuben}

欬而脉浮者。厚朴麻黄湯主之。脉沈者。澤漆湯主之。{\dengben}

	%校訂紀錄:由於我之前的疏忽,厚朴麻黄湯未加入類聚方。
	%20210102更新:此方類聚方廣義中无,我依據金匱1的排列次序將其列在澤漆湯前。

火逆上气。咽喉不利。止逆下气者。麥門冬湯主之。
	\footnote{「火逆」諸本均作「大逆」,編者改。}

肺癰。喘不得卧。葶藶大棗瀉肺湯主之。

欬而胸滿。振寒。脉數。咽乾。不渴。时出濁唾腥臭。久久吐膿如米粥者。为肺癰。桔梗湯主之。

欬而上气。此为肺胀。其人喘。目如脱狀。脉浮大者。越婢加半夏湯主之。
	\footnote{「欬而上气」吳本作「欬逆倚息」。}

肺胀。欬而上气。煩躁而喘。脉浮者。心下有水。小青龙加石膏湯主之。
%	\footnote{「躁」鄧本作「燥」。}

附方

肺痿。涎唾多。心中温温液液者。炙甘草湯主之。

肺痿。欬唾涎沫不止。咽燥而渴。生薑甘草湯主之。

肺痿。吐涎沫。桂枝去芍藥加皂莢湯主之。

欬而胸滿。振寒。脉數。咽乾。不渴。时出濁唾腥臭。久久吐膿如米粥者。为肺癰。桔梗白散主之。

葦莖湯。治欬。有微熱。煩滿。胸中甲錯。是为肺癰。

肺癰。胸滿胀。一身面目浮腫。鼻塞。清涕出。不聞香臭酸辛。欬逆上气。喘鳴迫塞。葶藶大棗瀉肺湯主之。

欬而上气。肺胀。其脉浮。心下有水气。脇下痛引缺盆。小青龙加石膏湯主之。

\chapter{奔豚气吐膿驚怖火邪}

師曰。病有奔豚。有吐膿。有驚怖。有火邪。此四部病。皆從驚发得之。

師曰。奔豚病。從少腹起。上衝咽喉。发作欲死。復還止。皆從驚恐得之。

奔豚。气上衝胸。腹痛。往來寒熱。奔豚湯主之。

燒針令其汗。針処被寒。核起而赤者。必发奔豚。气從少腹上衝心者。灸其核上各一壯。与桂枝加桂湯。117

发汗後。其人脐下悸。欲作奔豚。苓桂甘棗湯主之。65

夫嘔家有癰膿。不可治嘔。膿{\sungtpii 𥁞}自愈。376
	\footnote{本條原載於《嘔吐》篇。}

排膿散。
	\footnote{本條原載於《瘡癰》篇。}

排膿湯。
	\footnote{本條原載於《瘡癰》篇。}

寸口脉動而弱。動則为驚。弱則为悸。
	\footnote{本條原載於《驚悸吐衄下血胸滿瘀血》篇。}

火邪者。桂枝去芍藥加蜀漆牡蛎龙骨救逆湯主之。
	\footnote{唐弘宇按:本條原載於《驚悸吐衄下血胸滿瘀血》篇,本條実際内容与篇名不符,今移至此。}

傷寒。脉浮。醫以火迫劫之。亡陽。{\khaaitp 必}驚狂。卧起不安。桂枝去芍藥加蜀漆牡蛎龙骨救逆湯主之。112
	\footnote{第112至119條原載於《太陽病》篇。}

傷寒。其脉不弦緊而弱{\khaaitp 。弱}者必渴。被火必譫語。{\khaaitp 弱者。发熱。脉浮。解之当汗出愈。}113

太陽病。以火熏之。不得汗。其人必躁。到經不解。必清血。114
	\footnote{宋本「必清血」下有「名为火邪」四字。}
	
\hangindent 1em
\hangafter=0
太陽病。以火蒸之。不得汗者。其人必燥結。若不結。必下清血。其脉躁者。必发黄也。{\gaoben}114

脉浮。熱甚。而反灸之。此为実。実以虗治。因火而動。咽燥。必吐血。115

微數之脉。慎不可灸。因火为邪。則为煩逆。追虗逐実。血散脉中。火气雖微。内攻有力。焦骨傷筋。血難復也。116

\hangindent 1em
\hangafter=0
凡微數之脉。不可灸。因熱为邪。必致煩逆。内有損骨傷筋血枯之患。{\gaoben}116

脉浮。当以汗解。而反灸之。邪无從出。因火而盛。病從腰以下必重而痹。此为火逆。若欲自解。当先煩。煩乃有汗。隨汗而解。何以知之。脉浮。故知汗出当解。116

\hangindent 1em
\hangafter=0
脉当以汗解。反以灸之。邪无所去。因火而盛。病当必重。此为逆治。若欲解者。当发其汗而解也。{\gaoben}116

燒針令其汗。針処被寒。核起而赤者。必发奔豚。气從少腹上衝心者。灸其核上各一壯。与桂枝加桂湯。117

火逆。下之。因燒針。煩躁者。桂枝甘草龙骨牡蛎湯主之。118

傷寒。加温針必驚。119

\chapter{胸痹心痛短气}

師曰。夫脉当取太過不及。陽微陰弦。即胸痹而痛。所以然者。責其極虗也。今陽虗。知在上焦。所以胸痹心痛者。以其陰弦故也。

平人无寒熱。短气不足以息者。実也。

胸痹之病。喘息欬唾。胸背痛。短气。寸口脉沈而遲。関上小緊數。栝蔞薤白白酒湯主之。

胸痹。不得卧。心痛徹背者。栝蔞薤白半夏湯主之。

胸痹。心中痞。留气結在胸。胸滿。脇下逆{\khaaitp 气}搶心。枳実薤白桂枝湯主之。人参湯亦主之。

胸痹。胸中气塞。短气。茯苓杏仁甘草湯主之。橘{\khaaitp 皮}枳{\khaaitp 実生}薑湯亦主之。

胸痹緩急者。薏苡附子散主之。

心中痞。諸逆。心懸痛。桂枝生薑枳実湯主之。

心痛徹背。背痛徹心。烏頭赤石脂丸主之。

九痛丸。治九種心痛。{\khaaitp 一虫心痛。二疰心痛。三風心痛。四悸心痛。五食心痛。六飲心痛。七冷心痛。八熱心痛。九去來心痛。}
	\footnote{
		從「一虫心痛」至「九去來心痛」三十七字,吳本鄧本均无,劇《千金方》補。
	}

\chapter{腹滿寒疝宿食}

趺陽脉微弦。法当腹滿。不滿者必便難。兩胠疼痛。此虗寒從下上也。当以温藥服之。

病者腹滿。按之不痛{\khaaitp 者}为虗。痛者为実。可下之。舌黄未下者。下之黄自去。

\hangindent 1em
\hangafter=0
傷寒。腹滿。按之不痛者为虗。痛者为実。当下之。舌黄未下者。下之黄自去。宜大承气湯。{\yuhan}

腹滿时減。復如故。此为寒。当与温藥。

病者痿黄。躁而不渴。胸中寒実。而利不止者。死。

寸口脉弦者。即脇下拘急而痛。其人嗇嗇惡寒也。

夫中寒家。喜欠。其人清涕出。发熱。色和者。善嚏。

中寒。其人下利。以裏虗也。欲嚏不能。此人肚中寒。

夫瘦人繞脐痛。必有風冷。穀气不行。而反下之。其气必衝。不衝者。心下則痞。

病腹滿。发熱十日。脉浮而數。飲食如故。厚朴七物湯主之。

%待補充{\maijing}

\hangindent 1em
\hangafter=0
厚朴湯。治腹滿。发數十日。眿浮數。食飲如故。{\yixin}
	\footnote{醫心方此條中的厚朴湯,其組成与金匱厚朴三物湯相同。}

腹中寒气。雷鳴。切痛。胸脇逆滿。嘔吐。附子粳米湯主之。

腹滿。脉數。厚朴三物湯主之。{\wuben}

痛而閉者。厚朴三物湯主之。{\dengben}

病腹中滿痛者。此为実也。当下之。宜大柴胡湯。{\wuben}

按之心下滿痛者。此为実也。当下之。宜大柴胡湯。{\dengben}

腹滿不減。減不足言。当下之。宜{\khaaitp 大}承气湯。255

心胸中大寒痛。嘔。不能飲食。腹中寒。上衝皮起。出見有頭足。上下痛而不可觸近。大建中湯主之。

脇下偏痛。{\khaaitp 发熱。}其脉緊弦。此寒也。{\khaaitp 当}以温藥下之。宜大黄附子湯。
	\footnote{
		脉經无「发熱」二字。唐弘宇按:根據臨床,寒実内結,腹痛便祕証,有时可見发熱症狀,但发熱不一定是全身性的,可以在某一局部出現,故「发熱」亦可与上句連讀为一句。
	}

寒气厥逆。赤丸主之。

腹痛。脉弦而緊。弦則衛气不行。{\khaaitp 衛气不行}即惡寒。緊則不欲食。邪正相摶。即为寒疝。寒疝遶脐痛。若发則白汗出。手足厥冷。其脉沈弦者。大烏頭煎主之。

寒疝。腹中痛。及脇痛裏急者。当歸生薑羊肉湯主之。

寒疝。腹中痛。逆冷。手足不仁。若身疼痛。灸刺諸藥不能治。烏頭桂枝湯主之。

其脉數而緊乃弦。狀如弓弦。按之不移。脉數弦者。当下其寒。脉雙弦而遲者。必心下堅。脉大而緊者。陽中有陰。可下之。
	\footnote{
		「雙弦」鄧本作「緊大」。
	}

附方

烏頭湯。治寒疝。腹中絞痛。賊風入攻五臟。拘急不得轉側。发作有时。使人陰縮。手足厥逆。

寒疝。腹中痛者。柴胡桂枝湯主之。{\wuben}

柴胡桂枝湯。治心腹卒中痛者。{\dengben}

卒疝。走馬湯主之。{\wuben}

走馬湯。治中惡。心痛。腹胀。大便不通。{\dengben}

問曰。人病有宿食。何以別之。\\
師曰。寸口脉浮而大。按之反濇。尺中亦微而濇。故知有宿食。大承气湯主之。

脉緊如轉索无常者。有宿食也。

脉緊。頭痛。風寒。腹中有宿食不化也。
\footnote{一云寸口脉緊。}

脉數而滑者。実也。此有宿食。下之愈。宜大承气湯。

下利。不欲食者。有宿食也。当下之。宜大承气湯。

宿食在上脘。当吐之。宜瓜蒂散。

\chapter{五臓風寒積聚}

肺中風者。口燥而喘。身運而重。冒而腫胀。

肺中寒者。吐濁涕。

肺死臓。浮之虗。按之弱如葱枼。下无根者。死。

肝中風者。頭目瞤。兩脇痛。行常傴。令人嗜甘。

肝中寒者。兩臂不舉。舌本燥。喜太息。胸中痛。不得轉側。食則吐而汗出也。

肝死臓。浮之弱。按之如索不來。或曲如蛇行者。死。

肝著。其人常欲蹈其胸上。先未苦时。但欲飲熱。旋覆花湯主之。

心中風者。翕翕发熱。不能起。心中飢{\khaaitp 而欲食}。食即嘔。
	\footnote{
		吳本、千金方有「而欲食」三字,鄧本无。
	}

心中寒者。其人苦心如噉蒜狀。劇者心痛徹背。背痛徹心。譬如蠱注。其脉浮者。自吐乃愈。

心傷者。其人勞倦即頭面赤而下重。心中痛而自煩。发熱。当脐跳。其脉弦。此为心臓傷所致也。

心死臓。浮之実如麻豆。按之益躁疾者。死。

邪哭使魂魄不安者。血气少也。血气少者。屬於心。心气虗者。其人則畏。合目欲眠。夢遠行而精神離散。魂魄妄行。陰气衰者为癲。陽气衰者为狂。

脾中風者。翕翕发熱。形如醉人。腹中煩重。皮肉瞤瞤而短气。

脾死臓。浮之大堅。按之如覆杯潔潔。狀如搖者死。

趺陽脉浮而濇。浮則胃气强。濇則小便數。浮濇相摶。大便則堅。其脾为約。麻子仁丸主之。

腎著之病。其人身体重。腰中冷。如坐水中。形如水狀。反不渴。小便自利。飲食如故。病屬下焦。身勞汗出。衣裏冷濕。久久得之。腰以下冷痛。腹重如帶五千錢。甘{\khaaitp 草乾}薑{\khaaitp 茯}苓{\khaaitp 白}术湯主之。

腎死臓。浮之堅。按之亂如轉丸。益下入尺中者死。

問曰。三焦竭部。上焦竭善噫。何谓也。\\
師曰。上焦受中焦气未和。不能消穀。故令噫耳。下焦竭。即遺尿失便。其气不和。不能自禁制。不須治。久自愈。

師曰。熱在上焦者。因欬为肺痿。熱在中焦者。則为堅。熱在下焦者。則尿血。亦令淋祕不通。大腸有寒者。多鶩溏。有熱者。便腸垢。小腸有寒者。其人下重便血。有熱者必痔。

問曰。病有積。有聚。有䅽气。何谓也。\\
師曰。積者。臓病也。終不移。聚者。腑病也。发作有时。展轉痛移。为可治。䅽气者。脇下痛。按之則愈。復发为䅽气。諸積大法。脉來細而附骨者。乃積也。寸口。積在胸中。微出寸口。積在喉中。関上。積在脐傍。上関上。積在心下。微下関。積在少腹。尺中。積在气衝。脉出左。積在左。脉出右。積在右。脉兩出。積在中央。各以其部処之。

\chapter{痰飲欬嗽}

問曰。夫飲有四。何谓也。\\
師曰。有痰飲。有懸飲。有溢飲。有支飲。\\
問曰。四飲何以为異。\\
師曰。其人素盛今瘦。水走腸間。瀝瀝有聲。谓之痰飲。飲後水流在脇下。欬唾引痛。谓之懸飲。飲水流行。歸於四肢。当汗出而不汗出。身体疼重。谓之溢飲。其人欬逆。倚息。短气。不得卧。其形如腫。谓之支飲。
	\footnote{
		「瀝瀝」諸病源{\sungtpii 𠊱}論引作「漉漉」。
	}

水在心。心下堅築{\khaaitp 築}。短气。惡水。不欲飲。

水在肺。吐涎沫。欲飲水。

水在脾。少气。身重。

水在肝。脇下支滿。嚏而痛。

水在腎。心下悸。

夫心下有留飲。其人背寒冷如手大。

留飲者。脇下痛引缺盆。欬嗽則輒已。

胸中有留飲。其人短气而渴。四肢歷節痛。脉沈者。有留飲。

膈上之病。滿喘。欬唾。发則寒熱。背痛。腰疼。目泣自出。其人振振身瞤劇。必有伏飲。{\wuben}

膈上病痰。滿。喘。欬。吐。发則寒熱。背痛。腰疼。目泣自出。其人振振身瞤劇。必有伏飲。{\dengben}

夫病人卒飲水多。必暴喘滿。凡食少飲多。水停心下。甚者則悸。微者短气。

脉雙弦者。寒也。皆大下後喜虗。脉偏弦者。飲也。

肺飲不弦。但苦喘。短气。

支飲。亦喘而不能卧。加短气。其脉平也。

病痰飲者。当以温藥和之。

心下有痰飲。胸脇支滿。目胘。苓桂术甘湯主之。

夫短气。有微飲。当從小便去之。苓桂术甘湯主之。腎气丸亦主之。

病者脉伏。其人欲自利。利反快。雖利。心下續堅滿。此为留飲欲去故也。甘遂半夏湯主之。

脉浮而細滑。傷飲。
	\footnote{
	此條吳本无。
	}

脉弦數。有寒飲。冬夏難治。
	\footnote{
	此條吳本无。
	}

脉沈而弦者。懸飲内痛。
	\footnote{
	此條吳本无。
	}

病懸飲者。十棗湯主之。

病溢飲{\khaaitp 者}。当发其汗。宜大青龙湯。
%病溢飲者。当发其汗。大青龙湯主之。小青龙湯亦主之。
%	\footnote{
%		此條吳本作「病溢飲当发其汗宜大青龙湯」、「病溢飲者当发其汗宜小青龙湯」兩條。
%	}

膈間支飲。其人喘滿。心下痞堅。面色黎黑。其脉沈緊。得之數十日。醫吐下之不愈。木防己湯主之。虗者即愈。実者三日復发。復与不愈者。宜去石膏加茯苓芒硝湯。
	\footnote{
		吳本「虗者即愈実者三日復发復与不愈者宜去石膏加茯苓芒硝湯」位於煎服法之後。
	}

心下有支飲。其人苦冒眩。澤瀉湯主之。

支飲。胸滿者。厚朴大黄湯主之。

支飲。不得息。葶藶大棗瀉肺湯主之。

嘔家本渴。渴者为欲解。今反不渴。心下有支飲故也。小半夏湯主之。

腹滿。口舌乾燥。此腸間有水气。己椒藶黄丸主之。

卒嘔吐。心下痞。膈間有水。眩悸者。{\khaaitp 小}半夏加茯苓湯主之。

假令瘦人脐下悸。吐涎沫而癲眩。{\khaaitp 此}水也。五苓散主之。
	\footnote{
		「脐下悸」同《脉經》,吳本、鄧本作「脐下有悸」。
	}

附方

主心胸中有停痰宿水。自吐出水後。心胸間虗。气滿。不能食。消痰气。令能食。茯苓飲。

欬家。其脉弦。为有水。可与十棗湯。

夫有支飲家。欬煩。胸中痛者。不卒死。至一百日{\khaaitp 或}一歲。与十棗湯。

久欬數歲。其脉弱者可治。実大數者死。其脉虗者必苦冒。其人本有支飲在胸中故也。治屬飲家。

欬逆倚息。小青龙湯主之。{\wuben}

欬逆倚息。不得卧。小青龙湯主之。{\dengben}

青龙湯下已。多唾。口燥。寸脉沈。尺脉微。手足厥逆。气從少腹上衝胸咽。手足痹。其面翕然如醉。因復下流陰股。小便難。时復冒。可与茯苓桂枝五味子甘草湯。治其气衝。{\wuben}

青龙湯下已。多唾。口燥。寸脉沈。尺脉微。手足厥逆。气從少腹上衝胸咽。手足痹。其面翕熱如醉狀。因復下流陰股。小便難。时復冒者。与茯苓桂枝五味子甘草湯。治其气衝。{\dengben}

衝气即低。而反更欬滿者。因茯苓五味子甘草。去桂加乾薑細辛。以治其欬滿。{\wuben}

衝气即低。而反更欬。胸滿者。用桂苓五味甘草湯。去桂加乾薑細辛。以治其欬滿。{\dengben}

欬滿則止。而復更渴。衝气復发者。以細辛乾薑为熱藥。此法不当遂渴。而渴反止者。为支飲也。支飲法当冒。冒者必嘔。嘔者復内半夏。以去其水。{\wuben}

欬滿即止。而更復渴。衝气復发者。以細辛乾薑为熱藥也。服之当遂渴。而渴反止者。为支飲也。支飲者。法当冒。冒者必嘔。嘔者復内半夏。以去其水。{\dengben}

水去。嘔則止。其人形腫。可内麻黄。以其欲逐痹。故不内麻黄。乃内杏仁也。若逆而内麻黄者。其人必厥。所以然者。以其人血虗。麻黄发其陽故也。{\wuben}

水去。嘔止。其人形腫者。加杏仁主之。其証應内麻黄。以其人遂痹。故不内之。若逆而内之者。必厥。所以然者。以其人血虗。麻黄发其陽故也。{\dengben}

若面熱如醉狀者。此为胃中熱。上熏其面令熱。加大黄湯和之。{\wuben}

若面熱如醉。此为胃熱上衝熏其面。加大黄以利之。{\dengben}

先渴卻嘔。为水停心下。此屬飲家。小半夏加茯苓湯主之。{\wuben}

先渴後嘔。为水停心下。此屬飲家。小半夏茯苓湯主之。{\dengben}

\chapter{消渴小便{\khaaitp 不}利淋}

厥陰之为病。消渴。气上撞{\khaaitp 心}。心中疼熱。飢而不欲食。{\khaaitp 甚者}食則吐。下之利不止。326

寸口脉浮而遲。浮即为虗。遲即为勞。虗則衛气不足。勞則榮气竭。趺陽脉浮而數。浮即为气。數即消穀而大{\khaaitp 便}堅。气盛則溲數。溲數即堅。堅數相摶。即为消渴。
	\footnote{
		「大便堅」吳本作「矢堅」,鄧本作「大堅」,「便」字为編者所加。
	}

男子消渴。小便反多。以飲一斗。小便一斗。腎气丸主之。

脉浮。小便不利。微熱。消渴者。宜利小便。发汗。五苓散主之。

渴欲飲水。水入則吐者。名曰水逆。五苓散主之。

渴欲飲水不止者。文蛤散主之。

淋之为病。小便如粟狀。小腹弦急。痛引脐中。

趺陽脉數。胃中有熱。即消穀引食。大便必堅。小便即數。

淋家不可发汗。发汗必便血。84

小便不利者。有水气。其人若渴。栝蔞瞿麥丸主之。
	\footnote{
		「若渴」徐本作「苦渴」。
	}

小便不利。蒲灰散主之。滑石白魚散。茯苓戎鹽湯并主之。

渴欲飲水。口乾舌燥者。白虎加人参湯主之。

脉浮。发熱。渴欲飲水。小便不利者。豬苓湯主之。

\chapter{水气}

師曰。病有風水。有皮水。有正水。有石水。有黄汗。風水。其脉自浮。外証骨節疼痛。{\khaaitp 其人}惡風。皮水。其脉亦浮。外証胕腫。按之沒指。不惡風。其腹如鼓。不渴。当发其汗。正水。其脉沈遲。外証自喘。石水。其脉自沈。外証腹滿。不喘。黄汗。其脉沈遲。身{\khaaitp 体}发熱。胸滿。四肢頭面腫。久不愈。必致癰膿。

脉浮而洪。浮則为風。洪則为气。風气相擊。身体洪腫。汗出乃愈。惡風則虗。此为風水。不惡風者。小便通利。上焦有寒。其人多涎。此为黄汗。{\wuben}

脉浮而洪。浮則为風。洪則为气。風气相摶。風强則为癮疹。身体为癢。癢为泄風。久为痂癩。气强則为水。難以俛仰。風气相擊。身体洪腫。汗出則愈。惡風則虗。此为風水。不惡風者。小便通利。上焦有寒。其口多涎。此为黄汗。{\dengben}

%\hangindent 1em
%\hangafter=0
%脉浮而大。浮为風虗。大为气强。風气相摶。必成癮疹。身体为癢。癢者名泄風。久久为痂癩。(平脉法)

寸口脉沈滑者。中有水气。面目腫大。有熱。名曰風水。視人之目窠上微擁。如{\khaaitp 蠶}新卧起狀。其頸脉動。时时欬。按其手足上。陷而不起者。風水。
	\footnote{
		吳本无此條。
	}

太陽病。脉浮而緊。法当骨節疼痛。反不疼。身体反重而痠。其人不渴。汗出即愈。此为風水。惡寒者。此为極虗。发汗得之。渴而不惡寒者。此为皮水。身腫而冷。狀如周痹。胸中窒。不能食。反聚痛。暮躁不{\khaaitp 得}眠。此为黄汗。痛在骨節。欬而喘。不渴者。此为脾胀。其狀如腫。发汗即愈。然諸病此者。渴而下利。小便數者。皆不可发汗。

裏水者。一身面目自洪腫。其脉沈。小便不利。故令病水。假如小便自利。亡津液。故令渴也。{\wuben}

裏水者。一身面目黄腫。其脉沈。小便不利。故令病水。假如小便自利。此亡津液。故令渴也。越婢加术湯主之。{\dengben}

趺陽脉当伏。今反緊。本自有寒疝瘕。腹中痛。醫反下之。下之即胸滿短气。

趺陽脉当伏。今反數。本自有熱。消穀。小便數。今反不利。此欲作水。

寸口脉浮而遲。浮脉則熱。遲脉則潛。熱潛相摶。名曰沈。趺陽脉浮而數。浮脉即熱。數脉即止。熱止相摶。名曰伏。沈伏相摶。名曰水。沈則絡脉虗。伏則小便難。虗難相摶。水走皮膚。即为水矣。

寸口脉弦而緊。弦則衛气不行。{\khaaitp 衛气不行}即惡寒。水不沾流。走於腸間。

少陰脉緊而沈。緊則为痛。沈則为水。小便即難。脉得諸沈。当責有水。身体腫重。水病脉出者死。

夫水病人。目下有卧蠶。面目鲜澤。脉伏。其人消渴。病水腹水。小便不利。其脉沈絕者。有水。可下之。

問曰。病下利後。渴飲水。小便不利。腹滿因腫者。何也。\\
答曰。此法当病水。若小便自利及汗出者。自当愈。

心水者。其身重而少气。不得卧。煩而躁。其人陰腫。

肝水者。其腹大。不能自轉則。脇下腹痛。时时津液微生。小便續通。

肺水者。其身腫。小便難。时时鴨溏。

脾水者。其腹大。四肢苦重。津液不生。但苦少气。小便難。

腎水者。其腹大。脐腫。腰痛。不得尿。陰下濕如牛鼻上汗。其足逆冷。面反瘦。

師曰。諸有水者。腰以下腫。当利小便。腰以上腫。当发汗乃愈。

師曰。寸口脉沈而遲。沈則为水。遲則为寒。寒水相摶。趺陽脉伏。水穀不化。脾气衰則鶩溏。胃气衰則身腫。{\khaaitp 少陽脉卑。}少陰脉細。男子則小便不利。婦人則經水不通。經为血。血不利則为水。名曰血分。
	\footnote{
		「少陽脉卑」吳本无。
	}

問曰。病者苦水。面目身体四肢皆腫。小便不利。脉之不言水。反言胸中痛。气上衝咽。狀如炙肉。当微欬喘。審如師言。其脉何類。\\
師曰。寸口脉沈而緊。沈为水。緊为寒。沈緊相摶。結在関元。始时当微。年盛不覺。陽衰之後。榮衛相干。陽損陰盛。結寒微動。腎气上衝。喉咽塞噎。脇下急痛。醫以为留飲。而大下之。气擊不去。其病不除。後重吐之。胃家虗煩。咽燥欲飲水。小便不利。水穀不化。面目手足浮腫。又与葶藶丸下水。当时如小差。食飲過度。腫復如前。胸脇苦痛。象若奔豚。其水揚溢。則浮欬喘逆。当先攻擊衛气令止。乃治欬。欬止。其喘自差。先治新病。病当在後。

風水。脉浮。身重。汗出。惡風者。防己黄耆湯主之。腹痛加芍藥。

風水。惡風。一身悉腫。脉浮。不渴。續自汗出。无大熱。越婢湯主之。

皮水为病。四肢腫。水气在皮膚中。四肢聶聶動者。防己茯苓湯主之。

裏水。越婢加术湯主之。甘草麻黄湯亦主之。

水之为病。其脉沈小。屬少陰。浮者为風。无水。虗胀者为气。水。发其汗即已。脉沈者。宜附子麻黄湯。浮者。宜杏子湯。
	\footnote{
		「附子麻黄湯」同吳本,鄧本作「麻黄附子湯」,本方所用藥物、劑量與趙本《傷寒論》「麻黄附子甘草湯」同。
		唐弘宇按:杏子湯方未見,方名後小字註云:「恐是麻黄杏子甘草石膏湯」。可能是大青龙湯。
	}

厥而皮水者。蒲灰散主之。

問曰。黄汗之为病。身体腫。发熱。汗出而渴。狀如風水。汗沾衣。色正黄如檗汁。脉自沈。何從得之。\\
師曰。以汗出入水中浴。水從汗孔入得之。
	\footnote{
		「身体腫」《千金方》作「身体洪腫」。「汗出而渴」千金要方作「汗出不渴」。
	}

黄汗。黄耆芍藥桂枝苦酒湯主之。
	\footnote{
		「黄汗黄耆芍藥桂枝苦酒湯主之」同吳本,鄧本作「宜耆芍桂枝苦酒湯主之」,与上條合为一條。
	}

黄汗之病。兩脛自冷。假令发熱。此屬歷節。食已汗出。又身常暮{\khaaitp 卧}盜汗出者。此勞气也。若汗出已。反发熱者。久久其身必甲錯。发熱不止者。必生惡瘡。若身重。汗出已輒輕者。久久必身瞤。即胸中痛。又從腰以上必汗出。下无汗。腰髖弛痛。如有物在皮中狀。劇者不能食。身疼重。煩躁。小便不利。此为黄汗。桂枝加黄耆湯主之。

師曰。寸口脉遲而濇。遲則为寒。濇为血不足。趺陽脉微而遲。微則为气。遲則为寒。寒气不足。則手足逆冷。手足逆冷。則榮衛不利。榮衛不利。則腹滿脇鳴相逐。气轉膀胱。榮衛俱勞。陽气不通即身冷。陰气不通即骨痛。陽前通則惡寒。陰前通則痹不仁。陰陽相得。其气乃行。大气一轉。其气乃散。実則失气。虗則遺尿。名曰气分。

气分。心下堅。大如盤。邊如旋杯。水飲所作。桂枝去芍藥加麻黄細辛附子湯主之。

心下堅。大如盤。邊如旋盤。水飲所作。枳{\khaaitp 実}术湯主之。

附方

夫風水。脉浮为在表。其人或頭汗出。表无他病。病者但下重。故知從腰以上为和。腰以下当腫及陰。難以屈伸。防己黄耆湯主之。

\chapter{黄疸}

\hangindent 1em
\hangafter=0
論曰。黄有五種。有黄汗。黄疸。穀疸。酒疸。女勞疸。黄汗者。身体四肢微腫。胸滿。不渴。汗出如黄蘗汁。良由大汗出卒入水中所致。黄疸者。一身面目悉黄如橘。由暴得熱。以冷水洗之。熱因留胃中。食生黄瓜熏上所致。若成黑疸者多死。穀疸者。食畢頭眩。心忪怫鬱不安。而发黄。由失飢大食。胃气衝熏所致。酒疸者。心中懊痛。足脛滿。小便黄。面发赤斑黄黑。由大醉当風入水所致。女勞疸者。身目皆黄。发熱。惡寒。小腹滿急。小便難。由大勞大熱而交接竟入水所致。但依後方治之。{\qianjin}
	\footnote{
		此條金匱要略无,從千金要方補入。
	}

寸口脉浮而緩。浮則为風。緩則为痹。痹非中風。四肢苦煩。脾色必黄。瘀熱以行。

趺陽脉緊而數。數則为熱。熱則消穀。緊則为寒。食即为滿。尺脉浮为傷腎。趺陽脉緊为傷脾。風寒相摶。食穀即眩。穀气不消。胃中苦濁。濁气下流。小便不通。陰被其寒。熱流膀胱。身体{\sungtpii 𥁞}黄。名曰穀疸。額上黑。微汗出。手足中熱。薄暮即发。膀胱急。小便自利。名曰女勞疸。腹如水狀。不治。心中懊憹而熱。不能食。时欲吐。名曰酒疸。

陽明病。脉遲者。食難用飽。飽則发煩。頭眩。必小便難。此欲作穀疸。雖下之。腹滿如故。所以然者。脉遲故也。

夫病酒黄疸。必小便不利。其{\sungtpii 𠊱}心中熱。足下熱。是其証也。

酒黄疸者。或无熱。靖言了{\khaaitp 了}。腹滿欲吐。鼻燥。其脉浮者。先吐之。沈弦者。先下之。

酒疸。心中熱。欲嘔者。吐之愈。

酒疸下之。久久为黑疸。目青。面黑。心中如噉蒜虀狀。大便正黑。皮膚爪之不仁。其脉浮弱。雖黑。微黄。故知之。

師曰。病黄疸。发熱。煩喘。胸滿。口燥者。以病发时火劫其汗。兩熱所得。然黄家所得。從濕得之。一身{\sungtpii 𥁞}发熱而黄。肚熱。熱在裏。当下之。

脉沈。渴欲飲水。小便不利者。皆发黄。
	\footnote{
		「脉沈」吳本作「脉浮」。
	}

腹滿。舌痿黄燥。不得睡。屬黄家。

黄疸之病。当以十八日为期。治之十日以上差。反劇。为難治。

疸而渴者。其疸難治。疸而不渴者。其疸可治。发於陰部。其人必嘔。{\khaaitp 发於}陽部。其人振寒而发熱也。

穀疸之为病。寒熱不食。食即頭眩。心胸不安。久久发黄。为穀疸。茵陳蒿湯主之。

黄家。日晡所发熱。而反惡寒。此为女勞得之。膀胱急。少腹滿。身{\sungtpii 𥁞}黄。額上黑。足下熱。因作黑疸。其腹胀如水狀。大便必黑。时溏。此女勞之病。非水也。腹滿者難治。硝石礬石散主之。

酒黄疸。心中懊憹。或熱痛。栀子{\khaaitp 枳実豉}大黄湯主之。

諸病黄家。但利其小便。假令脉浮。当以汗解之。宜桂枝加黄耆湯主之。

諸黄。豬膏髮煎主之。

黄疸病。茵陳五苓散主之。

黄疸。腹滿。小便不利而赤。自汗出。此为表和裏実。当下之。宜大黄{\khaaitp 黄蘗栀子}硝石湯。

黄疸病。小便色不變。欲自利。腹滿而喘。不可除熱。熱除必噦。噦者。小半夏湯主之。

諸黄。腹痛而嘔者。宜柴胡湯。

男子黄。小便自利。当与虗勞小建中湯。

附方

諸黄。瓜蒂湯主之。

黄疸。麻黄淳酒湯主之。

\chapter{驚悸吐衄下血胸滿瘀血}

寸口脉動而弱。動則为驚。弱則为悸。

師曰。尺脉浮。目睛暈黄。衄未止。暈黄去。目睛慧了。知衄今止。
又曰。從春至夏衄者。太陽。從秋至冬衄者。陽明。

衄家不可发汗。汗出必額上促急{\khaaitp 緊}。直視不能眴。不得眠。86

病人面无色。无寒熱。脉沈弦者。衄。{\khaaitp 脉}浮弱。手按之絕者。下血。煩欬者。必吐血。

夫吐血。欬逆上气。其脉數而有熱。不得卧者。死。

夫酒客。欬者。必致吐血。此因極飲過度所致也。

寸口脉弦而大。弦則为減。大則为芤。減則为寒。芤則为虗。寒虗相摶。此名曰革。婦人則半產漏下。男子則亡血。

亡血不可攻其表。汗出則寒慄而振。87

病人胸滿。唇痿。舌青。口燥。但欲嗽水。不欲咽。无寒熱。脉微大來遲。腹不滿。其人言我滿。为有瘀血。

病者如熱狀。煩滿。口乾燥而渴。其脉反无熱。此为陰狀。是瘀血也。当下之。
	\footnote{
		本條下原有「火邪者」一條,此條与本章无関,故移至《奔豚气吐膿驚怖火邪》篇中。
	}

心下悸者。半夏麻黄丸主之。

吐血不止者。柏枼湯主之。

下血。先見血。後見便。此近血也。赤小豆当歸散主之。先見便。後見血。此遠血也。黄土湯主之。

附方

心气不足。吐血。衄血。瀉心湯主之。

\chapter{嘔吐噦下利}

夫嘔家有癰膿。不可治嘔。膿{\sungtpii 𥁞}自愈。376

先嘔卻渴者。此为欲解。先渴卻嘔者。为水停心下。此屬飲家。

嘔家本渴。今反不渴者。以心下有支飲故也。此屬支飲。

問曰。病人脉數。數为熱。当消穀引食。而反吐者。何也。\\
師曰。以发其汗。令陽微。膈气虗。脉乃數。數为客熱。不能消穀。胃中虗冷。故吐也。脉弦者虗也。胃气无餘。朝食暮吐。變为胃反。寒在於上。醫反下之。令脉反弦。故名曰虗。
	\footnote{
	「令脉」吳本作「今脉」。
	}

寸口脉微而數。微則无气。无气則榮虗。榮虗則血不足。血不足則胸中冷。

趺陽脉浮而濇。浮則为虗。濇則傷脾。脾傷則不磨。朝食暮吐。暮食朝吐。宿穀不化。名曰胃反。脉緊而濇。其病難治。
	\footnote{
	吳本此條与上條合为一條。
	}

病人欲吐者。不可下之。

噦而腹滿。視其前後。知何部不利。利之即愈。

嘔而胸滿者。{\khaaitp 吳}茱萸湯主之。

乾嘔。吐涎沫。頭痛者。{\khaaitp 吳}茱萸湯主之。378

嘔而腸鳴。心下痞者。半夏瀉心湯主之。

乾嘔而利者。黄芩加半夏生薑湯主之。

諸嘔吐。穀不得下者。小半夏湯主之。

嘔吐。而病在膈上。後思水者解。急与之。思水者。豬苓散主之。

嘔而脉弱。小便復利。身有微熱。見厥者。難治。四逆湯主之。377

嘔而发熱者。小柴胡湯主之。379

胃反。嘔吐者。大半夏湯主之。

食已即吐者。大黄甘草湯主之。

胃反。吐而渴欲飲水者。茯苓澤瀉湯主之。

吐後。渴欲得飲而貪水者。文蛤湯主之。兼主微風。脉緊。頭痛。

乾嘔。吐{\sungtpii 𠱘}。吐涎沫。半夏乾薑散主之。

病人胸中似喘不喘。似嘔不嘔。似噦不噦。徹心中憒憒然无奈者。生薑半夏湯主之。

乾嘔。噦。若手足厥{\khaaitp 冷}者。橘皮湯主之。

噦{\sungtpii 𠱘}者。橘皮竹茹湯主之。

夫六腑气絕於外者。手足寒。上气。腳縮。五臓气絕於内者。利不禁。下甚者。手足不仁。

下利。脉沈弦者。下重。脉大者。为未止。脉微弱數者。为欲自止。雖发熱。不死。

下利。手足厥{\khaaitp 冷}。无脉。{\khaaitp 当灸其厥陰。}灸之不温{\khaaitp 而脉不還}。反微喘者死。少陰負趺陽者为順。362

下利。有微熱而渴。脉弱者。今自愈。

下利。脉數。有微熱。汗出者。自愈。設{\khaaitp 脉}復緊。为未解。361

下利。脉數而渴者。今自愈。設不差。必清膿血。以有熱故也。

下利。脉反弦。发熱。身汗者。自愈。

下利气者。当利其小便。

下利。寸脉反浮數。尺中自濇者。必清膿血。

下利清穀。不可攻其表。汗出必胀滿。

下利。脉沈而遲。其人面少赤。身有微熱。下利清穀者。必鬱冒。汗出而解。其人必微厥。所以然者。其面戴陽。下虗故也。366
	\footnote{
		「必微厥」鄧本作「必微熱」。
	}

下利後。脉絕。手足厥{\khaaitp 冷}。晬时脉還。手足温者生。不還{\khaaitp 不温}者死。368

下利。腹胀滿。身体疼痛者。先温其裏。乃攻其表。温裏宜四逆湯。攻表宜桂枝湯。372

下利。三部脉皆平。按之心下堅者。急下之。宜大承湯。

下利。脉遲而滑者。実也。利未欲止。急下之。宜大承气湯。

下利。脉反滑{\khaaitp 者}。当有所去。下乃愈。宜大承气湯。

下利已差。至其时復发者。此为病不{\sungtpii 𥁞}。当{\khaaitp 復}下之。宜{\khaaitp 大}承气湯。

下利。譫語者。有燥屎也。宜{\khaaitp 小}承气湯。374

下利。便膿血者。桃花湯主之。

熱利下重者。白頭翁湯主之。371

下利後更煩。按之心下濡者。为虗煩也。栀子{\khaaitp 豉}湯主之。375

下利清穀。裏寒外熱。汗出而厥。通脉四逆湯主之。370

下利。肺痛。紫参湯主之。

气利。訶梨勒散主之。

附方

小承气湯。治大便不通。噦。數譫語。

乾嘔。下利。黄芩湯主之。{\scriptsize 玉函經云人参黄芩湯}

\chapter{瘡癰腸癰浸淫}

諸浮數脉。應当发熱。而反洒淅惡寒。若有痛処。当发其癰。

師曰。諸癰腫。欲知有膿无膿。以手掩腫上。熱者为有膿。不熱者为无膿。

腸癰之为病。其身甲錯。腹皮急。按之濡。如腫狀。腹无積聚。身无熱。脉數。此为腸内有{\khaaitp 癰}膿。薏苡{\khaaitp 仁}附子敗醬散主之。

腸癰者。少腹腫痞。按之即痛如淋。小便自調。时时发熱。自汗出。復惡寒。脉遲緊者。膿未成。可下之。当有血。脉洪數者。膿已成。不可下也。大黄牡丹湯主之。

問曰。寸口脉浮微而濇。法当亡血。若汗出。設不汗者云何。\\
答曰。若身有瘡。被刀器所傷。亡血故也。

病金瘡。王不留行散主之。
	\footnote{吳本此條与上條合为一條。}

排膿散。

排膿湯。

浸淫瘡。從口流向四肢者。可治。從四肢流來入口者。不可治。黄連粉主之。{\scriptsize 方未見。}
	\footnote{「黄連粉主之」鄧本作「浸淫瘡黄連粉主之」單獨列为一條。}

\chapter{趺蹶手指臂脛轉筋狐疝蛔虫}

師曰。病趺蹶。其人但能前。不能卻。刺腨入二寸。此太陽經傷也。

病人常以手指臂脛動。此人身体瞤瞤者。藜蘆甘草湯主之。{\scriptsize 方未見。}
	\footnote{
		「臂脛動」鄧本作「臂腫動」。
	}

轉筋之为病。其人臂腳直。脉上下行。微弦。轉筋入腹者。雞屎白散主之。

陰狐疝气者。偏有小大。时时上下。蜘蛛散主之。

問曰。病腹痛。有虫。其脉何以別之。\\
師曰。腹中痛。其脉当沈弦。反洪大。故有蛔虫。
	\footnote{「沈弦」吳本、鄧本均作「沈若弦」,編者改。}

蛔虫之为病。令人吐涎。心痛。发作有时。毒藥不止。甘草粉蜜湯主之。

蛔厥者。其人当吐蛔。今病者靜。而復时煩。此为臓寒。蛔上入膈。故煩。須臾復止。得食而嘔。又煩者。蛔聞食臭出。其人常自吐蛔。蛔厥者。烏梅丸主之。338

\chapter{雜療方
	\footnote{
		唐弘宇按:觀此章文法,不像仲景原文,本來已經刪除,後來讀到《諸病源{\sungtpii 𠊱}論》卷六解散病諸{\sungtpii 𠊱}第一寒食散发{\sungtpii 𠊱}引皇甫云:「然寒食藥者。世末知焉。或言華佗。或曰仲景。考之於実。佗之精微。方類單省。而仲景經有矦氏黑散。紫石英方。皆數種相出入。節度略同。然則寒食草石二方。出自仲景。非佗也。且佗之为治。或刳斷腸胃。滌洗五臟。不純任方也。仲景雖精。不及於佗。至於審方物之{\sungtpii 𠊱}。論草石之宜。亦妙絕乑醫。」據此段引文至少可以確定本章中的「紫石寒食散」是仲景方,所以为了穩妥起見,將本章保留。
	}
}

%宣通五臟虗熱。四时加減柴胡飲子。

%長服訶梨勒丸。

%三物備急丸。

%備急散。治人卒上气。呼吸气不得下。喘逆。

%紫石寒食散。治傷寒令已愈不復。

%救卒死方。

\chapter{婦人妊娠病}

師曰。脉婦人得平脉。陰脉小弱。其人渴。不能食。无寒熱。名为軀。桂枝湯主之。法六十日当有娠。設有醫治逆者。卻一月。加吐下者。則絕之。{\wuben}

師曰。婦人得平脉。陰脉小弱。其人渴。不能食。无寒熱。名妊娠。桂枝湯主之。於法六十日当有此証。設有醫治逆者。卻一月。加吐下者。則絕之。{\dengben}

婦人妊娠。經斷三月。而得漏下。下血四十日不止。胎欲動。在於脐上。此为妊娠。六月動者。前三月經水利时。胎也。下血者。後斷三月。衃也。所以下血不止者。其癥不去故也。当下其癥。宜桂枝茯苓丸。{\wuben}

婦人宿有癥病。經斷未及三月。而得漏下不止。胎動在脐上者。为癥痼害。妊娠六月動者。前三月經水利时。胎下血者。後斷三月。衃也。所以血不止者。其癥不去故也。当下其癥。桂枝茯苓丸主之。{\dengben}

婦人懷娠六七月。脉弦。发熱。其胎愈胀。腹痛。惡寒者。少腹如扇之狀。所以然者。子臓開故也。当以附子湯温其臓。
	\footnote{方未見。}
	\footnote{「愈胀」吳本、脉經作「踰腹」。}

師曰。婦人有漏下者。有半產後因續下血都不絕者。有妊娠下血者。假令妊娠腹中痛。为胞阻。膠艾湯主之。

婦人懷娠。腹中㽲痛。当歸芍藥散主之。

{\khaaitp 婦人}妊娠。嘔吐不止。乾薑人参半夏丸主之。

{\khaaitp 婦人}妊娠。小便難。飲食如故。{\khaaitp 当}歸{\khaaitp 貝}母苦参丸主之。

{\khaaitp 婦人}妊娠。有水气。身重。小便不利。洒淅惡寒。起即頭眩。葵子茯苓散主之。

婦人妊娠。宜服当歸散。

附方

妊娠養胎。白术散。

婦人傷胎。懷身。腹滿。不得小便。從腰以下重。如有水气狀。懷身七月。太陰当養不養。此心气実。当刺瀉勞宮及関元。小便利則愈。
	\footnote{吳本「傷胎」作「傷寒」。鄧本「小便利」作「小便微利」。}

\chapter{婦人產後病}

問曰。新產婦人有三病。一者病痙。二者病鬱冒。三者大便難。何谓也。\\
師曰。新產血虗。多汗出。喜中風。故令病痙。\\
{\khaaitp 問曰。何故鬱冒。}\\
{\khaaitp 師曰。}亡血復汗。寒多。故令鬱冒。\\
{\khaaitp 問曰。何故大便難。}\\
{\khaaitp 師曰。}亡津液。胃燥。故大便難。

產婦鬱{\khaaitp 冒}。其脉微弱。不能食。大便反堅。但頭汗出。所以然者。血虗而厥。厥而必冒。冒家欲解。必大汗出。以血虗下厥。孤陽上出。故但頭汗出。所以產婦喜汗出者。亡陰血虗。陽气獨盛。故当汗出。陰陽乃復。所以便堅者。嘔。不能食也。小柴胡湯主之。病解。能食。七八日。更发熱者。此为胃熱气実。大承气湯主之。{\wuben}

產婦鬱冒。其脉微弱。不能食。大便反堅。但頭汗出。所以然者。血虗而厥。厥而必冒。冒家欲解。必大汗出。以血虗下厥。孤陽上出。故頭汗出。所以產婦喜汗出者。亡陰血虗。陽气獨盛。故当汗出。陰陽乃復。大便堅。嘔。不能食。小柴胡湯主之。病解。能食。七八日。更发熱者。此为胃実。大承气湯主之。{\dengben}
	\footnote{從「病解能食」以下,鄧本獨立作一條。吳本本條与上條合为一條。}

{\khaaitp 婦人}產後。腹中㽲痛。当歸生薑羊肉湯主之。并治腹中寒疝。虗勞不足。

{\khaaitp 婦人}產後。腹痛。煩滿。不得卧。枳実芍藥散主之。

師曰。產婦腹痛。法当与枳実芍藥散。假令不愈者。此为腹中有乾血著脐下。与下瘀血湯服之。{\khaaitp 亦}主經水不利。

婦人產後七八日。无太陽証。少腹堅痛。此惡露不{\sungtpii 𥁞}。不大便四五日。趺陽脉微実。再倍其人发熱。日晡所煩躁者。不{\khaaitp 能}食。食即譫語。利之即愈。宜大承气湯。熱在裏。結在膀胱也。{\wuben}

{\khaaitp 婦人}產後七八日。无太陽証。少腹堅痛。此惡露不{\sungtpii 𥁞}。不大便。煩躁。发熱。切脉微実。再倍发熱。日晡时煩躁者。不{\khaaitp 能}食。食則譫語。至夜即愈。宜大承气湯主之。熱在裏。結在膀胱也。{\dengben}

婦人產得風。續之數十日不解。頭微痛。惡寒。时时有熱。心下堅。乾嘔。汗出。雖久。陽旦証續在耳。可与陽旦湯。{\wuben}

產後風。續之數十日不解。頭微痛。惡寒。时时有熱。心下悶。乾嘔。汗出。雖久。陽旦証續在耳。可与陽旦湯。{\dengben}

{\khaaitp 婦人}產後中風。发熱。面{\khaaitp 正}赤。喘而頭痛。竹枼湯主之。

婦人產中虗。煩亂。嘔{\sungtpii 𠱘}。安中益气。竹皮大丸主之。
	\footnote{脉經作「產中虗」,吳本、鄧本作「乳中虗」。}

{\khaaitp 婦人}產後下利。虗極。白頭翁加甘草阿膠湯主之。

附方

婦人在草蓐得風。四肢苦煩熱。皆自发露所为。頭痛者。与小柴胡湯。頭不痛。但煩者。与三物黄芩湯。{\wuben}

三物黄芩湯。治婦人在草蓐自发露得風。四肢苦煩熱。頭痛者。与小柴胡湯。頭不痛。但煩者。此湯主之。{\dengben}

治婦人產後。虗羸不足。腹中刺痛不止。吸吸少气。或苦少腹拘急攣痛引腰背。不能食飲。產後一月。日得服四五剂为善。令人强壯。内補当歸建中湯。{\wuben}

内補当歸建中湯。治婦人產後。虗羸不足。腹中刺痛不止。吸吸少气。或苦少腹拘急攣痛引腰背。不能食飲。產後一月。日得服四五剂为善。令人强壯。{\dengben}

\chapter{婦人雜病}

婦人中風七八日。續得寒熱。发作有时。經水適斷。此为熱入血室。其血必結。故使如瘧狀。发作有时。小柴胡湯主之。144

婦人傷寒。发熱。經水適來。晝日明了。暮則譫語。如見鬼狀。此为熱入血室。无犯胃气及上二焦。必自愈。145

婦人中風。发熱。惡寒。經水適來。得之七八日。熱除。脉遲。身涼。胸脇下滿。如結胸狀。其人譫語。此为熱入血室。当刺期門。隨其{\khaaitp 虗}実而取之。143

陽明病。下血。譫語者。此为熱入血室。但頭汗出。当刺期門。隨其実而瀉之。濈然汗出者愈。

婦人咽中如有炙臠。半夏厚朴湯主之。

婦人臓躁。喜悲傷。欲哭。象如神靈所作。數欠伸。甘草小麥大棗湯主之。

婦人吐涎沫。醫反下之。心下即痞。当先治其吐涎沫。宜小青龙湯。涎沫止。乃治痞。宜瀉心湯。

婦人之病。因虗稍入結气。为諸經水斷絕。至有歷年。血寒積結。胞門寒傷。經絡凝堅。在上嘔吐涎唾。久成肺癰。形体損分。在中盤結。繞脐寒疝。或兩脇疼痛。与臓相連。或結熱在中。痛在関元。脉數。无瘡。肌若魚鱗。时著男子。非止女身。在下未多。經{\sungtpii 𠊱}不勻。令陰掣痛。少腹惡寒。或引腰脊。下根气街。气衝急痛。膝脛疼煩。或奄忽眩冒。狀如厥癲。或有憂惨。悲傷多嗔。此皆帶下。非有鬼神。久則羸瘦。脉虗多寒。三十六病。千變万端。審脉陰陽。虗実緊弦。行其針藥。治危得安。其雖同病。脉各異源。子当辯記。勿谓不然。
	\footnote{「稍入」同吳本,鄧本作「積冷」。}

問曰。婦人年五十所。病下血。數十日不止。暮即发熱。少腹裏急。腹滿。手掌煩熱。唇口乾燥。何也。\\
師曰。此病屬帶下。何以故。曾經半產。瘀血在少腹不去。何以知之。其証唇口乾燥。故知之。当以温經湯主之。
	\footnote{
		「病下血」諸本均作「病下利」,編者改。吳本「少腹裏急」下有「痛」字。
	}

{\khaaitp 婦人}帶下。經水不利。少腹滿痛。經一月再見者。土瓜根散主之。

寸口脉弦而大。弦則为減。大則为芤。減則为寒。芤則为虗。寒虗相摶。此名曰革。婦人則半產漏下。旋覆花湯主之。

婦人陷經。漏下黑不解。膠薑湯主之。{\scriptsize 臣億等按。諸本无膠薑湯{\khaaitp 方}。恐是前妊娠中膠艾湯也。}

婦人少腹滿如敦狀。小便微難而不渴。生後者。此为水与血并結在血室也。大黄甘遂湯主之。

婦人經水不利。抵当湯主之。
	\footnote{鄧本「經水不利」四字下有「下」字。}

婦人經水閉不利。臓堅癖不止。中有乾血。下白物。礬石丸主之。

紅藍花酒。治婦人六十二種風。兼主腹中血气刺痛。
%唐弘宇:此條整合了吳本和鄧本。

婦人腹中諸疾痛。当歸芍藥散主之。

婦人腹中痛。小建中湯主之。

問曰。婦人病。飲食如故。煩熱不得卧。而反倚息者。何也。\\
師曰。此病轉胞。不得尿也。以胞系了戾。故致此病。但利小便則愈。宜腎气丸。
	\footnote{「宜腎气丸」後,鄧本有「主之」二字,吳本有「以中有茯苓故也」七字。}

温陰中坐藥。蛇床子散。

少陰脉滑而數者。陰中即生瘡。

{\khaaitp 婦人}陰中蝕瘡爛。狼牙湯洗之。

胃气下泄。陰吹而正喧。此穀气之実也。膏髮煎導之。

小兒疳虫蝕齒方。

%\part{輯佚}

\part{衍文}

問曰。証象陽旦。按法治之而增劇。厥逆。咽中乾。兩脛拘急而讝語。\\
師曰。言夜半手足当温。兩腳当伸。後如師言。何以知此。\\
答曰。寸口脉浮而大。浮{\khaaitp 則}为風。大{\khaaitp 則}为虗。風則生微熱。虗則兩脛攣。病形象桂枝。因加附子参其間。增桂令汗出。附子温經。亡陽故也。厥逆。咽中乾。煩燥。陽明内結。讝語。煩亂。更飲甘草乾薑湯。夜半陽气還。兩足当熱。脛尚微拘急。重与芍藥甘草湯。尔乃脛伸。以承气湯微溏。則止其讝語。故知病可愈。30

太陽病二日。反躁。反熨其背。而大汗出。大熱入胃。胃中水竭。躁煩。必发譫語。十餘日。振慄。自下利者。此为欲解也。故其汗從腰以下不得汗。欲小便不得。反嘔。欲失溲。足下惡風。大便硬。小便当數。而反不數及不多。大便已。頭卓然而痛。其人足心必熱。穀气下流故也。110

太陽病。中風。以火劫发汗。邪風被火熱。血气流溢。失其常度。兩陽相熏灼。其身发黄。陽盛則欲衄。陰虗{\khaaitp 則}小便難。陰陽俱虗竭。身体則枯燥。但頭汗出。齐頸而還。腹滿微喘。口乾咽爛。或不大便。久則譫語。甚者至噦。手足躁擾。捻衣摸床。小便利者。其人可治。111

脉按之來緩。时一止復來者。名曰結。又脉來動而中止。更來小數。中有還者反動。名曰結。陰也。脉來動而中止。不能自還。因而復動者。名曰代。陰也。得此脉者。必難治。178

問曰。上工治未病。何也。\\
師曰。夫治未病者。見肝之病。知肝傳脾。当先実脾。四季脾王不受邪。即勿補之。中工不曉相傳。見肝之病。不解実脾。惟治肝也。\\
夫肝之病。補用酸。助用焦苦。益用甘味之藥調之。酸入肝。焦苦入心。甘入脾。脾能傷腎。腎气微弱則水不行。水不行則心火气盛。心火气盛則傷肺。肺被傷則金气不行。金气不行則肝气盛。故実脾則肝自愈。此治肝補脾之要妙也。肝虛則用此法。実則不在用之。經曰。虛虛実実。補不足。損有餘。是其義也。餘藏准此。\\
夫人稟五常。因風气而生長。風气雖能生萬物。亦能害萬物。如水能浮舟。亦能覆舟。若五臟元真通暢。人即安和。客气邪風。中人多死。千般災難。不越三條。一者。經絡受邪。入臟腑。为内所因也。二者。四肢九竅。血脉相傳。壅塞不通。为外皮膚所中也。三者。房室金刃虫獸所傷。以此詳之。病由都{\sungtpii 𥁞}。若人能養慎。不令邪風干忤經絡。適中經絡。未流傳腑臟。即醫治之。四肢才覺重滯。即導引吐納鍼灸膏摩。勿令九竅閉塞。更能无犯王法。禽獸災傷。房室勿令竭之。服食節其冷熱苦酸辛甘。不遺形体有衰。病則无由入其腠理。腠者。是三焦通會元真之處。为血气所注。理者。是皮膚臟腑之文理也。

\end{document}

%字形
%
%実为与洒虗术体气処无当脉沈温別卧麥盖内弃泪时𥁞解发强枼栀鲜
%覺學舉
%納約結細緣緩縮純紅絕絞縱經絡續綱終繞總綠紙絲綿
%証許譫語設諸診訴謝辯訣記談諺謬誠靄諦訓認譚識説調論議誤訶
%齐脐剂
%腫種
%針鑠鎮鐘銓錯銖鋸錦鋭銅
%門瞤潤悶闔聞問閉開関間闕癇爛
%輿輗輒暈渾漸連軺載暫陳運轉輕輩
%尔弥
%飲飢飪飽餘蝕餅饐
%龙聋
%万厉蛎
%帶滯
%黄横
%长胀张
%参惨
%專轉傳
%𠊱矦
