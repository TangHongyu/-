
% main.tex
% Encoding: UTF-8


% preamble.tex
% Encoding: UTF-8

%“em”是相對长度單位,相当於所用字体中大寫“M”的寬度。
%“ex”是相對长度單位,相当於所用字体中小寫“x”的高度,此高度通常为字体尺寸的一半。

\documentclass[b5paper,twoside,zihao=-4,openany]{ctexbook}

%\usepackage{syntonly}
%	\syntaxonly

\usepackage{ctex}
	\ctexset{
		section = {
			format += \zihao{-4} \heiti \raggedright,
			name = {,.},
			number = \arabic{section},
			beforeskip = 1.0ex plus 0.2ex minus .2ex,
			afterskip = 1.0ex plus 0.2ex minus .2ex,
			aftername = \hspace{0.1em}
		},
		chapter = {
			format += \zihao{-3} \heiti,
			name = {,.},
			number = \arabic{chapter},
			beforeskip = 1.0ex plus 0.2ex minus .2ex,
			afterskip = 1.0ex plus 0.2ex minus .2ex,
			aftername = \hspace{0.1em}
		}
	}

\usepackage{xeCJK}
	\xeCJKsetup{
		PunctStyle = kaiming
	}
	\setCJKmainfont[Path=fonts/]{TH-Sung-TP0.ttf}
	\setCJKsansfont[Path=fonts/]{TH-Sung-TP0.ttf}
	\setCJKmonofont[Path=fonts/]{TH-Sung-TP0.ttf}
	\setCJKfamilyfont{sungtpii}[Path=fonts/]{TH-Sung-TP2.ttf}
	\setCJKfamilyfont{khaaitp}[Path=fonts/]{TH-Khaai-TP0.ttf}
	\setCJKfamilyfont{khaaitpii}[Path=fonts/]{TH-Khaai-TP2.ttf}
	
%	\setCJKmainfont{SimSun}
%	\setCJKsansfont{SimSun}
%	\setCJKmonofont{SimSun}

%\usepackage{fontspec}
%	\setmainfont{SimSun}
%	\setmainfont[Path=fonts/]{TH-Sy-P0}
%	\setsansfont{SimSun}

\usepackage{geometry}
	\geometry{left=2.5cm,right=2.5cm,top=2cm,bottom=2cm}

\usepackage[stable]{footmisc}

\usepackage{graphicx}
	\graphicspath{{pics/}}

\usepackage{xcolor}

\usepackage{tocloft}
	\cftbeforesecskip 0.5ex
	\cftbeforechapskip 0.5ex
	\cftbeforepartskip 1ex
	\cftsecindent 4em
	\cftchapindent 2em
	\cftpartindent 0em

\usepackage{makeidx}
	\makeindex

\usepackage{tcolorbox}
	\tcbuselibrary{skins, vignette, breakable, theorems, fitting}

\usepackage{lineno}

%\nofiles
%\punctstyle{kaiming}
%\raggedright
\pagestyle{headings}
\CJKsetecglue{}%完全禁用汉字与其他内容间的空格
\setlength{\parindent}{0em}%段落中第一行缩进量
\setlength{\parskip}{3ex}%段落间距
\setcounter{secnumdepth}{4}
\setcounter{tocdepth}{4}

\title{傷寒雜病論匯校本
%	\thanks{
%		
%	}
	\footnote{
		唐弘宇按:本論的原名現已不可考,可能本來就沒有名字。至於「张仲景方」、「张仲景辨傷寒」、「仲景全書」、「傷寒卒病論」等名,与「傷寒雜病論」一樣,皆是後人杜撰,沒有優劣之分。我選取「傷寒雜病論」作为書名的原因有二,一是這個名字已被大乑所熟知,二是這個書名概括了本論的兩大部分,即傷寒部分和雜病部分。有很多老師和同學給我提意見,説這個名字不好,應改成XXX。我則認为名字叫什麼不必太過追究,陸淵雷説的好:「其精華,在於諸方之症候用法」。
	}
}

\author{
	张仲景{\hfill}述\\
	王叔和{\hfill}撰次\\
	唐弘宇{\hfill}校訂
%	{\includegraphics[width=3em]{tanghongyu.jpg}}校訂
%\and
%	【編{\hfill}委】
%		\footnote{
%			按貢獻排序。
%		}\\
%	枼建橋\\
%	毛{\hfill}敏\\
%	胡曉帆
%	{\includegraphics[width=3em]{yejianqiao.jpg}}\\
%	{\includegraphics[width=3em]{maomin.jpg}}\\
%	{\includegraphics[width=3em]{huxiaofan.jpg}}
}

\date{\today}

\endinput

% newcommand.tex
% Encoding: UTF-8

%\renewcommand*{\sc}{\scriptsize}
%\renewcommand*{\fn}{\footnote}

\newcommand{\jing}{\hbox{\scalebox{0.6}[1]{纟}\kern-0.3em\scalebox{0.7}[1]{巠}}}%經
\newcommand{\qing}{\hbox{\scalebox{0.5}[1]{车}\kern-0.15em\scalebox{0.65}[1]{巠}}}%輕
\newcommand{\xu}{\hbox{\scalebox{0.6}[1]{纟}\kern-0.3em\scalebox{0.7}[1]{賣}}}%續
\newcommand{\rao}{\hbox{\scalebox{0.6}[1]{纟}\kern-0.3em\scalebox{0.7}[1]{堯}}}%繞

\newcommand{\wuben}{\colorbox{black}{\textcolor{white}{吳}}}
\newcommand{\dengben}{\colorbox{black}{\textcolor{white}{鄧}}}
\newcommand{\zhaoben}{\colorbox{black}{\textcolor{white}{趙}}}
\newcommand{\shenghui}{\colorbox{black}{\textcolor{white}{淳}}}
\newcommand{\qianjin}{\colorbox{black}{\textcolor{white}{千}}}
\newcommand{\yifang}{\colorbox{black}{\textcolor{white}{翼}}}
\newcommand{\chenben}{\colorbox{black}{\textcolor{white}{成}}}
\newcommand{\yuhan}{\colorbox{black}{\textcolor{white}{函}}}
\newcommand{\maijing}{\colorbox{black}{\textcolor{white}{脉}}}
\newcommand{\yixin}{\colorbox{black}{\textcolor{white}{心}}}

%\newcommand{\sungtpii}{\CJKfontspec[Path=fonts/]{TH-Sung-TP2.ttf}}
%\newcommand{\tshynpii}{\CJKfontspec[Path=fonts/]{TH-Tshyn-P2.ttf}}
%\newcommand{\tshynpxvi}{\CJKfontspec[Path=fonts/]{TH-Tshyn-P16.ttf}}
%\newcommand{\khaaitp}{\CJKfontspec[Path=fonts/]{TH-Khaai-TP0.ttf}}
%\newcommand{\khaaitpii}{\CJKfontspec[Path=fonts/]{TH-Khaai-TP2.ttf}}

\newcommand*{\khaai}{\CJKfamily{khaai}}
\newcommand*{\khaaiii}{\CJKfamily{khaaiii}}
\newcommand*{\sungii}{\CJKfamily{sungii}}

\endinput

\begin{document}

% frontmatter.tex
% Encoding: UTF-8

\frontmatter
\maketitle
\tableofcontents

\chapter{凡例}

\begin{itemize}
	
	\item 《傷寒雜病論匯校本》分为三個部分:第一部分,類聚方部分;第二部分,傷寒部分;第三部分,雜病部分。\\
	《類聚方》一書,是日本漢方醫學古方派代表人物吉益東洞的著作,此書分解《傷寒論》、《金匱要略》條文,以方剂为類目,匯集仲景相関論述,着意凸顯其「方証相對」之學術主張。本書的第一部分,即是以吉益東洞《類聚方》的形式,重新排列第二、第三部分的條文。方剂的先後次序基本按照《類聚方》,《類聚方》中缺失的方剂,則按照其在《傷寒雜病論》中出現的次序,排在最後。\\
	第一部分中的所有條文,与第二、第三部分的條文完全相同,但條文的註釋和校改記錄,僅在第二、第三部分有,第一部分不再出現。\\
	第二、第三部分僅有條文,无方剂,所有方剂僅出現在第一部分,所以對方剂的註釋,也僅出現在第一部分。
	
	\item 本書傷寒論部分所用的底本,是臺北故宮博物院文獻大樓藏明趙開美翻刻北宋元祐三年小字本《傷寒論》,雜病論部分所用的底本,是明吳遷钞本《金匱要略方》。本書所用的主校本有《千金翼方》、《金匱玉函經》、《脉經》。参校本有《千金要方》、《外臺祕要》、《聖惠方》、《醫心方》等。不用《康平本》、《康治本》、《桂林古本》。
	
	\item 主要參考書目:
		\begin{itemize}
			\item 《影印孫思邈本傷寒論校注考證》
			\item 《宋本傷寒論文獻史論》
			\item 《影印南朝祕本敦煌祕卷傷寒論校注考證》
			\item 《影印金匱玉函經校注考證》
			\item 《校勘元本影印明本金匱要略集》
			\item 《明洪武鈔本金匱要略方》
		\end{itemize}

	\item 古人校訂書籍,不修改原文,只在疑似有錯処出校注,這樣做是为了保存文獻原貌,防止由於自己理解錯誤而妄改原文。這些謹慎的作法,都是因为古代印刷技術、保存手段落後等原因,而採取的無奈之舉。在現代,保存古籍的任務有政府建立圖書館負責,而且以現代的保存手段,不太可能再有古籍消失。所以,我在校訂此書時,不考慮保存原貌的問題,直接修改原文,必要時出校記,這樣做是为了使學習、閲讀更加方便。
	
	\item 規範字形,將古書中的異體字、通假字,一律改为正字,部分字使用簡化字。請注意,「正字」不是臺灣正體字,「簡化字」也不是現行的大陸簡體字,宋本書籍中已有使用簡化字的先例。
	
	\item 《太平聖惠方》卷二論合和曰:「古方藥味,多以銖兩,及用水,皆言升數。年代綿歷浸遠,傳寫轉見乖訛。」(待補充)。大柴胡湯、柴胡桂枝乾薑湯在《傷寒論》和《金匱要略》中重出,其中柴胡的份量,趙本《傷寒論》作「半斤」,吳本《金匱要略》作「八兩」。为盡量統一全書單位,書中所有的「斤」單位,全部依此例轉換为「兩」單位,「一斤」合「十六兩」。
	
	\item 相比宋本系統,本書在行文上更簡潔,風格更接近《千金翼方》和《金匱玉函經》。舉一個例子,趙本第282條中的「下焦虗有寒」,《千金翼》作「下焦虗寒」,《聖惠方》作「下焦有虗寒」,它們表達的意思相同,但《千金翼》更簡潔,所以我選取了《千金翼》的説法。
	
	\item 
	
	\item 
	
	\item 
	
\end{itemize}























\endinput

\tableofcontents
\mainmatter

\part{類聚方}

\section{桂枝湯}

桂枝{\scriptsize 三兩去皮} 芍藥{\scriptsize 三兩} 甘草{\scriptsize 二兩炙} 生薑{\scriptsize 三兩切} 大棗{\scriptsize 十二枚擘}\\
右五味。㕮咀三味。以水七升。微火煮取三升。去滓。適寒温。服一升。服已須臾。歠熱稀粥一升餘。以助藥力。温覆令一时許。遍身漐漐微似有汗者益佳。不可令如水流漓。病必不除。若一服汗出病差。停後服。不必{\sungii 𥁞}剂。若不汗。更服依前法。又不汗。後服小促其間。半日許令三服{\sungii 𥁞}。若病重者。一日一夜服。周时觀之。服一剂{\sungii 𥁞}。病証猶在者。更作服。若汗不出。乃服至二三剂。禁生冷。粘滑。肉面。五辛。酒酪。臭惡等物。{\zhaoben}
	\footnote{
		唐弘宇按:桂枝湯煎服法,趙本、《玉函》、《千金翼》文義基本相同,但行文風格上,《玉函》、《千金翼》行文簡潔,宋本系統明顯比唐代的這兩个版本囉嗦,且能看出有被修改過的痕跡。比如本論其它方剂的服法都作「温服」,唯獨宋本桂枝湯服法作「適寒温服」,再比如「酪」這種食物在张仲景时代還未被发明,等等。但由於此段文字各版本差異太大,很難互校,故直接照錄趙本。
	}

太陽中風。{\khaai 脉}陽浮而陰弱。陽浮者熱自发。陰弱者汗自出。嗇嗇惡寒。淅淅惡風。翕翕发熱。鼻鳴。乾嘔。桂枝湯主之。12

太陽病。发熱。汗出。此为榮弱衛强。故使汗出。欲救邪風。宜桂枝湯。95

太陽病。頭痛。发熱。汗出。惡風。桂枝湯主之。13

太陽病。下之。其气上衝者。可与桂枝湯。不衝者。不可与之。15

太陽病三日。已发汗吐下温針而不觧。此为壞病。桂枝湯不復中与也。觀其脉証。知犯何逆。隨証治之。16

桂枝湯本为觧肌。若其人脉浮緊。发熱。无汗。不可与也。常須識此。勿令誤也。16

酒客不可与桂枝湯。得之則嘔。以酒客不喜甘故也。17

服桂枝湯吐者。其後必吐膿血。19

太陽病。初服桂枝湯。反煩不觧者。当先刺風池風府。卻与桂枝湯即愈。24

服桂枝湯。大汗出。脉洪大者。与桂枝湯。若形如瘧。一日再发者。汗出便觧。宜桂枝二麻黄一湯。25

太陽病。外証未觧。其脉浮弱。当以汗觧。宜桂枝湯。42

太陽病。外証未觧者。不可下。下之为逆。欲觧外者。宜桂枝湯。44

太陽病。先发汗不觧而下之。其脉浮者不愈。浮为在外。而反下之。故令不愈。今脉浮。故在外。当觧其外則愈。宜桂枝湯。45

病常自汗出者。此为榮气和。衛气不和也。榮行脉中。衛行脉外。復发其汗。衛和則愈。宜桂枝湯。53

病人臓无他病。时发熱。自汗出。而不愈者。此衛气不和也。先其时发汗則愈。宜桂枝湯。54

傷寒。不大便六七日。頭痛。有熱者。与承气湯。其小便清者。此为不在裏。續在表也。当发其汗。頭痛者必衄。宜桂枝湯。56

傷寒。发汗已觧。半日許復煩。脉浮數者。可復发汗。宜桂枝湯。57

傷寒。醫下之。續得下利。清穀不止。身体疼痛。急当救裏。後身体疼痛。清便自調。急当救表。救裏宜四逆湯。救表宜桂枝湯。91

傷寒。大下後。復发汗。心下痞。惡寒者。表未觧也。不可攻痞。当先觧表。表觧乃可攻痞。觧表宜桂枝湯。攻痞宜大黄黄連瀉心湯。164

陽明病。脉遲。雖汗出。不惡寒。其身必重。短气。腹滿而喘。有潮熱。如此者。其外为觧。可攻其裏。若手足濈然汗出者。此大便已堅。{\khaai 大}承气湯主之。若汗多。微发熱。惡寒者。为外未觧。{\khaai 桂枝湯主之。}其熱不潮。未可与承气湯。若腹大滿。而不大便者。可与小承气湯。微和其胃气。勿令至大下。208

陽明病。脉遲。汗出多。微惡寒者。表未觧也。可发汗。宜桂枝湯。234

病者煩熱。汗出即觧。復如瘧狀。日晡所发者。屬陽明。脉実者。当下之。脉浮虗者。当发其汗。下之宜{\khaai 大}承气湯。发汗宜桂枝湯。240

太陰病。脉浮者。可发汗。宜桂枝湯。276

下利。腹{\khaai 胀}滿。身体疼痛者。先温其裏。乃攻其表。温裏宜四逆湯。攻表宜桂枝湯。372

吐利止。而身痛不休者。当消息和觧其外。宜桂枝湯小和之。387

師曰。脉婦人得平脉。陰脉小弱。其人渴。不能食。无寒熱。名为軀。桂枝湯主之。法六十日当有娠。設有醫治逆者。卻一月。加吐下者。則絕之。

婦人產得風。續之數十日不觧。頭微痛。惡寒。时时有熱。心下堅。乾嘔。汗出。雖久。陽旦証續在耳。可与陽旦湯。

\section{桂枝加桂湯}

桂枝{\scriptsize 五兩} 芍藥{\scriptsize 三兩} 生薑{\scriptsize 三兩切} 甘草{\scriptsize 二兩炙} 大棗{\scriptsize 十二枚擘}\\
右五味。以水七升。煮取三升。去滓。温服一升。本云桂枝湯。今加桂滿五兩。所以加桂者,以能瀉奔豚气也。

燒針令其汗。針処被寒。核起而赤者。必发奔豚。气從少腹上衝心者。灸其核上各一壯。与桂枝加桂湯。117

\section{桂枝加芍藥湯}

桂枝{\scriptsize 三兩} 芍藥{\scriptsize 六兩} 甘草{\scriptsize 二兩炙} 大棗{\scriptsize 十二枚擘} 生薑{\scriptsize 三兩切}\\
右五味。以水七升。煮取三升。去滓。分温三服。本云桂枝湯。今加芍藥。

{\khaai 本}太陽病。醫反下之。因尔腹滿时痛者。屬太陰。桂枝加芍藥湯主之。大実痛者。桂枝加大黄湯主之。279

\section{桂枝去芍藥湯}

桂枝{\scriptsize 三兩} 甘草{\scriptsize 二兩炙} 生薑{\scriptsize 三兩切} 大棗{\scriptsize 十二枚擘}\\
右四味。以水七升。煮取三升。去滓。温服一升。本云桂枝湯。今去芍藥。

太陽病。下之。脉促。胸滿者。桂枝去芍藥湯主之。若微{\khaai 惡}寒者。桂枝去芍藥加附子湯主之。21.22

\section{桂枝加葛根湯}

葛根{\scriptsize 四兩} 芍藥{\scriptsize 二兩} 生薑{\scriptsize 三兩切} 甘草{\scriptsize 二兩炙} 大棗{\scriptsize 十二枚擘} 桂枝{\scriptsize 二兩去皮}\\
右七味。以水一斗。先煮麻黄。葛根。減二升。去上沫。内諸藥。煮取三升。去滓。温服一升。覆取微似汗。不需啜粥。餘如桂枝湯法將息及禁忌。{\scriptsize 臣億等謹按。仲景本論。太陽中風自汗用桂枝。傷寒无汗用麻黄。今証云汗出惡風。而方中有麻黄。恐非本意也。第三卷有葛根湯証云无汗惡風。正与此方同是合用麻黄也。此云桂枝加葛根湯。恐是桂枝中但加葛根耳。}

太陽病。項背强几几。反汗出。惡風。桂枝{\khaai 加葛根}湯主之。14

\section{栝蔞桂枝湯}

栝蔞根{\scriptsize 二兩} 桂枝{\scriptsize 三兩去皮} 芍藥{\scriptsize 三兩} 甘草{\scriptsize 二兩炙} 生薑{\scriptsize 三兩切} 大棗{\scriptsize 十二枚擘}\\
右六味。㕮咀。以水九升。煮取三升。去滓。分温三服。取微汗。汗不出。食頃。啜熱粥发之。

太陽病。其証備。身体强。几几然。脉反沈遲。此为痙。栝蔞桂枝湯主之。

\section{桂枝加黄耆湯}

桂枝{\scriptsize 三兩去皮} 生薑{\scriptsize 三兩} 芍藥{\scriptsize 三兩} 甘草{\scriptsize 二兩炙} 大棗{\scriptsize 十二枚擘} 黄耆{\scriptsize 二兩}\\
右六味。㕮咀。以水八升。煮取三升。去滓。温服一升。須臾。飲熱稀粥一升餘。以助藥力。温覆取微汗。若不汗者更服。

黄汗之病。兩脛自冷。假令发熱。此屬歷節。食已汗出。又身常暮{\khaai 卧}盜汗出者。此勞气也。若汗出已。反发熱者。久久其身必甲錯。发熱不止者。必生惡瘡。若身重。汗出已輒輕者。久久必身瞤瞤。即胸中痛。又從腰以上必汗出。下无汗。腰髖弛痛。如有物在皮中狀。劇者不能食。身疼重。煩躁。小便不利。此为黄汗。桂枝加黄耆湯主之。

諸病黄家。但利其小便。假令脉浮。当以汗觧之。宜桂枝加黄耆湯主之。

\section{桂枝加大黄湯}

桂枝{\scriptsize 三兩去皮} 大黄{\scriptsize 二兩} 芍藥{\scriptsize 六兩} 生薑{\scriptsize 三兩切} 甘草{\scriptsize 二兩炙} 大棗{\scriptsize 十二枚擘}\\
右六味。以水七升。煮取三升。去滓。温服一升。日三服。

{\khaai 本}太陽病。醫反下之。因尔腹滿时痛者。屬太陰。桂枝加芍藥湯主之。大実痛者。桂枝加大黄湯主之。279

\section{桂枝加芍藥生薑人参湯}

桂枝{\scriptsize 三兩去皮} 芍藥{\scriptsize 四兩} 甘草{\scriptsize 二兩炙} 人参{\scriptsize 三兩} 大棗{\scriptsize 十二枚擘} 生薑{\scriptsize 四兩切}\\
右六味。以水一斗二升。煮取三升。去滓。温服一升。本云桂枝湯。今加芍藥。生薑。人参。

发汗後。身体疼痛。其脉沈遲。桂枝加芍藥生薑人参湯主之。62

\section{桂枝加厚朴杏仁湯}

喘家作桂枝湯。加厚朴杏仁佳。18

太陽病。下之。微喘者。表未觧也。桂枝{\khaai 加厚朴杏仁}湯主之。43

\section{烏頭桂枝湯}

烏頭{\scriptsize 五枚実者去角}\\
右一味。以蜜二斤。煎減半。去滓。以桂枝湯五合觧之。令得一升許。初服二合。不知。即服三合。又不知。復更加至五合。其知者如醉狀。得吐者为中病。

寒疝。腹中痛。逆冷。手足不仁。若身疼痛。灸刺諸藥不能治。烏頭桂枝湯主之。

\section{桂枝加附子湯}

桂枝{\scriptsize 三兩} 芍藥{\scriptsize 三兩} 甘草{\scriptsize 三兩炙} 生薑{\scriptsize 三兩切} 大棗{\scriptsize 十二枚擘} 附子{\scriptsize 一枚炮去皮破八片}\\
右六味。以水七升。煮取三升。去滓。温服一升。本云桂枝湯。今加附子。

太陽病。发汗。遂漏不止。其人惡風。小便難。四肢微急。難以屈伸。桂枝加附子湯主之。20

\section{桂枝去芍藥加附子湯}

桂枝{\scriptsize 三兩} 甘草{\scriptsize 二兩炙} 生薑{\scriptsize 三兩切} 大棗{\scriptsize 十二枚擘} 附子{\scriptsize 一枚炮去皮破八片}\\
右五味。以水七升。煮取三升。去滓。温服一升。本云桂枝湯。今去芍藥。加附子。

太陽病。下之。脉促。胸滿者。桂枝去芍藥湯主之。若微{\khaai 惡}寒者。桂枝去芍藥加附子湯主之。21.22

\section{桂枝附子湯}

桂枝{\scriptsize 四兩去皮} 附子{\scriptsize 三枚炮去皮破} 生薑{\scriptsize 三兩切} 大棗{\scriptsize 十二枚擘} 甘草{\scriptsize 二兩炙}\\
右五味。以水六升。煮取二升。去滓。分温三服。

傷寒八九日。風濕相摶。身体疼煩。不能自轉側。不嘔。不渴。脉浮虗而濇者。桂枝附子湯主之。若其人大便堅。小便自利者。术附子湯主之。174

\section{术附子湯}

附子{\scriptsize 三枚炮去皮破} 白术{\scriptsize 四兩} 生薑{\scriptsize 三兩切} 甘草{\scriptsize 二兩炙} 大棗{\scriptsize 十二枚擘}\\
右五味。以水六升。煮取二升。去滓。分温三服。\\
初一服。其人身如痹。半日許。復服之。三服都{\sungii 𥁞}。其人如冒狀。勿怪。此以附子术并走皮内。逐水气。未得除。故使之耳。法当加桂四兩。此本一方二法。以大便堅。小便自利。故去桂也。以大便不堅。小便不利。当加桂。附子三枚恐多也。虗弱家及產婦宜減服之。

傷寒八九日。風濕相摶。身体疼煩。不能自轉側。不嘔。不渴。脉浮虗而濇者。桂枝附子湯主之。若其人大便堅。小便自利者。术附子湯主之。174

治風虗。頭重眩。苦極。不知食味。暖肌補中。益精气。术附子湯。

\section{甘草附子湯}

甘草{\scriptsize 二兩炙} 附子{\scriptsize 二枚炮} 白术{\scriptsize 三兩} 桂枝{\scriptsize 四兩}\\ 
右四味。以水六升。煮取三升。去滓。温服一升。日三服。初服得微汗即止。能食。汗止復煩者。將服五合。恐一升多者。後服六七合愈。

風濕相摶。骨節疼煩。掣痛。不得屈伸。近之則痛劇。汗出。短气。小便不利。惡風。不欲去衣。或身微腫。甘草附子湯主之。175

\section{桂枝去桂加茯苓术湯}

芍藥{\scriptsize 三兩} 甘草{\scriptsize 二兩炙} 生薑{\scriptsize 三兩切} 白术{\scriptsize 三兩} 茯苓{\scriptsize 三兩} 大棗{\scriptsize 十二枚}\\
右六味。以水八升。煮取三升。去滓。温服一升。小便利則愈。本云桂枝湯。今去桂枝。加茯苓。白术。

服桂枝湯。{\khaai 或}下之。仍頭項强痛。翕翕发熱。无汗。心下滿。微痛。小便不利。桂枝去桂加茯苓术湯主之。28

\section{桂枝去芍藥加麻黄細辛附子湯}

桂枝{\scriptsize 三兩去皮} 生薑{\scriptsize 三兩切} 甘草{\scriptsize 二兩炙} 大棗{\scriptsize 十二枚擘} 麻黄{\scriptsize 二兩去節} 細辛{\scriptsize 二兩} 附子{\scriptsize 一枚炮去皮破八片}\\
右七味。㕮咀。以水七升。先煮麻黄再沸。去上沫。内諸藥。煮取二升。去滓。分温三服。当汗出。如虫行皮中即愈。

气分。心下堅。大如盤。邊如旋杯。水飲所作。桂枝去芍藥加麻黄細辛附子湯主之。

\section{桂枝去芍藥加皂莢湯}

桂枝{\scriptsize 三兩去皮} 生薑{\scriptsize 三兩切} 甘草{\scriptsize 二兩炙} 大棗{\scriptsize 十二枚擘} 皂莢{\scriptsize 一枚去皮子炙焦}\\
右五味。㕮咀。以水七升。微微火煮取三升。去滓。分温三服。

肺痿。吐涎沫。桂枝去芍藥加皂莢湯主之。

\section{桂枝加龙骨牡蛎湯}

桂枝{\scriptsize 三兩去皮} 芍藥{\scriptsize 三兩} 生薑{\scriptsize 三兩切} 甘草{\scriptsize 二兩炙} 大棗{\scriptsize 十二枚擘} 龙骨{\scriptsize 二兩熬} 牡蛎{\scriptsize 二兩熬}\\
右七味。㕮咀。以水七升。煮取三升。去滓。分温三服。

夫失精家。少腹弦急。陰頭寒。目眩。髮落。脉極虗芤遲。为清穀。亡血。失精。脉得諸芤動微緊。男子失精。女子夢交。桂枝加龙骨牡蛎湯主之。天雄散亦主之。

\section{桂枝去芍藥加蜀漆牡蛎龙骨救逆湯}

桂枝{\scriptsize 三兩去皮} 甘草{\scriptsize 二兩炙} 生薑{\scriptsize 三兩切} 大棗{\scriptsize 十二枚擘} 牡蛎{\scriptsize 五兩熬} 蜀漆{\scriptsize 三兩洗去腥} 龙骨{\scriptsize 四兩}\\
右七味。以水一斗二升{\footnote{吳本作「八升」。}}。先煮蜀漆。減二升。内諸藥。煮取三升。去滓。温服一升。本云桂枝湯。今去芍藥。加蜀漆牡蛎龙骨。

傷寒。脉浮。醫以火迫劫之。亡陽。{\khaai 必}驚狂。卧起不安。桂枝去芍藥加蜀漆牡蛎龙骨救逆湯主之。112

火邪者。桂枝去芍藥加蜀漆牡蛎龙骨救逆湯主之。

\section{桂枝甘草龙骨牡蛎湯}

桂枝{\scriptsize 一兩去皮} 甘草{\scriptsize 二兩炙} 牡蛎{\scriptsize 二兩熬} 龙骨{\scriptsize 二兩}\\
右四味。以水五升。煮取二升半。去滓。温服八合。日三服。

火逆。下之。因燒針。煩躁者。桂枝甘草龙骨牡蛎湯主之。118

\section{桂枝二麻黄一湯}

桂枝{\scriptsize 一兩十七銖去皮} 芍藥{\scriptsize 一兩六銖} 麻黄{\scriptsize 十六銖去節} 生薑{\scriptsize 一兩六銖切} 杏人{\scriptsize 十六个去皮尖} 甘草{\scriptsize 一兩二銖炙} 大棗{\scriptsize 五枚擘}\\
右七味。以水五升。先煮麻黄一二沸。去上沫。内諸藥。煮取二升。去滓。温服一升。日再服。本云桂枝湯二分。麻黄湯一分。合为二升。分再服。今合为一方。將息如前法。{\scriptsize 臣億等謹按。桂枝湯方。桂枝芍藥生薑各三兩。甘草二兩。大棗十二枚。麻黄湯方。麻黄三兩。桂枝二兩。甘草一兩。杏仁七十个。今以算法約之。桂枝湯取十二分之五。即得桂枝芍藥生薑各一兩六銖。甘草二十銖。大棗五枚。麻黄湯取九分之二。即得麻黄十六銖。桂枝十株三分銖之二。收之得十一銖。甘草五銖三分銖之一。收之得六銖。杏仁十五个九分枚之四。收之得十六个。二湯所取相合。即共得桂枝一兩十七銖。麻黄十六銖。生薑芍藥各一兩六銖。甘草一兩二銖。大棗五枚。杏仁十六个。合方。}{\zhaoben}

服桂枝湯。大汗出。脉洪大者。与桂枝湯。若形如瘧。一日再发者。汗出便觧。宜桂枝二麻黄一湯。25

\hangindent 1em
\hangafter=0
凡大汗出復後。脉洪大。形如瘧。一日再发。汗出便觧。更与桂枝麻黄湯。{\yixin}

\section{桂枝二越婢一湯}

桂枝{\scriptsize 十八銖去皮} 芍藥{\scriptsize 十八銖} 麻黄{\scriptsize 十八銖} 甘草{\scriptsize 十八銖炙} 大棗{\scriptsize 四枚擘} 生薑{\scriptsize 一兩二銖切} 石膏{\scriptsize 二十四銖碎棉裹}\\
右七味。以水五升。煮麻黄一二沸。去上沫。内諸藥。煮取二升。去滓。温服一升。本云。当裁为越婢湯桂枝湯。合{\khaai 之}飲一升。今合为一方。桂枝湯二分。越婢湯一分。{\scriptsize 臣億等謹按。桂枝湯方。桂枝芍藥生薑各三兩。甘草二兩。大棗十二枚。越婢湯方。麻黄二兩。生薑三兩。甘草二兩。石膏八兩。大棗十五枚。今以算法約之。桂枝湯取四分之一。即得桂枝芍藥生薑各十八銖。甘草十二銖。大棗三枚。越婢湯取八分之一。即得麻黄十八銖。生薑九銖。甘草六銖。石膏二十四銖。大棗一枚八分之七。棄之。二湯所取相合。即共得桂枝芍藥甘草麻黄各十八銖。生薑一兩三銖。石膏二十四銖。大棗四枚。合方。舊云桂枝三。今取四分之一。即当云桂枝二也。越婢湯方。見仲景雜方中。外臺祕要一云起脾湯。}

太陽病。发熱。惡寒。熱多寒少。脉微弱者。此无陽也。不可{\khaai 復}发汗。{\khaai 宜桂枝二越婢一湯。}27

\section{桂枝麻黄各半湯}

桂枝{\scriptsize 一兩十六銖} 芍藥{\scriptsize 一兩} 生薑{\scriptsize 一兩切} 甘草{\scriptsize 一兩炙} 麻黄{\scriptsize 一兩去節} 大棗{\scriptsize 四枚擘} 杏仁{\scriptsize 二十四枚去皮尖兩仁者}\\
右七味。以水五升。先煮麻黄一二沸。去上沫。内諸藥。煮取一升八合。去滓。温服六合。本云桂枝湯三合。麻黄湯三合。并为六合。頓服。將息如上法。{\scriptsize 臣億等謹按。桂枝湯方。桂枝芍藥生薑各三兩。甘草二兩。大棗十二枚。麻黄湯方。麻黄三兩。桂枝二兩。甘草一兩。杏仁七十个。今以算法約之。二湯各取三分之一。即得桂枝一兩十六銖。芍藥生薑甘草各一兩。大棗四枚。杏仁二十三个零三分枚之一。收之得二十四个。合方。詳此方乃三分之一。非各半也。宜云合半湯。}

太陽病。得之八九日。如瘧狀。发熱。惡寒。熱多寒少。其人不嘔。清便續自可。一日再三发。脉微緩者。为欲愈也。脉微而惡寒者。此为陰陽俱虗。不可復{\khaai 吐下}发汗也。面反有熱色者。未欲觧也。以其不能得汗出。身必癢。宜桂枝麻黄各半湯。23

\section{小建中湯}

桂枝{\scriptsize 三兩去皮} 甘草{\scriptsize 二兩炙} 大棗{\scriptsize 十二枚擘} 芍藥{\scriptsize 六兩} 生薑{\scriptsize 三兩切} 膠飴{\scriptsize 一升}\\
右六味。{\khaai 㕮咀。}以水七升。{\khaai 先}煮{\khaai 五味。}取三升。去滓。内{\khaai 膠}飴。令消\footnote{「令消」同吳本,宋本作「更上微火消觧」。}。温服一升。日三服。{\scriptsize 嘔家不可用建中湯。以甜故也。千金療男女因積冷气滯。或大病後不復常。苦四肢沈重。骨肉痠疼。吸吸少气。行動喘乏。胸滿。气急。腰背强痛。心中虗悸。咽乾。唇燥。面体少色。或飲食无味。脇肋腹脹。頭重不舉。多卧少起。甚者積年。輕者百日。漸致瘦弱。五臟气竭。則難可復常。六脉俱不足。虗寒乏气。少腹拘急。羸脊百病。名曰黄耆建中湯。又有人参二兩。}

傷寒。陽脉濇。陰脉弦。法当腹中急痛。先与小建中湯。不差者。与小柴胡湯。100

傷寒二三日。心中悸而煩者。小建中湯主之。102

虗勞。裏急。悸。衄。腹中痛。夢失精。四肢痠疼。手足煩熱。咽乾口燥。小建中湯主之。

男子黄。小便自利。当与虗勞小建中湯。

婦人腹中痛。小建中湯主之。

\section{黄耆建中湯}

黄耆{\scriptsize 三兩} 桂枝{\scriptsize 三兩去皮} 生薑{\scriptsize 三兩切} 芍藥{\scriptsize 六兩} 甘草{\scriptsize 二兩炙} 大棗{\scriptsize 十二枚擘} 膠飴{\scriptsize 一升}\\
右七味。㕮咀。以水七升。先煮六味。取三升。去滓。内膠飴。令消。温服一升。日三服。

虗勞。裏急。諸不足。黄耆建中湯主之。

\section{黄耆桂枝五物湯}

黄耆{\scriptsize 三兩} 芍藥{\scriptsize 三兩} 桂枝{\scriptsize 三兩去皮} 生薑{\scriptsize 六兩切} 大棗{\scriptsize 十二枚擘}\\
右五味。㕮咀。以水六升。煮取二升。去滓。温服七合。日三服。{\scriptsize 一方有人参。}

血痹。陰陽俱微。寸口関上微。尺中小緊。外証身体不仁。如風狀。黄耆桂枝五物湯主之。

\section{黄耆芍藥桂枝苦酒湯}

黄耆{\scriptsize 五兩} 芍藥{\scriptsize 二兩} 桂枝{\scriptsize 三兩去皮}\\
右三味。㕮咀。以苦酒一升。水七升相和。煮取三升。去滓。温服一升。当心煩。服至六七日乃觧。若心煩不止者。以苦酒阻故也。{\scriptsize 一方用美清醯代苦酒。}

問曰。黄汗之为病。身体腫。发熱。汗出而渴。狀如風水。汗沾衣。色正黄如檗汁。脉自沈。何從得之。\\
師曰。以汗出入水中浴。水從汗孔入得之。

黄汗。黄耆芍藥桂枝苦酒湯主之。

\section{桂枝甘草湯}

桂枝{\scriptsize 四兩去皮} 甘草{\scriptsize 二兩炙}\\
右二味。以水三升。煮取一升。去滓。頓服。

发汗過多。其人叉手自冒心。心下悸。欲得按者。桂枝甘草湯主之。64

\section{半夏散及湯}

半夏{\scriptsize 洗} 桂枝{\scriptsize 去皮} 甘草{\scriptsize 炙}\\
右三味。等分。各別擣篩已。合治之。白飲和服方寸匕。日三服。若不能散服者。以水一升。煎七沸。内散兩方寸匕。更煮三沸。下火。令小冷。少少咽之。半夏有毒。不当散服。

少陰病。咽中痛。半夏散及湯主之。313

\section{桂枝人参湯}

桂枝{\scriptsize 四兩別切} 甘草{\scriptsize 四兩炙} 白术{\scriptsize 三兩} 人参{\scriptsize 三兩} 乾薑{\scriptsize 三兩}\\
右五味。以水九升。先煮四味。取五升。内桂。更煮取三升。去滓。温服一升。日再夜一服。

太陽病。外証未除。而數下之。遂挾熱而利。利下不止。心下痞堅。表裏不觧。桂枝人参湯主之。163

\section{理中湯(人参湯)*}

人参{\scriptsize 三兩} 乾薑{\scriptsize 三兩} 甘草{\scriptsize 三兩炙} 白术{\scriptsize 三兩}\\
右四味。擣篩。蜜和为丸。如雞子黄許大。以沸湯數合。和一丸。研碎。温服之。日三夜二服。腹中未熱。益至三四丸。然不及湯。\\
湯法。以四物。依兩數切。用水八升。煮取三升。去滓。温服一升。日三服。

{\khaai 若}脐上築者。腎气動也。去术。加桂四兩。\\
{\khaai 若}吐多者。去术。加生薑三兩。\\
{\khaai 若}下{\khaai 利}多者。復用术。\\
{\khaai 若}悸者。加茯苓二兩。\\
{\khaai 若}渴{\khaai 欲得水}者。加术至四兩半。\\
{\khaai 若}腹中痛者。加人参至四兩半。\\
{\khaai 若}寒者。加乾薑至四兩半。\\
{\khaai 若}腹滿者。去术。加附子一枚。\\

傷寒。服湯藥。下利不止。心下痞堅。服瀉心湯已。復以他藥下之。利不止。醫以理中与之。利益甚。理中者。理中焦。此利在下焦。赤石脂禹餘糧湯主之。復不止者。当利小便。159

霍亂。頭痛。发熱。身疼痛。熱多。欲飲水者。五苓散主之。寒多。不用水者。理中湯主之。386

大病差後。其人喜唾。久不了了者。胃上有寒。当温之。宜理中丸。396

胸痹。心中痞。留气結在胸。胸滿。脇下逆搶心。枳実薤白桂枝湯主之。理中湯亦主之。

\section{茯苓杏仁甘草湯}

茯苓{\scriptsize 三兩} 杏仁{\scriptsize 五十个去皮尖} 甘草{\scriptsize 一兩炙}\\
右三味。㕮咀。以水一斗。煮取五升。去滓。温服一升。日三服。不差。更合服。

胸痹。胸中气塞。短气。茯苓杏仁甘草湯主之。橘皮枳実生薑湯亦主之。

\section{茯苓戎鹽湯}

茯苓{\scriptsize 八兩} 白术{\scriptsize 二兩} 戎鹽{\scriptsize 彈丸大一枚}\\
右三味。㕮咀。以水七升。煮取三升。去滓。分温三服。

小便不利。蒲灰散主之。滑石白魚散。茯苓戎鹽湯并主之。

\section{葵子茯苓散}

葵子{\scriptsize 一升} 茯苓{\scriptsize 三兩}\\
右二味。杵为散。飲服方寸匕。日三服。小便利則愈。

婦人妊娠。有水气。身重。小便不利。洒淅惡寒。起即頭眩。葵子茯苓散主之。

\section{甘草乾薑茯苓白术湯}

甘草{\scriptsize 二兩炙} 乾薑{\scriptsize 四兩} 茯苓{\scriptsize 四兩} 白术{\scriptsize 二兩}\\
右四味。㕮咀。以水五升。煮取三升。去滓。分温三服。腰中即温。

腎著之病。其人身体重。腰中冷。如坐水中。形如水狀。反不渴。小便自利。食飲如故。病屬下焦。身勞汗出。衣裏冷濕。久久得之。腰以下冷痛。腹重如帶五千錢。甘草乾薑茯苓白术湯主之。

\section{苓桂术甘湯}

茯苓{\scriptsize 四兩} 桂枝{\scriptsize 三兩去皮} 白术{\scriptsize 二兩
	\footnote{
		「二兩」同趙本,《金匱要略》吳本、鄧本均作「三兩」。
	}
} 甘草{\scriptsize 二兩炙}\\
右四味。以水六升。煮取三升。去滓。分温三服。

傷寒吐下发汗後。心下逆滿。气上衝胸。起則頭眩。其脉沈緊。发汗則動經。身为振搖。苓桂术甘湯主之。67

心下有痰飲。胸脇支滿。目眩。苓桂术甘湯主之。

夫短气。有微飲。当從小便去之。苓桂术甘湯主之。腎气丸亦主之。

\section{苓桂甘棗湯}

茯苓{\scriptsize 八兩} 桂枝{\scriptsize 四兩去皮} 甘草{\scriptsize 二兩炙} 大棗{\scriptsize 十五枚擘}\\
右四味。以甘爤水一斗。先煮茯苓。減二升。内諸藥。煮取三升。去滓。温服一升。日三服。作甘爤水法。取水二斗。置大盆内。以杓揚之。水上有珠子五六千顆相逐。取用之。

发汗後。其人脐下悸。欲作奔豚。苓桂甘棗湯主之。65

\section{茯苓桂枝五味子甘草湯}

茯苓{\scriptsize 四兩} 桂枝{\scriptsize 四兩去皮} 五味子{\scriptsize 八兩碎} 甘草{\scriptsize 三兩炙}\\
右四味。㕮咀。以水八升。煮取三升。去滓。分温三服。

青龙湯下已。多唾。口燥。寸脉沈。尺脉微。手足厥逆。气從少腹上衝胸咽。手足痹。其人面翕然如醉。因復下流陰股。小便難。时復冒。可与茯苓桂枝五味子甘草湯。治其气衝。

\section{茯苓桂枝五味子甘草湯去桂加乾薑細辛}

茯苓{\scriptsize 四兩} 五味子{\scriptsize 半升碎} 甘草{\scriptsize 一兩炙} 乾薑{\scriptsize 一兩} 細辛{\scriptsize 一兩}\\
右五味。㕮咀。以水八升。煮取三升。去滓。分温三服。{\wuben}

茯苓{\scriptsize 四兩} 甘草{\scriptsize 三兩} 乾薑{\scriptsize 三兩} 細辛{\scriptsize 三兩} 五味子{\scriptsize 半升}\\
右五味。以水八升。煮取三升。去滓。温服半升。日三。{\dengben}

衝气即低。而反更欬滿者。因茯苓五味子甘草。去桂加乾薑細辛。以治其欬滿。{\wuben}

衝气即低。而反更欬。胸滿者。用桂苓五味甘草湯。去桂加乾薑細辛。以治其欬滿。{\dengben}

\section{}

欬滿則止。而復更渴。衝气復发者。以細辛乾薑为熱藥也。此法不当逐渴。而渴反止者。为支飲也。支飲法当冒。冒者必嘔。嘔者復内半夏。以去其水。

\section{}

水去。嘔則止。其人形腫。可内麻黄。以其欲逐痹。故不内麻黄。乃内杏仁也。若逆而内麻黄者。其人必厥。所以然者。以其人血虗。麻黄发其陽故也。

\section{}

若面熱如醉狀者。此为胃中熱。上熏其面令熱。加大黄湯和之。

\section{澤瀉湯}

澤瀉{\scriptsize 五兩} 白术{\scriptsize 二兩}\\
右二味。㕮咀。以水二升。煮取一升。去滓。分温再服。

心下有支飲。其人苦冒眩。澤瀉湯主之。

\section{茯苓澤瀉湯}

茯苓{\scriptsize 八兩} 澤瀉{\scriptsize 四兩} 甘草{\scriptsize 二兩炙} 桂枝{\scriptsize 二兩去皮} 白术{\scriptsize 三兩} 生薑{\scriptsize 四兩切}\\
右六味。㕮咀。以水一斗。煮取三升。内澤瀉。再煮。取二升半。去滓。温服八合。日三服。

胃反。吐而渴欲飲水者。茯苓澤瀉湯主之。

%外臺載有不同條文,日後補充。

\section{茯苓甘草湯}

茯苓{\scriptsize 二兩} 桂枝{\scriptsize 二兩去皮} 甘草{\scriptsize 一兩炙} 生薑{\scriptsize 三兩切}\\
右四味。以水四升。煮取二升。去滓。分温三服。

傷寒。汗出而渴者。五苓散主之。不渴者。茯苓甘草湯主之。73

傷寒。厥而心下悸。宜先治水。当与茯苓甘草湯。卻治其厥。不尔。水漬入胃。必作利也。356

\section{五苓散}

豬苓{\scriptsize 十八銖去皮} 澤瀉{\scriptsize 一兩六銖} 白术{\scriptsize 十八銖} 茯苓{\scriptsize 十八銖} 桂枝{\scriptsize 半兩去皮}\\
右五味。擣为散。以白飲和服方寸匕。日三服。多飲暖水。汗出愈。如法將息。

太陽病发汗後。大汗出。胃中乾。煩躁不得眠。其人欲飲水。当稍飲之。令胃气和則愈。若脉浮。小便不利。微熱。消渴者。五苓散主之。71

发汗已。脉浮數。煩渴者。五苓散主之。72

傷寒。汗出而渴者。五苓散主之。不渴者。茯苓甘草湯主之。73

中風。发熱。六七日不觧而煩。有表裏証。渴欲飲水。水入則吐。此为水逆。五苓散主之。74

病在陽。当以汗觧。反以水潠之若灌之。其熱被劫不得去。益煩。皮上粟起。意欲飲水。反不渴。宜服文蛤散。若不差。与五苓散。141

本以下之。故心下痞。与瀉心湯。痞不觧。其人渴而口燥{\khaai 煩}。小便不利。五苓散主之。156

太陽病。寸{\khaai 口}緩。関{\khaai 上小}浮。尺{\khaai 中}弱。其人发熱。汗出。復惡寒。不嘔。但心下痞者。此为醫下之故也。若不下。其人不惡寒而渴者。此轉屬陽明。小便數者。大便必堅。不更衣十日。无所苦也。{\khaai 渴}欲飲水者。少少与之。但以法救之。渴者。宜五苓散。244

霍亂。頭痛。发熱。身疼痛。熱多。欲飲水者。五苓散主之。寒多。不用水者。理中湯主之。386

假令瘦人脐下悸。吐涎沫而癲眩。{\khaai 此}水也。五苓散主之。

脉浮。小便不利。微熱。消渴者。宜利小便。发汗。五苓散主之。

\section{茵陳五苓散}

茵陳蒿末{\scriptsize 五分} 五苓散{\scriptsize 五分}\\
右二物。和。先食飲服方寸匕。日三服。

黄疸病。茵陳五苓散主之。

\section{豬苓湯}

豬苓{\scriptsize\khaai 一兩去皮}{ }茯苓{\scriptsize 一兩}{ }澤瀉{\scriptsize 一兩}{ }阿膠{\scriptsize 一兩}{ }滑石{\scriptsize 一兩碎}\\
右五味。以水四升。先煮四味。取二升。去滓。内阿膠烊消。温服七合。日三服。

若脉浮。发熱。渴欲飲水。小便不利者。豬苓湯主之。223

陽明病。汗出多而渴者。不可与豬苓湯。以汗多。胃中燥。豬苓湯復利其小便故也。224

少陰病。下利六七日。欬而嘔。渴。心煩不得眠。豬苓湯主之。319

\section{豬苓散}

豬苓{\scriptsize\khaai 去皮} 茯苓{ }白术{\scriptsize 各等分}\\
右三味。杵为散。飲服方寸匕。日三服。

嘔吐。而病在膈上。後思水者觧。急与之。思水者。豬苓散主之。

\section{牡蛎澤瀉散}

牡蛎{\scriptsize 熬} 澤瀉 蜀漆{\scriptsize 煖水洗去腥} 葶藶子{\scriptsize 熬} 商陸根{\scriptsize 熬} 海藻{\scriptsize 洗去鹹} 栝蔞根{\scriptsize 各等分}\\
右七味。異擣。下篩为散。更於臼中治之。白飲和服方寸匕。日三服。小便利。止後服。

大病差後。從腰以下有水气者。牡蛎澤瀉散主之。395

\section{腎气丸}

乾地黄{\scriptsize 八兩} 薯蕷{\scriptsize 四兩} 山茱萸{\scriptsize 四兩} 澤瀉{\scriptsize 三兩} 茯苓{\scriptsize 三兩} 牡丹皮{\scriptsize 三兩} 桂枝{\scriptsize 一兩} 附子{\scriptsize 一兩炮}\\
右八味。末之。煉蜜和丸梧子大。酒下十五丸。日再服。{\khaai 加至二十五丸。}

治腳气上入。少腹不仁。服八味丸。

虗勞。腰痛。少腹拘急。小便不利者。八味腎气丸主之。

夫短气。有微飲。当從小便去之。苓桂术甘湯主之。腎气丸亦主之。

男子消渴。小便反多。以飲一斗。小便一斗。腎气丸主之。

問曰。婦人病。飲食如故。煩熱不得卧。而反倚息者。何也。\\
師曰。此名轉胞。不得尿也。以胞系了戾。故致此病。但利小便則愈。宜腎气丸。

\section{栝蔞瞿麥丸}

栝蔞根{\scriptsize 二兩} 茯苓{\scriptsize 三兩} 薯蕷{\scriptsize 三兩} 附子{\scriptsize 大者一枚炮去皮} 瞿麥{\scriptsize 一兩}\\
右五味。末之。煉蜜和为丸梧子大。飲服三丸。日三服。不知。增至七八丸。以小便利。腹中温为知。

小便不利者。有水气。其人若渴。栝蔞瞿麥丸主之。

\section{麻黄湯}

麻黄{\scriptsize 三兩去節} 桂枝{\scriptsize 二兩去皮} 甘草{\scriptsize 一兩炙} 杏仁{\scriptsize 七十个去皮尖}\\
右四味。以水九升。先煮麻黄。減二升。去上沫。内諸藥。煮取二升半。去滓。温服八合。覆取微似汗。不須啜粥。餘如桂枝法將息。

太陽病。頭痛。发熱。身疼。腰痛。骨節疼痛。惡風。无汗而喘。麻黄湯主之。35

太陽与陽明合病。喘而胸滿者。不可下。宜麻黄湯。36

太陽病。十日已去。脉浮細而嗜卧者。外已觧也。設胸滿脇痛者。与小柴胡湯。脉{\khaai 但}浮者。与麻黄湯。37

太陽病。脉浮緊。无汗。发熱。身疼痛。八九日不觧。表証仍在。此当发其汗。服藥已。微除。其人发煩目暝。劇者必衄。衄乃觧。所以然者。陽气重故也。麻黄湯主之。46

脉浮者。病在表。可发汗。宜麻黄湯。51

{\khaai 太陽病。}脉浮數者。可发汗。宜麻黄湯。52

傷寒。脉浮緊。不发汗。因致衄者。宜麻黄湯。55

{\khaai 寸口}脉浮而緊。浮則为風。緊則为寒。風則傷衛。寒則傷榮。榮衛俱病。骨節煩疼。当发其汗。{\khaai 宜麻黄湯。}

陽明中風。脉弦浮大。而短气。腹都滿。脇下及心痛。久按之。气不通。鼻乾。不得汗。嗜卧。一身及目悉黄。小便難。有潮熱。时时噦。耳前後腫。刺之小差。外不觧。病過十日。脉續浮者。与{\khaai 小}柴胡湯。脉但浮。无餘証者。与麻黄湯。不尿。腹滿加噦者。不治。231.232

陽明病。脉浮。无汗而喘者。发汗則愈。宜麻黄湯。235

\section{麻黄加术湯}

麻黄{\scriptsize 三兩去節} 桂枝{\scriptsize 二兩去皮} 甘草{\scriptsize 一兩炙} 杏仁{\scriptsize 七十个去皮尖} 白术{\scriptsize 四兩}\\
右五味。㕮咀。以水九升。先煮麻黄一二沸。去上沫。内諸藥。煮取二升。去滓。温服八合。覆取微似汗。

濕家。身煩疼。可与麻黄加术湯。发其汗为宜。慎不可以火攻之。

\section{甘草麻黄湯}

甘草{\scriptsize 二兩炙} 麻黄{\scriptsize 四兩去節}\\
右二味。㕮咀。以水五升。先煮麻黄{\khaai 再沸}。去上沫。内甘草。煮取三升。去滓。温服一升。重覆汗出。不汗再服。慎風寒。

裏水。越婢加术湯主之。甘草麻黄湯亦主之。

\section{麻黄附子甘草湯(附子麻黄湯)}

麻黄{\scriptsize 二兩去節} 甘草{\scriptsize 二兩炙} 附子{\scriptsize 一枚炮去皮破八片}\\
右三味。以水七升。先煮麻黄一兩沸。去上沫。内諸藥。煮取三升。去滓。温服一升。日三服。{\zhaoben}

附子{\scriptsize 一枚炮去皮破八片} 麻黄{\scriptsize 二兩去節} 甘草{\scriptsize 二兩炙}\\
右三味。㕮咀。以水七升。先煮麻黄再沸。去上沫。内諸藥。煮取二升半。去滓。温服八分。日三服。{\wuben}

少陰病。得之二三日。麻黄附子甘草湯微发汗。以二三日无{\khaai 裏}証。故微发汗。302

水之为病。其脉沈小。屬少陰。浮者为風。无水。虗胀者为气。水。发其汗即已。脉沈者。宜附子麻黄湯。浮者。宜杏子湯。

\section{麻黄細辛附子湯}

麻黄{\scriptsize 二兩去節} 細辛{\scriptsize 二兩} 附子{\scriptsize 一枚炮去皮破八片}\\
右三味。以水一斗。先煮麻黄。減二升。去上沫。内諸藥。煮取三升。去滓。温服一升。日三服。

少陰病。始得之。反发熱。脉沈者。麻黄細辛附子湯主之。301

\section{麻杏甘石湯}

麻黄{\scriptsize 四兩去節} 杏仁{\scriptsize 五十个去皮尖} 甘草{\scriptsize 二兩炙} 石膏{\scriptsize 八兩碎綿裹}\\
右四味。以水七升。煮麻黄。減二升。去上沫。内諸藥。煮取二升。去滓。温服一升。

发汗後。不可更行桂枝湯。汗出而喘。无大熱者。可与麻杏甘石湯。63

下後。不可更行桂枝湯。汗出而喘。无大熱者。可与麻杏甘石湯。162

\section{麻杏薏甘湯}

麻黄{\scriptsize 二兩去節} 杏仁{\scriptsize 三十个去皮尖} 薏苡仁{\scriptsize 一兩} 甘草{\scriptsize 一兩炙}\\
右四味。㕮咀。以水四升。先煮麻黄一二沸。去上沫。内諸藥。煮取二升。去滓。分温再服。{\wuben}

病者一身{\sungii 𥁞}疼。发熱。日晡所劇者。此名風濕。此病傷於汗出当風。或久傷取冷所致也。可与麻杏薏甘湯。

\section{牡蛎湯}

牡蛎{\scriptsize 四兩熬} 麻黄{\scriptsize 四兩去節} 甘草{\scriptsize 二兩炙} 蜀漆{\scriptsize 三兩洗去腥}\\
右四味。㕮咀。以水八升。先煮蜀漆。麻黄。去上沫。得六升。内諸藥。煮取二升。去滓。温服一升。吐則勿更服。{\scriptsize 見外臺}

治牡瘧。牡蛎湯。

\section{麻黄淳酒湯}

麻黄{\scriptsize 三兩去節綿裹}\\
右一味。以美清酒五升。煮取二升半。去滓。頓服{\sungii 𥁞}。冬月用酒。春月用水煮之。

黄疸。麻黄淳酒湯主之。

\section{半夏麻黄丸}

半夏{\scriptsize 洗} 麻黄{\scriptsize 去節{ }等分}\\
右二味。末之。煉蜜和丸如小豆大。飲服三丸。日三服。

心下悸者。半夏麻黄丸主之。

\section{小青龙湯*}

麻黄{\scriptsize 三兩去節} 芍藥{\scriptsize 三兩} 細辛{\scriptsize 三兩} 乾薑{\scriptsize 三兩} 甘草{\scriptsize 三兩炙} 桂枝{\scriptsize 三兩去皮} 五味子{\scriptsize 半升} 半夏{\scriptsize 半升洗}\\
右八味。以水一斗。先煮麻黄。減二升。去上沫。内諸藥。煮取三升。去滓。温服一升。

渴者。去半夏。加栝蔞根三兩。\\
微利者。去麻黄。加蕘花。如一雞子。熬令赤色。\\
噎者。去麻黄。加附子一枚炮。\\
小便不利。少腹滿者。去麻黄。加茯苓四兩。\\
喘者。去麻黄。加杏仁半升。去皮尖。\\
蕘花不治利。麻黄主喘。今此語反之。疑非仲景意。{\scriptsize 臣億等謹按。小青龙湯。大要治水。又按本草。蕘花下十二水。若水去。利則止也。又按千金。形腫者應内麻黄。乃内杏仁者。以麻黄发其陽故也。以此証之。豈非仲景意也。}

傷寒。表不觧。心下有水气。乾嘔。发熱而欬。或渴。或利。或噎。或小便不利。少腹滿。或喘。小青龙湯主之。40

傷寒。心下有水气。欬而微喘。发熱。不渴。服湯已而渴者。此寒去。为欲觧。小青龙湯主之。41

病溢飲者。当发其汗。大青龙湯主之。小青龙湯亦主之。

欬逆倚息。小青龙湯主之。

婦人吐涎沫。醫反下之。心下即痞。当先治其吐涎沫。宜小青龙湯。涎沫止。乃治痞。宜瀉心湯。

\section{小青龙加石膏湯}

麻黄{\scriptsize 三兩去節} 芍藥{\scriptsize 三兩} 桂枝{\scriptsize 三兩} 細辛{\scriptsize 三兩} 甘草{\scriptsize 三兩炙} 乾薑{\scriptsize 三兩} 五味子{\scriptsize 半升} 半夏{\scriptsize 半升洗} 石膏{\scriptsize 二兩碎}\\
右九味。㕮咀。以水一斗。先煮麻黄。減二升。去上沫。内諸藥。取三升。去滓。强人服一升。羸者減之。日三服。小兒服四合。

肺胀。欬而上气。煩躁而喘。脉浮者。心下有水。小青龙加石膏湯主之。

欬而上气。肺胀。其脉浮。心下有水气。脇下痛引缺盆。小青龙加石膏湯主之。

\section{大青龙湯}

麻黄{\scriptsize 六兩去節} 桂枝{\scriptsize 二兩去皮} 甘草{\scriptsize 二兩炙} 杏仁{\scriptsize 四十枚去皮尖} 生薑{\scriptsize 三兩切} 大棗{\scriptsize 十枚擘} 石膏{\scriptsize 如雞子大碎}\\
右七味。以水九升。先煮麻黄。減二升。去上沫。内諸藥。煮取三升。去滓。温服一升。取微似汗。汗出多者。温粉粉之。一服汗者。停後服。若復服。汗多亡陽。遂虗。惡風。煩躁。不得眠也。

太陽中風。脉浮緊。发熱。惡寒。身疼痛。不汗出而煩躁者。大青龙湯主之。若脉微弱。汗出。惡風者。不可服之。服之則厥。筋愓肉瞤。此为逆也。38

傷寒。脉浮緩。身不疼。但重。乍有輕时。无少陰証者。大青龙湯发之。39

%病溢飲{\khaai 者}。当发其汗。宜大青龙湯。
病溢飲者。当发其汗。大青龙湯主之。小青龙湯亦主之。

\section{文蛤湯}

文蛤{\scriptsize 五兩} 麻黄{\scriptsize 三兩去節} 甘草{\scriptsize 三兩炙} 杏仁{\scriptsize 五十枚去皮尖} 石膏{\scriptsize 五兩碎} 大棗{\scriptsize 十二枚擘} 生薑{\scriptsize 三兩切}\\
右七味。㕮咀。以水六升。煮取二升。去滓。温服一升。汗出愈。

吐後。渴欲得飲而貪水者。文蛤湯主之。兼主微風。脉緊。頭痛。

\section{文蛤散}

文蛤{\scriptsize 五兩}\\
右一味。杵为散。以沸湯五合。和服方寸匕。{\dengben}

文蛤{\scriptsize 五兩}\\
右一味。为散。以沸湯和一方寸匕服湯。用五合。{\zhaoben}

病在陽。当以汗觧。反以水潠之若灌之。其熱被劫不得去。益煩。皮上粟起。意欲飲水。反不渴。宜服文蛤散。若不差。与五苓散。141

渴欲飲水不止者。文蛤散主之。

\section{越婢湯}

麻黄{\scriptsize 六兩去節} 石膏{\scriptsize 八兩碎} 生薑{\scriptsize 三兩切} 大棗{\scriptsize 十五枚擘} 甘草{\scriptsize 二兩炙}\\
右五味。㕮咀。以水六升。先煮麻黄{\khaai 再沸}。去上沫。内諸藥。煮取三升。去滓。分温三服。惡風者。加附子一枚炮。{\scriptsize 古今錄驗云。風水加术四兩。}

風水。惡風。一身悉腫。脉浮。不渴。續自汗出。无大熱。越婢湯主之。

\section{越婢加术湯}

麻黄{\scriptsize 六兩去節} 石膏{\scriptsize 八兩} 生薑{\scriptsize 三兩切} 甘草{\scriptsize 二兩炙} 大棗{\scriptsize 十五枚擘} 白术{\scriptsize 四兩}\\
右六味。㕮咀。以水六升。先煮麻黄再沸。去上沫。内諸藥。煮取三升。去滓。分温三服。惡風加附子一枚炮。

裏水者。一身面目自洪腫。其脉沈。小便不利。故令病水。假如小便自利。亡津液。故令渴也。
	\footnote{
		鄧本本條末有「越婢加术湯主之」七字,吳本无。
	}

裏水。越婢加术湯主之。甘草麻黄湯亦主之。

治肉極。熱則身体津{\khaai 液}脱。腠理開。汗大泄。厉風气。下焦腳弱。越婢加术湯。

\section{越婢加半夏湯}

麻黄{\scriptsize 六兩去節} 石膏{\scriptsize 八兩碎} 生薑{\scriptsize 三兩切} 大棗{\scriptsize 十五枚擘} 甘草{\scriptsize 二兩炙} 半夏{\scriptsize 半升洗}\\
右六味。㕮咀。以水六升。先煮麻黄再沸。去上沫。内諸藥。煮取三升。去滓。分温三服。

欬逆倚息。此为肺胀。其人喘。目如脱狀。脉浮大者。越婢加半夏湯主之。

\section{葛根湯}

葛根{\scriptsize 四兩} 麻黄{\scriptsize 三兩去節} 桂枝{\scriptsize 二兩去皮} 生薑{\scriptsize 三兩切} 甘草{\scriptsize 二兩炙} 芍藥{\scriptsize 二兩} 大棗{\scriptsize 十二枚擘}\\
右七味。以水一斗。先煮麻黄。葛根。减二升。去白沫。内諸藥。煮取三升。去滓。温服一升。覆取微似汗。餘如桂枝法將息及禁忌。諸湯皆倣此。

太陽病。項背强几几。无汗。惡風。葛根湯主之。31

太陽与陽明合病。而自利{\khaai 者}。葛根湯主之。不下利。但嘔者。葛根加半夏湯主之。32.33

太陽病。无汗。而小便反少。气上衝胸。口噤不得語。欲作剛痙。葛根湯主之。

\section{葛根加半夏湯}

葛根{\scriptsize 四兩} 麻黄{\scriptsize 三兩去節} 甘草{\scriptsize 二兩炙} 芍藥{\scriptsize 二兩} 桂枝{\scriptsize 二兩去皮} 生薑{\scriptsize 二兩切} 半夏{\scriptsize 半升洗} 大棗{\scriptsize 十二枚擘}\\
右八味。以水一斗。先煮葛根。麻黄。减二升。去白沫。内諸藥。煮取三升。去滓。温服一升。覆取微似汗。

太陽与陽明合病。而自利{\khaai 者}。葛根湯主之。不下利。但嘔者。葛根加半夏湯主之。32.33

\section{葛根芩連湯}

葛根{\scriptsize 八兩} 甘草{\scriptsize 二兩炙} 黄芩{\scriptsize 三兩} 黄連{\scriptsize 三兩}\\
右四味。以水八升。先煮葛根。减二升。内諸藥。煮取二升。去滓。分温再服。

太陽病。桂枝証。醫反下之。遂利不止。脉促者。表未觧也。喘而汗出者。葛根芩連湯主之。34

\section{小柴胡湯*}

柴胡{\scriptsize 八兩} 黄芩{\scriptsize 三兩} 人参{\scriptsize 三兩} 甘草{\scriptsize 三兩炙} 生薑{\scriptsize 三兩切} 大棗{\scriptsize 十二枚擘} 半夏{\scriptsize 半升洗}\\
右七味。以水一斗二升。煮取六升。去滓。再煎。取三升。温服一升。日三服。

若胸中煩而不嘔者。去半夏。人参。加栝蔞実一枚。\\
若渴者。去半夏。加人参合前成四兩半。栝蔞根四兩。\\
若腹中痛者。去黄芩,加芍藥三兩。\\
若脇下痞堅者。去大棗。加牡蛎六兩。\\
若心下悸。小便不利者。去黄芩。加茯苓四兩。\\
若不渴。外有微熱者。去人参。加桂三兩。温覆。微发其汗。\\
若欬者。去人参。大棗。生薑。加五味子半升。乾薑二兩。

太陽病。十日已去。脉浮細而嗜卧者。外已觧也。設胸滿脇痛者。与小柴胡湯。脉{\khaai 但}浮者。与麻黄湯。37

血弱气{\sungii 𥁞}。腠理開。邪气因入。与正气相摶。結於脇下。正邪分爭。往來寒熱。休作有时。默默不欲飲食。臓腑相連。其痛必下。邪高痛下。故使嘔也。小柴胡湯主之。服柴胡湯已。渴者。屬陽明。以法治之。97

傷寒五六日。中風。往來寒熱。胸脇苦滿。默默不欲飲食。心煩。喜嘔。或胸中煩而不嘔。或渴。或腹中痛。或脇下痞堅。或心下悸。小便不利。或不渴。外有微熱。或欬。小柴胡湯主之。96

得病六七日。脉遲浮弱。惡風寒。手足温。醫再三下之。不能食。其人脇下滿{\khaai 痛}。面目及身黄。頸項强。小便難。与柴胡湯後必下重。本渴。飲水而嘔。柴胡{\khaai 湯}不復中与也。食穀者噦。98

傷寒四五日。身熱。惡風。頸項强。脇下滿。手足温而渴。小柴胡湯主之。99

傷寒。陽脉濇。陰脉弦。法当腹中急痛。先与小建中湯。不差者。与小柴胡湯。100

傷寒中風。有柴胡証。但見一証便是。不必悉具。101

凡柴胡湯証而下之。柴胡証不罷者。復与柴胡湯。必蒸蒸而振。卻发熱汗出而觧。101

太陽病。過經十餘日。反再三下之。後四五日。柴胡証仍在。先与小柴胡湯。嘔不止。心下急。其人鬱鬱微煩者。为未觧。与大柴胡湯下之則愈。103

傷寒十三日不觧。胸脇滿而嘔。日晡所发潮熱{\khaai 。已}而微利。此本当柴胡湯下之。不得利。今反利者。知醫以丸藥下之。非其治也。潮熱者。実也。先宜服小柴胡湯以觧其外。後以柴胡加芒硝湯主之。104

婦人中風七八日。續得寒熱。发作有时。經水適斷。此为熱入血室。其血必結。故使如瘧狀。发作有时。小柴胡湯主之。144

傷寒五六日。頭汗出。微惡寒。手足冷。心下滿。口不欲食。大便堅。其脉細。此为陽微結。必有表。復有裏。沈亦为病在裏。汗出为陽微。假令純陰結。不得有外証。悉入在裏。此为半在外半在裏。脉雖沈緊。不得为少陰病。所以然者。陰不得有汗。今頭汗出。故知非少陰也。可与{\khaai 小}柴胡湯。設不了了者。得屎而觧。148

傷寒五六日。嘔而发熱。柴胡湯証具。而以他藥下之。柴胡証仍在者。復与柴胡湯。此雖已下之。不为逆。必蒸蒸而振。卻发熱汗出而觧。若心下滿而堅痛者。此为結胸。宜大陷胸湯。若但滿而不痛者。此为痞。柴胡{\khaai 湯}不復中与也。宜半夏瀉心湯。149

陽明病。发潮熱。大便溏。小便自可。胸脇滿不去。小柴胡湯主之。229

陽明病。脇下堅滿。不大便而嘔。舌上{\khaai 白}胎者。可与小柴胡湯。上焦得通。津液得下。胃气因和。身濈然汗出而觧。230

陽明中風。脉弦浮大。而短气。腹都滿。脇下及心痛。久按之。气不通。鼻乾。不得汗。嗜卧。一身及目悉黄。小便難。有潮熱。时时噦。耳前後腫。刺之小差。外不觧。病過十日。脉續浮者。与{\khaai 小}柴胡湯。脉但浮。无餘証者。与麻黄湯。不尿。腹滿加噦者。不治。231.232

太陽病不觧。轉入少陽。脇下堅滿。乾嘔。不能食。往來寒熱。尚未吐下。其脉沈緊。可与小柴胡湯。266

嘔而发熱者。小柴胡湯主之。379

傷寒差已後。更发熱者。小柴胡湯主之。脉浮者。以汗觧之。脉沈実者。以下觧之。394

諸黄。腹痛而嘔者。宜柴胡湯。

產婦鬱{\khaai 冒}。其脉微弱。不能食。大便反堅。但頭汗出。所以然者。血虗而厥。厥而必冒。冒家欲觧。必大汗出。以血虗下厥。孤陽上出。故但頭汗出。所以產婦喜汗出者。亡陰血虗。陽气獨盛。故当汗出。陰陽乃復。所以便堅者。嘔。不能食也。小柴胡湯主之。病觧。能食。七八日。更发熱者。此为胃熱气実。大承气湯主之。

婦人多在草蓐得風。四肢苦煩熱。皆自发露所为。頭痛者。与小柴胡湯。頭不痛。但煩者。与三物黄芩湯。

\section{柴胡加芒硝湯}

柴胡{\scriptsize 二兩十六銖} 黄芩{\scriptsize 一兩} 人参{\scriptsize 一兩} 甘草{\scriptsize 一兩炙} 生薑{\scriptsize 一兩切} 半夏{\scriptsize 二十銖本云五枚洗} 大棗{\scriptsize 四枚擘} 芒硝{\scriptsize 二兩}\\
右八味。以水四升。煮取二升。去滓。内芒硝。更煮微沸。分温再服。不觧更作。{\scriptsize 臣億等謹按。金匱玉函方中无芒硝。別一方云。以水七升。下芒硝二合。大黄四兩。桑螵蛸五枚。煮取一升半。服五合。微下即愈。本云。柴胡再服。以觧其外。餘二升。加芒硝大黄桑螵蛸也。}

傷寒十三日不觧。胸脇滿而嘔。日晡所发潮熱{\khaai 。已}而微利。此本当柴胡湯下之。不得利。今反利者。知醫以丸藥下之。非其治也。潮熱者。実也。先宜服小柴胡湯以觧其外。後以柴胡加芒硝湯主之。104

\section{柴胡加大黄芒硝桑螵蛸湯}

右前七味。以水四升。煮取二升。去滓。下芒硝大黄桑螵蛸。煮取一升半。去滓。温服五合。微下即愈。本方柴胡湯。再服以觧其外。餘一服。加芒硝大黄桑螵蛸。{\yuhan}

右以前七味。以水七升。下芒硝三合。大黄四分。桑螵蛸五枚。煮取一升半。去滓。温服五合。微下即愈。本云柴胡湯。再服以觧其外。餘二升。加芒硝大黄桑螵蛸也。{\yifang}

\section{小柴胡去半夏加栝蔞湯}

柴胡{\scriptsize 八兩} 人参{\scriptsize 三兩} 黄芩{\scriptsize 三兩} 甘草{\scriptsize 三兩炙} 栝蔞根{\scriptsize 四兩} 生薑{\scriptsize 二兩切} 大棗{\scriptsize 十二枚擘}\\
右七味。㕮咀。以水一斗二升。煮取六升。去滓。再煎取三升。温服一升。日三。{\scriptsize 見外臺經心錄治勞瘧}

瘧病发渴者。与小柴胡去半夏加栝蔞湯。

\section{柴胡桂枝湯}

桂枝{\scriptsize 一兩半去皮} 黄芩{\scriptsize 一兩半} 人参{\scriptsize 一兩半} 甘草{\scriptsize 一兩炙} 半夏{\scriptsize 二合半洗} 芍藥{\scriptsize 一兩半} 大棗{\scriptsize 六枚擘} 生薑{\scriptsize 一兩半切} 柴胡{\scriptsize 四兩}\\
右九味。以水七升。煮取三升。去滓。温服一升。本云人参湯。作如桂枝法。加半夏柴胡黄芩。復如柴胡法。今用人参作半剂。

傷寒六七日。发熱。微惡寒。肢節煩疼。微嘔。心下支結。外証未去者。柴胡桂枝湯主之。146

发汗多。亡陽。狂語者。不可下。与柴胡桂枝湯。和其榮衛。以通津液。後自愈。

寒疝。腹中痛者。柴胡桂枝湯主之。

\section{柴胡桂枝乾薑湯}

柴胡{\scriptsize 八兩} 桂枝{\scriptsize 三兩去皮} 乾薑{\scriptsize 二兩} 栝蔞根{\scriptsize 四兩} 黄芩{\scriptsize 三兩} 牡蛎{\scriptsize 二兩熬} 甘草{\scriptsize 二兩炙}\\
右七味。㕮咀。以水一斗二升。煮取六升。去滓。再煎。取三升。温服一升。日三服。初服微煩。{\khaai 復服。}汗出{\khaai 便}愈。

傷寒五六日。已发汗而復下之。胸脇滿。微結。小便不利。渴而不嘔。但頭汗出。往來寒熱。心煩。此为未觧。柴胡桂枝乾薑湯主之。147

柴胡桂薑湯。{\scriptsize 此方治寒多微有熱。或但寒不熱。服一剂如神。故錄之。}

\section{柴胡加龙骨牡蛎湯}

柴胡{\scriptsize 四兩} 龙骨{\scriptsize 一兩半} 黄芩{\scriptsize 一兩半} 生薑{\scriptsize 一兩半切} 鈆丹{\scriptsize 一兩半} 人参{\scriptsize 一兩半} 桂枝{\scriptsize 一兩半去皮} 茯苓{\scriptsize 一兩半} 半夏{\scriptsize 二合洗} 大黄{\scriptsize 二兩} 牡蛎{\scriptsize 一兩半熬} 大棗{\scriptsize 六枚擘}\\
右十二味。以水八升。煮取四升。内大黄。切如碁子。更煮一兩沸。去滓。温服一升。本云柴胡湯。今加龙骨等。

傷寒八九日。下之。胸滿。煩。驚。小便不利。譫語。一身{\khaaiii 𥁞}{\khaai 重。}不可轉側。柴胡加龙骨牡蛎湯主之。107

\section{大柴胡湯}

柴胡{\scriptsize 八兩} 黄芩{\scriptsize 三兩} 芍藥{\scriptsize 三兩} 半夏{\scriptsize 半升洗} 生薑{\scriptsize 五兩切} 枳実{\scriptsize 四枚炙} 大棗{\scriptsize 十二枚擘} {\khaai 大黄{\scriptsize 二兩}}\\
右七味。以水一斗二升。煮取六升。去滓。再煎。{\khaai 取三升。}温服一升。日三服。一方加大黄二兩。若不加恐不名大柴胡也。

太陽病。過經十餘日。反再三下之。後四五日。柴胡証仍在。先与小柴胡湯。嘔不止。心下急。其人鬱鬱微煩者。为未觧。与大柴胡湯下之則愈。103

傷寒十餘日。熱結在裏。復往來寒熱者。与大柴胡湯。但結胸。无大熱者。此为水結在胸脇。{\khaai 但}頭微汗出。大陷胸湯主之。136

傷寒。发熱。汗出不觧。心中痞堅。嘔吐。下利。大柴胡湯主之。165

病腹中滿痛者。此为実也。当下之。宜大柴胡湯。{\wuben}

\section{白虎湯}

知母{\scriptsize 六兩} 石膏{\scriptsize 十六兩碎綿裹} 甘草{\scriptsize 二兩炙} 粳米{\scriptsize 六合}\\
右四味。以水一斗。煮米熟。湯成。去滓。温服一升。日三服。{\scriptsize 臣億等謹按。前篇云。熱結在裏。表裏俱熱者。白虎湯主之。又云。其表不觧。不可与白虎湯。此云脉浮滑。表有熱。裏有寒者。必表裏字差矣。又陽明一証云。脉浮遲。表熱裏寒。四逆湯主之。又少陰一証云。裏寒外熱。通脉四逆湯主之。以此表裏自差明矣。千金翼云白通湯。非也。}

傷寒。脉浮滑。此以表有熱。裏有寒。白虎湯主之。176

三陽合病。腹滿。身重。難以轉側。口不仁。面垢。譫語。遺尿。发汗則譫語{\khaai 甚}。下之則額上生汗。手足厥冷。自汗。白虎湯主之。219

傷寒。脉滑而厥者。裏有熱也。白虎湯主之。350

\section{白虎加人参湯}

知母{\scriptsize 六兩} 石膏{\scriptsize 十六兩
	\footnote{
	「十六兩」鄧本作「一斤」,編者改为「十六兩」,吳本作「一升」。
	}
碎綿裹} 甘草{\scriptsize 二兩炙} 粳米{\scriptsize 六合} 人参{\scriptsize 三兩}\\
右五味。以水一斗。煮米熟。湯成。去滓。温服一升。日三服。


服桂枝湯。大汗出{\khaai 後}。大煩渴不觧。脉洪大者。白虎{\khaai 加人参}湯主之。26

傷寒若吐若下後。七八日不觧。熱結在裏。表裏俱熱。时时惡風。大渴。舌上乾燥而煩。欲飲水數升。白虎{\khaai 加人参}湯主之。168

傷寒。无大熱。口燥渴。心煩。背微惡寒。白虎{\khaai 加人参}湯主之。169

傷寒。脉浮。发熱。无汗。其表不觧。不可与白虎湯。渴欲飲水。无表証者。白虎{\khaai 加人参}湯主之。170

若渴欲飲水。口乾舌燥者。白虎{\khaai 加人参}湯主之。222

太陽中熱者。暍是也。其人汗出。惡寒。身熱而渴。白虎{\khaai 加人参}湯主之。

\section{白虎加桂枝湯}

知母{\scriptsize 六兩} 甘草{\scriptsize 二兩炙} 石膏{\scriptsize 十六兩碎綿裹} 粳米{\scriptsize 六合} 桂枝{\scriptsize 三兩去皮}\\
右五味。㕮咀以水一斗二升。煮米熟。去滓。煎取三升。温服一升。日三服。汗出愈。

温瘧者。其脉如平。身无寒。但熱。骨節疼煩。时嘔。白虎加桂枝湯主之。

\section{小承气湯}

大黄{\scriptsize 四兩} 厚朴{\scriptsize 二兩炙去皮} 枳実{\scriptsize 三枚大者炙}\\
右三味。以水四升。煮取一升二合。去滓。分温二服。初服湯。当更衣。不爾者。{\sungii 𥁞}飲之。若更衣者。勿服之。

傷寒。不大便六七日。頭痛。有熱者。与承气湯。其小便清者。此为不在裏。續在表也。当发其汗。頭痛者必衄。宜桂枝湯。56

陽明病。脉遲。雖汗出。不惡寒。其身必重。短气。腹滿而喘。有潮熱。如此者。其外为觧。可攻其裏。若手足濈然汗出者。此大便已堅。{\khaai 大}承气湯主之。若汗多。微发熱。惡寒者。为外未觧。{\khaai 桂枝湯主之。}其熱不潮。未可与承气湯。若腹大滿。而不大便者。可与小承气湯。微和其胃气。勿令至大下。208

陽明病。潮熱。大便微堅者。可与{\khaai 大}承气湯。不堅者。不可与之。若不大便六七日。恐有燥屎。欲知之法。可少与小承气湯。若腹中轉失气者。此有燥屎也。乃可攻之。若不轉失气者。此但頭堅後溏。不可攻之。攻之必腹滿。不能食也。欲飲水者。与水即噦。其後发熱者。必大便復堅而少也。以小承气湯和之。若不轉失气者。慎不可攻之。209

陽明病。其人多汗。津液外出。胃中燥。大便必堅。堅則譫語。{\khaai 小}承气湯主之。{\khaai 若一服譫語止。莫復服。}213

陽明病。譫語。发潮熱。脉滑疾者。{\khaai 小}承气湯主之。因与承气湯一升。腹中轉失气者。復与一升。若不轉失气者。勿更与之。明日又不大便。脉反微濇者。此为裏虗。为難治。不可復与承气湯。214

太陽病吐下发汗後。微煩。小便數。大便因堅。可与小承气湯。和之則愈。250

得病二三日。脉弱。无太陽柴胡証。煩躁。心下堅。至四日。雖能食。以{\khaai 小}承气湯少与。微和之。令小安。至六日。与承气湯一升。若不大便六七日。小便少者。雖不大便。但頭堅後溏。未定成堅。攻之必溏。当須小便利。屎定堅。乃可攻之。宜{\khaai 大}承气湯。251

下利。譫語者。有燥屎也。宜{\khaai 小}承气湯。374

小承气湯。治大便不通。噦。數譫語。

\section{厚朴三物湯}

厚朴{\scriptsize 八兩炙} 大黄{\scriptsize 四兩} 枳実{\scriptsize 五枚炙}\\
右藥㕮咀。以水一斗二升。先煮二味。取五升。内大黄。煮取三升。去滓。温服一升。腹中轉動更服。不動勿服。

腹滿。脉數。厚朴三物湯主之。

\section{厚朴七物湯}

厚朴{\scriptsize 八兩炙} 甘草{\scriptsize 三兩炙} 大黄{\scriptsize 三兩} 大棗{\scriptsize 十枚擘} 枳実{\scriptsize 五枚炙} 桂枝{\scriptsize 二兩去皮} 生薑{\scriptsize 五兩切}\\
右藥㕮咀。以水一斗。煮取四升。去滓。温服八合。日三服。嘔者加半夏五合。下利者去大黄。寒冷多者加生薑至八兩。

病腹滿。发熱十日。脉浮而數。飲食如故。厚朴七物湯主之。

\section{大承气湯}

大黄{\scriptsize 四兩酒洗} 厚朴{\scriptsize 八兩炙去皮} 枳実{\scriptsize 五枚炙} 芒硝{\scriptsize 三合}\\
右四味。以水一斗。先煮二物。取五升。去滓。内大黄。更煮取二升。去滓。内芒硝。更上微火一兩沸。分温再服。得下。餘勿服。

陽明病。脉遲。雖汗出。不惡寒。其身必重。短气。腹滿而喘。有潮熱。如此者。其外为觧。可攻其裏。若手足濈然汗出者。此大便已堅。{\khaai 大}承气湯主之。若汗多。微发熱。惡寒者。为外未觧。{\khaai 桂枝湯主之。}其熱不潮。未可与承气湯。若腹大滿。而不大便者。可与小承气湯。微和其胃气。勿令至大下。208

陽明病。潮熱。大便微堅者。可与{\khaai 大}承气湯。不堅者。不可与之。若不大便六七日。恐有燥屎。欲知之法。可少与小承气湯。若腹中轉失气者。此有燥屎也。乃可攻之。若不轉失气者。此但頭堅後溏。不可攻之。攻之必腹滿。不能食也。欲飲水者。与水即噦。其後发熱者。必大便復堅而少也。以小承气湯和之。若不轉失气者。慎不可攻之。209

傷寒。吐下後未觧。不大便五六日。至十餘日。其人日晡所发潮熱。不惡寒。獨語。如見鬼{\khaai 神之}狀。若劇者。发則不識人。順衣妄撮。怵惕不安。微喘。直視。脉弦者生。濇者死。{\khaai 若}微者。但发熱。譫語。{\khaai 大}承气湯主之。若一服利。止後服。212

陽明病。譫語。有潮熱。反不能食者。{\khaai 胃中}必有燥屎五六枚。若能食者。但堅耳。{\khaai 大}承气湯主之。215

汗出。譫語者。以有燥屎在胃中。此風也。{\khaai 須下者。}過經乃可下之。下之若早。語言必亂。以表虗裏実故也。下之則愈。宜{\khaai 大}承气湯。217

二陽并病。太陽証罷。但发潮熱。手足漐漐汗出。大便難。而譫語者。下之則愈。宜{\khaai 大}承气湯。220

陽明病。下之。心中懊憹而煩。胃中有燥屎者。可攻。其人腹微滿。頭堅後溏者。不可攻之。若有燥屎者。宜{\khaai 大}承气湯。238

病者煩熱。汗出即觧。復如瘧狀。日晡所发者。屬陽明。脉実者。当下之。脉浮虗者。当发其汗。下之宜{\khaai 大}承气湯。发汗宜桂枝湯。240

大下後。六七日不大便。煩不觧。腹滿痛者。此有燥屎。所以然者。本有宿食故也。宜{\khaai 大}承气湯。241

病者小便不利。大便乍難乍易。时有微熱。怫㥜不能卧者。有燥屎故也。宜{\khaai 大}承气湯。242

得病二三日。脉弱。无太陽柴胡証。煩躁。心下堅。至四日。雖能食。以{\khaai 小}承气湯少与。微和之。令小安。至六日。与承气湯一升。若不大便六七日。小便少者。雖不大便。但頭堅後溏。未定成堅。攻之必溏。当須小便利。屎定堅。乃可攻之。宜{\khaai 大}承气湯。251

傷寒六七日。目中不了了。睛不和。无表{\khaai 裏}証。大便難。身微熱者。此为実也。急下之。宜{\khaai 大}承气湯。252

陽明病。发熱。汗多者。急下之。宜{\khaai 大}承气湯。253

发汗不觧。腹滿痛者。急下之。宜{\khaai 大}承气湯。254

腹滿不減。減不足言。当下之。宜{\khaai 大}承气湯。255

脉滑而數者。有宿食也。当下之。宜{\khaai 大}承气湯。256

少陰病。得之二三日。口燥。咽乾者。急下之。宜{\khaai 大}承气湯。320

少陰病。{\khaai 下}利清水。色青者。心下必痛。口乾燥者。急下之。宜{\khaai 大}承气湯。321

少陰病六七日。腹滿。不大便者。急下之。宜{\khaai 大}承气湯。322

下利。三部脉皆平。按之心下堅者。急下之。宜{\khaai 大}承气湯。

下利。脉遲而滑者。{\khaai 内}実也。利未欲止。当下之。宜{\khaai 大}承气湯。

問曰。人病有宿食。何以別之。\\
師曰。寸口脉浮大。按之反濇。尺中亦微而濇。故知有宿食。当下之。宜{\khaai 大}承气湯。

下利。不欲食者。有宿食也。当下之。宜{\khaai 大}承气湯。

下利{\khaai 已}差。至其时復发者。此为病不{\sungii 𥁞}。当復下之。宜{\khaai 大}承气湯。

下利。脉反滑{\khaai 者}。当有所去。下乃愈。宜大承气湯。

病腹中滿痛者。为実。当下之。宜大承气湯。

脉雙弦而遲。心下堅。脉大而緊者。陽中有陰也。可下之。宜{\khaai 大}承气湯。

剛痙为病。胸滿。口噤。卧不著席。腳攣急。其人必齘齒。可与大承气湯。

產婦鬱{\khaai 冒}。其脉微弱。不能食。大便反堅。但頭汗出。所以然者。血虗而厥。厥而必冒。冒家欲觧。必大汗出。以血虗下厥。孤陽上出。故但頭汗出。所以產婦喜汗出者。亡陰血虗。陽气獨盛。故当汗出。陰陽乃復。所以便堅者。嘔。不能食也。小柴胡湯主之。病觧。能食。七八日。更发熱者。此为胃熱气実。大承气湯主之。

婦人產後七八日。无太陽証。少腹堅痛。此惡露不{\sungii 𥁞}。不大便四五日。趺陽脉微実。再倍其人发熱。日晡所煩躁者。不食。食即譫語。利之即愈。宜大承气湯。熱在裏。結在膀胱也。

\section{大黄黄連瀉心湯
	\footnote{
		唐弘宇按:「大黄黄連瀉心湯」与「瀉心湯」,可能是一方的不同名称,也可能是兩个不同方剂。我採取保守処理,即把他們看作兩个方剂:大黄黄連瀉心湯,无黄芩,麻沸湯漬服,瀉熱力量較弱,主治心下熱痞証;瀉心湯,有黄芩,水煎服,瀉熱力量較强,主治血熱妄行吐衄。第156條、第164條、「婦人吐涎沫」條,雖云「瀉心湯」,但其主治是「心下痞」,可知此三條中的「瀉心湯」是大黄黄連瀉心湯的簡称,故將其歸在大黄黄連瀉心湯下。
	}
}

大黄{\scriptsize 二兩} 黄連{\scriptsize 一兩}\\
右二味。以麻沸湯漬之。須臾。絞去滓。分温再服。{\scriptsize 臣億等看詳。大黄黄連瀉心湯。諸本皆二味。又後附子瀉心湯用大黄黄連黄芩附子。恐是前方中亦有黄芩。後但加附子也。故後云附子瀉心湯。本云加附子是也。}

心下痞。按之濡。其脉関上浮者。大黄{\khaai 黄連}瀉心湯主之。154

本以下之。故心下痞。与瀉心湯。痞不觧。其人渴而口燥{\khaai 煩}。小便不利。五苓散主之。156

傷寒。大下後。復发汗。心下痞。惡寒者。表未觧也。不可攻痞。当先觧表。表觧乃可攻痞。觧表宜桂枝湯。攻痞宜大黄黄連瀉心湯。164

婦人吐涎沫。醫反下之。心下即痞。当先治其吐涎沫。宜小青龙湯。涎沫止。乃治痞。宜瀉心湯。

\section{瀉心湯}

大黄{\scriptsize 二兩} 黄連{\scriptsize 一兩} 黄芩{\scriptsize 一兩}\\
右三味。㕮咀。以水三升。煮取一升。頓服。亦治霍亂。{\scriptsize 傷寒論以麻沸湯漬服之。見千金。}

治心气不足。吐血。衄血。瀉心湯。

\section{附子瀉心湯}

大黄{\scriptsize 二兩} 黄連{\scriptsize 一兩} 黄芩{\scriptsize 一兩} 附子{\scriptsize 一枚炮去皮破別煮取汁}\\
右四味。切三味。以麻沸湯二升漬之。須臾。絞去滓。内附子汁。分温再服。

心下痞。而復惡寒。汗出者。附子瀉心湯主之。155

\section{大黄附子湯}

大黄{\scriptsize 三兩} 附子{\scriptsize 三枚炮去皮破} 細辛{\scriptsize 二兩}\\
右三味。㕮咀。以水五升。煮取二升。去滓。分温三服。若强人煮取二升半。分三服。服後如人行四五里。進一服。

脇下偏痛。发熱。其脉緊弦。此寒也。以温藥下之。宜大黄附子湯。

\section{大黄甘草湯}

大黄{\scriptsize 四兩} 甘草{\scriptsize 一兩炙}\\
右二味。㕮咀。以水三升。煮取一升。去滓。分温再服。

食已即吐者。大黄甘草湯主之。

\section{調胃承气湯}

大黄{\scriptsize 四兩去皮清酒洗} 甘草{\scriptsize 二兩炙} 芒硝{\scriptsize 半升}\\
右三味。以水三升。煮取一升。去滓。内芒硝。更上火微煮令沸。少少温服之。

傷寒。脉浮。自汗出。小便數。心煩。微惡寒。腳攣急。反与桂枝湯。欲攻其表。得之便厥。咽乾。煩躁。吐{\sungii 𠱘}者。当作甘草乾薑湯。以復其陽。若厥愈。足温者。更作芍藥甘草湯与之。其腳即伸。若胃气不和。譫語者。少与{\khaai 調胃}承气湯。若重发汗。復加燒針者。四逆湯主之。29

发汗不觧。反惡寒者。虗故也。芍藥甘草附子湯主之。不惡寒。但熱者。実也。当和胃气。宜調胃承气湯。68.70

太陽病未觧。脉陰陽俱微。必先振汗出而觧。但陽{\khaai 脉}微者。先汗之而觧。但陰{\khaai 脉}微者。先下之而觧。汗之宜桂枝湯。下之宜{\khaai 調胃}承气湯。94

傷寒十三日。過經。譫語者。内有熱也。当以湯下之。小便利者。大便当堅。而反{\khaai 下}利。脉調和者。知醫以丸藥下之。非其治也。自利者。脉当微厥。今反和者。此为内実也。{\khaai 調胃}承气湯主之。105

太陽病。過經十餘日。心下温温欲吐。而胸中痛。大便反溏。腹微滿。鬱鬱微煩。先{\khaai 此}时自極吐下者。与{\khaai 調胃}承气湯。若不尔者。不可与。但欲嘔。胸中痛。微溏者。此非柴胡湯証。以嘔。故知極吐下也。123

陽明病。不吐不下。心煩者。可与{\khaai 調胃}承气湯。207

太陽病三日。发汗不觧。蒸蒸发熱者。{\khaai 屬胃也。調胃}承气湯主之。248

傷寒吐後。腹胀滿者。与{\khaai 調胃}承气湯。249

\section{橘皮大黄朴消湯}

橘皮{\scriptsize 一兩} 朴硝{\scriptsize 一两} 大黄{\scriptsize 二兩}\\
右三味。切。以水一大升。煮至小升。去滓。頓服即消。

%本方位於飲食禁忌一章,條文未收錄。

\section{桃仁承气湯}

桃仁{\scriptsize 五十枚去皮尖} 大黄{\scriptsize 四兩} 桂枝{\scriptsize 二兩去皮} 甘草{\scriptsize 二兩炙} 芒硝{\scriptsize 二兩}\\
右五味。以水七升。煮取二升半。去滓。内芒硝。更上火。微沸下火。先食温服五合。日三服。当微利。

太陽病不觧。熱結膀胱。其人如狂。血自下。下之即愈。其外不觧者。尚未可攻。当先觧其外。{\khaai 宜桂枝湯。}外觧已。{\khaai 但}少腹急結者。乃可攻之。宜桃仁承气湯。106

\section{大黄牡丹湯}

大黄{\scriptsize 四兩} 牡丹{\scriptsize 一兩} 桃仁{\scriptsize 五十枚去皮尖} 瓜子{\scriptsize 半升} 芒硝{\scriptsize 三合}\\
右五味。㕮咀。以水六升。煮取一升。去滓。内芒硝。再煎一沸。頓服之。有膿当下。如无。当下血。

腸癰者。少腹腫痞。按之即痛如淋。小便自調。时时发熱。自汗出。復惡寒。脉遲緊者。膿未成。可下之。当有血。脉洪數者。膿已成。不可下也。大黄牡丹湯主之。

\section{大黄甘遂湯}

大黄{\scriptsize 四兩} 甘遂{\scriptsize 二兩} 阿膠{\scriptsize 二兩}\\
右三味。㕮咀。以水三升。煮取一升。去滓。頓服。其血当下。

婦人少腹滿如敦狀。小便微難而不渴。生後者。此为水与血并結在血室也。大黄甘遂湯主之。

\section{下瘀血湯}

大黄{\scriptsize 二兩} 桃仁{\scriptsize 三十枚去皮} 䗪虫{\scriptsize 二十枚熬去足}\\
右三味。末之。煉蜜和为四丸。以酒一升。煎一丸。取八合。頓服之。新血利下如豚肝。

師曰。產婦腹痛。法当与枳実芍藥散。假令不愈者。此为腹中有乾血著脐下。与下瘀血湯服之。{\khaai 亦}主經水不利若瘀血。

\section{抵当湯}

水蛭{\scriptsize 三十个熬} 蝱虫{\scriptsize 三十个去翅足熬} 桃仁{\scriptsize 二十个去皮尖} 大黄{\scriptsize 三兩酒洗}\\
右四味。以水五升。煮取三升。去滓。温服一升。不下更服。{\zhaoben}

水蛭{\scriptsize 三十枚熬} 蝱虫{\scriptsize 三十枚去足翅熬} 桃仁{\scriptsize 二七枚去皮尖熬} 大黄{\scriptsize 三兩}\\
右四味。㕮咀。以水五升。煮取三升。去滓。温服一升。当血下。不下再服。\\
亦治男子膀胱滿急。有瘀血者。{\wuben}

水蛭{\scriptsize 三十个熬} 蝱虫{\scriptsize 三十枚熬去翅足} 桃仁{\scriptsize 廿个去皮尖} 大黄{\scriptsize 三兩酒浸}\\
右四味。为末。以水五升。煮取三升。去滓。温服一升。{\dengben}

太陽病六七日。表証仍在。脉微而沈。反不結胸。其人发狂。此熱在下焦。少腹当堅滿。小便自利者。下血乃愈。所以然者。以太陽隨經。瘀熱在裏故也。抵当湯主之。124

太陽病。身黄。脉沈結。少腹堅。小便不利者。为无血也。小便自利。其人如狂者。血証諦也。抵当湯主之。125

陽明証。其人喜忘者。必有畜血。所以然者。本有久瘀血。故令喜忘。屎雖堅。大便反易。其色必黑。抵当湯主之。237

病人无表裏証。发熱七八日。雖脉浮數。可下之。{\khaai 宜大柴胡湯。}假令下已。脉數不觧。合熱則消穀善飢。至六七日。不大便者。有瘀血。宜抵当湯。若脉數不觧。而下不止。必挾熱。便膿血。257.258

婦人經水不利。抵当湯主之。

\section{抵当丸}

水蛭{\scriptsize 二十个熬} 蝱虫{\scriptsize 二十个去翅足熬} 桃仁{\scriptsize 二十五个去皮尖} 大黄{\scriptsize 三兩}\\
右四味。擣。分四丸。以水一升。煮一丸。取七合服之。晬时当下血。若不下者更服。

傷寒。有熱。少腹滿。應小便不利。今反利者。为有血也。当下之。宜抵当丸。126

\section{土瓜根散}

土瓜根{\scriptsize 三分} 芍藥{\scriptsize 三分} 桂枝{\scriptsize 三分去皮} 䗪虫{\scriptsize 三分熬}\\
右四味。杵为散。酒服方寸匕。日三服。{\khaai 陰顛腫亦主之。}

婦人帶下。經水不利。少腹滿痛。經一月再見者。土瓜根散主之。

\section{甘草湯}

甘草{\scriptsize 二兩}\\
右一味。以水三升。煮取一升半。去滓。温服七合。日再服。

少陰病二三日。咽痛者。可与甘草湯。不差者。与桔梗湯。311

\section{桔梗湯}

桔梗{\scriptsize 一兩} 甘草{\scriptsize 二兩}\\
右二味。以水三升。煮取一升。去滓。分温再服。

桔梗{\scriptsize 一兩} 甘草{\scriptsize 二兩炙}\\
右二味。㕮咀。以水三升。煮取一升。去滓。分温再服。則吐膿也。{\scriptsize 亦治㗋痹。}{\wuben}

少陰病二三日。咽痛者。可与甘草湯。不差者。与桔梗湯。311

欬而胸滿。振寒。脉數。咽乾。不渴。时出濁唾腥臭。久久吐膿如米粥者。为肺癰。桔梗湯主之。

\section{排膿湯}

甘草{\scriptsize 二兩炙} 桔梗{\scriptsize 三兩} 生薑{\scriptsize 一兩切} 大棗{\scriptsize 十枚擘}\\
右四味。㕮咀。以水三升。煮取一升。去滓。温服五合。日再服。

\section{芍藥甘草湯}

芍藥{\scriptsize 四兩} 甘草{\scriptsize 四兩炙}\\
右二味。以水三升。煮取一升五合。去滓。分温再服。

傷寒。脉浮。自汗出。小便數。心煩。微惡寒。腳攣急。反与桂枝湯。欲攻其表。得之便厥。咽乾。煩躁。吐{\sungii 𠱘}者。当作甘草乾薑湯。以復其陽。若厥愈。足温者。更作芍藥甘草湯与之。其腳即伸。若胃气不和。譫語者。少与{\khaai 調胃}承气湯。若重发汗。復加燒針者。四逆湯主之。29

\section{芍藥甘草附子湯}

芍藥{\scriptsize 三兩} 甘草{\scriptsize 三兩炙} 附子{\scriptsize 一枚炮去皮破八片}\\
右三味。以水五升。煮取一升五合。去滓。分温三服。疑非仲景方。

发汗不觧。反惡寒者。虗故也。芍藥甘草附子湯主之。不惡寒。但熱者。実也。当和胃气。宜調胃承气湯。68.70

\section{甘遂半夏湯}

甘遂{\scriptsize 大者三枚} 半夏{\scriptsize 十二枚洗以水一升煮取半升去滓} 芍藥{\scriptsize 五枚} 甘草{\scriptsize 如指大一枚炙一本无}\\
右四味。㕮咀。以水二升。煮取半升。去滓。以蜜半升。和藥汁。煮取八合。頓服之。

病者脉伏。其人欲自利。利反快。雖利。心下續堅滿。此为留飲欲去故也。甘遂半夏湯主之。

\section{甘草小麥大棗湯}

甘草{\scriptsize 三兩炙} 小麥{\scriptsize 一升} 大棗{\scriptsize 十枚擘}\\
右三味。㕮咀。以水六升。煮取三升。去滓。分温三服。亦補脾气。

婦人臓躁。喜悲傷。欲哭。象如神靈所作。數欠伸。甘草小麥大棗湯主之。

\section{甘草粉蜜湯}

甘草{\scriptsize 二兩炙} 粉{\scriptsize 一兩} 蜜{\scriptsize 四兩}\\
右三味。㕮咀。以水三升。先煮甘草。取二升。去滓。内粉。蜜。攪令和。煎如薄粥。温服一升。差即止。

蛔虫之为病。令人吐涎。心痛。发作有时。毒藥不止。甘草粉蜜湯主之。

\section{生薑甘草湯}

生薑{\scriptsize 五兩切} 人参{\scriptsize 二兩} 甘草{\scriptsize 四兩炙} 大棗{\scriptsize 十五枚擘}\\
右四味。㕮咀。以水七升。煮取三升。去滓。分温三服。

肺痿。欬唾涎沫不止。咽燥而渴。生薑甘草湯主之。

\section{甘草乾薑湯}

甘草{\scriptsize 四兩炙} 乾薑{\scriptsize 二兩}\\
右二味。以水三升。煮取一升五合。去滓。分温再服。

甘草{\scriptsize 四兩炙} 乾薑{\scriptsize 二兩}\\
右二味。㕮咀。以水四升。煮取一升半。去滓。分温再服。服湯已。小温覆之。若渴者。屬消渴。{\wuben}

傷寒。脉浮。自汗出。小便數。心煩。微惡寒。腳攣急。反与桂枝湯。欲攻其表。得之便厥。咽乾。煩躁。吐{\sungii 𠱘}者。当作甘草乾薑湯。以復其陽。若厥愈。足温者。更作芍藥甘草湯与之。其腳即伸。若胃气不和。譫語者。少与{\khaai 調胃}承气湯。若重发汗。復加燒針者。四逆湯主之。29

肺痿。吐涎沫。而不能欬者。其人不渴。必遺尿。小便數。所以然者。以上虗不能制下故也。此为肺中冷。必眩。甘草乾薑湯以温其病。

\section{乾薑附子湯}

乾薑{\scriptsize 一兩} 附子{\scriptsize 一枚生用去皮切八片}\\
右二味。以水三升。煮取一升。去滓。頓服。

下之後。復发汗。晝日煩躁不得眠。夜而安靜。不嘔。不渴。无表証。脉沈微。身无大熱。乾薑附子湯主之。61

\section{四逆湯}

甘草{\scriptsize 二兩炙} 乾薑{\scriptsize 一兩半} 附子{\scriptsize 一枚生用去皮破八片}\\
右三味。{\khaai 㕮咀。}以水三升。煮取一升二合。去滓。分温再服。强人可大附子一枚。乾薑三兩。

傷寒。脉浮。自汗出。小便數。心煩。微惡寒。腳攣急。反与桂枝湯。欲攻其表。得之便厥。咽乾。煩躁。吐{\sungii 𠱘}者。当作甘草乾薑湯。以復其陽。若厥愈。足温者。更作芍藥甘草湯与之。其腳即伸。若胃气不和。譫語者。少与{\khaai 調胃}承气湯。若重发汗。復加燒針者。四逆湯主之。29

傷寒。醫下之。續得下利。清穀不止。身体疼痛。急当救裏。後身体疼痛。清便自調。急当救表。救裏宜四逆湯。救表宜桂枝湯。91

病发熱。頭痛。脉反沈。若不差。身体疼痛。当救其裏。宜四逆湯。92

{\khaai 陽明病。}脉浮而遲。表熱裏寒。下利清穀者。四逆湯主之。225

自利。不渴者。屬太陰。以其臓有寒故也。当温之。宜四逆輩。277

少陰病。脉沈者。急温之。宜四逆湯。323

少陰病。其人飲食入則吐。心中温温欲吐。復不能吐。始得之。手足寒。脉弦遲。此胸中実。不可下也。当吐之。若膈上有寒飲。乾嘔者。不可吐。当温之。宜四逆湯。324

大汗出。熱不去。内拘急。四肢疼。{\khaai 又}下利。厥逆而惡寒。四逆湯主之。353

大汗{\khaai 出}若大下利。而厥冷者。四逆湯主之。354

下利。腹{\khaai 胀}滿。身体疼痛者。先温其裏。乃攻其表。温裏宜四逆湯。攻表宜桂枝湯。372

嘔而脉弱。小便復利。身有微熱。見厥者。難治。四逆湯主之。377

吐利。汗出。发熱。惡寒。四肢拘急。手足厥冷。四逆湯主之。388

既吐且利。小便復利。而大汗出。下利清穀。裏寒外熱。脉微欲絕。四逆湯主之。388

\section{通脉四逆湯*}

甘草{\scriptsize 二兩炙} 附子{\scriptsize 大者一枚生用去皮破八片} 乾薑{\scriptsize 三兩强人可四兩}\\
右三味。以水三升。煮取一升二合。去滓。分温再服。其脉即出者愈。

{\khaai 若}面赤者。加蔥{\khaai 白}九莖。\\
{\khaai 若}腹痛者。加芍藥二兩。\\
{\khaai 若}嘔者。加生薑二兩。\\
{\khaai 若}咽痛者。加桔梗一兩。\\
{\khaai 若}利止而脉不出者。加人参二兩。\\
{\khaai 病皆与方相應者乃服之。}
	\footnote{
		「加芍藥」《玉函》同,其餘諸本均作「去蔥加芍藥」。「加桔梗」前諸本均有「去芍藥」三字,編者刪。「加人参」前諸本均有「去桔梗」三字,編者刪。「病皆」至「服之」十字,《玉函》、成本无。
	}

少陰病。下利清穀。裏寒外熱。手足厥逆。脉微欲絕。身反不惡寒。其人面赤。或腹痛。或乾嘔。或咽痛。或利止而脉不出。通脉四逆湯主之。317

下利清穀。裏寒外熱。汗出而厥。通脉四逆湯主之。370

\section{四逆加人参湯}

甘草{\scriptsize 二兩炙} 附子{\scriptsize 一枚生去皮破八片} 乾薑{\scriptsize 一兩半} 人参{\scriptsize 一兩}\\
右四味。以水三升。煮取一升二合。去滓。分温再服。

惡寒。脉微。而復利。利止。亡血也。四逆加人参湯主之。385

\section{茯苓四逆湯}

茯苓{\scriptsize 四兩} 人参{\scriptsize 一兩} 附子{\scriptsize 一枚生用去皮破八片} 甘草{\scriptsize 二兩炙} 乾薑{\scriptsize 一兩半}\\
右五味。以水五升。煮取三升。去滓。温服七合。日二服。

发汗若下之。{\khaai 病仍}不觧。煩躁。茯苓四逆湯主之。69

\section{通脉四逆加豬膽汁湯}

甘草{\scriptsize 二兩炙} 乾薑{\scriptsize 三兩强人可四兩} 附子{\scriptsize 大者一枚生去皮破八片} 豬膽汁{\scriptsize 半合}\\
右四味。以水三升。煮取一升二合。去滓。内豬膽汁。分温再服。其脉即來。无豬膽。以羊膽代之。

吐利已斷。汗出而厥。四肢拘急不觧。脉微欲絕。通脉四逆加豬膽汁湯主之。390

\section{白通湯}

葱白{\scriptsize 四莖} 乾薑{\scriptsize 一兩} 附子{\scriptsize 一枚生去皮破八片}\\
右三味。以水三升。煮取一升。去滓。分温再服。

少陰病。下利。白通湯主之。314

少陰病。下利。脉微。服白通湯。利不止。厥逆。无脉。乾嘔。煩者。白通加豬膽汁湯主之。服湯脉暴出者死。微{\khaai 微}續{\khaai 出}者生。315

\section{白通加豬膽汁湯}

蔥白{\scriptsize 四莖} 乾薑{\scriptsize 一兩} 附子{\scriptsize 一枚生去皮破八片} 人尿{\scriptsize 五合} 豬膽汁{\scriptsize 一合}\\
右五味。以水三升。煮取一升。去滓。内膽汁。人尿。和令相得。分温再服。若无膽亦可用。

少陰病。下利。脉微。服白通湯。利不止。厥逆。无脉。乾嘔。煩者。白通加豬膽汁湯主之。服湯脉暴出者死。微{\khaai 微}續{\khaai 出}者生。315

\section{玄武湯*}

茯苓{\scriptsize 三兩} 芍藥{\scriptsize 三兩} 生薑{\scriptsize 三兩} 白术{\scriptsize 二兩} 附子{\scriptsize 一枚。炮。去皮。破八片}\\
右五味。以水八升。煮取三升。去滓。温服七合。日三服。

若欬者。加五味子半升。細辛一兩。乾薑一兩。\\
若小便自利者。去茯苓。\\
若不利者。去芍藥。加乾薑二兩。\\
若嘔者。去附子。加生薑至八兩。

太陽病。发汗。汗出不觧。其人仍发熱。心下悸。頭眩。身瞤動。振振欲躃地。玄武湯主之。82

少陰病。二三日不已。至四五日。腹痛。小便不利。四肢沈重疼痛而利。此为有水气。其人或欬。或小便{\khaai 自}利。或下利。或嘔。玄武湯主之。316

\section{附子湯}

附子{\scriptsize 二枚炮去皮破八片} 茯苓{\scriptsize 三兩} 人参{\scriptsize 二兩} 白术{\scriptsize 四兩} 芍藥{\scriptsize 三兩}\\
右五味。以水八升。煮取三升。去滓。温服一升。日三服。

少陰病。得之一二日。口中和。其背惡寒者。当灸之。附子湯主之。304

少陰病。身体痛。手足寒。骨節痛。脉沈者。附子湯主之。305

婦人懷娠六七月。脉弦。发熱。其胎愈胀。腹痛。惡寒者。少腹如扇之狀。所以然者。子臓開故也。当以附子湯温其臓。{\scriptsize 方未見。}

\section{附子粳米湯}

附子{\scriptsize 一枚炮去皮破八片} 半夏{\scriptsize 半升洗} 甘草{\scriptsize 一兩炙} 大棗{\scriptsize 十枚擘} 粳米{\scriptsize 半升}\\
右五味。㕮咀。以水八升。煮米熟。湯成去滓。温服一升。日三服。

腹中寒气。雷鳴。切痛。胸脇逆滿。嘔吐。附子粳米湯主之。

\section{赤丸}

茯苓{\scriptsize 四兩} 半夏{\scriptsize 四兩洗一方用桂} 細辛{\scriptsize 一兩千金作人参} 烏頭{\scriptsize 二兩炮去皮} 附子{\scriptsize 二兩炮去皮} 射罔{\scriptsize 一枚如棗大}\\
右六味。末之。内真朱为色。煉蜜和丸。如麻子大。先食酒飲服一丸。日再夜一服。不知。二丸为度。

寒气厥逆。赤丸主之。

\section{大烏頭煎}

烏頭{\scriptsize 十五枚熬黑不㕮咀}\\
右一味。以水三升。煮取一升。去滓。内蜜二升。煎。令水气{\sungii 𥁞}。取二升。强人服七合。弱人服五合。不差。明日更服。慎不可一日再服。

寸口脉弦而緊。弦則衛气不行。衛气不行即惡寒。緊則不欲食。弦緊相摶。即为寒疝。寒疝繞脐痛。若发則白汗出。手足厥寒。其脉沈弦者。大烏頭煎主之。

\section{烏頭湯}

烏頭{\scriptsize 五枚㕮咀以蜜二升煎取一升即取烏頭} 甘草{\scriptsize 炙} 麻黄{\scriptsize 三兩去節} 芍藥{\scriptsize 三兩} 黄耆{\scriptsize 三兩}\\
右五味。㕮咀四味。以水三升。煮取一升。去滓。内蜜煎中。更煎之。服七合。不知。{\sungii 𥁞}服之。
\footnote{本方吳本甘草麻黄无剂量,鄧本甘草无剂量。}

病歷節。疼痛。不可屈伸。烏頭湯主之。

烏頭湯。治腳气。疼痛。不可屈伸。

烏頭湯。治寒疝。腹中絞痛。賊風入腹。攻五臟。拘急不得轉側。叫呼。发作有时。使人陰縮。手足厥逆。

\section{薏苡仁附子散}

薏苡仁{\scriptsize 十五兩} 大附子{\scriptsize 十枚炮}\\
右二味。杵为散。服方寸匕。日三服。

胸痹緩急者。薏苡仁附子散主之。

\section{薏苡{\khaai 仁}附子敗醬散}

薏苡仁{\scriptsize 十分} 附子{\scriptsize 二分炮去皮} 敗醬{\scriptsize 五分}\\
右三味。杵为末。取方寸匕。以水二升。煎取一升。頓服之。小便当下。

腸癰之为病。其身甲錯。腹皮急。按之濡。如腫狀。腹无積聚。身无熱。脉數。此为腸内有{\khaai 癰}膿。薏苡{\khaai 仁}附子敗醬散主之。

\section{天雄散}

天雄{\scriptsize 三兩炮去皮} 白术{\scriptsize 八兩} 桂枝{\scriptsize 六兩} 龙骨{\scriptsize 三兩}\\
右四味。杵为散。酒服半錢匕。不知。稍增之。

夫失精家。少腹弦急。陰頭寒。目眩。髮落。脉極虗芤遲。为清穀。亡血。失精。脉得諸芤動微緊。男子失精。女子夢交。桂枝加龙骨牡蛎湯主之。天雄散亦主之。

\section{蜀漆散}

蜀漆{\scriptsize 洗去腥} 雲母{\scriptsize 燒之三日三夜} 龙骨{\scriptsize 等分}\\
右三味。杵为散。未发前以漿水服半錢。温瘧加蜀漆半分。臨发时服一錢匕。{\scriptsize 一方雲母作雲石。}

瘧。多寒者。名曰牡瘧。蜀漆散主之。

\section{栀子豉湯}

栀子{\scriptsize 十四个擘} 香豉{\scriptsize 四合綿裹}\\
右二味。以水四升。先煮栀子。得二升半。内豉。煮取一升半。去滓。分为二服。温進一服。得吐者。止後服。

发汗吐下後。虗煩。不得眠。若劇者。反覆顛倒。心中懊憹。栀子{\khaai 豉}湯主之。若少气者。栀子甘草{\khaai 豉}湯主之。若嘔者。栀子生薑{\khaai 豉}湯主之。76

发汗若下之。煩熱。胸中窒者。栀子{\khaai 豉}湯主之。77

傷寒五六日。大下之後。身熱不去。心中結痛者。未欲觧也。栀子{\khaai 豉}湯主之。78

凡用栀子湯。其人微溏者。不可与服之。81

陽明病。脉浮緊。咽乾。口苦。腹滿而喘。发熱。汗出。不惡寒。反惡熱。身重。若发汗則躁。心憒憒。反譫語。若加温針。必怵惕。煩躁。不得眠。若下之。則胃中空虗。客气動膈。心中懊憹。舌上胎者。栀子{\khaai 豉}湯主之。221

陽明病。下之。其外有熱。手足温。不結胸。心中懊憹。飢不能食。但頭汗出。栀子{\khaai 豉}湯主之。228

下利後更煩。按之心下濡者。为虗煩也。栀子{\khaai 豉}湯主之。375

\section{栀子甘草豉湯}

栀子{\scriptsize 十四个擘} 甘草{\scriptsize 二兩炙} 香豉{\scriptsize 四合綿裹}\\
右三味。以水四升。先煮栀子甘草。取二升半。内豉。煮取一升半。去滓。分二服。温進一服。得吐者。止後服。

发汗吐下後。虗煩。不得眠。若劇者。反覆顛倒。心中懊憹。栀子{\khaai 豉}湯主之。若少气者。栀子甘草{\khaai 豉}湯主之。若嘔者。栀子生薑{\khaai 豉}湯主之。76

\section{栀子生薑豉湯}

栀子{\scriptsize 十四个擘} 生薑{\scriptsize 五兩} 香豉{\scriptsize 四合綿裹}\\
右三味。以水四升。先煮栀子生薑。取二升半。内豉。煮取一升半。去滓。分二服。温進一服。得吐者。止後服。

发汗吐下後。虗煩。不得眠。若劇者。反覆顛倒。心中懊憹。栀子{\khaai 豉}湯主之。若少气者。栀子甘草{\khaai 豉}湯主之。若嘔者。栀子生薑{\khaai 豉}湯主之。76

\section{枳実栀子湯}

枳実{\scriptsize 三枚炙} 栀子{\scriptsize 十四个擘} 豉{\scriptsize 一升綿裹}\\
右三味。以清漿水七升。空煮取四升。内枳実栀子。煮取二升。下豉。更煮五六沸。去滓。温分再服。覆令微似汗。若有宿食者。内大黄如博碁子五六枚。服之愈。

大病差後勞復者。枳実栀子湯主之。393

\section{栀子枳実豉大黄湯}

栀子{\scriptsize 十四枚擘} 枳実{\scriptsize 五枚炙} 豉{\scriptsize 一升綿裹} 大黄{\scriptsize 一兩}\\
右四味。㕮咀。以水六升。煮取二升。去滓。分温三服。

酒黄疸。心中懊憹。或熱痛。栀子{\khaai 枳実豉}大黄湯主之。

\section{大黄黄蘗栀子硝石湯}

大黄{\scriptsize 四兩} 黄蘗{\scriptsize 四兩} 栀子{\scriptsize 十五枚擘} 硝石{\scriptsize 四兩}\\
右四味。㕮咀。以水六升。煮取二升。去滓。内硝。更煮取一升。頓服。

黄疸。腹滿。小便不利而赤。自汗出。此为表和裏実。当下之。宜大黄{\khaai 黄蘗栀子}硝石湯。

\section{茵陳蒿湯}

茵陳蒿{\scriptsize 六兩} 栀子{\scriptsize 十四枚擘} 大黄{\scriptsize 二兩去皮}\\
右三味。以水一斗二升。先煮茵陳。減六升。内二味。煮取三升。去滓。分三服。小便当利。尿如皂莢汁狀。色正赤。一宿腹減。黄從小便去也。

陽明病。发熱。汗出者。此为熱越。不能发黄也。但頭汗出。身无汗。齐頸而還。小便不利。渴引水漿者。此为瘀熱在裏。身必发黄。茵陳{\khaai 蒿}湯主之。236

傷寒七八日。身黄如橘。小便不利。腹微滿者。茵陳{\khaai 蒿}湯主之。260

穀疸之为病。寒熱不食。食即頭眩。心胸不安。久久发黄。为穀疸。茵陳蒿湯主之。

\section{栀子蘗皮湯}

肥栀子{\scriptsize 十五个擘} 甘草{\scriptsize 一兩炙} 黄蘗{\scriptsize 二兩}\\
右三味。以水四升。煮取一升半。去滓。分温再服。

傷寒。身黄。发熱。栀子蘗皮湯主之。261

\section{栀子厚朴湯}

栀子{\scriptsize 十四个擘} 厚朴{\scriptsize 四兩炙去皮} 枳実{\scriptsize 四枚水浸炙令黄}\\
右三味。以水三升半。煮取一升半。去滓。分二服。温進一服。得吐者。止後服。

傷寒下後。煩而腹滿。卧起不安。栀子厚朴湯主之。79

\section{栀子乾薑湯}

栀子{\scriptsize 十四个擘} 乾薑{\scriptsize 二兩}\\
右二味。以水三升半。煮取一升半。去滓。分二服。一服得吐者。止後服。

傷寒。醫以丸藥大下之。身熱不去。微煩。栀子乾薑湯主之。80

\section{酸棗{\khaai 仁}湯}

酸棗仁{\scriptsize 二升} 甘草{\scriptsize 一兩炙} 知母{\scriptsize 二兩} 茯苓{\scriptsize 二兩} 芎藭{\scriptsize 二兩}\\
右五味。㕮咀。以水八升。煮酸棗仁。得六升。内諸藥。煮取三升。去滓。分温三服。{\scriptsize 深師有生薑二兩。}

虗勞。虗煩。不得眠。酸棗{\khaai 仁}湯主之。

\section{大陷胸湯}

大黄{\scriptsize 六兩去皮酒洗} 芒硝{\scriptsize 一升} 甘遂末{\scriptsize 一錢匕}\\
右三味。以水六升。煮大黄。取二升。去滓。内芒硝。煮兩沸。内甘遂末。温服一升。得快利。止後服。

太陽病。脉浮而動數。浮則为風。數則为熱。動則为痛。數則为虗。頭痛。发熱。微盜汗出。而反惡寒。其表未觧。醫反下之。動數變遲。膈内拒痛。胃中空虗。客气動膈。短气。躁煩。心中懊憹。陽气内陷。心下因堅。則为結胸。大陷胸湯主之。若不結胸。但頭汗出。餘処无汗。齐頸而還。小便不利。身必发黄。134

傷寒六七日。結胸熱実。脉沈緊。心下痛。按之如石堅。大陷胸湯主之。135

傷寒十餘日。熱結在裏。復往來寒熱者。与大柴胡湯。但結胸。无大熱者。此为水結在胸脇。{\khaai 但}頭微汗出。大陷胸湯主之。136

太陽病。重发汗而復下之。不大便五六日。舌上燥而渴。日晡所小有潮熱。從心下至少腹堅滿而痛不可近。大陷胸湯主之。137

傷寒五六日。嘔而发熱。柴胡湯証具。而以他藥下之。柴胡証仍在者。復与柴胡湯。此雖已下之。不为逆。必蒸蒸而振。卻发熱汗出而觧。若心下滿而堅痛者。此为結胸。宜大陷胸湯。若但滿而不痛者。此为痞。柴胡{\khaai 湯}不復中与也。宜半夏瀉心湯。149

\section{大陷胸丸}

大黄{\scriptsize 八兩} 葶藶子{\scriptsize 半升熬} 芒硝{\scriptsize 半升} 杏仁{\scriptsize 半升去皮尖熬黑}\\
右四味。擣篩二味。内杏仁芒硝。合研如脂。和散。取如彈丸一枚。別擣甘遂末一錢匕。白蜜二合。水二升。煮取一升。温頓服之。一宿乃下。如不下。更服。取下为效。禁如藥法。

結胸者。項亦强。如柔痙狀。下之則和。宜大陷胸丸。131

\section{小陷胸湯}

黄連{\scriptsize 一兩} 半夏{\scriptsize 半升洗} 栝蔞実{\scriptsize 大者一枚}\\
右三味。以水六升。先煮栝蔞。取三升。去滓。内諸藥。煮取二升。去滓。分温三服。

小結胸者。正在心下。按之則痛。其脉浮滑。小陷胸湯主之。138

\section{枳実薤白桂枝湯}

枳実{\scriptsize 四枚炙} 厚朴{\scriptsize 四兩炙} 薤白{\scriptsize 八兩切} 桂枝{\scriptsize 一兩去皮} 栝蔞実{\scriptsize 一枚擣}\\
右五味。㕮咀。以水五升。先煮枳実。厚朴。取二升。去滓。内諸藥。煮三沸。去滓。分温三服。

胸痹。心中痞。留气結在胸。胸滿。脇下逆搶心。枳実薤白桂枝湯主之。理中湯亦主之。

\section{栝蔞薤白白酒湯}

栝蔞実{\scriptsize 一枚擣} 薤白{\scriptsize 半升切} 白酒{\scriptsize 七升}\\
右三味。同煮。取二升。去滓。分温再服。

胸痹之病。喘息欬唾。胸背痛。短气。寸口脉沈而遲。関上小緊數。栝蔞薤白白酒湯主之。

\section{栝蔞薤白半夏湯}

栝蔞実{\scriptsize 一枚擣} 薤白{\scriptsize 三兩切} 半夏{\scriptsize 半升洗切} 白酒{\scriptsize 一斗}\\
右四味。同煮。取四升。去滓。温服一升。日三服。

胸痹。不得卧。心痛徹背者。栝蔞薤白半夏湯主之。

\section{瓜蒂散}

瓜蒂{\scriptsize 一分熬黄} 赤小豆{\scriptsize 一分}\\
右二味。各別擣篩。为散已。合治之。取一錢匕。以香豉一合。用熱湯七合。煮作稀糜。去滓。取汁和散。温頓服之。不吐者。少少加。得快吐乃止。

病如桂枝証。頭不痛。項不强。寸{\khaai 口}脉微浮。胸中痞堅。气上衝咽㗋。不得息。此为胸有寒。当吐之。宜瓜蒂散。166

病者手足厥冷。脉乍緊。邪結在胸中。心下滿而煩。飢不能食。病在胸中。当吐之。宜瓜蒂散。355

宿食在上脘。当吐之。宜瓜蒂散。

\section{小半夏湯}

半夏{\scriptsize 一升洗} 生薑{\scriptsize 八兩}\\
右二味。切。以水七升。煮取一升半。去滓。分温再服。

嘔家本渴。渴者为欲觧。今反不渴。心下有支飲故也。小半夏湯主之。

黄疸病。小便色不變。欲自利。腹滿而喘。不可除熱。熱除必噦。噦者。小半夏湯主之。

諸嘔吐。穀不得下者。小半夏湯主之。

\section{小半夏加茯苓湯}

半夏{\scriptsize 一升洗} 生薑{\scriptsize 八兩} 茯苓{\scriptsize 三兩一方四兩}\\
右三味。切。以水七升。煮取一升五合。去滓。分温再服。

卒嘔吐。心下痞。膈間有水。眩悸者。小半夏加茯苓湯主之。

先渴卻嘔。为水停心下。此屬飲家。小半夏加茯苓湯主之。

\section{大半夏湯}

半夏{\scriptsize 三升洗完用} 人参{\scriptsize 三兩切} 白蜜{\scriptsize 一升}\\
右三味。以泉水一斗二升和蜜揚之二百四十遍煮藥。取二升半。去滓。温服一升。餘分再服。

胃反。嘔吐者。大半夏湯主之。

%千金方、外臺載有本方不同的條文,以後補充。

\section{生薑半夏湯}

生薑汁{\scriptsize 一升} 半夏{\scriptsize 半升洗切}\\
右二味。以水三升。煮半夏。取二升。内生薑汁。煮取一升半。去滓。小冷。分四服。日三夜一服。若一服止。停後服。

病人胸中似喘不喘。似嘔不嘔。似噦不噦。徹心中憒憒然无奈者。生薑半夏湯主之。

\section{苦酒湯}

半夏{\scriptsize 洗破如棗核十四枚} 雞子{\scriptsize 一枚去黄内上苦酒著雞子殼中}\\
右二味。内半夏著苦酒中。以雞子殼置刀環中。安火上。令三沸。去滓。少少含嚥之。不差。更作三剂。

少陰病。咽中傷。生瘡。不能語言。聲不出者。苦酒湯主之。312

\section{半夏厚朴湯}

半夏{\scriptsize 一升洗} 厚朴{\scriptsize 三兩炙} 茯苓{\scriptsize 四兩} 生薑{\scriptsize 五兩切} 乾蘇枼{\scriptsize 二兩}\\
右五味。㕮咀。以水七升。煮取四升。去滓。分温四服。日三夜一服。{\scriptsize 一作治胸滿。心下堅。咽中怗怗。如有炙肉。吐之不出。吞之不下。}

婦人咽中如有炙臠。半夏厚朴湯主之。

\section{半夏乾薑散}

半夏{\scriptsize 洗} 乾薑{\scriptsize 各等分}\\
右二味。杵为散。取方寸匕。漿水一升半。煎取七合。頓服之。

乾嘔。吐{\sungii 𠱘}。吐涎沫。半夏乾薑散主之。

\section{厚朴生薑半夏甘草人参湯}

厚朴{\scriptsize 八兩炙去皮} 生薑{\scriptsize 八兩切} 半夏{\scriptsize 半升洗} 甘草{\scriptsize 二兩} 人参{\scriptsize 一兩}\\
右五味。以水一斗。煮取三升。去滓。温服一升。日三服。

发汗後。腹胀滿者。厚朴{\khaai 生薑半夏甘草人参}湯主之。66

\section{乾薑人参半夏丸}

乾薑{\scriptsize 一兩} 人参{\scriptsize 一兩} 半夏{\scriptsize 半兩洗}\\
右三味。末之。以生薑汁和为丸。如梧子大。飲服一丸。日三服。

婦人妊娠。嘔吐不止。乾薑人参半夏丸主之。

\section{乾薑黄芩黄連人参湯}

乾薑{\scriptsize 三兩} 黄芩{\scriptsize 三兩} 黄連{\scriptsize 三兩} 人参{\scriptsize 三兩}
右四味。以水六升。煮取二升。去滓。分温再服。

傷寒。本自寒下。醫復吐{\khaai 下}之。寒格。更逆吐{\khaai 下}。食入即出。乾薑黄芩黄連人参湯主之。359

\section{黄連湯}

黄連{\scriptsize 三兩} 甘草{\scriptsize 三兩炙} 乾薑{\scriptsize 三兩} 桂枝{\scriptsize 三兩} 人参{\scriptsize 二兩} 半夏{\scriptsize 半升洗} 大棗{\scriptsize 十二枚擘}
右七味。以水一斗。煮取六升。去滓。分温五服。晝三夜二。疑非仲景方。

傷寒。胸中有熱。胃中有邪气。腹中痛。欲嘔吐。黄連湯主之。173

\section{半夏瀉心湯}

半夏{\scriptsize 半升洗} 黄芩{\scriptsize 三兩} 乾薑{\scriptsize 三兩} 人参{\scriptsize 三兩} 甘草{\scriptsize 三兩炙} 黄連{\scriptsize 一兩} 大棗{\scriptsize 十二枚擘}\\
右七味。以水一斗。煮取六升。去滓。再煎。取三升。温服一升。日三服。{\zhaoben}

半夏{\scriptsize 半升洗} 黄芩{\scriptsize 二兩} 人参{\scriptsize 二兩} 甘草{\scriptsize 二兩炙} 乾薑{\scriptsize 二兩} 黄連{\scriptsize 一兩} 大棗{\scriptsize 十二枚擘}\\
右七味。㕮咀。以水一斗。煮取六升。去滓。再煎。取三升。温服一升。日三服。{\wuben}

傷寒五六日。嘔而发熱。柴胡湯証具。而以他藥下之。柴胡証仍在者。復与柴胡湯。此雖已下之。不为逆。必蒸蒸而振。卻发熱汗出而觧。若心下滿而堅痛者。此为結胸。宜大陷胸湯。若但滿而不痛者。此为痞。柴胡{\khaai 湯}不復中与也。宜半夏瀉心湯。149

嘔而腸鳴。心下痞者。半夏瀉心湯主之。

\section{甘草瀉心湯}

甘草{\scriptsize 四兩炙} 黄芩{\scriptsize 三兩} 人参{\scriptsize 三兩} 乾薑{\scriptsize 三兩} 黄連{\scriptsize 一兩} 大棗{\scriptsize 十二枚擘} 半夏{\scriptsize 半升洗}\\
右七味。㕮咀。以水一斗。煮取六升。去滓。再煎。温服一升。日三服。{\scriptsize 臣億等謹按。上生薑瀉心湯法。本云理中人参黄芩湯。今詳瀉心以療痞。痞气因发陰而生。是半夏生薑甘草瀉心三方皆本於理中也。其方必各有人参。今甘草瀉心湯中无者。脱落之也。又按。千金并外臺祕要。治傷寒䘌食用此方。皆有人参。知脱落无疑。}
	\footnote{
		錢超塵「按:今考《千金要方》卷九甘草瀉心湯方,宋臣小註云「加人参三兩乃是」,又考《千金要方》卷十傷寒不发汗變成狐惑病第四亦載此方,今錄之如下:半夏{\scriptsize 半升}{ }黄芩{ }人参{ }乾薑{\scriptsize 各三兩}{ }黄連{\scriptsize 一兩}{ }甘草{\scriptsize 三兩}{ }大棗{\scriptsize 十二枚}。又考《外臺祕要》卷二傷寒狐惑病方亦載此方,錄之如下:半夏{\scriptsize 半升}{ }黄芩{\scriptsize 三兩}{ }人参{\scriptsize 三兩}{ }乾薑{\scriptsize 三兩}{ }黄連{\scriptsize 一兩}{ }甘草{\scriptsize 四兩炙}{ }大棗{\scriptsize 十二枚擘}。綜觀《千金要方》、《外臺祕要》所載之甘草瀉心湯方,確有人参三兩无疑,林億等所校是也。北宋校正醫書局所校之《傷寒論》每有極精闢処,此即其中一例也。」
	}

傷寒中風。醫反下之。其人下利。日數十行。穀不化。腹中雷鳴。心下痞堅而滿。乾嘔。心煩。不{\khaai 能}得安。醫見心下痞。谓病不{\sungii 𥁞}。復下之。其痞益甚。此非結熱。但以胃中虗。客气上逆。故使之堅。甘草瀉心湯主之。158

狐惑之为病。狀如傷寒。默默欲眠。目不得閉。卧起不安。蝕於㗋为惑。蝕於陰为狐。不欲飲食。惡聞食臭。其面目乍赤乍黑乍白。蝕於上部則聲喝。甘草瀉心湯主之。蝕於下部則咽乾。苦参湯洗之。蝕於肛者。雄黄熏之。

\section{生薑瀉心湯}

生薑{\scriptsize 四兩切} 甘草{\scriptsize 三兩炙} 人参{\scriptsize 三兩} 乾薑{\scriptsize 一兩} 黄芩{\scriptsize 三兩} 半夏{\scriptsize 半升洗} 黄連{\scriptsize 一兩} 大棗{\scriptsize 十二枚擘}\\
右八味。以水一斗。煮取六升。去滓。再煎。取三升。温服一升。日三服。\\
附子瀉心湯。本云加附子。半夏瀉心湯。甘草瀉心湯。同体別名耳。生薑瀉心湯。本云理中人参黄芩湯。去桂枝术。加黄連。并瀉肝法。

傷寒。汗出。觧之後。胃中不和。心下痞堅。乾噫食臭。脇下有水气。腹中雷鳴而利。生薑瀉心湯主之。157

\section{旋覆代赭湯}

旋覆花{\scriptsize 三兩} 人参{\scriptsize 二兩} 生薑{\scriptsize 五兩} 代赭{\scriptsize 一兩} 甘草{\scriptsize 三兩炙} 半夏{\scriptsize 半升洗} 大棗{\scriptsize 十二枚擘}\\
右七味。以水一斗。煮取六升。去滓。再煎。取三升。温服一升。日三服。

傷寒。发汗{\khaai 若}吐{\khaai 若}下。觧後。心下痞堅。噫气不除者。旋覆代赭湯主之。161

\section{吳茱萸湯}

吳茱萸{\scriptsize 一升{\khaai 洗}} 人参{\scriptsize 三兩} 生薑{\scriptsize 六兩切} 大棗{\scriptsize 十二枚擘}\\
右四味。以水七升。煮取二升。去滓。温服七合。日三服。{\zhaoben}
	\footnote{
		「以水七升」、「煮取二升」同趙本,吳本分別作「以水五升」、「煮取三升」。唐弘宇按:吳本誤。漢代十合为一升,若每服七合,則三服約等於二升。
	}

食穀欲嘔者。屬陽明。吳茱萸湯主之。{\khaai 得湯反劇者。屬上焦。}243

少陰病。吐利。手足逆{\khaai 冷}。煩躁欲死者。吳茱萸湯主之。309

乾嘔。吐涎沫。頭痛者。吳茱萸湯主之。378

嘔而胸滿者。吳茱萸湯主之。

\section{大建中湯}

蜀椒{\scriptsize 二合汗} 乾薑{\scriptsize 四兩} 人参{\scriptsize 二兩}\\
右三味。{\khaai 㕮咀。}以水四升。煮取二升。去滓。内膠飴一升。微火煎取一升半。分温在服。如一炊頃。可飲粥二升後更服。当一日食糜。温覆之。

心胸中大寒痛。嘔。不能飲食。腹中寒。上衝皮起。出見有頭足。上下痛而不可觸近。大建中湯主之。

\section{黄連阿膠湯}

黄連{\scriptsize 四兩} 黄芩{\scriptsize 二兩} 芍藥{\scriptsize 二兩} 雞子黄{\scriptsize 二枚} 阿膠{\scriptsize 三兩一云三挺}\\
右五味。以水六升。先煮三物。取二升。去滓。内膠。烊{\sungii 𥁞}。小冷。内雞子黄。攪令相得。温服七合。日三服。

少陰病。得之二三日以上。心中煩。不得卧。黄連阿膠湯主之。303

\section{黄芩湯}

黄芩{\scriptsize 三兩} 芍藥{\scriptsize 二兩} 甘草{\scriptsize 二兩炙} 大棗{\scriptsize 十二枚擘}\\
右四味。以水一斗。煮取三升。去滓。温服一升。日再夜一服。

太陽与少陽合病。自下利者。与黄芩湯。若嘔者。与黄芩加半夏生薑湯。172

\section{黄芩加半夏生薑湯}

黄芩{\scriptsize 三兩} 芍藥{\scriptsize 二兩} 甘草{\scriptsize 二兩炙} 大棗{\scriptsize 十二枚擘} 半夏{\scriptsize 半升洗} 生薑{\scriptsize 一兩半一方三兩切}\\
右六味。以水一斗。煮取三升。去滓。温服一升。日再夜一服。

太陽与少陽合病。自下利者。与黄芩湯。若嘔者。与黄芩加半夏生薑湯。172

乾嘔而利者。黄芩加半夏生薑湯主之。

\section{黄芩湯}

黄芩{\scriptsize 三兩} 人参{\scriptsize 三兩} 乾薑{\scriptsize 三兩} 桂枝{\scriptsize 二兩去皮} 大棗{\scriptsize 十二枚擘} 半夏{\scriptsize 半升洗}\\
右六味。㕮咀。以水七升。煮取三升。去滓。分温三服。

乾嘔。下利。黄芩湯主之。{\scriptsize 玉函經云人参黄芩湯}

\section{三物黄芩湯}

黄芩{\scriptsize 一兩} 苦参{\scriptsize 二兩} 乾地黄{\scriptsize 四兩}\\
右藥㕮咀。以水八升。煮取二升。去滓。温服一升。多吐下虫。{\scriptsize 見千金。}

婦人多在草蓐得風。四肢苦煩熱。皆自发露所为。頭痛者。与小柴胡湯。頭不痛。但煩者。与三物黄芩湯。

\section{白頭翁湯}

白頭翁{\scriptsize 二兩} 黄蘗{\scriptsize 三兩} 黄連{\scriptsize 三兩} 秦皮{\scriptsize 三兩}\\
右四味。以水七升。煮取二升。去滓。温服一升。不愈。更服一升。

熱利下重者。白頭翁湯主之。371

下利。欲飲水者。为有熱也。白頭翁湯主之。373

\section{白頭翁加甘草阿膠湯}

白頭翁{\scriptsize 二兩} 黄連{\scriptsize 三兩} 蘗皮{\scriptsize 三兩} 秦皮{\scriptsize 三兩} 甘草{\scriptsize 二兩炙} 阿膠{\scriptsize 二兩}\\
右六味。㕮咀。以水七升。煮取二升半。去滓。内膠。令消{\sungii 𥁞}。分温三服。

婦人產後。下利。虗極。白頭翁加甘草阿膠湯主之。

\section{木防己湯}

木防己{\scriptsize 三兩} 桂枝{\scriptsize 二兩去皮} 石膏{\scriptsize 如雞子大十二枚} 人参{\scriptsize 四兩}\\
右四味。㕮咀。以水六升。煮取二升。去滓。分温再服。

膈間支飲。其人喘滿。心下痞堅。面色黎黑。其脉沈緊。得之數十日。醫吐下之不愈。木防己湯主之。虗者即愈。実者三日復发。復与不愈者。宜去石膏加茯苓芒硝湯。

\section{木防己去石膏加茯苓芒硝湯}

木防己{\scriptsize 二兩} 桂枝{\scriptsize 二兩去皮} 人参{\scriptsize 四兩} 茯苓{\scriptsize 四兩} 芒硝{\scriptsize 三合}\\
右五味。㕮咀。以水六升。煮取二升。去滓。内芒硝。再微煎。分温再服。微利則愈。

膈間支飲。其人喘滿。心下痞堅。面色黎黑。其脉沈緊。得之數十日。醫吐下之不愈。木防己湯主之。虗者即愈。実者三日復发。復与不愈者。宜去石膏加茯苓芒硝湯。

\section{防己茯苓湯}

防己{\scriptsize 五兩
	\footnote{
		「五兩」同吳本,鄧本作「三兩」。
	}
} 黄耆{\scriptsize 三兩} 桂枝{\scriptsize 三兩去皮} 茯苓{\scriptsize 六兩} 甘草{\scriptsize 二兩炙}\\
右五味。㕮咀。以水六升。煮取二升。去滓。分温再服。

皮水为病。四肢腫。水气在皮膚中。四肢聶聶動者。防己茯苓湯主之。

\section{防己黄耆湯*}

防己{\scriptsize 四兩} 黄耆{\scriptsize 五兩} 甘草{\scriptsize 二兩炙} 白术{\scriptsize 三兩} 生薑{\scriptsize 二兩切} 大棗{\scriptsize 十二枚擘}\\
右六味。㕮咀。以水七升。煮取二升。去滓。分温三服。喘者加麻黄。胃中不和者加芍藥。气上衝者加桂。下有陳寒者加細辛。服後当如虫行皮中。從腰以上如冰。後坐被上。又以一被繞腰以温下。令微汗。差。{\scriptsize 腰以上。疑作腰以下。}{\wuben}

風濕。脉浮。身重。汗出。惡風者。防己黄耆湯主之。

風水。脉浮。身重。汗出。惡風者。防己黄耆湯主之。腹痛者。加芍藥。

夫風水。脉浮为在表。其人或頭汗出。表无他病。病者但下重。故知從腰以上为和。腰以下当腫及陰。難以屈伸。防己黄耆湯主之。

\section{枳実术湯}

枳実{\scriptsize 七枚} 白术{\scriptsize 二兩}\\
右二味。㕮咀。以水五升。煮取三升。去滓。分温三服。腹中軟。即当散也。

心下堅。大如盤。邊如旋盤。水飲所作。枳実术湯主之。

\section{枳実芍藥散}

枳実{\scriptsize 燒令黑勿令太過} 芍藥{\scriptsize 等分}\\
右二味。杵为散。服方寸匕。日三服。并主癰膿。以麥屑粥下之。

婦人產後。腹痛。煩滿。不得卧。枳実芍藥散主之。

師曰。產婦腹痛。法当与枳実芍藥散。假令不愈者。此为腹中有乾血著脐下。与下瘀血湯服之。{\khaai 亦}主經水不利若瘀血。

\section{排膿散}

枳実{\scriptsize 十六枚炙} 芍藥{\scriptsize 六分} 桔梗{\scriptsize 二分}\\
右三味。杵为散。取雞子黄一枚。取散与雞黄等揉合令相得。飲和服之。日一服。

\section{桂枝生薑枳実湯}

桂枝{\scriptsize 三兩去皮} 枳実{\scriptsize 伍枚炙} 生薑{\scriptsize 三兩切}\\
右三味。㕮咀。以水六升。煮取三升。去滓。分温三服。

心中痞。諸逆。心懸痛。桂枝生薑枳実湯主之。

\section{橘{\khaai 皮}枳{\khaai 実生}薑湯}

橘皮{\scriptsize 十六兩} 枳実{\scriptsize 二兩炙} 生薑{\scriptsize 八兩切}\\
右三味。㕮咀。以水五升。煮取二升。去滓。分温再服

胸痹。胸中气塞。短气。茯苓杏仁甘草湯主之。橘皮枳実生薑湯亦主之。

\section{橘皮湯}

橘皮{\scriptsize 四兩} 生薑{\scriptsize 八兩}\\
右二味。切。以水七升。煮取三升。去滓。温服一升。下咽即愈。

乾嘔。噦。若手足厥者。橘皮湯主之。

\section{橘皮竹茹湯}

橘皮{\scriptsize 二升} 竹茹{\scriptsize 三升} 大棗{\scriptsize 三十枚擘} 生薑{\scriptsize 八兩切} 甘草{\scriptsize 五兩炙} 人参{\scriptsize 一兩}\\
右六味。㕮咀。以水一斗。煮取三升。去滓。温服一升。日三服

噦{\sungii 𠱘}者。橘皮竹茹湯主之。

\section{茯苓飲}

茯苓{\scriptsize 三兩} 人参{\scriptsize 三兩} 白术{\scriptsize 三兩} 生薑{\scriptsize 四兩} 枳実{\scriptsize 二兩} 橘皮{\scriptsize 一兩半}\\
右六味。㕮咀。以水六升。煮取一升八合。去滓。分温三服。如人行八九里進之。

主心胸中有停痰宿水。自吐出水後。心胸間虗。气滿。不能食。消痰气。令能食。茯苓飲。

\section{桂枝茯苓丸}

桂枝{\scriptsize 去皮} 茯苓{ }牡丹{\scriptsize 去心} 桃仁{\scriptsize 去皮尖熬} 芍藥{\scriptsize 各等分}\\
右五味。末之。煉蜜为丸。如兔屎大。每日{\khaai 食前服}一丸。不知。加至三丸。

婦人妊娠。經斷三月。而得漏下。下血四十日不止。胎欲動。在於脐上。此为妊娠。六月動者。前三月經水利时。胎也。下血者。後斷三月。衃也。所以下血不止者。其癥不去故也。当下其癥。宜桂枝茯苓丸。

\section{膠艾湯}

阿膠{\scriptsize 二兩} 芎窮{\scriptsize 二兩} 甘草{\scriptsize 二兩炙} 艾枼{\scriptsize 三兩} 当歸{\scriptsize 三兩} 芍藥{\scriptsize 四兩} 乾地黄{\scriptsize 四兩}\\
右七味。㕮咀。以水五升清酒三升合煮。取三升。去滓。内膠。令消{\sungii 𥁞}。温服一升。日三服。不差更作。

師曰。婦人有漏下者。有半產後因續下血都不絕者。有妊娠下血者。假令妊娠腹中痛。为胞阻。膠艾湯主之。

\section{赤石脂禹餘糧湯}

赤石脂{\scriptsize 十六兩碎} 太一禹餘糧{\scriptsize 十六兩碎}\\
右二味。以水六升。煮取二升。去滓。分温三服。

傷寒。服湯藥。下利不止。心下痞堅。服瀉心湯已。復以他藥下之。利不止。醫以理中与之。利益甚。理中者。理中焦。此利在下焦。赤石脂禹餘糧湯主之。復不止者。当利小便。159

\section{桃花湯}

赤石脂{\scriptsize 十六兩一半全用一半篩末} 乾薑{\scriptsize 一兩} 粳米{\scriptsize 一升}\\
右三味。以水七升。煮米令熟。去滓。温服七合。内赤石脂末方寸匕。日三服。若一服愈。餘勿服。

少陰病。下利。便膿血。桃花湯主之。306

少陰病二三日至四五日。腹痛。小便不利。下利不止。便膿血。桃花湯主之。307

下利。便膿血者。桃花湯主之。

\section{蜜煎}

食蜜{\scriptsize 七合}\\
右一味。於銅器内。微火煎。当需凝如飴狀。攪之勿令焦著。欲可丸。并手撚作挺。令頭鋭。大如指。長二寸許。当熱时急作。冷則鞕。以内榖道中。以手急抱。欲大便时乃去之。疑非仲景意。已試甚良。

陽明病。{\khaai 自}汗出。若发汗。小便自利者。此为{\khaai 津液}内竭。雖堅。不可攻之。当須自欲大便。宜蜜煎。導而通之。若土瓜根及豬膽汁。皆可以導。233

\section{麻子仁丸}

麻子仁{\scriptsize 二升} 芍藥{\scriptsize 八兩} 枳実{\scriptsize 八兩炙} 大黄{\scriptsize 十六兩去皮} 厚朴{\scriptsize 一尺炙去皮} 杏仁{\scriptsize 一升去皮尖熬別作脂}\\
右六味。蜜和丸。如梧桐子大。飲服十丸。日三服。漸加。以知为度。{\zhaoben}

麻子仁{\scriptsize 一升} 芍藥{\scriptsize 八兩} 枳実{\scriptsize 十六兩炙} 大黄{\scriptsize 十六兩} 厚朴{\scriptsize 一尺炙} 杏仁{\scriptsize 一升去皮尖熬焦}\\
右六味。末之。鍊蜜和丸。如梧子大。飲服十丸。日三服。漸加。以知为度。{\wuben}

趺陽脉浮而濇。浮則胃气强。濇則小便數。浮濇相摶。大便則堅。其脾为約。麻子仁丸主之。247

\section{防己椒目葶藶大黄丸}

防己{\scriptsize 一兩} 椒目{\scriptsize 一兩} 葶藶{\scriptsize 一兩熬} 大黄{\scriptsize 一兩}\\
右四味。末之。蜜和丸如梧桐子大。先食飲服一丸。日三服。稍增。口中有津液止。渴者。加芒硝半兩

腹滿。口舌乾燥。此腸間有水气。防己椒目葶藶大黄丸主之。

\section{葶藶大棗瀉肺湯}

葶藶{\scriptsize 熬令黄色。擣丸如彈丸大} 大棗{\scriptsize 十二枚擘}\\
右先以水三升煮棗。取二升。去棗。内葶藶。煮取一升。頓服之。

肺癰。喘不得卧。葶藶大棗瀉肺湯主之。

肺癰。胸滿胀。一身面目浮腫。鼻塞。清涕出。不聞香臭酸辛。欬逆上气。喘鳴迫塞。葶藶大棗瀉肺湯主之。

支飲。不得息。葶藶大棗瀉肺湯主之。

\section{十棗湯}

芫花{\scriptsize 熬} 甘遂{ }大戟{\scriptsize 熬}\\
右三味。擣篩。以水一升五合。煮大棗十枚。煮取八合。去滓。内藥末。强人一錢匕。羸人服半錢。平旦服之。不下者。明日更加半錢。下後。糜粥自養。

太陽中風。下利。嘔{\sungii 𠱘}。表觧乃可攻之。其人漐漐汗出。发作有时。頭痛。心下痞堅滿。引脇下痛。乾嘔。短气。汗出。不惡寒。此为表觧裏未和。十棗湯主之。152

病懸飲者。十棗湯主之。

欬家。其脉弦。为有水。可与十棗湯。

夫有支飲家。欬煩。胸中痛者。不卒死。至一百日一歲。与十棗湯。

\section{桔梗白散(三物白散)}

桔梗{\scriptsize 三分} 貝母{\scriptsize 三分} 巴豆{\scriptsize 一分去皮心熬研如脂}\\
右三味为散。强人飲服半錢匕。羸者減之。病在膈上者吐出。在膈下者瀉出。若下多不止。飲冷水一杯則定。

寒実結胸。无熱証者。与三物白散。141

欬而胸滿。振寒。脉數。咽乾。不渴。时出濁唾腥臭。久久吐膿如米粥者。为肺癰。桔梗白散主之。

\section{走馬湯}

巴豆{\scriptsize 二枚去皮心熬} 杏仁{\scriptsize 二枚{\khaai 去皮尖}}\\
右二味。取綿纏。槌令碎。熱湯二合。捻取白汁飲之。当下。老小量之。通治飛尸鬼{\khaai 疰}擊病。{\scriptsize {\khaai 并見外臺。}}

卒疝。走馬湯主之。

%\section{三物備急丸(備急散)}
%《傷寒論》与《金匱要略》中均未見備急丸,但是有三物備急丸,另外吳本有備急散,鄧本无。回去再查一下《類聚方》。

%大黄{\scriptsize 一兩} 乾薑{\scriptsize 一兩} 巴豆{\scriptsize 一兩。去皮心熬別研如脂}\\
%右藥各須精新。先擣大黄乾薑为末。研巴豆内中。合治一千杵。用为散。蜜和为丸亦佳。密器中貯之。莫令歇。主心腹諸卒暴百病。若中惡客忤。心腹脹滿。卒痛如錐刀刺痛。气急。口噤。停尸。卒死者。以煖水若酒。服大豆許三四丸。或不下。捧頭起。灌令下咽。須臾差。如未差。更与三丸。当腹中鳴。即吐下。便差。若口噤。亦須折齒灌之。{\scriptsize 見千金。云司空裴秀为散用。亦可先和成汁。乃傾口中。令從齒間得入。至食驗。}{\wuben}

\section{礬石湯}

礬石{\scriptsize 二兩} 
右一味。以漿水一斗五升。煎三五沸。浸腳良。

礬石湯。治腳气衝心。

\section{硝石礬石散}

硝石{ }礬石{\scriptsize 燒各等分} 
右二味。为散。以大麥粥汁。和服方寸匕。日三服。病隨大小便去。小便正黄。大便正黑。是{\sungii 𠊱}也。

黄家。日晡所发熱。而反惡寒。此为女勞得之。膀胱急。少腹滿。身{\sungii 𥁞}黄。額上黑。足下熱。因作黑疸。其腹胀如水狀。大便必黑。时溏。此女勞之病。非水也。腹滿者難治。硝石礬石散主之。

\section{礬石丸}

礬石{\scriptsize 三分燒} 杏仁{\scriptsize 一分去皮尖熬}\\
右二味。末之。煉蜜和丸。如棗核大。内臟中劇者。再内之。

婦人經水閉不利。臓堅癖不止。中有乾血。下白物。礬石丸主之。

\section{蛇床子散}

蛇床子仁\\
右一味。末之。以白粉少許。和令相得。如棗大。棉裹。内之。自然温矣。

温陰中坐藥。蛇床子散。

\section{竹枼石膏湯}

竹枼{\scriptsize 二把} 半夏{\scriptsize 半升洗
	\footnote{
		「半升洗」千金方作「一升洗」。
	}
} 麥門冬{\scriptsize 一升去心} 甘草{\scriptsize 二兩炙} 人参{\scriptsize 二兩
	\footnote{
		「二兩」《玉函》作「三兩」
	}
} 石膏{\scriptsize 十六兩碎} 粳米{\scriptsize 半升}\\
右七味。以水一斗。煮取六升。去滓。内粳米。煮米熟。湯成。去米。温服一升。日三服。

傷寒觧後。虗羸少气。气逆欲吐。竹枼石膏湯主之。397

\section{麥門冬湯}

麥門冬{\scriptsize 七升去心} 半夏{\scriptsize 一升洗} 人参{\scriptsize 二兩} 甘草{\scriptsize 二兩炙} 粳米{\scriptsize 三合} 大棗{\scriptsize 十二枚擘}\\
右六味。㕮咀。以水一斗二升。煮取六升。去滓。温服一升。日三夜一服。

火逆上气。咽㗋不利。止逆下气者。麥門冬湯主之。

\section{雄黄}

雄黄一味为末。筒瓦二枚合之。燒。向肛熏之。

狐惑之为病。狀如傷寒。默默欲眠。目不得閉。卧起不安。蝕於㗋为惑。蝕於陰为狐。不欲飲食。惡聞食臭。其面目乍赤乍黑乍白。蝕於上部則聲喝。甘草瀉心湯主之。蝕於下部則咽乾。苦参湯洗之。蝕於肛者。雄黄熏之。

\section{頭風摩散}

大附子{\scriptsize 一枚炮去皮} 鹽{\scriptsize 等分}\\
右二味。为散。沐了。以方寸匕摩疢上。令藥力行。

頭風摩散方。

\section{皂莢丸}

皂莢{\scriptsize 一挺刮去皮炙焦去子}\\
右一味。末之。蜜丸梧桐子大。以棗膏和湯服三丸。日三夜一服。{\wuben}

皂莢{\scriptsize 八兩刮去皮用酥炙}\\
右一味。末之。蜜丸梧子大。以棗膏和湯服三丸。日三夜一服。{\dengben}

欬逆。气上衝。唾濁。但坐不得卧。皂莢丸主之。

\section{葦湯(葦莖湯)}

葦枼{\scriptsize 二升切} 薏苡仁{\scriptsize 半升} 桃仁{\scriptsize 五十枚去皮尖} 瓜瓣{\scriptsize 半升}\\
右四味。以水一斗。先煮葦。得五升。去滓。内諸藥。煮取二升。分温再服。当吐如膿。{\scriptsize 見千金。}{\wuben}

葦莖{\scriptsize 二升} 薏苡仁{\scriptsize 半升} 桃仁{\scriptsize 五十枚} 瓜瓣{\scriptsize 半升}\\
右四味。以水一斗。先煮葦莖。得五升。去滓。内諸藥。煮取二升。服一升。再服。当吐如膿。{\dengben}

治肺癰。葦湯。

%葦莖湯。治欬。有微熱。煩滿。胸中甲錯。是为肺癰。

\section{当歸生薑羊肉湯}

当歸{\scriptsize 三兩} 生薑{\scriptsize 五兩切} 羊肉{\scriptsize 十六兩}\\
右三味。㕮咀。以水八升。煮取三升。去滓。温服七合。日三服。若寒多者。加生薑成十六兩。痛多而嘔者。加橘皮二兩。术一兩。加生薑者。亦加水五升。煮取三升二合。服之。

寒疝。腹中痛。及脇痛。裏急者。当歸生薑羊肉湯主之。

婦人產後。腹中㽲痛。当歸生薑羊肉湯主之。并治腹中寒疝。虗勞不足。

\section{蒲灰散}

蒲灰{\scriptsize 七分} 滑石{\scriptsize 三分}\\
右二味。杵为散。飲服方寸匕。日三服。

小便不利。蒲灰散主之。滑石白魚散。茯苓戎鹽湯并主之。

厥而皮水者。蒲灰散主之。

\section{滑石白魚散}

滑石{\scriptsize 二分} 亂髮{\scriptsize 二分燒} 白魚{\scriptsize 二分}\\
右三味。杵为散。飲服方寸匕。日三服。

小便不利。蒲灰散主之。滑石白魚散。茯苓戎鹽湯并主之。

\section{豬膏髮煎}

猪膏{\scriptsize 八兩} 亂髮{\scriptsize 如雞子大三枚}\\
右二味。和膏中煎之。髪消藥成。分再服。病從小便去。

諸黄。豬膏髮煎主之。

胃气下泄。陰吹而正喧。此穀气之実也。膏髮煎導之。

\section{柏枼湯}

柏葉{\scriptsize 三兩} 艾{\scriptsize 三把} 乾薑{\scriptsize 三兩}\\
右三味。㕮咀。以水五升。取馬通汁一升。合煮取一升。去滓。分温再服。

吐血不止者。柏枼湯主之。

\section{黄土湯}

甘草{\scriptsize 三兩炙} 乾地黄{\scriptsize 三兩} 白术{\scriptsize 三兩} 附子{\scriptsize 三兩炮去皮破八片} 阿膠{\scriptsize 三兩} 黄芩{\scriptsize 三兩} 竃中黄土{\scriptsize 八兩}\\
右七味。㕮咀。以水八升。煮取三升。去滓。分温二服。

下血。先見血。後見便。此近血也。先見便。後見血。此遠血也。遠血。黄土湯主之。近血。赤小豆当歸散主之。

\section{雞屎白散}

鷄屎白\\
右一位。为散。取方寸匕。以水六合和温服。

轉筋之为病。其人臂腳直。脉上下行。微弦。轉筋入腹者。雞屎白散主之。

\section{蜘蛛散}

蜘蛛{\scriptsize 十四枚熬焦} 桂枝{\scriptsize 半兩去皮}\\
右二味。为散。取八分一匕。飲和服。日再服。蜜丸亦得。

陰狐疝气者。偏有大小。时时上下。蜘蛛散主之。

\section{当歸芍藥散}

当歸{\scriptsize 四兩} 芍藥{\scriptsize 十六兩} 茯苓{\scriptsize 四兩} 白术{\scriptsize 四兩} 澤瀉{\scriptsize 八兩} 芎藭{\scriptsize 八兩一作三兩}\\
右六味。杵为散。取方寸匕。飲和服。酒和。日三服。

婦人懷娠。腹中㽲痛。当歸芍藥散主之。

婦人腹中諸疾痛。当歸芍藥散主之。

\section{当歸貝母苦参丸}

当歸{\scriptsize 四兩} 貝母{\scriptsize 四兩} 苦参{\scriptsize 四兩}\\
右三味。末之。煉蜜和丸如小豆大。飲服三丸。加至十丸。男子加滑石半兩。

婦人妊娠。小便難。飲食如故。当歸貝母苦参丸主之。

\section{狼牙湯}

狼牙{\scriptsize 三兩}\\
右一味。㕮咀。以水四升。煮取半升。以棉裹筋。大如繭。浸湯。瀝陰中。日四遍。

婦人陰中蝕瘡爛。狼牙湯洗之。

\section{小兒疳虫蝕齒方}

雄黄{ }葶藶{\scriptsize 各少許}\\
右二味。末之。取臘月豬脂合鎔。以槐枝棉裹頭四五枚。點藥烙之。{\scriptsize 疑非仲景方。}

\section{炙甘草湯}

甘草{\scriptsize 四兩炙} 生薑{\scriptsize 三兩切} 人参{\scriptsize 二兩} 生地黄{\scriptsize 十六兩} 桂枝{\scriptsize 三兩去皮} 阿膠{\scriptsize 二兩} 麥門冬{\scriptsize 半升去心} 麻仁{\scriptsize 半升} 大棗{\scriptsize 三十枚擘}\\
右九味。以清酒七升。水八升。先煮八味。取三升。去滓。内膠。烊消{\sungii 𥁞}。温服一升。日三服。一名復脉湯。

傷寒。脉結代。心動悸。炙甘草湯主之。177

虗勞不足。汗出而悶。脉結。心悸。行動如常。不出百日。危急者。十一日死。炙甘草湯主之。

肺痿。涎唾多。心中温温液液者。炙甘草湯主之。

\section{当歸四逆湯}

当歸{\scriptsize 三兩} 桂枝{\scriptsize 三兩去皮} 芍藥{\scriptsize 三兩} 細辛{\scriptsize 三兩} 甘草{\scriptsize 二兩炙} 通草{\scriptsize 二兩} 大棗{\scriptsize 二十五枚擘一法十二枚}\\
右七味。以水八升。煮取三升。去滓。温服一升。日三服。

手足厥寒。脉細欲絕。当歸四逆湯主之。若其人内有久寒。当歸四逆加吳茱萸生薑湯主之。351.352

\section{当歸四逆加吳茱萸生薑湯}

当歸{\scriptsize 三兩} 芍藥{\scriptsize 三兩} 甘草{\scriptsize 二兩炙} 通草{\scriptsize 二兩} 桂枝{\scriptsize 三兩去皮} 細辛{\scriptsize 三兩} 生薑{\scriptsize 八兩切} 吳茱萸{\scriptsize 二升} 大棗{\scriptsize 二十五枚擘}\\
右九味。以水六升清酒六升合。煮取五升。去滓。分温五服。{\scriptsize 一方水酒各四升。}

手足厥寒。脉細欲絕。当歸四逆湯主之。若其人内有久寒。当歸四逆加吳茱萸生薑湯主之。351.352

\section{麻黄連軺赤小豆湯}

傷寒。瘀熱在裏。身必发黄。麻黄連軺赤小豆湯主之。262

\section{四逆散*}

甘草{\scriptsize 炙} 枳実{\scriptsize 破水漬炙乾} 柴胡{ }芍藥\\
右四味。各十分。擣篩。白飲和服方寸匕。日三服。 

欬者。加五味子乾薑各五分。并主下利。\\
悸者。加桂枝五分。\\
小便不利者。加茯苓五分。\\
腹中痛者。加附子一枚炮令坼。\\
泄利下重者。先以水五升。煮薤白三升。煮取三升。去滓。以散三方寸匕。内湯中。煮取一升半。分温再服。

少陰病。四逆。其人或欬。或悸。或小便不利。或腹中痛。或泄利下重。四逆散主之。318

\section{射干麻黄湯}

射干{\scriptsize 十三枚一法三枚} 麻黄{\scriptsize 四兩去節} 生薑{\scriptsize 四兩切} 細辛{\scriptsize 三兩} 紫菀{\scriptsize 三兩} 款冬花{\scriptsize 三兩} 五味子{\scriptsize 半升} 半夏{\scriptsize 大者八枚洗一法半升} 大棗{\scriptsize 七枚擘}\\
右九味。㕮咀。以水一斗二升。先煮麻黄兩沸。去上沫。内諸藥。煮取三升。去滓。分温三服。

欬而上气。㗋中水雞聲。射干麻黄湯主之。

\section{桂枝芍藥知母湯}

桂枝{\scriptsize 四兩去皮} 芍藥{\scriptsize 三兩} 甘草{\scriptsize 二兩炙} 麻黄{\scriptsize 二兩去節} 生薑{\scriptsize 五兩切} 白术{\scriptsize 五兩} 知母{\scriptsize 四兩} 防風{\scriptsize 四兩} 附子{\scriptsize 二兩炮去皮破}\\
右九味。㕮咀。以水七升。煮取二升。去滓。温服七合。日三服。

諸肢節疼痛。身体魁羸。腳腫如脱。頭眩短气。温温欲吐。桂枝芍藥知母湯主之。

\section{内補当歸建中湯*}

当歸{\scriptsize 四兩} 桂枝{\scriptsize 去皮三兩} 芍藥{\scriptsize 六兩} 生薑{\scriptsize 三兩切} 甘草{\scriptsize 二兩炙} 大棗{\scriptsize 十二枚擘}\\
右六味。㕮咀。以水八升。煮取二升。去滓。分温三服。一日令{\sungii 𥁞}。若大虗。加飴糖六兩。湯成内之。於火上煖令飴消。若无生薑。以乾薑代之。若其人去血過多。崩傷内衄不止。加地黄六兩。阿膠二兩。合八種。湯成。去滓。内阿膠。若无当歸。以芎藭代之。

治婦人產後。虗羸不足。腹中刺痛不止。吸吸少气。或苦少腹拘急攣痛引腰背。不能食飲。產後一月。日得服四五剂为善。令人强壯。内補当歸建中湯。

\section{續命湯}

麻黄{\scriptsize 三兩去節} 桂枝{\scriptsize 三兩去皮} 当歸{\scriptsize 三兩} 人参{\scriptsize 三兩} 石膏{\scriptsize 三兩碎綿裹} 乾薑{\scriptsize 三兩} 甘草{\scriptsize 三兩炙} 芎藭{\scriptsize 一兩} 杏仁{\scriptsize 四十枚去皮尖}\\
右九味。㕮咀。以水一斗。煮取四升。去滓。分温一服。当小汗。薄覆脊。憑几坐。汗出則愈。不汗更服。无所禁。勿当風。并治但伏不得卧。欬逆上气。面目洪腫。

續命湯。治中風痱。身体不能自收。口不能言。冒昧不知痛処。或拘急。不得轉側。{\scriptsize 姚云。与大續命同。兼治婦人產後去血者。及老人小兒。}

\section{烏梅丸}

烏梅{\scriptsize 三百枚} 細辛{\scriptsize 六兩} 乾薑{\scriptsize 十兩} 黄連{\scriptsize 十六兩} 当歸{\scriptsize 四兩} 附子{\scriptsize 六枚炮去皮} 蜀椒{\scriptsize 四兩去目及閉口者汗} 桂枝{\scriptsize 六兩去皮} 人参{\scriptsize 六兩} 黄蘗{\scriptsize 六倆}\\
右十一味。{\khaai 各}異擣篩。合治之。以苦酒漬烏梅一宿。去核。蒸之五斗米下。飯熟。擣成泥。和藥相得。内臼中。与蜜杵三千下。丸如梧桐子大。先食飲。服十丸。日三服。稍加{\khaai 至}二十丸。
	\footnote{
		「附子六枚」同吳本,趙本作「附子六兩」。「杵三千下」同吳本,趙本作「杵兩千下」。
	}

傷寒。脉微而厥。至七八日。膚冷。其人躁。无暫安时者。此为臓厥。非蛔厥也。蛔厥者。其人当吐蛔。今病者靜。而復时煩。此为臓寒。蛔上入膈。故煩。須臾復止。得食而嘔。又煩者。蛔聞食臭出。其人常自吐蛔。蛔厥者。烏梅丸主之。338

\section{大黄䗪虫丸}

大黄{\scriptsize 十分蒸} 黄芩{\scriptsize 二兩} 甘草{\scriptsize 三兩炙} 桃仁{\scriptsize 一升去皮尖熬} 杏仁{\scriptsize 一升同上法} 芍藥{\scriptsize 四兩} 乾地黄{\scriptsize 十兩} 乾漆{\scriptsize 一兩熬} 蝱虫{\scriptsize 一升去翅足熬} 水蛭{\scriptsize 一百枚熬} 蛴螬{\scriptsize 一升熬} 䗪虫{\scriptsize 半升熬}\\
右十二味。末之。煉蜜和丸小豆大。酒飲服五丸。日三服。

五勞。虗極。羸瘦。腹滿。不能飲食。食傷。憂傷。飲傷。房室傷。飢傷。勞傷。經絡榮衛气傷。内有乾血。肌膚甲錯。兩目黯黑。緩中補虗。大黄䗪虫丸主之。

\section{麻黄升麻湯}

麻黄{\scriptsize 二兩半去節} 升麻{\scriptsize 一兩一分} 当歸{\scriptsize 一兩一分} 知母{\scriptsize 十八銖} 黄芩{\scriptsize 十八銖} 萎蕤{\scriptsize 十八銖一作菖蒲} 芍藥{\scriptsize 六銖} 天門冬{\scriptsize 六銖去心} 桂枝{\scriptsize 六銖去皮} 茯苓{\scriptsize 六銖} 甘草{\scriptsize 六銖炙} 石膏{\scriptsize 六銖碎綿裹} 白术{\scriptsize 六銖} 乾薑{\scriptsize 六銖}\\
右十四味。以水一斗。先煮麻黄一兩沸。去上沫。内諸藥。煮取三升。去滓。分温三服。相去如炊三斗米頃令{\sungii 𥁞}。汗出愈。

傷寒六七日。大下後。{\khaai 寸}脉沈遲。手足厥逆。下部脉不至。咽㗋不利。唾膿血。泄利不止者。为難治。麻黄升麻湯主之。357

\section{豬膚湯}

豬膚{\scriptsize 十六兩}\\
右一味。以水一斗。煮取五升。去滓。内白蜜一升。白粉五合。熬香。和令相得。温分六服。

少陰病。下利。咽痛。胸滿。心煩。豬膚湯主之。310

\section{燒裩散}

婦人中裩近陰処燒灰\\
右一味。水和服方寸匕。日三。小便即利。陰頭微腫。此为愈。婦人病。取男子裩燒服。

傷寒陰易之为病。其人身体重。少气。少腹裏急。或引陰中拘攣。熱上衝胸。頭重不欲舉。眼中生眵。{\khaai 眼胞赤。}膝脛拘急。燒裩散主之。392

\section{瓜蒂湯}

瓜蒂{\scriptsize 二七枚}\\
右一味。以水一升。煮取五合。去滓。頓服。

太陽中暍。身熱疼重。而脉微弱。此以夏月傷冷水。水行皮膚中所致也。瓜蒂湯主之。

諸黄。瓜蒂湯主之。

\section{百合知母湯}

百合{\scriptsize 七枚擘} 知母{\scriptsize 三兩切}\\
右二味。先以水洗百合。漬一宿。当白沫出。去其水。更以泉水二升。煮取一升。去滓。別以泉水二升煮知母。取一升。去滓。後合和。重煎取一升五合。分温再服。

治百合病。发汗後者。百合知母湯。

\section{百合滑石代赭湯}

百合{\scriptsize 七枚擘} 滑石{\scriptsize 三兩碎綿裹} 代赭{\scriptsize 如彈丸一枚碎綿裹}\\
右三味。先以水洗百合。漬一宿。当白沫出。去其水。更以泉水二升。煮取一升。去滓。別以泉水二升煮滑石。代赭。取一升。去滓。後合和。重煎取一升五合。分温再服。

治百合病。下之後者。百合滑石代赭湯。

\section{百合雞子湯}

百合{\scriptsize 七枚擘} 雞子黄{\scriptsize 一枚}\\
右二味。先以水洗百合。漬一宿。当白沫出。去其水。更以泉水二升。煮取一升。去滓。内雞子黄。攪令調。分温再服。

治百合病。吐之後者。百合雞子湯。

\section{百合地黄湯}

百合{\scriptsize 七枚擘} 生地黄汁{\scriptsize 一升}\\
右二味。先以水洗百合。漬一宿。当白沫出。去其水。更以泉水二升。煮取一升。去滓。内地黄汁。煮取一升五合。分温再服。中病勿更服。大便当如漆。

治百合病。不經吐下发汗。病形如初者。百合地黄湯。

\section{百合洗方}

百合{\scriptsize 一升}\\
右一味。以水一斗。漬之一宿。以洗身。洗已。食煮餅。勿与鹽豉也。

治百合病。一月不觧。變成渴者。百合洗方。渴不差者。栝蔞牡蛎散。

\section{栝蔞牡蛎散}

栝蔞根{ }牡蛎{\scriptsize 熬等分}\\
右二味。杵为散。飲服方寸匕。日三服。

治百合病。一月不觧。變成渴者。百合洗方。渴不差者。栝蔞牡蛎散。

\section{百合滑石散}

百合{\scriptsize 一兩炙} 滑石{\scriptsize 三兩}\\
右二味。杵为散。飲服方寸匕。日三服。当微利者止。勿服之。熱則除。

治百合病。變发熱者。百合滑石散。

\section{}

治百合病。變腹中滿痛者。但取百合根隨多少。熬令黄色。擣篩为散。飲服方寸匕。日三。滿消痛止。

\section{苦参湯}

狐惑之为病。狀如傷寒。默默欲眠。目不得閉。卧起不安。蝕於㗋为惑。蝕於陰为狐。不欲飲食。惡聞食臭。其面目乍赤乍黑乍白。蝕於上部則聲喝。甘草瀉心湯主之。蝕於下部則咽乾。苦参湯洗之。蝕於肛者。雄黄熏之。

\section{赤小豆当歸散}

赤小豆{\scriptsize 三升浸令芽出曝乾} 当歸{\scriptsize 三兩}\\
右二味。杵为散。漿水服方寸匕。日三服。

病者脉數。无熱。微煩。默默。但欲卧。汗出。初得之三四日。目赤如鳩眼。七八日目四眥黑。若能食者。膿已成也。赤小豆当歸散主之。

下血。先見血。後見便。此近血也。先見便。後見血。此遠血也。遠血。黄土湯主之。近血。赤小豆当歸散主之。

\section{升麻鱉甲湯}

升麻{\scriptsize 二兩} 当歸{\scriptsize 一兩} 蜀椒{\scriptsize 一兩汗} 鱉甲{\scriptsize 如手大一片炙} 甘草{\scriptsize 二兩炙} 雄黄{\scriptsize 半兩研}\\
右六味。㕮咀。以水四升。煮取一升。去滓。頓服之。老小再服。取汗。陰毒去雄黄。蜀椒。{\scriptsize 肘後。千金。陽毒用升麻湯。无鱉甲。有桂。陰毒用甘草湯。无雄黄。}

陽毒之为病。面赤斑斑如錦文。㗋咽痛。唾膿血。五日可治。七日不可治。陰毒之为病。面目青。身痛。狀如被打。㗋咽痛。死生与陽毒同。升麻鱉甲湯并主之。

%\section{升麻鱉甲湯去雄黄蜀椒}

\section{鱉甲煎丸}

鱉甲{\scriptsize 十二分炙} 烏扇{\scriptsize 三分燒} 黄芩{\scriptsize 三分} 柴胡{\scriptsize 六分} 鼠婦{\scriptsize 三分熬} 乾薑{\scriptsize 三分} 大黄{\scriptsize 三分} 芍藥{\scriptsize 五分} 桂枝{\scriptsize 三分去皮} 葶藶{\scriptsize 一分熬} 石韋{\scriptsize 三分去毛} 厚朴{\scriptsize 三分} 牡丹{\scriptsize 五分去心} 瞿麥{\scriptsize 二分} 紫威{\scriptsize 三分} 半夏{\scriptsize 一分洗} 人参{\scriptsize 一分} 䗪虫{\scriptsize 五分熬} 阿膠{\scriptsize 三分炙} 蜂窠{\scriptsize 四分熬} 赤硝{\scriptsize 十二分} 蜣蜋{\scriptsize 六分熬} 桃仁{\scriptsize 二分去皮尖熬焦}\\
右二十三味。为末。取煆竈下灰一斗。清酒一斛五斗浸灰。{\sungii 𠊱}酒{\sungii 𥁞}一半。着鱉甲於中煮。令泛爛如膠漆。絞取汁。内諸藥煎。为丸如梧桐子大。空心服七丸。日三服。{\scriptsize 千金用鱉甲十二片。又有海藻三分。大戟一分。䗪虫五分。无鼠婦赤硝二味。以鱉甲煎。和諸藥为丸。}

問曰。瘧以月一日发。当以十五日愈。設不差。当月{\sungii 𥁞}觧也。如其不差。当云何。\\
師曰。此結为癥瘕。名曰瘧母。急治之。宜鱉甲煎丸。

\section{矦氏黑散}

菊花{\scriptsize 四十分} 白术{\scriptsize 十分} 細辛{\scriptsize 三分} 茯苓{\scriptsize 三分} 牡蛎{\scriptsize 三分熬} 桔梗{\scriptsize 八分} 防風{\scriptsize 十分} 人参{\scriptsize 三分} 礬石{\scriptsize 三分熬} 黄芩{\scriptsize 五分} 当歸{\scriptsize 三分} 乾薑{\scriptsize 三分} 芎窮{\scriptsize 三分} 桂枝{\scriptsize 三分去皮}\\
右十四味。杵为散。酒服方寸匕。日一服。初服二十日。温酒下之。禁一切魚肉大蒜。常宜冷食。六十日止。即藥積在腹中不下也。熱食即下矣。冷食自能助藥力。{\scriptsize 外臺有鐘乳礬石各三分。无桔梗。}

大風。四肢煩重。心中惡寒不足者。矦氏黑散主之。{\scriptsize 外臺治風癲。}

\section{風引湯}

大黄{\scriptsize 四兩} 乾薑{\scriptsize 四兩} 龙骨{\scriptsize 四兩} 桂枝{\scriptsize 三兩去皮} 甘草{二兩炙} 牡蛎{\scriptsize 二兩熬} 凝水石{\scriptsize 六兩} 滑石{\scriptsize 六兩} 赤石脂{\scriptsize 六兩} 白石脂{\scriptsize 六兩} 石膏{\scriptsize 六兩} 紫石英{\scriptsize 六兩}\\
右十二味。杵粗。篩。以韋囊盛之。取三指撮井華水三升。煮三沸。去滓。温服一升。{\scriptsize 深師云。治大人風引。少小驚癇。瘈瘲日數十发。醫所不療。除熱方。巢源。腳气宜風引湯。}

風引湯。除熱。主癱癇。

\section{防己地黄湯}

防己{\scriptsize 一分} 桂枝{\scriptsize 三分去皮} 防風{\scriptsize 三分} 甘草{\scriptsize 二分炙}\\
右四味。㕮咀。以酒一杯。漬之一宿。絞取汁。取生地黄二斤。㕮咀。蒸之如斗米飯久。以銅器盛其汁。更絞地黄等汁。和分再服。

病如狂狀。妄行。獨語不休。无寒熱。其脉浮。防己地黄湯主之。

%\hangindent 1em
%\hangafter=0
%言語狂錯。眼目茫茫。或見鬼。精神昏亂。防己地黄湯。{\qianjin}

\section{三黄湯*}

麻黄{\scriptsize 去節五分} 獨活{\scriptsize 四分} 細辛{\scriptsize 二分} 黄耆{\scriptsize 二分} 黄芩{\scriptsize 三分}\\
右五味。㕮咀。以水六升。煮取二升。去滓。分温三服。一服小汗。兩服大汗。心熱加大黄二分。腹痛加枳実一枚。气逆加人参三分。悸加牡蛎三分。渴加栝蔞根三分。先有寒加附子一枚。{\scriptsize 見千金。}

治中風。手足拘急。百節疼痛。煩熱心亂。惡寒。經日不欲飲食。三黄湯。

\section{薯蕷丸}

薯蕷{\scriptsize 三十分} 当歸{\scriptsize 十分} 桂枝{\scriptsize 十分去皮} 麹{\scriptsize 十分} 乾地黄{\scriptsize 十分} 大豆黄卷{\scriptsize 十分} 甘草{\scriptsize 二十八分炙} 人参{\scriptsize 七分} 芎窮{\scriptsize 六分} 芍藥{\scriptsize 六分} 白术{\scriptsize 六分} 麥門冬{\scriptsize 六分去心} 杏仁{\scriptsize 六分去皮尖熬} 柴胡{\scriptsize 五分} 桔梗{\scriptsize 五分} 茯苓{\scriptsize 五分} 阿膠{\scriptsize 七分炙} 乾薑{\scriptsize 三分} 白斂{\scriptsize 二分} 防風{\scriptsize 六分} 大棗{\scriptsize 百枚为膏}\\
右二十一味。末之。煉蜜和丸。如彈子大。空腹。酒服一丸。一百丸为剂。
	\footnote{
		「阿膠七分炙」吳本作「阿膠炙各七分」,鄧本作「阿膠七分」。
	}

虗勞。諸不足。風气百疾。薯蕷丸主之。

\section{獺肝散}

獺肝一具。炙乾。末之。水服方寸匕。日三服。{\scriptsize 見肘後。恐非仲景方。}{\wuben}

%獺肝{\scriptsize 一具}\\
%陰乾。末之。水服方寸匕。日三服。{\dengben}
%手邊暫时沒有資料,此方等開學後去學校查書再校。

治冷勞。又主鬼疰。一門相染。獺肝散。

\section{厚朴麻黄湯}

厚朴{\scriptsize 五兩炙} 麻黄{\scriptsize 四兩去節} 石膏{\scriptsize 如雞子大碎} 杏仁{\scriptsize 半升去皮尖} 乾薑{\scriptsize 二兩} 細辛{\scriptsize 二兩} 小麥{\scriptsize 一升} 五味子{\scriptsize 半升} 半夏{\scriptsize 半升洗}\\
右九味。㕮咀。以水一斗二升。先煮小麥熟。去滓。内諸藥。煮取三升。去滓。温服一升。日三服。

上气。脉浮者。厚朴麻黄湯主之。脉沈者。澤漆湯主之。

\section{澤漆湯}

澤瀉{\scriptsize 四十八兩\footnote{「四十八兩」吳本、鄧本均作「三斤」,編者改。}以東流水五斗煮取一斗五升} 半夏{\scriptsize 半升洗} 紫参{\scriptsize 五兩一作紫菀} 生薑{\scriptsize 五兩切} 白前{\scriptsize 五兩} 甘草{\scriptsize 三兩炙} 黄芩{\scriptsize 三兩} 人参{\scriptsize 三兩} 桂枝{\scriptsize 三兩去皮}\\
右九味。㕮咀。内澤漆汁中煮取五升。去滓。温服五合。至夜{\sungii 𥁞}。

上气。脉浮者。厚朴麻黄湯主之。脉沈者。澤漆湯主之。{\wuben}

欬而脉浮者。厚朴麻黄湯主之。脉沈者。澤漆湯主之。{\dengben}

\section{奔豚湯}

甘草{\scriptsize 二兩{\khaai 炙}} 芎藭{\scriptsize 二兩} 当歸{\scriptsize 二兩} 半夏{\scriptsize 四兩{\khaai 洗}} 黄芩{\scriptsize 二兩} 生葛{\scriptsize 五兩} 芍藥{\scriptsize 二兩} 生薑{\scriptsize 四兩{\khaai 切}} 甘李根白皮{\scriptsize 一升{\khaai 切}}

右九味。{\khaai 㕮咀。}以水二斗。煮取五升。{\khaai 去滓。}温服一升。日三夜一服。

奔豚。气上衝胸。腹痛。往來寒熱。奔豚湯主之。

\section{烏頭赤石脂丸}

烏頭{\scriptsize 一分炮去皮} 附子{\scriptsize 半兩炮去皮一法一分} 赤石脂{\scriptsize 一兩一法二分} 乾薑{\scriptsize 一兩一法二分} 蜀椒{\scriptsize 一兩汗一法二分}\\
右五味。末之。蜜丸如梧子大。先食服一丸。日三服。不知。稍增之。

心痛徹背。背痛徹心。烏頭赤石脂丸主之。

\section{九痛丸}

附子{\scriptsize 三兩炮去皮} 巴豆{\scriptsize 一兩去皮心熬研如脂} 生狼牙{\scriptsize 一兩炙令香} 人参{\scriptsize 一兩} 乾薑{\scriptsize 一兩} 吳茱萸{\scriptsize 一兩}\\
右六味。末之。煉蜜和丸如梧子大。酒下。强人初服三丸。日一服。弱者二丸。兼治卒中惡。腹脹痛。口不能言。又治連年積冷。流注心胸痛。并冷腫上气。落馬。墜車。血疾等。皆主之。禁口如常法。{\wuben}

{\dengben}%待錄入

九痛丸。治九種心痛。{\khaai 一虫心痛。二疰心痛。三風心痛。四悸心痛。五食心痛。六飲心痛。七冷心痛。八熱心痛。九去來心痛。}

\section{温經湯}

吳茱萸{\scriptsize 三兩} 当歸{\scriptsize 二兩} 芎藭{\scriptsize 二兩} 芍藥{\scriptsize 二兩} 麥門冬{\scriptsize 一升去心} 人参{\scriptsize 二兩} 桂枝{\scriptsize 二兩去皮} 阿膠{\scriptsize 二兩} 牡丹{\scriptsize 二兩去心} 生薑{\scriptsize 二兩} 甘草{\scriptsize 二兩炙} 半夏{\footnote{吳本无半夏。}}{\scriptsize 半升}\\
右十二味。㕮咀。以水一斗。煮取三升。去滓。分温三服。\\
亦主婦人少腹寒。久不作軀。兼主崩中去血。或月水來過多。及過期不來。\footnote{「作軀」同吳本,鄧本作「受胎」。}

問曰。婦人年五十所。病下血。數十日不止。暮即发熱。少腹裏急{\khaai 痛}。腹滿。手掌煩熱。唇口乾燥。何也。\\
師曰。此病屬帶下。何以故。曾經半產。瘀血在少腹不去。何以知之。其証唇口乾燥。故知之。当以温經湯主之。

\section{旋覆花湯}

旋覆花{\scriptsize 三兩} 蔥{\scriptsize 十四莖} 新絳{\scriptsize 少許}\\
右三味。以水三升。煮取一升。去滓。頓服之。

肝著。其人常欲蹈其胸上。先未苦时。但欲飲熱。旋覆花湯主之。

寸口脉弦而大。弦則为減。大則为芤。減則为寒。芤則为虗。寒虗相摶。脉即名为革。婦人則半產漏下。旋覆花湯主之。

\section{厚朴大黄湯}

厚朴{\scriptsize 一尺去皮炙} 大黄{\scriptsize 六兩} 枳実{\scriptsize 四枚炙}\\
右三味。㕮咀。以水五升。煮取二升。去滓。分温再服。
%厚朴一尺可能是錯誤,等開學後查。2021年3月4日查,吳本、鄧本均作一尺。

支飲。胸滿者。厚朴大黄湯主之。

\section{紫参湯}

紫参{\scriptsize 八兩} 甘草{\scriptsize 三兩炙}\\
右二味。㕮咀。以水五升。先煮紫参。取二升。内甘草。煮取一升半。去滓。分温三服。{\scriptsize 疑非仲景方。}

下利。肺痛。紫参湯主之。

\section{訶梨勒散}

訶梨勒{\scriptsize 十枚以麵裹煻灰火中煨之令麵熟去核}\\
右一味。細为散。粥飲和。頓服之。{\scriptsize 疑非仲景方。}

气利。訶梨勒散主之。

\section{王不留行散}

王不留行{\scriptsize 十分八月八日采之} 蒴藋細枼{\scriptsize 十分七月七日采之} 桑東南根{\scriptsize 如指大白皮十分三月三日采} 甘草{\scriptsize 十八分炙} 蜀椒{\scriptsize 三分去目及閉口者汗} 黄芩{\scriptsize 二分} 乾薑{\scriptsize 二分} 芍藥{\scriptsize 二分} 厚朴{\scriptsize 二分炙}\\
右九味。桑東南根以上三味。燒为灰。存性。勿令灰過。各別杵篩。合治之为散。病者与方寸匕服之。小瘡即粉之。中大瘡但服之。產後亦可服。如風寒。桑根勿取之。前三物皆陰乾百日。

病金瘡。王不留行散主之。

\section{当歸散}

当歸{\scriptsize 十六兩} 黄芩{\scriptsize 十六兩} 芍藥{\scriptsize 十六兩} 芎窮{\scriptsize 十六兩} 白术{\scriptsize 八兩}\\
右五味。杵为散。酒飲服方寸匕。日再服。妊娠常服即易產。胎无苦疾。產後百病悉主之。

婦人妊娠。宜服当歸散。

\section{白术散}

白术{\scriptsize 四分} 芎窮{\scriptsize 四分} 蜀椒{\scriptsize 三分汗} 牡蛎{\scriptsize 二分熬} \\
右四味。杵为散。酒服一錢匕。日三夜一服。但苦痛。加芍藥。心下毒痛。倍加芎窮。心煩吐痛。不能食飲。加細辛一兩。半夏錢大二十枚。服之後。更以醋漿水服之。若嘔。亦以醋漿水服之。復不觧者。小麥汁服之。已後若渴者。大麥粥服之。病雖愈。{\sungii 𥁞}服之勿置。{\scriptsize 見外臺出古今錄驗。}

妊娠養胎。白术散主之。

\section{竹枼湯*}

竹枼{\scriptsize 一把} 葛根{\scriptsize 三兩} 防風{\scriptsize 一兩} 桔梗{\scriptsize 一兩} 桂枝{\scriptsize 一兩去皮} 人参{\scriptsize 一兩} 甘草{\scriptsize 一兩炙} 附子{\scriptsize 一枚炮去皮破八片} 大棗{\scriptsize 十五枚擘} 生薑{\scriptsize 五兩切}\\
右十味。㕮咀。以水一斗。煮取二升半。去滓。分温三服。温覆使汗出。頸項强。用大附子一枚。破之如豆大。煎藥。揚去沫。嘔者。加半夏半升洗。

婦人產後。中風。发熱。面{\khaai 正}赤。喘而頭痛。竹枼湯主之。

\section{竹皮大丸*}

生竹茹{\scriptsize 二分} 石膏{\scriptsize 二分研} 桂枝{\scriptsize 一分去皮} 甘草{\scriptsize 七分炙} 白薇{\scriptsize 一分}\\
右五味。末之。棗肉和丸如彈丸大。以飲服一丸。日三夜二服。有熱者。倍白薇。煩喘者。加柏実一分。

婦人產中虗。煩亂。嘔{\sungii 𠱘}。安中益气。竹皮大丸主之。

\section{紅藍花酒}

紅藍花{\scriptsize 一大兩}\\
右一味。以酒一大升。煎强半。頓服。不止再服。{\scriptsize 疑非仲景方。}{\wuben}

紅藍花{\scriptsize 一兩}\\
右一味。以酒一大升。煎減半。頓服一半。未止再服。{\scriptsize 疑非仲景方。}{\dengben}

治婦人六十二種風。兼主腹中血气刺痛。紅藍花酒。

\section{禹餘糧丸{\scriptsize (方闕)}}

汗家。重发汗。必恍惚心亂。小便已。陰疼。与禹餘糧丸。88

\section{杏子湯{\scriptsize (方闕)}}

水之为病。其脉沈小。屬少陰。浮者为風。无水。虗胀者为气。水。发其汗即已。脉沈者。宜附子麻黄湯。浮者。宜杏子湯。

\section{黄連粉{\scriptsize (方闕)}}

浸淫瘡。從口流向四肢者。可治。從四肢流來入口者。不可治。黄連粉主之。{\scriptsize 方未見。}

\section{藜蘆甘草湯{\scriptsize (方闕)}}

病人常以手指臂脛動。此人身体瞤瞤者。藜蘆甘草湯主之。{\scriptsize 方未見。}

\section{膠薑湯{\scriptsize (方闕)}}

婦人陷經。漏下黑不觧。膠薑湯主之。

\part{傷寒論}

%\chapter{辨脉法}

%\chapter{平脉法}

%\chapter{傷寒例}

\chapter{辨太陽病}

太陽之为病。{\khaai 脉浮。}頭項强痛。而惡寒。1
	\footnote{
		趙本、成本、《聖惠方》有「脉浮」二字,《玉函》、《千金翼》无。
	}

太陽病。发熱。汗出。惡風。脉緩者。为中風。2

%太陽中風。发熱而惡寒。0
%
%\hangindent 1em
%\hangafter=0
%太陽中風。发熱而惡寒。宜桂枝湯。{\shenghui}

太陽病。或已发熱。或未发熱。必惡寒。体痛。嘔{\sungii 𠱘}。脉陰陽俱緊者。为傷寒。3

%傷寒一日。太陽脉弱。至四日。太陰脉大。0
%	\footnote{
%		此條見《千金翼》卷九第二、《玉函》卷二第三,《脉經》、趙本、成本无。
%	}

傷寒一日。太陽受之。脉若靜者。为不傳。頗欲吐。若躁煩。脉數急者。乃为傳。4

傷寒{\khaai 二三日}。陽明少陽証不見者。为不傳。5
	\footnote{
		「陽明少陽証」除趙本外其它版本均作「其二陽証」。
	}

傷寒三日。陽明脉大{\khaai 者。为欲傳}。186

傷寒三日。少陽脉小者。为欲已。271

%太陽病三四日。不吐下。見芤乃汗之。0
%	\footnote{
%	此條見《千金翼》卷九太陽病用桂枝湯法第一、《玉函》卷二辨太陽病形証第三,趙本、成本、《脉經》无。
%	}

太陽病。发熱而渴。不惡寒者。为温病。若发汗已。身灼熱者。为風温。6

風温{\khaai 之}为病。脉陰陽俱浮。自汗出。身重。多眠睡。鼻息必鼾。語言難出。若被下者。小便不利。直視。失溲。若被火者。微发黄{\khaai 色}。劇則如驚癇。时痸瘲。若火熏之。一逆尚引日。再逆促命期。6
	\footnote{
		「痸瘲」同趙本,玉函作「掣縱发作」。
	}

病有发熱而惡寒者。发於陽也。不熱而惡寒者。发於陰也。发於陽者七日愈。发於陰者六日愈。以陽數七。陰數六故也。7
	\footnote{
		此條为《金匱玉函經》太陽病篇第一條。
	}

太陽病。頭痛。至七日以上自愈者。其經{\sungii 𥁞}故也。若欲作再經者。当針足陽明。使經不傳則愈。8
	\footnote{
		「以上自愈者」同趙本、《千金翼》卷九,《玉函》卷六、《千金翼》卷十作「自当愈」,《玉函》卷二作「有当愈者」。
	}

太陽病欲觧时。從巳{\sungii 𥁞}未。9
	\footnote{
		「從巳{\sungii 𥁞}未」同《千金翼》、《玉函》,趙本、成本作「從巳至未上」。
		唐弘宇按:
	}

風家。表觧而不了了者。十二日愈。10

病人身大熱。反欲得衣者。熱在皮膚。寒在骨髓也。身大寒。反不欲近衣者。寒在皮膚。熱在骨髓也。11
	\footnote{
		此條見趙本、成本、《玉函》卷二,《千金翼》无。関於此條是否为仲景原文,歷來註家聚訟不已,考趙本卷二子目,亦无此條之提示,山田正珍云:「此條不似仲景氏辭气,疑是古語。」唐弘宇按:此條以「皮膚」、「骨髓」劃分人体内外層次,這是扁鵲醫學的特徵,我認为此條可能是作者引用古語,也可能是後人將古語沾益於此。
	}

太陽中風。{\khaai 脉}陽浮而陰弱。陽浮者熱自发。陰弱者汗自出。嗇嗇惡寒。淅淅惡風。翕翕发熱。鼻鳴。乾嘔。桂枝湯主之。12
	\footnote{
		「脉陽浮而陰弱」《千金翼》卷九、《脉經》卷七第二、《玉函》卷二第三作「陽浮而陰濡弱」,《玉函》卷五第十四作「脉陽浮而陰濡弱」,宋本、成本作「陽浮而陰弱」,《千金方》作「其脉陽浮而陰弱」,《聖惠方》卷八辨太陽病形証作「脉其陽浮而弱」。
%		孫世揚説:「嗇嗇」今諺称「冷瑟瑟」。
	}

太陽病。发熱。汗出。此为榮弱衛强。故使汗出。欲救邪風。宜桂枝湯。95

太陽病。頭痛。发熱。汗出。惡風。桂枝湯主之。13

太陽病。項背强几几。反汗出。惡風。桂枝{\khaai 加葛根}湯主之。14
%	\footnote{
%		孫世揚説:「几几」当作「掔掔」,《説文觧字》段玉裁註云:「掔之言緊也。」。
%	}

太陽病。下之。其气上衝者。可与桂枝湯。不衝者。不可与之。15

太陽病三日。已发汗吐下温針而不觧。此为壞病。桂枝湯不復中与也。觀其脉証。知犯何逆。隨証治之。16

桂枝湯本为觧肌。若其人脉浮緊。发熱。无汗。不可与也。常須識此。勿令誤也。16

酒客不可与桂枝湯。得之則嘔。以酒客不喜甘故也。17

喘家作桂枝湯。加厚朴杏仁佳。18

服桂枝湯吐者。其後必吐膿血。19

太陽病。发汗。遂漏不止。其人惡風。小便難。四肢微急。難以屈伸。桂枝加附子湯主之。20

太陽病。下之。脉促。胸滿者。桂枝去芍藥湯主之。若微{\khaai 惡}寒者。桂枝去芍藥加附子湯主之。21.22
	\footnote{
		「微惡寒」同《玉函》卷二第三、成本,趙本无「惡」字。
		趙本第21條末有「促一作縱」小字註釋。
	}

太陽病。得之八九日。如瘧狀。发熱。惡寒。熱多寒少。其人不嘔。清便續自可。一日再三发。脉微緩者。为欲愈也。脉微而惡寒者。此为陰陽俱虗。不可復{\khaai 吐下}发汗也。面反有熱色者。未欲觧也。以其不能得汗出。身必癢。宜桂枝麻黄各半湯。23
	\footnote{
		「續自可」《玉函》卷二第三作「自調」。
		「面反有熱色」《聖惠方》作「面色赤有熱」。
	}

太陽病。初服桂枝湯。反煩不觧者。当先刺風池風府。卻与桂枝湯即愈。24

服桂枝湯。大汗出。脉洪大者。与桂枝湯。若形如瘧。一日再发者。汗出便觧。宜桂枝二麻黄一湯。25

%\hangindent 1em
%\hangafter=0
%凡大汗出復後。脉洪大。形如瘧。一日再发。汗出便觧。更与桂枝麻黄湯。{\yixin}
%
%\hangindent 1em
%\hangafter=0
%大汗出後。脉猶洪大。形如瘧。日一发。汗出便觧方。{\yixin}

服桂枝湯。大汗出{\khaai 後}。大煩渴不觧。脉洪大者。白虎{\khaai 加人参}湯主之。26
	\footnote{
		「大汗出後」同趙本、成本、《玉函》卷二第三,《脉經》、《玉函》卷六第十九作「大汗出」,外臺作「大汗後」。
		「脉洪大者」同趙本、成本、《外臺》,《脉經》、《玉函》卷六第十九作「若脉洪大」,《玉函》卷二第三作「若脉洪大者」。
		「白虎加人参湯主之」同趙本卷二、成本卷二、《玉函》卷二第三,趙本卷八、《外臺》作「屬白虎加人参湯」,《脉經》作「屬白虎湯」,《玉函》卷六第十九作「屬白虎湯証」,《千金翼》作「与白虎湯」。
	}

太陽病。发熱。惡寒。熱多寒少。脉微弱者。此无陽也。不可{\khaai 復}发汗。{\khaai 宜桂枝二越婢一湯。}27

服桂枝湯。{\khaai 或}下之。仍頭項强痛。翕翕发熱。无汗。心下滿。微痛。小便不利。桂枝去桂加茯苓术湯主之。28
	\footnote{
		「白术」《脉經》作「术」。
	}

傷寒。脉浮。自汗出。小便數。心煩。微惡寒。腳攣急。反与桂枝湯。欲攻其表。得之便厥。咽乾。煩躁。吐{\sungii 𠱘}者。当作甘草乾薑湯。以復其陽。若厥愈。足温者。更作芍藥甘草湯与之。其腳即伸。若胃气不和。譫語者。少与{\khaai 調胃}承气湯。若重发汗。復加燒針者。四逆湯主之。29

太陽病。項背强几几。无汗。惡風。葛根湯主之。31

太陽与陽明合病。而自利{\khaai 者}。葛根湯主之。不下利。但嘔者。葛根加半夏湯主之。32.33
	\footnote{
		「而自利者」趙本作「者必自下利」,《脉經》作「而自利不嘔者」。
		唐弘宇按:此二條参照第172條的結構,合为一條。
	}

太陽病。桂枝証。醫反下之。遂利不止。脉促者。表未觧也。喘而汗出者。葛根芩連湯主之。34

太陽病。頭痛。发熱。身疼。腰痛。骨節疼痛。惡風。无汗而喘。麻黄湯主之。35

太陽与陽明合病。喘而胸滿者。不可下。宜麻黄湯。36

太陽病。十日已去。脉浮細而嗜卧者。外已觧也。設胸滿脇痛者。与小柴胡湯。脉{\khaai 但}浮者。与麻黄湯。37
	\footnote{
		「脉但浮」《脉經》、《玉函》作「脉浮」。
	}

太陽中風。脉浮緊。发熱。惡寒。身疼痛。不汗出而煩躁者。大青龙湯主之。若脉微弱。汗出。惡風者。不可服之。服之則厥。筋愓肉瞤。此为逆也。38
	\footnote{
		「煩躁者」《脉經》卷七第二、《玉函》卷二第三、卷五第十四作「煩躁頭痛」。
		「愓」同《千金方》、成本,《千金翼》、《脉經》、趙本作「惕」,《説文》:「惕,敬也。」此処應为「愓」,「愓」与「蕩」同,为動蕩之意,「筋愓」者,謂筋脉跳動也。
	}

傷寒。脉浮緩。身不疼。但重。乍有輕时。无少陰証者。大青龙湯发之。39
	\footnote{
		「乍」《聖惠方》卷八辨太陽病形証作「或」。
	}

傷寒。表不觧。心下有水气。乾嘔。发熱而欬。或渴。或利。或噎。或小便不利。少腹滿。或喘。小青龙湯主之。40

傷寒。心下有水气。欬而微喘。发熱。不渴。服湯已而渴者。此寒去。为欲觧。小青龙湯主之。41

太陽病。外証未觧。其脉浮弱。当以汗觧。宜桂枝湯。42

太陽病。下之。微喘者。表未觧也。桂枝{\khaai 加厚朴杏仁}湯主之。43
	\footnote{
%		「表未觧也」同趙本卷七辨可发汗篇、《千金方》卷九第五,《聖惠方》作「外未觧也」,趙本第四十三條、成本、《玉函》、《脉經》作「表未觧故也」。
		「桂枝加厚朴杏仁湯」一云「麻黄湯」。
	}

太陽病。外証未觧者。不可下。下之为逆。欲觧外者。宜桂枝湯。44
%	\footnote{
%		「欲觧外者宜桂枝湯」《脉經》无。
%	}

太陽病。先发汗不觧而下之。其脉浮者不愈。浮为在外。而反下之。故令不愈。今脉浮。故在外。当觧其外則愈。宜桂枝湯。45

太陽病。脉浮緊。无汗。发熱。身疼痛。八九日不觧。表証仍在。此当发其汗。服藥已。微除。其人发煩目暝。劇者必衄。衄乃觧。所以然者。陽气重故也。麻黄湯主之。46
%	\footnote{
%		「身疼痛」《千金翼》卷九第二、《脉經》卷七第二、《玉函》卷二第三、《玉函》卷五第十四、《外臺》卷三作「其身疼痛」,《聖惠方》卷八作「身痛」。
%	}

太陽病。脉浮緊。发熱。身无汗。自衄者愈。47

二陽并病。太陽初得病时。发其汗。汗先出{\khaai 。復}不徹。因轉屬陽明。續自微汗出。不惡寒。若太陽病証不罷者。不可下。下之为逆。如此者可小发汗。設面色緣緣正赤者。陽气怫鬱不得越。当觧之熏之。当汗不汗。其人躁煩。不知痛処。乍在腹中。乍在四肢。按之不可得。其人短气。但坐。以汗出不徹故也。更发汗則愈。何以知汗出不徹。以脉濇故知之。48

脉浮數者。法当汗出而愈。若下之。身体重。心悸者。不可发汗。当自汗出而觧。所以然者。尺中脉微。此裏虗。須表裏実。津液和。即自汗出愈。49

脉浮而緊。法当身体疼痛。当以汗觧之。假令尺中脉遲者。不可发汗。何以知然。以榮气不足。血少故也。50

脉浮者。病在表。可发汗。宜麻黄湯。51
	\footnote{
		「麻黄湯」一云「桂枝湯」。
	}

{\khaai 太陽病。}脉浮數者。可发汗。宜麻黄湯。52
	\footnote{
		「麻黄湯」一云「桂枝湯」。
	}

病常自汗出者。此为榮气和。衛气不和也。榮行脉中。衛行脉外。復发其汗。衛和則愈。宜桂枝湯。53
	\footnote{
		「此为榮气和衛气不和也」《玉函》作「此为營气与衛气不和也」,
		《脉經》作「此为榮气和榮气和而外不觧此衛不和也」,
		《千金方》作「此为榮气和榮气和而外不觧此为衛不和也」,
		《千金翼》作「此为榮气和衛气不和故也」,
		《聖惠方》作「此为榮气和衛气不和」,
		趙本作「此为榮气和榮气和者外不谐以衛气不共榮气谐和故尔」。
	}

病人臓无他病。时发熱。自汗出。而不愈者。此衛气不和也。先其时发汗則愈。宜桂枝湯。54

傷寒。脉浮緊。不发汗。因致衄者。宜麻黄湯。55

{\khaai 寸口}脉浮而緊。浮則为風。緊則为寒。風則傷衛。寒則傷榮。榮衛俱病。骨節煩疼。当发其汗。{\khaai 宜麻黄湯。}
	\footnote{
		此條趙本六經篇无,見於趙本卷一辨脉法、趙本卷七辨可发汗篇、《脉經》卷七第二、《玉函》卷二第二、《玉函》卷五第十四、《聖惠方》卷八辨傷寒脉{\sungii 𠊱}、《聖惠方》卷八辨可发汗形証。
	}

傷寒。不大便六七日。頭痛。有熱者。与承气湯。其小便清者。此为不在裏。續在表也。当发其汗。頭痛者必衄。宜桂枝湯。56
	\footnote{
		「与承气湯」《玉函》卷二第三作「未可与承气湯」、《玉函》卷五第十四作「不可与承气湯」。
		「其小便清者」《脉經》、《千金翼》卷九作「其大便反青」,玉函作「其小便反清」,外臺作「其人小便反清者」。
	}

傷寒。发汗已觧。半日許復煩。脉浮數者。可復发汗。宜桂枝湯。57
	\footnote{
		「半日許復煩」《聖惠方》卷八辨太陽病形証作「半日後復煩躁」、《千金方》卷九发汗吐下後作「半日許復心煩熱」。
	}

凡病。若发汗。若吐。若下。若亡血。{\khaai 内}无津液而陰陽自和者。必自愈。58

大下後。復发汗。其人小便不利。此亡津液。勿治之。得小便利。必自愈。59

下之後。復发汗。必振寒。脉微細。所以然者。内外俱虗故也。60

下之後。復发汗。晝日煩躁不得眠。夜而安靜。不嘔。不渴。无表証。脉沈微。身无大熱。乾薑附子湯主之。61

发汗後。身体疼痛。其脉沈遲。桂枝加芍藥生薑人参湯主之。62

发汗後。不可更行桂枝湯。汗出而喘。无大熱者。可与麻杏甘石湯。63
	\footnote{
		唐弘宇按:「发汗後」下諸本均有「不可更行桂枝湯」七字,唯獨《醫心方》此條作「治傷寒。发汗出而喘。无大熱」,文意更通順。
	}

发汗過多。其人叉手自冒心。心下悸。欲得按者。桂枝甘草湯主之。64

发汗後。其人脐下悸。欲作奔豚。苓桂甘棗湯主之。65

发汗後。腹胀滿者。厚朴{\khaai 生薑半夏甘草人参}湯主之。66

傷寒吐下发汗後。心下逆滿。气上衝胸。起則頭眩。其脉沈緊。发汗則動經。身为振搖。苓桂术甘湯主之。67
	\footnote{
		「吐下发汗後」玉函作「若吐若下若发汗後」,宋本作「若吐若下後」。「白术」脉經作「术」。
	}

发汗不觧。反惡寒者。虗故也。芍藥甘草附子湯主之。不惡寒。但熱者。実也。当和胃气。宜調胃承气湯。68.70
	\footnote{
		「調胃承气湯」除宋本外其它版本均作「小承气湯」。
	}

发汗若下之。{\khaai 病仍}不觧。煩躁。茯苓四逆湯主之。69

太陽病发汗後。大汗出。胃中乾。煩躁不得眠。其人欲飲水。当稍飲之。令胃气和則愈。若脉浮。小便不利。微熱。消渴者。五苓散主之。71
	\footnote{
		「其人欲飲水当稍飲之」趙本作「欲得飲水者少少与飲之」。錢超塵「按:作『稍』是。《説文觧字》云:『稍,物出有漸也。』魏晉前『稍』字的意義与現代漢語『逐漸地』相当。『稍飲水』指病人逐漸地飲水,『少少飲水』与『稍飲水』意近而別。『稍飲』的詞義重點指飲水的過程,『少飲』指飲水量少。按:当作『稍』,作『少』誤。」
		趙本本條末有「即豬苓散是」小字註釋。
	}

发汗已。脉浮數。煩渴者。五苓散主之。72

傷寒。汗出而渴者。五苓散主之。不渴者。茯苓甘草湯主之。73

中風。发熱。六七日不觧而煩。有表裏証。渴欲飲水。水入則吐。此为水逆。五苓散主之。74
	\footnote{
		根據宋本子目小註,第74條「下別有三病証」。趙本第75條无方,是为証,第76條有方,是为法,則趙本74條後僅有一証。《玉函》、成本第75條分为兩條,第76條亦分为兩條,則74條後有三証,与子目之説相符,故從。《玉函》部分條文排序優於趙本,於此可見一斑。
	}

未持脉时。病人叉手自冒心。師因教试令欬。而不即欬者。此必兩耳聋无所聞也。所以然者。以重发汗。虗故也。75

发汗後。飲水多者必喘。以水灌之亦喘。75

发汗後。水藥不得入口。为逆。{\khaai 若更发汗。必吐下不止。}76

发汗吐下後。虗煩。不得眠。若劇者。反覆顛倒。心中懊憹。栀子{\khaai 豉}湯主之。若少气者。栀子甘草{\khaai 豉}湯主之。若嘔者。栀子生薑{\khaai 豉}湯主之。76

发汗若下之。煩熱。胸中窒者。栀子{\khaai 豉}湯主之。77
	\footnote{
		「胸中窒者」同趙本、成本、《玉函》卷二第三,《脉經》卷七第八「窒」作「塞」,《千金方》卷九第九作「胸中窒气逆搶心者」。
	}

傷寒五六日。大下之後。身熱不去。心中結痛者。未欲觧也。栀子{\khaai 豉}湯主之。78

傷寒下後。煩而腹滿。卧起不安。栀子厚朴湯主之。79
	\footnote{
		「煩而」宋本作「心煩」。
	}

傷寒。醫以丸藥大下之。身熱不去。微煩。栀子乾薑湯主之。80

凡用栀子湯。其人微溏者。不可与服之。81
	\footnote{
		「其人」趙本作「病人舊」。
	}

太陽病。发汗。汗出不觧。其人仍发熱。心下悸。頭眩。身瞤動。振振欲躃地。玄武湯主之。82
	\footnote{
		「躃」趙本作「擗」,編者改。《集韻》:「躃,仆也」;《龙龕手鑑》:「躃,倒也」。
	}

咽㗋乾燥者。不可发汗。83

淋家不可发汗。发汗必便血。84

瘡家。雖身疼痛。不可发汗。汗出則痙。85

衄家不可发汗。汗出必額上促急{\khaai 緊}。直視不能眴。不得眠。86
	\footnote{
		「額上促急緊」《金匱》吳本同,趙本作「額上陷脉急緊」,《脉經》、《玉函》作「額陷脉上促急而緊」,《千金翼》作「額上促急」。唐弘宇按:趙本、《脉經》、《玉函》此條文意不通,從《金匱》吳本。
	}

亡血家不可发汗。汗出則寒慄而振。87

汗家。重发汗。必恍惚心亂。小便已。陰疼。与禹餘糧丸。88
	\footnote{
		趙本本條末有「方本闕」小字註釋。
	}

\hangindent 1em
\hangafter=0
凡失血者。不可发汗。发汗必恍惚心亂。{\shenghui}88

病者有寒。復发汗。胃中冷。必吐蛔。89
	\footnote{
		趙本「吐蛔」下有「一作{\sungii 𠱘}」小字註釋。
	}

本发汗。而復下之。为逆。若先发汗。治不为逆。本先下之。而反汗之。为逆。若先下之。治不为逆。90

傷寒。醫下之。續得下利。清穀不止。身体疼痛。急当救裏。後身体疼痛。清便自調。急当救表。救裏宜四逆湯。救表宜桂枝湯。91
%	\footnote{
%		《脉經》卷七第二病可发汗証、宋本卷七辨可发汗作「下利後身体疼痛清便自調急当救表宜桂枝湯」,
%		《脉經》卷七第八病发汗吐下以後証作「傷寒醫下之續得下利清穀不止身体疼痛急当救裏身体疼痛清便自調急当救表救裏宜四逆湯救表宜桂枝湯」,
%		《脉經》卷七病可温証第九作「傷寒醫下之續得下利清穀不止身体疼痛急当救裏宜温之以四逆湯」,
%		《玉函》卷二辨太陽病形証第三、趙本第九十一條、趙本卷十辨发汗吐下後、成本第九十一條、成本卷十辨发汗吐下後作「傷寒醫下之續得下利清穀不止身体疼痛急当救裏後身疼痛清便自調急当救表救裏宜四逆湯救表宜桂枝湯」,
%		《玉函》卷六辨可温病形証第二十作「傷寒醫下之而續得下利清穀不止身体疼痛急当救裏宜温之以四逆湯」,
%		《玉函》卷六辨发汗吐下後病形証治第十九作「傷寒醫下之續得下利清穀不止身体疼痛急当救裏後身体疼痛清便自調急当救表救裏宜四逆湯救表宜桂枝湯」。
%	}

病发熱。頭痛。脉反沈。若不差。身体疼痛。当救其裏。宜四逆湯。92

太陽病。先下而不愈。因復发汗。表裏俱虗。其人因冒。冒家当汗出自愈。所以然者。汗出表和故也。裏未和。然後{\khaai 復}下之。93
	\footnote{
		「裏未和」同趙本卷二辨太陽病脉証并治第五、《玉函》卷三,趙本卷十第二十二作「得表和」,《玉函》卷六、《脉經》卷七第八、《千金翼》卷十作「表和」。
	}

太陽病未觧。脉陰陽俱微。必先振汗出而觧。但陽{\khaai 脉}微者。先汗之而觧。但陰{\khaai 脉}微者。先下之而觧。汗之宜桂枝湯。下之宜{\khaai 調胃}承气湯。94
	\footnote{
		「陰陽俱微」《千金翼》卷九第一、《脉經》卷七第七、《玉函》卷二第三及卷五第十八、趙本、成本作「陰陽俱停」,《脉經》卷七第七作「陰陽俱沈」。
		「停」字下林億孫奇等註云「一作微」。唐弘宇按:「脉停」意義不明。程應旄《傷寒後條辨》、錢潢《傷寒溯源集》、《醫宗金鑑》訓为「停止」,成无己訓为「勻停」、「調和」。查《内經》、《難經》、《傷寒論》均无停脉,且後句的「但陽脉微」、「但陰脉微」当是承接前句而來,故依宋臣所校,將此処改为「微」。
		「調胃承气湯」一云「大柴胡湯」。
	}

血弱气{\sungii 𥁞}。腠理開。邪气因入。与正气相摶。結於脇下。正邪分爭。往來寒熱。休作有时。默默不欲飲食。臓腑相連。其痛必下。邪高痛下。故使嘔也。小柴胡湯主之。服柴胡湯已。渴者。屬陽明。以法治之。97
	\footnote{
		趙本「故使嘔也」下有「一云臓腑相違其病必下脇膈中痛」小字註釋。
	}

得病六七日。脉遲浮弱。惡風寒。手足温。醫再三下之。不能食。其人脇下滿{\khaai 痛}。面目及身黄。頸項强。小便難。与柴胡湯後必下重。本渴。飲水而嘔。柴胡{\khaai 湯}不復中与也。食穀者噦。98

傷寒五六日。中風。往來寒熱。胸脇苦滿。默默不欲飲食。心煩。喜嘔。或胸中煩而不嘔。或渴。或腹中痛。或脇下痞堅。或心下悸。小便不利。或不渴。外有微熱。或欬。小柴胡湯主之。96
	\footnote{
		「傷寒五六日中風往來寒熱」同《千金翼》卷九第四、趙本、成本,《脉經》、《玉函》卷五第十三作「中風往來寒熱傷寒五六日已後」,《玉函》卷二第三作「中風五六日傷寒往來寒熱」。
		「外有微熱」同《千金翼》、《脉經》、《玉函》,趙本作「身有微热」。
	}

傷寒四五日。身熱。惡風。頸項强。脇下滿。手足温而渴。小柴胡湯主之。99

傷寒。陽脉濇。陰脉弦。法当腹中急痛。先与小建中湯。不差者。与小柴胡湯。100
	\footnote{
	「小柴胡湯」《聖惠方》作「大柴胡湯」。
	}

傷寒中風。有柴胡証。但見一証便是。不必悉具。101
	\footnote{
	「有柴胡証」《玉函》作「有小柴胡証」。
	}

凡柴胡湯証而下之。柴胡証不罷者。復与柴胡湯。必蒸蒸而振。卻发熱汗出而觧。101

傷寒二三日。心中悸而煩者。小建中湯主之。102

太陽病。過經十餘日。反再三下之。後四五日。柴胡証仍在。先与小柴胡湯。嘔不止。心下急。其人鬱鬱微煩者。为未觧。与大柴胡湯下之則愈。103
	\footnote{
		「嘔不止心下急」同趙本、《外臺》卷一,《千金翼》卷九第四、《脉經》卷七第八、《玉函》卷二第三、卷六第十九作「嘔止小安」。
	}

傷寒十三日不觧。胸脇滿而嘔。日晡所发潮熱{\khaai 。已}而微利。此本当柴胡湯下之。不得利。今反利者。知醫以丸藥下之。非其治也。潮熱者。実也。先宜服小柴胡湯以觧其外。後以柴胡加芒硝湯主之。104
	\footnote{
	「已而微利」《外臺》卷一論傷寒日數病源作「畢而微利」。
	「此本当柴胡湯下之」同《脉經》卷七第八,《千金翼》卷九第四作「此本当柴胡下之」,《玉函》卷六第十九作「此証当柴胡湯下之」,趙本、成本、《外臺》卷一、《玉函》卷二第三作「此本柴胡証」。
	}

柴胡加大黄芒硝桑螵蛸湯。0
	\footnote{
		趙本无此條,《千金翼》、《玉函》有方无証。
	}

傷寒十三日。過經。譫語者。内有熱也。当以湯下之。小便利者。大便当堅。而反{\khaai 下}利。脉調和者。知醫以丸藥下之。非其治也。自利者。脉当微厥。今反和者。此为内実也。{\khaai 調胃}承气湯主之。105

太陽病不觧。熱結膀胱。其人如狂。血自下。下之即愈。其外不觧者。尚未可攻。当先觧其外。{\khaai 宜桂枝湯。}外觧已。{\khaai 但}少腹急結者。乃可攻之。宜桃仁承气湯。106
	\footnote{
		「下之即愈」同《脉經》卷七第七,《千金翼》卷九第五、《脉經》卷七第二、《玉函》卷二第三、卷五第十四、卷五第八作「下者即愈」,趙本、成本作「下者愈」。
		趙本「宜桃核承气湯」下有「後云觧外宜桂枝湯」小字註釋。
	}

傷寒八九日。下之。胸滿。煩。驚。小便不利。譫語。一身{\khaaiii 𥁞}{\khaai 重。}不可轉側。柴胡加龙骨牡蛎湯主之。107
	\footnote{
		「一身{\sungii 𥁞}重不可轉側」同《玉函》卷二第三、趙本、成本、《外臺》卷一,《千金翼》卷九第四、《脉經》卷七第八、《玉函》卷六第十九、《聖惠方》卷八作「一身不可轉側」。
	}

傷寒。腹滿。譫語。寸口脉浮而緊。此为肝乘脾。名曰縱。当刺期門。108

傷寒。发熱。嗇嗇惡寒。其人大渴。欲飲水者。其腹必滿。自汗出。小便利。其病欲觧。此为肝乘肺。名曰横。当刺期門。109
	\footnote{
		唐弘宇按:本條後,第110、111條为衍文,移至衍文部分,第112至119條,其内容不是太陽病,移至《奔豚气吐膿驚怖火邪》篇。
	}

%傷寒。脉浮。醫以火迫劫之。亡陽。{\khaai 必}驚狂。卧起不安。桂枝去芍藥加蜀漆牡蛎龙骨救逆湯主之。112
%
%傷寒。其脉不弦緊而弱{\khaai 。弱}者必渴。被火必譫語。{\khaai 弱者。发熱。脉浮。觧之当汗出愈。}113
%
%太陽病。以火熏之。不得汗。其人必躁。到經不觧。必清血。114
%	\footnote{
%		宋本「必清血」下有「名为火邪」四字。
%	}
%	
%\hangindent 1em
%\hangafter=0
%太陽病。以火蒸之。不得汗者。其人必燥結。若不結。必下清血。其脉躁者。必发黄也。{\shenghui}114
%
%脉浮。熱甚。而反灸之。此为実。実以虗治。因火而動。咽燥。必吐血。115
%
%微數之脉。慎不可灸。因火为邪。則为煩逆。追虗逐実。血散脉中。火气雖微。内攻有力。焦骨傷筋。血難復也。116
%
%\hangindent 1em
%\hangafter=0
%凡微數之脉。不可灸。因熱为邪。必致煩逆。内有損骨傷筋血枯之患。{\shenghui}116
%
%脉浮。当以汗觧。而反灸之。邪无從出。因火而盛。病從腰以下必重而痹。此为火逆。若欲自觧。当先煩。煩乃有汗。隨汗而觧。何以知之。脉浮。故知汗出当觧。116
%
%\hangindent 1em
%\hangafter=0
%脉当以汗觧。反以灸之。邪无所去。因火而盛。病当必重。此为逆治。若欲觧者。当发其汗而觧也。{\shenghui}116
%
%燒針令其汗。針処被寒。核起而赤者。必发奔豚。气從少腹上衝心者。灸其核上各一壯。与桂枝加桂湯。117
%
%火逆。下之。因燒針。煩躁者。桂枝甘草龙骨牡蛎湯主之。118
%
%傷寒。加温針必驚。119

太陽病。当惡寒。发熱。今自汗出。反不惡寒。发熱。関上脉細數。此醫吐之過也。一二日吐之者。腹中飢。口不能食。三四日吐之者。不喜糜粥。欲食冷食。朝食暮吐。此醫吐之所致也。此为小逆。120

太陽病吐之。但太陽病当惡寒。今反不惡寒。不欲近衣。此为吐之内煩也。121

病人脉數。數为熱。当消穀引食。而反吐者。以醫发其汗。令陽气微。膈气虗。脉則为數。數为客熱。不能消穀。胃中虗冷。故吐也。122

太陽病。過經十餘日。心下温温欲吐。而胸中痛。大便反溏。腹微滿。鬱鬱微煩。先{\khaai 此}时自極吐下者。与{\khaai 調胃}承气湯。若不尔者。不可与。但欲嘔。胸中痛。微溏者。此非柴胡湯証。以嘔。故知極吐下也。123
	\footnote{
		「心下温温欲吐」《玉函》「温温」作「嗢嗢」。「嗢」讀作[wà]。《説文》:「嗢,咽也」,吞咽之意。《漢語大詞典》:「嗢嗢,象聲詞,反胃欲嘔的聲音。」錢超塵説:「欲吐而吐不暢快的樣子曰嗢嗢欲吐」,他認为《玉函》用「嗢」是本字,趙本作「温」是假借字。
		唐弘宇按:錢老對「嗢嗢欲吐」的觧釋沒有証據支持,我比較認同《漢語大詞典》的觧釋。「心下温温欲吐」有兩種斷句方式:第一種將「嗢嗢欲吐」作一句讀,模擬欲嘔吐时发出的聲音;第二種將「心下温温」四字作一句讀,形容一種主觀感覺。若將「温温欲吐」作一句讀,則剩下一个孤立的「心下」,這顯然不合理,所以我認为「心下温温」四字当作一句讀。我推測「温温」應該是一種方言,即使在今天,方言也經常有无法用正字書寫的情況,只能用借字代替,此処的「温温」應該就是這樣一種情況。
		《金匱》肺痿篇:「肺痿。涎唾多。心中温温液液者。炙甘草湯主之。」很明顯此條中的「心中温温液液」就是一種主觀感覺,「液液」的部首也是水部,可証明我的推測。
		《金匱》歷節篇:「諸肢節疼痛。身体魁羸。腳腫如脱。頭眩短气。温温欲吐。桂枝芍藥知母湯主之。」此條的「温温」或可作「嗢嗢」。
	}

太陽病六七日。表証仍在。脉微而沈。反不結胸。其人发狂。此熱在下焦。少腹当堅滿。小便自利者。下血乃愈。所以然者。以太陽隨經。瘀熱在裏故也。抵当湯主之。124
	\footnote{
		「瘀熱在裏」《千金翼》保元堂本、《千金翼》世補齋本作「瘀血在裏」。
		%「抵当湯主之」《千金翼》卷九太陽病雜療法第七作「宜下之以抵当湯」。
	}

太陽病。身黄。脉沈結。少腹堅。小便不利者。为无血也。小便自利。其人如狂者。血証諦也。抵当湯主之。125

傷寒。有熱。少腹滿。應小便不利。今反利者。为有血也。当下之。宜抵当丸。126
	\footnote{
		「当下之」三字下,趙本、成本、《玉函》卷二第三、《千金翼》卷九第七有「不可餘藥」四字,《脉經》、《玉函》卷五第十八、《千金方》、《千金翼》卷十、《聖惠方》、趙本卷九均无。
	}

太陽病。小便利者。以飲水多。必心下悸。小便少者。必苦裏急也。127

問曰。病有結胸。有臓結。其狀何如。\\
答曰。按之痛。寸口脉浮。関上自沈。为結胸。128\\
問曰。何谓臓結。\\
答曰。如結胸狀。飲食如故。时时下利。寸口脉浮。関上細沈而緊。为臓結。舌上白胎滑者。難治。129

臓結无陽証。不往來寒熱。其人反靜。舌上胎滑者。不可攻也。130
	\footnote{
		趙本「不往來寒熱」下有「一云寒而不熱」小字註釋。
	}

病发於陽。而反下之。熱入因作結胸。病发於陰。而反下之。因作痞。所以成結胸者。以下之太早故也。131
	\footnote{
		趙本「病发於陰而反下之」下有「一云汗出」小字註釋。
	}

結胸者。項亦强。如柔痙狀。下之則和。宜大陷胸丸。131
	\footnote{
		唐弘宇按:根據宋本子目小註,第131條「前後有結胸臓結病六証」。趙本第131條之前的第128、129、130條为結胸臓結証,之後的第132、133條为結胸証,前後共五証,非六証。成本第128、129條合为一條,其餘同趙本,共四証,非六証。《玉函》卷三第四第131條分为兩條,則131條前有四証,後有二証,共六証,与子目之説相合。
		又,第128、129條文義相連,故將其合为一條。
	}

結胸証。脉浮大者。不可下。下之則死。132

結胸証悉具。而煩躁者死。133

太陽病。脉浮而動數。浮則为風。數則为熱。動則为痛。數則为虗。頭痛。发熱。微盜汗出。而反惡寒。其表未觧。醫反下之。動數變遲。膈内拒痛。胃中空虗。客气動膈。短气。躁煩。心中懊憹。陽气内陷。心下因堅。則为結胸。大陷胸湯主之。若不結胸。但頭汗出。餘処无汗。齐頸而還。小便不利。身必发黄。134
	\footnote{
		「膈内拒痛」《脉經》卷七第八、《千金翼》卷九第六作「頭痛即眩」,《玉函》卷三第四、卷六第十九作「頭痛則眩」。
		趙本「膈内拒痛」下有「一云頭痛即眩」小字註釋。
	}

傷寒六七日。結胸熱実。脉沈緊。心下痛。按之如石堅。大陷胸湯主之。135

傷寒十餘日。熱結在裏。復往來寒熱者。与大柴胡湯。但結胸。无大熱者。此为水結在胸脇。{\khaai 但}頭微汗出。大陷胸湯主之。136
	\footnote{
		「熱結在裏」《千金翼》卷九第四作「邪气結在裏」。
		此條趙本卷九辨可下篇、《外臺》卷二分作兩條。
	}

太陽病。重发汗而復下之。不大便五六日。舌上燥而渴。日晡所小有潮熱。從心下至少腹堅滿而痛不可近。大陷胸湯主之。137
	\footnote{
		趙本「日晡所小有潮熱」下有「一云日晡所发心胸大煩」小註。
	}

小結胸者。正在心下。按之則痛。其脉浮滑。小陷胸湯主之。138

太陽病二三日。不能卧。但欲起。心下必結。脉微弱者。此本寒也。而反下之。利止者。必結胸。未止者。四五日復下之。此挾熱利也。139
	\footnote{
		「此本寒也」同《千金翼》卷九第六、《脉經》卷七第八、《玉函》卷三第四、卷六第十九,趙本、成本作「此本有寒分也」,《外臺》卷二作「本有久寒也」。
	}

太陽病。下之。其脉促。不結胸者。此为欲觧。脉浮者。必結胸。脉緊者。必咽痛。脉弦者。必兩脇拘急。脉細數者。頭痛未止。脉沈緊者。必欲嘔。脉沈滑者。挾熱利。脉浮滑者。必下血。140
	\footnote{
		趙本「脉促」下有「一作縱」小字註釋。
	}

病在陽。当以汗觧。反以水潠之若灌之。其熱被劫不得去。益煩。皮上粟起。意欲飲水。反不渴。宜服文蛤散。若不差。与五苓散。141
	\footnote{
		「反以水潠之若灌之」《千金翼》卷九第六、《脉經》卷七第十四作「而反以水噀之若灌之」,《玉函》卷三第四、卷六第二十七作「而反以水潠之若灌之」,趙本、成本、《外臺》卷二作「反以冷水潠之」。
		《説文》:「潠,含水噴也。」《古今韻會舉要》:「潠,噴水也,亦作噀。」《龙龕手鑑》:「噀,俗;{\sungii 𠹀},正也,与潠同。」
		唐弘宇按:關於此條的「若」字,錢超塵説:「若,選擇連詞,義为或、或者。」我認为錢老師的説法不正確。%待補充
		「其熱被劫不得去」同趙本、成本、《玉函》卷三第四,《脉經》卷七第十四、《玉函》卷六第二十七、《千金翼》卷九第六、《外臺》卷二作「其熱卻不得去」。
		「益煩」同《千金翼》卷九第六、《脉經》卷七第十四、《玉函》卷三第四、卷六第二十七,趙本、成本、《外臺》卷二作「弥更益煩」。唐弘宇按:「彌」、「更」、「益」三字意義重復。
	}

寒実結胸。无熱証者。与三物白散。141
	\footnote{
		此條趙本、成本、《脉經》、《玉函》均与上條連寫为一條,《千金翼》卷九第六、《外臺》卷二單獨列为一條。
		「与三物白散」趙本作「与三物小陷胸湯白散亦可服」,下有「一云与三物小白散」小字註釋。
	}

太陽与少陽并病。頭項强痛。或眩冒。时如結胸。心下痞堅。当刺大椎第一間。肺腧。肝腧。慎不可发汗。发汗則譫語。脉弦。譫語五日不止者。当刺期門。142

婦人中風。发熱。惡寒。經水適來。得之七八日。熱除。脉遲。身涼。胸脇下滿。如結胸狀。譫語。此为熱入血室。当刺期門。隨其{\khaai 虗}実而取之。143
	\footnote{
		「隨其虗実而取之」同《千金翼》卷九第七、《脉經》卷七第十三、《玉函》卷三第四,趙本、《玉函》卷六第二十六作「隨其実而取之」,成本作「隨其実而瀉之」。
	}

婦人中風七八日。續得寒熱。发作有时。經水適斷。此为熱入血室。其血必結。故使如瘧狀。发作有时。小柴胡湯主之。144

婦人傷寒。发熱。經水適來。晝日明瞭。暮則譫語。如見鬼狀。此为熱入血室。无犯胃气及上二焦。必自愈。145
	\footnote{
		「明瞭」同趙本、成本、《玉函》卷三第四,《千金翼》卷九第七作「了了」。《玉篇》:「瞭,目明也。」《廣韻》:「瞭,目睛明也。」
		「二焦」《脉經》卷七作「三焦」。
	}

傷寒六七日。发熱。微惡寒。肢節煩疼。微嘔。心下支結。外証未去者。柴胡桂枝湯主之。146

\hangindent 1em
\hangafter=0
发汗多。亡陽。狂語者。不可下。与柴胡桂枝湯。和其榮衛。以通津液。後自愈。
	\footnote{
		此條不見於趙本六經篇,而見於趙本辨发汗後病脉証并治第十七、《脉經》卷七病发汗以後証第三、《千金翼》卷九太陽病用柴胡湯法第四。「狂語」趙本、成本作「譫語」。
	}

傷寒五六日。已发汗而復下之。胸脇滿。微結。小便不利。渴而不嘔。但頭汗出。往來寒熱。心煩。此为未觧。柴胡桂枝乾薑湯主之。147

\hangindent 1em
\hangafter=0
傷寒六日。已发汗及下之。其人胸脇滿。大腸微結。小腸不利而不嘔。但頭汗出。往來寒熱而煩。此为未觧。宜小柴胡桂枝湯。{\shenghui}147

傷寒五六日。頭汗出。微惡寒。手足冷。心下滿。口不欲食。大便堅。其脉細。此为陽微結。必有表。復有裏。沈亦为病在裏。汗出为陽微。假令純陰結。不得有外証。悉入在裏。此为半在外半在裏。脉雖沈緊。不得为少陰病。所以然者。陰不得有汗。今頭汗出。故知非少陰也。可与{\khaai 小}柴胡湯。設不了了者。得屎而觧。148

傷寒五六日。嘔而发熱。柴胡湯証具。而以他藥下之。柴胡証仍在者。復与柴胡湯。此雖已下之。不为逆。必蒸蒸而振。卻发熱汗出而觧。若心下滿而堅痛者。此为結胸。宜大陷胸湯。若但滿而不痛者。此为痞。柴胡{\khaai 湯}不復中与也。宜半夏瀉心湯。149

太陽与少陽并病。而反下之。{\khaai 成}結胸。心下堅。下利不止。水漿不下。其人心煩。150

脉浮緊。而反下之。緊反入裏。則作痞。按之自濡。但气痞耳。151

太陽中風。下利。嘔{\sungii 𠱘}。表觧乃可攻之。其人漐漐汗出。发作有时。頭痛。心下痞堅滿。引脇下痛。乾嘔。短气。汗出。不惡寒。此为表觧裏未和。十棗湯主之。152

太陽病。醫发其汗。遂发熱。惡寒。復下之。則心下痞。此表裏俱虗。陰陽气并竭。无陽則陰獨。復加燒針。因胸煩。面色青黄。膚瞤者。難治。今色微黄。手足温者。易愈。153

心下痞。按之濡。其脉関上浮者。大黄{\khaai 黄連}瀉心湯主之。154

心下痞。而復惡寒。汗出者。附子瀉心湯主之。155

本以下之。故心下痞。与瀉心湯。痞不觧。其人渴而口燥{\khaai 煩}。小便不利。五苓散主之。156
	\footnote{
		「口燥煩」《脉經》卷七第八无「煩」字。
	}

傷寒。汗出。觧之後。胃中不和。心下痞堅。乾噫食臭。脇下有水气。腹中雷鳴而利。生薑瀉心湯主之。157
%	\footnote{
%		「傷寒汗出觧之後」《千金方》卷九作「傷寒发汗後」,《聖惠方》卷八作「太陽病汗出後」。
%	}

傷寒中風。醫反下之。其人下利。日數十行。穀不化。腹中雷鳴。心下痞堅而滿。乾嘔。心煩。不{\khaai 能}得安。醫見心下痞。谓病不{\sungii 𥁞}。復下之。其痞益甚。此非結熱。但以胃中虗。客气上逆。故使之堅。甘草瀉心湯主之。158
%	\footnote{「故使之堅」《千金方》作「使之然也」。}

傷寒。服湯藥。下利不止。心下痞堅。服瀉心湯已。復以他藥下之。利不止。醫以理中与之。利益甚。理中者。理中焦。此利在下焦。赤石脂禹餘糧湯主之。復不止者。当利小便。159
	\footnote{
		「理中者理中焦」《千金翼》卷九第六作「理中治中焦」。
	}

傷寒。吐下{\khaai 後}发汗。虗煩。脉甚微。八九日。心下痞堅。脇下痛。气上衝咽㗋。眩冒。經脉動愓者。久而成痿。160

傷寒。发汗{\khaai 若}吐{\khaai 若}下。觧後。心下痞堅。噫气不除者。旋覆代赭湯主之。161

下後。不可更行桂枝湯。汗出而喘。无大熱者。可与麻杏甘石湯。162

太陽病。外証未除。而數下之。遂挾熱而利。利下不止。心下痞堅。表裏不觧。桂枝人参湯主之。163

傷寒。大下後。復发汗。心下痞。惡寒者。表未觧也。不可攻痞。当先觧表。表觧乃可攻痞。觧表宜桂枝湯。攻痞宜大黄黄連瀉心湯。164

傷寒。发熱。汗出不觧。心中痞堅。嘔吐。下利。大柴胡湯主之。165

病如桂枝証。頭不痛。項不强。寸{\khaai 口}脉微浮。胸中痞堅。气上衝咽㗋。不得息。此为胸有寒。当吐之。宜瓜蒂散。166
	\footnote{
		「寸口脉微浮」同《玉函》卷五第十六,《脉經》卷七第五作「寸口脉微細」,《千金翼》卷九第六作「脉微浮」,趙本、成本、《玉函》卷三第四作「寸脉微浮」。
	}

諸亡血。虗家。不可与瓜蒂散。
	\footnote{
		唐弘宇按:此條諸本均在瓜蒂散煎服法中,未入正文。根據宋本獨有的子目,第166條瓜蒂散証後註云:「下有不可与瓜蒂散証」,也就是説第166條後当有一條不可与瓜蒂散的條文,故編者將此條列为正文。
	}

病者脇下素有痞。連在脐傍。痛引少腹。入陰筋者。此为臓結。死。167
	\footnote{
		「入陰筋者」《玉函》作「入陰俠陰筋者」。
	}

傷寒若吐若下後。七八日不觧。熱結在裏。表裏俱熱。时时惡風。大渴。舌上乾燥而煩。欲飲水數升。白虎{\khaai 加人参}湯主之。168

\hangindent 1em
\hangafter=0
傷寒六日不觧。熱結在裏。但熱。时时惡風。大渴。舌乾。煩躁。宜白虎湯。{\shenghui}168

凡用白虎湯。立夏後至立秋前得用之。立秋後不可服。\\
春三月。病常苦裏冷。不可与白虎湯。与之則嘔利而腹痛。\\
諸亡血。虗家。不可与白虎湯。得之則腹痛而利。但当温之。
%	\footnote{
%		「春三月」宋本作「正月二月三月」。
%		「病常苦裏冷」宋本、千金作「尚凜冷」。
%		此條千金无。「得之則腹痛而利」玉函作「得之腹痛而利者」,宋本作「得之則腹痛利者」。「但当温之」玉函作「急当温之」,宋本作「但可温之可愈」。
%		唐弘宇按:此條位於趙本第168條後的白虎加人参湯煎服法中,未入正文;位於《千金翼》第卷九第七176條後的白虎湯煎服法中,未入正文;位於《玉函》卷三第四第170條後,为正文。《千金方》我手邊沒有可参考的善本。根據《宋本》的子目,168條後註云:「下有不可与白虎湯証」,也就是説第168條後当有一條不可与白虎湯的條文,故編者將此三條合为一條,列为正文。
%	}

傷寒。无大熱。口燥渴。心煩。背微惡寒。白虎{\khaai 加人参}湯主之。169

傷寒。脉浮。发熱。无汗。其表不觧。不可与白虎湯。渴欲飲水。无表証者。白虎{\khaai 加人参}湯主之。170

太陽与少陽合病。自下利者。与黄芩湯。若嘔者。与黄芩加半夏生薑湯。172

傷寒。胸中有熱。胃中有邪气。腹中痛。欲嘔吐。黄連湯主之。173

傷寒八九日。風濕相摶。身体疼煩。不能自轉側。不嘔。不渴。脉浮虗而濇者。桂枝附子湯主之。若其人大便堅。小便自利者。术附子湯主之。174

風濕相摶。骨節疼煩。掣痛。不得屈伸。近之則痛劇。汗出。短气。小便不利。惡風。不欲去衣。或身微腫。甘草附子湯主之。175

傷寒。脉浮滑。此以表有熱。裏有寒。白虎湯主之。176

傷寒。脉結代。心動悸。炙甘草湯主之。177
	\footnote{
		「心動悸」《玉函》卷三第四作「心中驚悸」。
		唐弘宇按:趙本此條後有第178條。按照北宋校正醫書局子目的撰寫体例,如果第178條存在,則本卷子目177條下当有「下有結代脉一証」小註,而事実是沒有。又查《千金翼》、《玉函》皆无第178條。「脉按之來緩时一止復來者名曰結」見於辨脉篇,其後的結代脉語句見於《脉經》、《千金方》。本條可能是匯集相関内容後,沾益与177條之後,故將其移至衍文部分。
		}

\chapter{辨陽明病}

陽明之为病。胃家実是也。180
	\footnote{
		趙本「実」下有「一作寒」小字註釋。
		唐弘宇按:趙本第179條为陽明病篇第一條,此條列於其後,《玉函》卷三第五、《千金翼》卷九第八此條列於陽明病篇第一條,在第179條之前。按照本論的編寫慣例,六經病的第一條当是本病提綱証,所以本篇第一條当是第180條。
	}

\hangindent 1em
\hangafter=0
傷寒二日。陽明受病。陽明者。胃中寒是也。宜桂枝湯。{\shenghui}180

問曰。病有太陽陽明。有正陽陽明。有少陽陽明。何谓也。\\
答曰。太陽陽明者。脾約是也。正陽陽明者。胃家実是也。微陽陽明者。发汗。利小便已。胃中燥。大便難是也。179
	\footnote{
		「少陽」同趙本,《千金翼》卷九陽明病狀第八、《玉函》卷三第五作「微陽」。
		「发汗利小便已」同趙本。《千金翼》卷九第八、《玉函》卷三第五作「发其汗若利其小便」。
		趙本「胃中燥」下有「煩実」二字,錢超塵説:「《金匱玉函經》、《千金翼方》均无『煩実』二字,義长。」從。
		趙本「脾約」下有「一云絡」小字註釋。
	}

問曰。何緣得陽明病。\\
答曰。太陽病。发汗若下之。亡其津液。胃中乾燥。因轉屬陽明。不更衣。{\khaai 内実。}大便難者。为陽明病也。181

問曰。陽明病外証云何。\\
答曰。身熱。汗出。而不惡寒。{\khaai 但}反惡熱。182

問曰。病有得之一日。不发熱而惡寒者。何。\\
答曰。然雖一日。惡寒自罷。即汗出。惡熱也。183\\
問曰。惡寒何故自罷。\\
答曰。陽明居中。主土。万物所歸。无所復傳。始雖惡寒。二日自止。此为陽明病也。184

{\khaai 本}太陽。初得病时。发其汗。汗先出{\khaai 。復}不徹。因轉屬陽明。185

\hangindent 1em
\hangafter=0
太陽病而发汗。汗雖出。復不觧。不觧者。轉屬陽明也。宜麻黄湯。{\shenghui}185

傷寒。发熱。无汗。嘔不能食。而反汗出濈濈然。是为轉屬陽明。185

傷寒。脉浮而緩。手足自温。是为系在太陰。太陰{\khaai 身}当发黄。若小便自利者。不能发黄。至七八日。大便堅者。为屬陽明。187

傷寒轉系陽明者。其人濈然微汗出也。188

陽明中風。口苦。咽乾。腹滿。微喘。发熱。惡寒。脉浮緊。若下之。則腹滿。小便難也。189

陽明病。能食为中風。不能食为中寒。190

陽明{\khaai 病。}中寒。不能食。小便不利。手足濈然汗出。此欲作堅瘕。必大便頭堅後溏。所以然者。胃中冷。水穀不別故也。191

\hangindent 1em
\hangafter=0
陽明中寒。不能食。小便不利。手足濈然汗出。欲作堅癥也。所以然者。胃中水穀不化故也。宜桃仁承气湯。{\shenghui}191

陽明病。初欲食。小便反不利。大便自調。其人骨節疼。翕翕如有熱狀。奄然发狂。濈然汗出而觧。此为水不勝穀气。与汗共并。脉緊則愈。192
	\footnote{
		「小便反不利」同趙本、成本,《千金翼》卷九第八、《玉函》卷三第五作「小便反不數」。
	}

陽明病欲觧时。從申{\sungii 𥁞}戍。193

陽明病。不能食。下之不觧。攻其熱必噦。所以然者。胃中虗冷故也。{\khaai 以其人本虗。故攻其熱必噦。}194

\hangindent 1em
\hangafter=0
陽明病。能食。下之不觧。其人不能食。攻其熱必噦者。胃中虗冷也。宜半夏湯。{\shenghui}194

陽明病。脉遲。食難用飽。飽則发煩。頭眩。必小便難。此欲作穀疸。雖下之。腹滿如故。所以然者。脉遲故也。195

\hangindent 1em
\hangafter=0
陽明病。脉遲。发熱。頭眩。小便難。此欲作榖疸。下之必腹滿。宜柴胡湯。{\shenghui}195

陽明病当多汗。而反无汗。其身如虫行皮中狀。此以久虗故也。196
	\footnote{
	《玉函》卷三第五、《千金翼》卷九第八「陽明病」前有「陽明病久久而堅者」八字。
	}

陽明病。反无汗。但小便利。二三日。嘔而欬。手足厥者。其人頭必痛。若不嘔。不欬。手足不厥者。頭不痛。197
	\footnote{
		趙本本條末有「一云冬陽明」小字註釋。
	}

陽明病。但頭眩。不惡寒。故能食而欬。其人咽必痛。若不欬者。咽不痛。198
	\footnote{
		趙本本條末有「一云冬陽明」小字註釋。
	}

陽明病。脉浮緊者。必潮熱。发作有时。{\khaai 脉}但浮者。必盜汗出。201
	\footnote{
		「脉但浮者」趙本、成本、《玉函》卷三第五、《千金翼》卷九第八均作「但浮者」,《聖惠方》卷八作「其脉浮者」。
	}

陽明病。无汗。小便不利。心中懊憹者。必发黄。199

\hangindent 1em
\hangafter=0
陽明病。无汗。小便不利。心中熱壅。必发黄也。宜茵陳湯。{\shenghui}199

陽明病。被火。額上微汗出。小便不利者。必发黄。200

\hangindent 1em
\hangafter=0
陽明病。被火灸。其額上微有汗出。小便不利者。必发黄也。宜茵陳湯。{\shenghui}200

陽明病。口燥。但欲漱水。不欲咽者。必衄。202

陽明病。本自汗出。醫復重发汗。病已差。其人微煩不了了者。此必大便堅故也。以亡津液。胃中{\khaai 乾}燥。故令其堅。当問其小便日幾行。若本日三四行。今日再行者。知必大便不久出。今为小便數少。津液当還入胃中。故知不久必大便也。203

傷寒。嘔多。雖有陽明証。不可攻之。204

陽明病。心下堅滿者。不可攻之。攻之遂利不止者死。利止者愈。205

陽明病。面合色赤者。不可攻之。{\khaai 攻之}必发熱。色黄。小便不利也。206
%	\footnote{
%		「面合色赤」錢超塵在出版时間較早的《影印孫思邈本校注考証》中説:「合,当也,應也。」又在後來出版的《宋本傷寒論文獻史論》中説:「合字訛,当作垢。」
%	}

陽明病。不吐不下。心煩者。可与{\khaai 調胃}承气湯。207
%	\footnote{
%		「不吐不下」同趙本、成本,《千金翼》卷九第八、《脉經》卷七第七、《玉函》卷三第五、卷五第十八作「不吐下」。
%	}

陽明病。脉遲。雖汗出。不惡寒。其身必重。短气。腹滿而喘。有潮熱。如此者。其外为觧。可攻其裏。若手足濈然汗出者。此大便已堅。{\khaai 大}承气湯主之。若汗多。微发熱。惡寒者。为外未觧。{\khaai 桂枝湯主之。}其熱不潮。未可与承气湯。若腹大滿。而不大便者。可与小承气湯。微和其胃气。勿令至大下。208
	\footnote{
		%「如此者其外为觧可攻其裏」趙本、《千金方》、《外臺》作「者此外欲觧可攻裏也」。
		「桂枝湯主之」趙本无。趙本「外未觧也」下有「一法与桂枝湯」小字註釋,南宋李檉以北宋校訂本《傷寒論》为底本撰《傷寒要旨藥方》卷二大承气湯第七十三條有「桂枝湯主之」,趙本卷九辨可下篇、《千金方》卷九宜下亦有「桂枝湯主之」。
	}

陽明病。潮熱。大便微堅者。可与{\khaai 大}承气湯。不堅者。不可与之。若不大便六七日。恐有燥屎。欲知之法。可少与小承气湯。若腹中轉失气者。此有燥屎也。乃可攻之。若不轉失气者。此但頭堅後溏。不可攻之。攻之必腹滿。不能食也。欲飲水者。与水即噦。其後发熱者。必大便復堅而少也。以小承气湯和之。若不轉失气者。慎不可攻之。209
	\footnote{
		「必大便復堅而少也」《脉經》卷七第六、《玉函》卷五第十七、《千金翼》卷九第八作「必復堅」,《玉函》卷三第五作「必復堅而少也」,《聖惠方》作「必腹堅脹」,趙本作「必大便復鞕而少也」,《千金方》卷九宜下作「大便必復堅」。
	}

\hangindent 1em
\hangafter=0
陽明病。有潮熱。大便堅。可与承气湯。若有結燥。乃可徐徐攻之。若无壅滯。不可攻之。攻之者。必腹滿。不能食。欲飲水者即噦。其{\sungii 𠊱}发熱。必腹堅胀。宜与小承气湯。{\shenghui}209

夫実則譫語。虗則鄭聲。鄭聲者。重語也。直視。譫語。喘滿者死。下利者亦死。210
	\footnote{
		「鄭聲者重語也」《外臺》卷一千金方六首作「鄭聲重語也」,寫作雙行小字註於「虗則鄭聲」下。
		錢超塵「按:『鄭聲者重語是也』七字乃註文也,後竄入正文,当依《外臺》正之。」
	}

发汗多。重发汗。{\khaai 此为}亡陽。{\khaai 若}譫語。脉短者死。脉自和者不死。211

傷寒。吐下後未觧。不大便五六日。至十餘日。其人日晡所发潮熱。不惡寒。獨語。如見鬼{\khaai 神之}狀。若劇者。发則不識人。順衣妄撮。怵惕不安。微喘。直視。脉弦者生。濇者死。{\khaai 若}微者。但发熱。譫語。{\khaai 大}承气湯主之。若一服利。止後服。212
	\footnote{
		「順衣妄撮怵惕不安」趙本作「循衣摸牀惕而不安」,其下有「一云順衣妄撮怵惕不安」小字註釋。
		「若微者」各本均无「若」字。唐弘宇按:此條「若劇者」与「若微者」为并列的兩種情況,故加「若」字。
	}

陽明病。其人多汗。津液外出。胃中燥。大便必堅。堅則譫語。{\khaai 小}承气湯主之。{\khaai 若一服譫語止。莫復服。}213
	\footnote{
		「若一服譫語止莫復服」《千金翼》卷九第八、《玉函》卷五第十八、《脉經》卷七第七、趙本卷九辨可下篇无。
	}

陽明病。譫語。发潮熱。脉滑疾者。{\khaai 小}承气湯主之。因与承气湯一升。腹中轉失气者。復与一升。若不轉失气者。勿更与之。明日又不大便。脉反微濇者。此为裏虗。为難治。不可復与承气湯。214
	\footnote{
		「譫語」同趙本、成本、《玉函》卷三第五、《脉經》卷七第七,《千金翼》卷九第八、《玉函》卷五第十八、《聖惠方》卷八辨陽明病形証作「譫語妄言」。
	}

陽明病。譫語。有潮熱。反不能食者。{\khaai 胃中}必有燥屎五六枚。若能食者。但堅耳。{\khaai 大}承气湯主之。215
	\footnote{
		「胃中」二字趙本、成本有,《千金翼》卷九第八、《脉經》卷七第七、《玉函》卷三第五、卷五第十八均无。
		錢超塵「按:宋本、成本之『胃中』二字誤衍,以《脉經》、《玉函》均无此二字可知也。」
		「大承气湯主之」同《玉函》卷三第五,趙本作「宜大承气湯下之」,《千金翼》卷九第八作「承气湯主之」,《脉經》卷七第七、趙本卷九作「屬承气湯証」,《玉函》卷五第十八作「屬承气湯」。
%		唐弘宇按:
	}

陽明病。下血。譫語者。此为熱入血室。但頭汗出者。当刺期門。隨其実而瀉之。濈然汗出則愈。216

汗出。譫語者。以有燥屎在胃中。此風也。{\khaai 須下者。}過經乃可下之。下之若早。語言必亂。以表虗裏実故也。下之則愈。宜{\khaai 大}承气湯。217
	\footnote{
		趙本「汗」字下有「汗一作卧」小字註釋,「大承气湯」下有「一云大柴胡湯」小字註釋。
	}

傷寒四五日。脉沈而喘滿。沈为在裏。反发其汗。津液越出。大便为難。表虗裏実。久則譫語。218

三陽合病。腹滿。身重。難以轉側。口不仁。面垢。譫語。遺尿。发汗則譫語{\khaai 甚}。下之則額上生汗。手足厥冷。自汗。白虎湯主之。219
	\footnote{
		趙本「面垢」下有「又作枯一云向經」小字註釋。
		「譫語甚」同《脉經》卷三第五,趙本、《玉函》、《千金翼》均无「甚」字。
		「自汗」同《脉經》卷七第八、《玉函》卷六第十九,趙本、成本、《玉函》卷三第五作「若自汗出者」。
	}

二陽并病。太陽証罷。但发潮熱。手足漐漐汗出。大便難。而譫語者。下之則愈。宜{\khaai 大}承气湯。220

陽明病。脉浮緊。咽乾。口苦。腹滿而喘。发熱。汗出。不惡寒。反惡熱。身重。若发汗則躁。心憒憒。反譫語。若加温針。必怵惕。煩躁。不得眠。若下之。則胃中空虗。客气動膈。心中懊憹。舌上胎者。栀子{\khaai 豉}湯主之。221
	\footnote{
		孫世揚:「舌上胎者」之「胎」当作「菭」,《説文觧字》:「菭,水衣也。」
	}

\hangindent 1em
\hangafter=0
陽明病。脉浮。咽乾。口苦。腹滿。汗出而喘。不惡寒。反惡熱。心躁。譫語。不得眠。胃虗。客熱。舌燥。宜栀子湯。{\shenghui}221

若渴欲飲水。口乾舌燥者。白虎{\khaai 加人参}湯主之。222

若脉浮。发熱。渴欲飲水。小便不利者。豬苓湯主之。223
	\footnote{
		《聖惠方》卷八辨陽明病形証本條首有「陽明病」三字。
	}

陽明病。汗出多而渴者。不可与豬苓湯。以汗多。胃中燥。豬苓湯復利其小便故也。224

\hangindent 1em
\hangafter=0
陽明病。汗出多而渴者。不可与豬苓湯。汗多者。胃中燥也。汗少者。宜与豬苓湯利其小便。{\shenghui}224

{\khaai 陽明病。}脉浮而遲。表熱裏寒。下利清穀者。四逆湯主之。225

{\khaai 陽明病。}若胃中虗冷。不能食者。飲水即噦。226
	\footnote{
		《脉經》卷七第十四、《聖惠方》卷八有「陽明病」三字。
	}

脉浮。发熱。口乾。鼻燥。能食者。則衄。227

\hangindent 1em
\hangafter=0
脉浮。发熱。口鼻中燥。能食者。必衄。宜黄芩湯。{\shenghui}227

陽明病。下之。其外有熱。手足温。不結胸。心中懊憹。飢不能食。但頭汗出。栀子{\khaai 豉}湯主之。228

陽明病。发潮熱。大便溏。小便自可。胸脇滿不去。小柴胡湯主之。229

\hangindent 1em
\hangafter=0
陽明病。发潮熱。大便溏。小便自利。胸脇煩滿不止。宜小柴胡湯。{\shenghui}229

陽明病。脇下堅滿。不大便而嘔。舌上{\khaai 白}胎者。可与小柴胡湯。上焦得通。津液得下。胃气因和。身濈然汗出而觧。230

\hangindent 1em
\hangafter=0
陽明病。脇下堅滿。大便祕而嘔。口燥。宜柴胡湯。{\shenghui}230

陽明中風。脉弦浮大。而短气。腹都滿。脇下及心痛。久按之。气不通。鼻乾。不得汗。嗜卧。一身及目悉黄。小便難。有潮熱。时时噦。耳前後腫。刺之小差。外不觧。病過十日。脉續浮者。与{\khaai 小}柴胡湯。脉但浮。无餘証者。与麻黄湯。不尿。腹滿加噦者。不治。231.232
	\footnote{
		「腹都滿」諸本均同。錢超塵説:「考趙開美本《傷寒論》詞例,均作『腹滿』,除此例之外,无更作『腹都滿』者。『都』字衍。」
	}

\hangindent 1em
\hangafter=0
陽明病。中風。其脉浮大。短气。心痛。鼻乾。嗜卧。不得汗。一身悉黄。小便難。有潮熱而噦。耳前後腫。刺之雖小差。外若不觧。宜柴胡湯。{\shenghui}231.232

陽明病。{\khaai 自}汗出。若发汗。小便自利者。此为{\khaai 津液}内竭。雖堅。不可攻之。当須自欲大便。宜蜜煎。導而通之。若土瓜根及豬膽汁。皆可以導。233

陽明病。脉遲。汗出多。微惡寒者。表未觧也。可发汗。宜桂枝湯。234

陽明病。脉浮。无汗而喘者。发汗則愈。宜麻黄湯。235
	\footnote{
		「而喘者」同趙本、成本,《千金翼》卷九第八、《玉函》卷三第五、卷五第十四、《聖惠方》卷八辨太陽病形証均作「其人必喘」。
	}

陽明病。发熱。汗出者。此为熱越。不能发黄也。但頭汗出。身无汗。齐頸而還。小便不利。渴引水漿者。此为瘀熱在裏。身必发黄。茵陳{\khaai 蒿}湯主之。236
	\footnote{
		「熱越」《聖惠方》卷八作「熱退」。
	}

\hangindent 1em
\hangafter=0
陽明病。发熱而汗出。此为熱退。不能发黄也。但頭汗出。身体无汗。小便不利。渴引水漿。此为瘀熱在裏。必发黄也。宜茵陳湯。{\shenghui}

\hangindent 1em
\hangafter=0
陽明病。但頭汗出。其身无汗。小便不利。渴汁水漿。此为瘀熱在裏。身必发黄。宜急下之。{\shenghui}

陽明証。其人喜忘者。必有畜血。所以然者。本有久瘀血。故令喜忘。屎雖堅。大便反易。其色必黑。抵当湯主之。237

陽明病。下之。心中懊憹而煩。胃中有燥屎者。可攻。其人腹微滿。頭堅後溏者。不可攻之。若有燥屎者。宜{\khaai 大}承气湯。238
	\footnote{
		「頭堅後溏者」同《千金翼》卷九第八、《脉經》卷七第八、《玉函》卷三第五、卷六第十九,趙本、成本作「初頭鞕後必溏」。
	}

病者五六日不大便。繞脐痛。煩躁。发作有时。此有燥屎。故使不大便也。239

病者煩熱。汗出則觧。復如瘧狀。日晡所发者。屬陽明。脉実者。当下之。脉浮虗者。当发其汗。下之宜{\khaai 大}承气湯。发汗宜桂枝湯。240
	\footnote{
		「日晡所发者」同《千金翼》卷七第八、《脉經》卷七第七,《脉經》卷七第二作「日晡所发熱」,趙本、成本、《玉函》卷三第五、卷五第十四作「日晡所发熱者」,《玉函》卷五第十四作「日晡发熱者」。
		「大承气湯」一云「大柴胡湯」。
		錢超塵「按:《聖惠方》此條脱文雖多,但仍可知为240條之殘文无疑。《聖惠方》此條尚存之文字为:『汗出後則暫觧,日晡則復发。脉実者,当宜下之。』雖为殘文,仍可從中窺見《聖惠方》卷八確系南朝醫師祕傳之本。歷來校勘家及研治《傷寒論》者,多不注意此本之價值,以为殘缺甚多,无可多論,其実不然也。其中有多処對研治《傷寒論》之流傳、正其舛誤、補其方剂之未備,啓迪多多,此本於《傷寒論》文獻之研究,有待予以較高評價。」
	}

%\hangindent 1em
%\hangafter=0
%陽明病。脉実者当下。脉浮虗者当汗。下者宜承气湯。汗者宜桂枝湯。{\shenghui}240
%
%\hangindent 1em
%\hangafter=0
%傷寒病。五六日不大便。繞脐痛。煩躁。汗出者。此为有結。汗出後則暫觧。日晡則復发。脉実者。当宜下之。{\shenghui}239.240

大下後。六七日不大便。煩不觧。腹滿痛者。此有燥屎。所以然者。本有宿食故也。宜{\khaai 大}承气湯。241

病者小便不利。大便乍難乍易。时有微熱。怫㥜不能卧者。有燥屎故也。宜{\khaai 大}承气湯。242
	\footnote{
		「怫㥜」《玉函》卷五第十八、《千金翼》卷九第八作「怫鬱」,《脉經》卷七第七、《玉函》卷三第五、赵本、趙本、成本作「喘冒」,趙本「喘冒」二字下有「一作怫鬱」小字註釋,《醫心方》作「沸胃」。《集韻》:怫㥜,心不安也。
		唐弘宇按:
	}

食穀欲嘔者。屬陽明。吳茱萸湯主之。{\khaai 得湯反劇者。屬上焦。}243
	\footnote{
		「得湯反劇者屬上焦」八字《千金翼》卷九第八在煎服法中。
	}

太陽病。寸{\khaai 口}緩。関{\khaai 上小}浮。尺{\khaai 中}弱。其人发熱。汗出。復惡寒。不嘔。但心下痞者。此为醫下之故也。若不下。其人不惡寒而渴者。此轉屬陽明。小便數者。大便必堅。不更衣十日。无所苦也。{\khaai 渴}欲飲水者。少少与之。但以法救之。渴者。宜五苓散。244
	\footnote{
		「太陽病」《千金翼》卷九第八作「陽明病」。
		「但以法救之渴者」同趙本。
	}

脉陽微而汗出少者。为自和。汗出多者。为太過。陽脉実。因发其汗。出多者。亦为太過。太過者。陽絕於内。亡津液。大便因堅。245
	\footnote{
		趙本「为自和」下有「一作如」小字註釋。
	}

脉浮而芤。浮{\khaai 則}为陽。芤{\khaai 則}为陰。浮芤相摶。胃气生熱。其陽則絕。246
	\footnote{
		唐弘宇按:此條僅为《千金方》一段中的一句。
	}

趺陽脉浮而濇。浮則胃气强。濇則小便數。浮濇相摶。大便則堅。其脾为約。麻子仁丸主之。247

太陽病三日。发汗不觧。蒸蒸发熱者。{\khaai 屬胃也。調胃}承气湯主之。248

傷寒吐後。腹{\khaai 胀}滿者。与{\khaai 調胃}承气湯。249

太陽病吐下发汗後。微煩。小便數。大便因堅。可与小承气湯。和之則愈。250

得病二三日。脉弱。无太陽柴胡証。煩躁。心下堅。至四日。雖能食。以{\khaai 小}承气湯少与。微和之。令小安。至六日。与承气湯一升。若不大便六七日。小便少者。雖不大便。但頭堅後溏。未定成堅。攻之必溏。当須小便利。屎定堅。乃可攻之。宜{\khaai 大}承气湯。251
	\footnote{
		「雖不大便」同《千金翼》卷九第八、《脉經》卷七第六、《玉函》卷五第十七、趙本卷九辨可下篇,趙本辨陽明病篇作「雖不受食」,《玉函》卷五第十八、卷三第五、成本作「雖不能食」。
		趙本「雖不受食」下有「一云不大便」小字註釋。
		「但頭堅後溏」同《千金翼》卷九第八、《脉經》卷七第六、《玉函》卷三第五、卷五第十七、卷五第十八,趙本、成本作「但初頭鞕後必溏」。
	}

傷寒六七日。目中不了了。睛不和。无表{\khaai 裏}証。大便難。身微熱者。此为実也。急下之。宜{\khaai 大}承气湯。252
	\footnote{
		「无表裏証」《聖惠方》卷八作「无外証」。
		「大承气湯」一云「大柴胡湯」。
	}

\hangindent 1em
\hangafter=0
傷寒六七日。目中瞳子不明。无外証。大便難。微熱者。此为実。宜急下之。{\shenghui}252

陽明病。发熱。汗多者。急下之。宜{\khaai 大}承气湯。253
	\footnote{
		趙本「大承气湯」下有「一云大柴胡湯」小字註釋。
	}

发汗不觧。腹滿痛者。急下之。宜{\khaai 大}承气湯。254
	\footnote{
		「大承气湯」一云「大柴胡湯」。
	}

腹滿不減。減不足言。当下之。宜{\khaai 大}承气湯。255
	\footnote{
		「大承气湯」一云「大柴胡湯」。
	}

陽明与少陽合病而利。脉不負者为順。負者。失也。互相克賊。名为負。256

脉滑而數者。有宿食也。当下之。宜{\khaai 大}承气湯。256
	\footnote{
		「大承气湯」一云「大柴胡湯」。
	}

病人无表裏証。发熱七八日。雖脉浮數。可下之。{\khaai 宜大柴胡湯。}假令下已。脉數不觧。合熱則消穀善飢。至六七日。不大便者。有瘀血。宜抵当湯。若脉數不觧。而下不止。必挾熱。便膿血。257.258
	\footnote{
		「宜大柴胡湯」同《玉函》卷五第十八、趙本卷九,《脉經》卷七第七作「屬大柴胡湯証」,《千金翼》卷九第八无。
		「下已」同《千金翼》卷九第八、《玉函》卷三第五、《脉經》卷七第八,趙本、成本、《玉函》卷六第十九作「已下」,錢超塵「按:作『下已』義长。已,止也。」從。
	}

傷寒发汗已。身目为黄。所以然者。寒濕相摶。在裏不觧故也。以为非瘀熱。而不可下。当於寒濕中求之。259
	\footnote{
		趙本「寒濕」下有「一作温」小字註釋。
		「以为非瘀熱而不可下当於寒濕中求之」同《玉函》卷三第五,趙本、成本作「以为不可下也於寒濕中求之」,《脉經》卷七第三、《千金翼》卷九第八、《玉函》卷六第十九无。
	}

傷寒七八日。身黄如橘。小便不利。腹微滿者。茵陳{\khaai 蒿}湯主之。260
	\footnote{
		「腹微滿者」同趙本、成本,《千金翼》卷九第八作「其腹微滿」,《脉經》卷七第七、《玉函》卷三第五作「少腹微滿」,《玉函》卷五第十八作「小腹微滿」,《聖惠方》卷八辨可下形証作「其腹微滿者」。
	}

傷寒。身黄。发熱。栀子蘗皮湯主之。261
	\footnote{
		「身黄发熱」同《玉函》卷三第五、趙本,《千金翼》卷九第八作「其人发黄」。
	}

傷寒。瘀熱在裏。身必发黄。麻黄連軺赤小豆湯主之。262
	\footnote{
		「連軺」《千金翼》卷九第八作「連翹」。
	}

\chapter{辨少陽病}

少陽之为病。口苦。咽乾。目眩。263

\hangindent 1em
\hangafter=0
傷寒三日。少陽受病。口苦乾燥。目眩。宜柴胡湯。{\shenghui}263

少陽中風。兩耳无所聞。目赤。胸中滿而煩。不可吐下。吐下則悸而驚。264

傷寒。脉弦細。頭痛。发熱者。屬少陽。少陽不可发汗。发汗則譫語。此屬胃。胃和則愈。胃不和。煩而悸。265

太陽病不觧。轉入少陽。脇下堅滿。乾嘔。不能食。往來寒熱。尚未吐下。其脉沈緊。可与小柴胡湯。266

{\khaai 少陽病。}若已吐下发汗温針。{\khaai 譫語。}柴胡湯証罷。此为壞病。知犯何逆。以法治之。267

三陽合病。脉浮大。上関上。但欲寐。目合則汗。268

傷寒六七日。无大熱。其人躁煩。此为陽去入陰故也。269

傷寒三日。三陽为{\sungii 𥁞}。三陰当受邪。其人反能食。而不嘔。此为三陰不受邪也。270

少陽病欲觧时。從寅{\sungii 𥁞}辰。272

\chapter{辨太陰病}

太陰之为病。腹滿而吐。食不下。下之益甚。时腹自痛。胸下堅結。273
	\footnote{
		唐弘宇按:本條《聖惠方》作:「傷寒四日。太陰受病。腹滿。吐食。下之益甚。时时腹痛。心胸堅滿。若脉浮者。可发其汗。沈者宜攻其裏也。发汗者宜桂枝湯。攻裏者宜承气湯。」《醫心方》作:「凡病。腹滿。吐食。下之益甚。」可見《醫心方》与《聖惠方》條文的一部分完全相同。《醫心方》与《聖惠方》都是非常古老的版本,且都未經過宋臣校正,這兩个版本的文字一致,應該不是巧合。
	}

\hangindent 1em
\hangafter=0
太陰之为病。腹滿而吐。食不下。自利益甚。时腹自痛。若下之。必胸下結鞕。{\zhaoben}273

%\hangindent 1em
%\hangafter=0
%傷寒四日。太陰受病。腹滿。吐食。下之益甚。时时腹痛。心胸堅滿。若脉浮者。可发其汗。沈者宜攻其裏也。发汗者宜桂枝湯。攻裏者宜承气湯。{\shenghui}273.276
%
%\hangindent 1em
%\hangafter=0
%凡病。腹滿。吐食。下之益甚。{\yixin}273

太陰病。脉浮者。可发汗。宜桂枝湯。276

太陰中風。四肢煩疼。{\khaai 脉}陽微陰濇而长者。为欲愈。274

太陰病欲觧时。從亥{\sungii 𥁞}丑。275

自利。不渴者。屬太陰。以其臓有寒故也。当温之。宜四逆輩。277

傷寒。脉浮而緩。手足自温者。系在太陰。太陰{\khaai 身}当发黄。若小便自利者。不能发黄。至七八日。雖暴煩。下利日十餘行。必自止。所以自止者。脾家実。腐穢当去故也。278
	\footnote{
		「腐穢当去」同《千金翼》卷十第一、《玉函》卷四第七、趙本、成本,《聖惠方》卷八辨太陰病形証作「腐穢已去」。
		錢超塵「按:作『已』義长。」
	}

\hangindent 1em
\hangafter=0
傷寒。脉浮而緩。手足自温。是为系在太陰。小便不利。其人当发黄。宜茵陳湯。\\
太陰病不觧。雖暴煩。下利十餘行而自止。所以自止者。脾家実。腐穢已去故也。宜橘皮湯。278{\shenghui}

{\khaai 本}太陽病。醫反下之。因尔腹滿时痛者。屬太陰。桂枝加芍藥湯主之。大実痛者。桂枝加大黄湯主之。279

太陰为病。脉弱。其人續自便利。設当行大黄芍藥者。宜減之。以其人胃气弱。易動故也。280
	\footnote{
		「太陰为病」《千金翼》卷十第一作「人无陽証」。
		趙本「易動故也」下有「下利者先煎芍藥三沸」小字註釋。
	}

\chapter{辨少陰病}

少陰之为病。脉微細。但欲寐。281

少陰病。欲吐不吐。心煩。但欲寐。五六日。自利而渴者。屬少陰。虗故引水自救。若小便色白者。少陰病形悉具。所以然者。以下焦虗寒。不能制水。故白也。282
	\footnote{
		「欲吐不吐心煩」同《玉函》卷四第八、趙本、成本,《千金翼》卷十第二作「欲吐而不煩」,《聖惠方》卷八作「其人欲吐而不煩」。
		「所以然者」趙本、成本作「小便白者」。
		「下焦虗寒」同《千金翼》卷十第二,《聖惠方》作「下焦有虗寒」,《玉函》卷四第八、趙本、成本作「下焦虗有寒」。
	}

\hangindent 1em
\hangafter=0
傷寒五日。少陰受病。其脉微細。但欲寐。其人欲吐而不煩。五日自利而渴者。屬陰虗。故引水以自救。小便白而利者。下焦有虗寒。故不能制水。而小便白也。宜龙骨牡蛎湯。{\shenghui}281.282

病人脉陰陽俱緊。反汗出者。为亡陽。此屬少陰。法当咽痛。而復吐利。283

少陰病。欬而下利。譫語者。被火气劫故也。小便必難。以强責少陰汗也。284

\hangindent 1em
\hangafter=0
少陰病。欬而下利。譫語。是为心臓有積熱故也。小便必難。宜服豬苓湯。{\shenghui}284

少陰病。脉細沈數。病为在裏。不可发汗。285

少陰病。脉微。不可发汗。亡陽故也。陽已虗。尺中弱濇者。復不可下之。286

少陰病。脉緊。至七八日。{\khaai 自}下利。脉暴微。手足反温。脉緊反去。此为欲觧。雖煩。下利。必自愈。287

少陰病。下利。若利{\khaai 自}止。惡寒而踡。手足温者。可治。288

少陰病。惡寒而踡。时自煩。欲去衣被者。可治。289
	\footnote{
		「可治」《千金翼》卷十第二作「不可治」。
	}

少陰中風。脉陽微陰浮者。为欲愈。290

少陰病欲觧时。從子{\sungii 𥁞}寅。291

少陰病。吐利。手足不逆{\khaai 冷}。反发熱者。不死。脉不至者。灸少陰七壯。292
	\footnote{
		趙本「脉不至者」下有「至一作足」小字註釋。
	}

\hangindent 1em
\hangafter=0
少陰病。吐利。手足逆而发熱。脉不足者。灸其少陰。{\shenghui}

少陰病八九日。一身手足{\sungii 𥁞}熱者。以熱在膀胱。必便血。293

少陰病。但厥。无汗。而强发之。必動其血。未知從何道出。或從口鼻。或從{\khaai 耳}目出。是为下厥上竭。为難治。294

少陰病。惡寒。身踡而利。手足逆{\khaai 冷}者。不治。295

少陰病。下利止而頭眩。时时自冒者死。297

少陰病。吐利。躁煩。四逆者死。296

少陰病。四逆。惡寒而踡。脉不至。不煩而躁者死。298
	\footnote{
		趙本本條末有「一作吐利而躁逆者死」小字註釋。
	}

少陰病六七日。息高者死。299

少陰病。脉微細沈。但欲卧。汗出。不煩。自欲吐。{\khaai 至}五六日。自利。復煩燥。不得卧寐者死。300

少陰病。始得之。反发熱。脉沈者。麻黄細辛附子湯主之。301

少陰病。得之二三日。麻黄附子甘草湯微发汗。以二三日无{\khaai 裏}証。故微发汗。302
	\footnote{
		「无裏証」同《玉函》卷四第八、成本,《脉經》卷七第三、《千金翼》卷十第二、趙本作「无証」。
	}

少陰病。得之二三日以上。心中煩。不得卧。黄連阿膠湯主之。303

少陰病。得之一二日。口中和。其背惡寒者。当灸之。附子湯主之。304

少陰病。身体痛。手足寒。骨節痛。脉沈者。附子湯主之。305

少陰病。下利。便膿血。桃花湯主之。306

少陰病二三日至四五日。腹痛。小便不利。下利不止。便膿血。桃花湯主之。307

少陰病。下利。便膿血者。可刺。308

少陰病。吐利。手足逆{\khaai 冷}。煩躁欲死者。吳茱萸湯主之。309

少陰病。下利。咽痛。胸滿。心煩。豬膚湯主之。310

少陰病二三日。咽痛者。可与甘草湯。不差者。与桔梗湯。311

少陰病。咽中傷。生瘡。不能語言。聲不出者。苦酒湯主之。312

少陰病。咽中痛。半夏散及湯主之。313

少陰病。下利。白通湯主之。314

少陰病。下利。脉微。服白通湯。利不止。厥逆。无脉。乾嘔。煩者。白通加豬膽汁湯主之。服湯脉暴出者死。微{\khaai 微}續{\khaai 出}者生。315
	\footnote{
		「利不止」《脉經》卷七第十八作「下利止」,《聖惠方》卷八作「止後」。
		「服湯脉暴出者死微微續出者生」同《聖惠方》卷八,《脉經》卷七第十八作「服湯藥其脉暴出者死微細者生」。
	}

\hangindent 1em
\hangafter=0
少陰病。下利。服白通湯。止後。厥逆。无脉。煩躁者。宜白通豬苓湯。其脉暴出者死。微微續出者生。{\shenghui}315

少陰病。二三日不已。至四五日。腹痛。小便不利。四肢沈重疼痛而利。此为有水气。其人或欬。或小便{\khaai 自}利。或下利。或嘔。玄武湯主之。316

少陰病。下利清穀。裏寒外熱。手足厥逆。脉微欲絕。身反不惡寒。其人面赤。或腹痛。或乾嘔。或咽痛。或利止{\khaai 而}脉不出。通脉四逆湯主之。317
	\footnote{
		「身反不惡寒」同《玉函》卷四第八、趙本、成本,《千金翼》卷十第二、《聖惠方》卷八作「身反惡寒」。
		「或利止而脉不出」《聖惠方》卷八作「或时利止而脉不出者」。
	}

少陰病。四逆。其人或欬。或悸。或小便不利。或腹中痛。或泄利下重。四逆散主之。318

少陰病。下利六七日。欬而嘔。渴。心煩不得眠。豬苓湯主之。319

\hangindent 1em
\hangafter=0
少陰病。下利。欬而嘔。煩渴。不得眠卧。宜豬苓湯。{\shenghui}319

少陰病。得之二三日。口燥。咽乾者。急下之。宜{\khaai 大}承气湯。320

少陰病。{\khaai 下}利清水。色青者。心下必痛。口乾燥者。急下之。宜{\khaai 大}承气湯。321
	\footnote{
		「色青者」趙本、《玉函》卷四第八作「色純青」。
		「急下之」同《玉函》卷四第八、《聖惠方》卷八、《外臺》卷一,《脉經》卷七第七、趙本、《玉函》卷五第十八、《千金翼》卷十第二作「可下之」,《千金翼》卷十宜下作「宜下之」。
		趙本「大承气湯」下有「一法用大柴胡」小字註釋。
	}

少陰病六七日。腹滿。不大便者。急下之。宜{\khaai 大}承气湯。322

少陰病。脉沈者。急温之。宜四逆湯。323

少陰病。其人飲食入則吐。心中温温欲吐。復不能吐。始得之。手足寒。脉弦遲。此胸中実。不可下也。当吐之。若膈上有寒飲。乾嘔者。不可吐。当温之。宜四逆湯。324

少陰病。下利。脉微濇。嘔而汗出。必數更衣。反少者。当温其上。灸之。325
	\footnote{
		「灸之」二字下,趙本、脉經有「一云灸厥陰可五十狀」小字註釋。
	}

\chapter{辨厥陰病
	\footnote{
		趙本此篇篇名下有「厥利嘔噦附合一十九法方一十六首」小字註釋。趙本第326至381條都歸於此篇之内。
		章太炎説:「近世間流行之《傷寒論》誤將厥利嘔噦列入厥陰篇中,殊失仲景立論之本旨。其実厥陰篇中,僅首條提綱,及條上著有厥陰病三字者,乃为厥陰病之本病,其餘厥利嘔噦諸條,当照《金匱玉函經》,与霍亂、勞復、陰陽易等另列一篇,庶幾无誤。」從。
	}
}

厥陰之为病。消渴。气上撞心。心中疼熱。飢而不欲食。{\khaai 甚者}食則吐{\khaai 蛔}。下之利不止。326
	\footnote{
		「甚者食則吐蛔」同《玉函》卷四第九,《脉經》卷七第六作「甚者則欲吐」,《千金翼》卷十第三作「甚者則欲吐蛔」,趙本、成本作「食則吐蛔」。
	}

厥陰中風。其脉微浮为欲愈。不浮为未愈。327

厥陰病欲觧时。從丑{\sungii 𥁞}卯。328

厥陰病。渴欲飲水者。少少与之即愈。329

\hangindent 1em
\hangafter=0
傷寒六日。渴欲飲水者。宜豬苓湯。{\shenghui}329

\chapter{辨厥利嘔噦}

諸四逆。厥者。不可下之。虗家亦然。330
	\footnote{
		「不可下之」同趙本卷六、卷九、成本、《脉經》卷七第六、《千金翼》卷十第三、《玉函》卷四第十、卷五第十七,《聖惠方》卷八辨不可下形証作「不可下也」,《千金翼》卷十忌下第五作「忌下」,趙本卷八辨不可吐篇作「不可吐也」,《脉經》卷七第四病不可吐証、《聖惠方》卷八辨不可吐形証作「不可吐之」。
	}

傷寒。先厥後发熱而利者。必自止。見厥復利。331

傷寒。始发熱六日。厥反九日而利。凡厥利者。当不能食。今反能食。恐为除中。食以黍餅。不发熱者。知胃气尚在。必愈。恐暴熱來出而復去也。後日脉之。其熱續在者。期之旦日夜半愈。所以然者。本发熱六日。厥反九日。復发熱三日。并前六日。亦为九日。与厥相應。故期之旦日夜半愈。後三日脉之而脉數。其熱不罷者。此为熱气有餘。必发癰膿。332
	\footnote{
		趙本「除中」下有「一云消中」小字註釋。
	}

傷寒。脉遲六七日。而反与黄芩湯徹其熱。脉遲为寒。而与黄芩湯復除其熱。腹中應冷。当不能食。今反能食。此为除中。必死。333

傷寒。先厥後发熱。下利必自止。而反汗出。咽中痛者。其㗋为痹。发熱。无汗。而利必自止。若不止。必便膿血。便膿血者。其㗋不痹。334

傷寒一二日至四五日。厥者。必发熱。前熱者後必厥。厥深者熱亦深。厥微者熱亦微。厥應下之。而反发汗者。必口傷爛赤。335
	\footnote{
		「前熱者後必厥」同趙本、成本、《玉函》卷四第十,《脉經》卷七第一、《玉函》卷五第十三、《千金翼》卷十第三作「前厥者後必熱」。錢超塵「按:当作『前熱者後必厥』,此論熱厥,熱在前而厥在後,凡『厥』字在前者,皆誤。」從。
	}

凡厥者。陰陽气不相順接。便为厥。厥者。手足逆冷是也。337

傷寒。病厥五日。熱亦五日。設六日当復厥。不厥者自愈。厥終不過五日。以熱五日。故知自愈。336

傷寒。脉微而厥。至七八日。膚冷。其人躁。无暫安时者。此为臓厥。非蛔厥也。蛔厥者。其人当吐蛔。今病者靜。而復时煩。此为臓寒。蛔上入膈。故煩。須臾復止。得食而嘔。又煩者。蛔聞食臭出。其人常自吐蛔。蛔厥者。烏梅丸主之。338

傷寒。熱少。厥微。指頭寒。默默不欲食。煩躁。數日。小便利。色白者。此熱除也。欲得食。其病为愈。若厥而嘔。胸脇煩滿者。其後必便血。339
	\footnote{
		趙本「指」下有「一作稍」小字註釋。
	}

病者手足厥冷。言我不結胸。小腹滿。按之痛。此冷結在膀胱関元也。340
	\footnote{
		「小腹」《千金翼》卷十第三作「少腹」。
	}

傷寒。发熱四日。厥反三日。復{\khaai 发}熱四日。厥少熱多。其病当愈。四日至七日。熱不除者。必便膿血。341

傷寒。厥四日。熱反三日。復厥五日。其病为進。寒多熱少。陽气退。故为進。342

傷寒六七日。脉微。手足厥{\khaai 冷}。煩躁。灸其厥陰。厥不還者死。343
	\footnote{
		「脉微」同趙本、成本,《脉經》卷七第十一、《玉函》卷四第十、《千金翼》卷十第十一作「其脉微」,《千金翼》卷十第三作「其脉數」,《聖惠方》卷八辨可灸形証作「脉數」。
	}

傷寒。{\khaai 发熱。}下利。厥逆。躁不得卧者死。344
	\footnote{
		「发熱」同趙本、成本、《玉函》卷四第十,《脉經》卷七第八、《千金翼》卷十第三无。
	}

傷寒。发熱。下利至{\khaai 甚。}厥不止者死。345

傷寒。六七日不利。忽发熱而利。其人汗出不止者死。有陰无陽故也。346

\hangindent 1em
\hangafter=0
傷寒。厥逆。六七日不利。便发熱而利者生。其人汗出。利不止者死。但有陰无陽故也。{\maijing}346

傷寒五六日。不結胸。腹濡。脉虗。復厥者。不可下。此为亡血。{\khaai 下之}死。347

傷寒。发熱而厥七日。下利者。为難治。348

傷寒。脉促。手足厥逆。可灸之。349

傷寒。脉滑而厥者。裏有熱也。白虎湯主之。350

手足厥寒。脉細欲絕。当歸四逆湯主之。若其人内有久寒。当歸四逆加吳茱萸生薑湯主之。351.352

大汗出。熱不去。内拘急。四肢疼。{\khaai 又}下利。厥逆而惡寒。四逆湯主之。353
	\footnote{
		「又下利」《脉經》作「下利」,《千金翼》作「若下利」。
	}

大汗{\khaai 出}若大下利。而厥冷者。四逆湯主之。354

病者手足厥冷。脉乍緊。邪結在胸中。心下滿而煩。飢不能食。病在胸中。当吐之。宜瓜蒂散。355

傷寒。厥而心下悸。宜先治水。当与茯苓甘草湯。卻治其厥。不尔。水漬入胃。必作利也。356

傷寒六七日。大下後。{\khaai 寸}脉沈遲。手足厥逆。下部脉不至。咽㗋不利。唾膿血。泄利不止者。为難治。麻黄升麻湯主之。357

傷寒四五日。腹中痛。若轉气下趨少腹者。为欲自利也。358

傷寒。本自寒下。醫復吐{\khaai 下}之。寒格。更逆吐{\khaai 下}。食入即出。乾薑黄芩黄連人参湯主之。359

下利。有微熱而渴。脉弱者自愈。360

下利。脉數。有微熱。汗出者。自愈。設{\khaai 脉}復緊。为未觧。361
	\footnote{
		「有微熱汗出者」同《玉函》卷四第十,趙本、成本作「有微熱汗出」,《千金翼》卷十第三作「若微发熱汗出者」。
		「設脉復緊」同《千金翼》卷十第三,趙本、成本、《玉函》卷四第十作「設復緊」。
	}

下利。手足厥{\khaai 冷}。无脉。{\khaai 当灸其厥陰。}灸之不温{\khaai 而脉不還}。反微喘者死。少陰負趺陽者为順。362

下利。寸脉反浮數。尺中自濇者。必清膿血。363

下利清穀。不可攻表。汗出必胀滿。364

下利。脉沈弦者下重。脉大者为未止。脉微弱數者为欲自止。雖发熱。不死。365

下利。脉沈而遲。其人面少赤。身有微熱。下利清穀者。必鬱冒。汗出而觧。其人必微厥。所以然者。其面戴陽。下虗故也。366

下利。脉反數而渴者。今自愈。設不差。必清膿血。以有熱故也。367

下利後。脉絕。手足厥{\khaai 冷}。晬时脉還。手足温者生。不還{\khaai 不温}者死。368

傷寒。下利日十餘行。脉反実者死。369

下利清穀。裏寒外熱。汗出而厥。通脉四逆湯主之。370

熱利下重者。白頭翁湯主之。371

下利。欲飲水者。为有熱也。白頭翁湯主之。373

下利。腹{\khaai 胀}滿。身体疼痛者。先温其裏。乃攻其表。温裏宜四逆湯。攻表宜桂枝湯。372

下利。譫語者。有燥屎也。宜{\khaai 小}承气湯。374

下利後更煩。按之心下濡者。为虗煩也。栀子{\khaai 豉}湯主之。375

嘔家有癰膿。不可治嘔。膿{\sungii 𥁞}自愈。376

嘔而发熱者。小柴胡湯主之。379

嘔而脉弱。小便復利。身有微熱。見厥者。難治。四逆湯主之。377

乾嘔。吐涎沫。頭痛者。吳茱萸湯主之。378
	\footnote{
		「頭痛者」同《趙本》、成本,《玉函》卷四第十、《千金翼》卷十第三作「而復頭痛」。
	}

傷寒。大吐大下之。極虗。復極汗者。其人外气怫鬱。復与之水。以发其汗。因得噦。所以然者。胃中寒冷故也。380

傷寒。噦而腹滿。視其前後。知何部不利。利之即愈。381

\chapter{辨霍亂}

問曰。病有霍亂者何。\\
答曰。嘔吐而利。此名霍亂。382

問曰。病发熱。頭痛。身疼。惡寒。吐利者。此屬何病。\\
答曰。此名霍亂。霍亂自吐下。又利止。復更发熱也。383
	\footnote{
		「霍亂自吐下又利止」同趙本,《脉經》卷八第四作「霍亂吐利止」,《千金翼》卷十霍亂病狀第六作「霍亂吐下利止」,《玉函》卷四第十一作「吐下利止」。
	}

傷寒。其脉微濇。本是霍亂。今是傷寒。卻四五日。至陰經上。轉入陰。必利。本嘔。下利者。不治。若其人似欲大便。而反失气。仍不利者。此屬陽明。便必堅。十三日愈。所以然者。經{\sungii 𥁞}故也。384
%	\footnote{「必利」同趙本、成本,《脉經》卷八第四作「必吐利」,《玉函》卷四第十一、《千金翼》卷十霍亂病狀第六作「当利」。}

下利後当便堅。堅則能食者愈。今反不能食。到後經中。頗能食。復過一經能食。過之一日当愈。若不愈。不屬陽明也。384

惡寒。脉微。而復利。利止。亡血也。四逆加人参湯主之。385
	\footnote{
		趙本「脉微」下有「一作緩」小字註釋。
		「利止亡血也」同趙本、成本、《玉函》卷四第十一,《千金翼》卷十霍亂病狀第六作「利止必亡血」
	}

霍亂。頭痛。发熱。身疼痛。熱多。欲飲水者。五苓散主之。寒多。不用水者。理中湯主之。386
	\footnote{
		「理中湯」同《千金翼》卷十霍亂病狀第六、《玉函》卷四第十一,趙本、成本作「理中丸」。唐弘宇按:霍亂病勢急劇,当用湯剂,故從《千金翼》、《玉函》。
	}

吐利止。而身痛不休者。当消息和觧其外。宜桂枝湯小和之。387

吐利。汗出。发熱。惡寒。四肢拘急。手足厥冷。四逆湯主之。388

既吐且利。小便復利。而大汗出。下利清穀。裏寒外熱。脉微欲絕。四逆湯主之。388

吐利已斷。汗出而厥。四肢拘急不觧。脉微欲絕。通脉四逆加豬膽汁湯主之。390
	\footnote{
		「吐利已斷」同《千金方》、趙本,《金匱要略》吳本、鄧本均作「吐已下斷」。
	}

吐利发汗後。其人脉平。小煩者。以新虗不勝穀气故也。391

\chapter{辨陰易病已後勞復
	\footnote{
		「陰易」除千金翼外其它版本均作「陰陽易」。
	}
}

傷寒陰易之为病。其人身体重。少气。少腹裏急。或引陰中拘攣。熱上衝胸。頭重不欲舉。眼中生眵。{\khaai 眼胞赤。}膝脛拘急。燒裩散主之。392
	\footnote{
		「眼中生眵」同《千金方》卷十第三,趙本、成本、《玉函》卷四第十二作「眼中生花」,《外臺》卷二作「眼中生䁾」。
		趙本「眼中生花」下有「花一作䏧」小字註釋。錢超塵説:「中國所藏五部宋本及日本内閣文庫坊刻本皆誤作「䏧」,日本安正本改作「眵」,是。」從。
		「眼胞赤」同《玉函》卷四第十二,趙本、成本、《千金方》卷十第三、《外臺》卷二均无,《千金翼》卷十第七作「痂胞赤」。
	}

大病差後勞復者。枳実栀子湯主之。393

傷寒差已後。更发熱者。小柴胡湯主之。脉浮者。以汗觧之。脉沈実者。以下觧之。394
	\footnote{
		趙本「脉沈実」下有「一作緊」小字註釋。
	}

大病差後。從腰以下有水气者。牡蛎澤瀉散主之。395

傷寒觧後。虗羸少气。气逆欲吐。竹枼石膏湯主之。397

大病差後。其人喜唾。久不了了者。胃上有寒。当温之。宜理中丸。396
	\footnote{
		「胃上有寒」千金翼作「胸上有寒」。
	}

病人脉已觧。而日暮微煩者。以病新差。人强与穀。脾胃气尚弱。不能消穀。故令微煩。損穀即愈。398

\chapter{发汗吐下後}

发汗後。身熱。又重发汗。胃中虗冷。必反吐也。0

大下後。口燥者。裏虗故也。0

发汗多。亡陽。狂語者。不可下。与柴胡桂枝湯。和其榮衛。以通津液。後自愈。

%\chapter{可与不可}

夫以为疾病至急。倉卒尋按。要者難得。故重集諸可与不可方治。比之三陰三陽篇中。此易見也。又时有不止是三陽三陰。出在諸可与不可中也。
	\footnote{
		錢超塵:『此五十六字叔和語,又見《金匱玉函經》卷五。王叔和《脉經》卷七第一節至第十八節为《傷寒論》條文,皆按可与不可排列,叔和謂之「諸可与不可」,是为叔和第一次整理者。然後又按三陽三陰整理,是为第二次整理。王叔和第三次整理者,見宋本《傷寒論》之辨不可发汗病脉証并治第十五至辨发汗吐下後病脉証并治第二十二。「重集」的原因是這些條文比放在三陰三陽中便於尋按。有些條文为三陰三陽所无,而是「出在諸可与不可中」,即出於《脉經》卷七諸可与不可。這類條文为數不多,故云「时有」。因可見《脉經》卷七諸可与不可基本为仲景《傷寒論》原始結構。換言之,《傷寒論》原始結構非按三陰三陽排列,而是按照「諸可与不可」結構排列,從兩漢文史及醫學文獻考察,辨証治療均按可与不可理法辨治,如《蘇武傳》、《華佗傳》等。王叔和拘於《素問熱論》一日傳一經之説,以为傷寒病亦如此除在《傷寒例》撰文觧説外,又按照三陰三陽排列《傷寒論》經文。王叔和理觧三陰三陽的意義与後世不同,王氏理觧的三陰三陽是日傳一經的概念,這種痕跡在淳化本《傷寒論》和《病源》裏有明顯的反應。叔和又依当时辨証施治習慣,又進行第三次整理,形成諸可与不可諸條。』
	}

\part{雜病論}
%2021.3.13已將傷寒部分誤改的「或」都改回「若」,金匱部分大致看了一下,應該沒有誤改,以後還需仔細校對一遍。

\chapter{臓腑經絡先後}

問曰。病人有气色見於面部。願聞其説。\\
師曰。鼻頭色青。腹中痛。苦冷者死。鼻頭色微黑者。有水气。色黄者。胸上有寒。色白者。亡血也。設微赤。非时者死。其目正圓者。痙。不治。又色青为痛。色黑为勞。色赤为風。色黄者便難。色鲜明者有留飲。
	\footnote{
		鄧本本段前有「問曰上工治未病」一段,吳本在臓腑經絡先後篇之前,未入正文,且此段文字都是无用不実之词,故移至衍文部分。
	}

師曰。病人語聲寂寂然。喜驚呼者。骨節間病。語聲喑喑然不徹者。心膈間病。語聲啾啾然細而长者。頭中病。

師曰。息搖肩者。心中堅。息引胸中上气者。欬。息张口短气者。肺痿唾沫。

師曰。吸而微數。其病在中焦。実也。当下之即愈。虗者不治。在上焦者。其吸促。在下焦者。其吸遠。此皆難治。呼吸動搖振振者。不治。

師曰。寸口脉動者。因其王时而動。假令肝王色青。四时各隨其色。肝色青而反色白。非其时色脉。皆当病。

問曰。有未至而至。有至而不至。有至而不去。有至而太過。何谓也。\\
師曰。冬至之後。甲子夜半少陽起。少陽之时陽始生。天得温和。以未得甲子。天因温和。此为未至而至也。以得甲子而天未温和。此为至而不至也。以得甲子而天大寒不觧。此为至而不去也。以得甲子而天温如盛夏五六月时。此为至而太過也。

師曰。病人脉浮者在前。其病在表。浮者在後。其病在裏。腰痛背强不能行。必短气而極也。

問曰。經云。厥陽獨行。何谓也。\\
師曰。此为有陽无陰。故称厥陽。

問曰。寸脉沈大而滑。沈則为実。滑則为气。実气相摶。血气入臓即死。入腑即愈。此为卒厥。何谓也。\\
師曰。唇口青。身冷。为入臓即死。如身和。汗自出。为入腑即愈。

問曰。脉脱。入臓即死。入腑即愈。何谓也。\\
師曰。非为一病。百病皆然。譬如浸淫瘡。從口起流向四肢者。可治。從四肢流來入口者。不可治。{\khaai 諸}病在外者可治。入裏者即死。

問曰。陽病十八。何谓也。\\
師曰。頭痛。項。腰。脊。臂。腳掣痛。\\
問曰。陰病十八。何谓也。\\
師曰。欬。上气。喘。噦。咽。腸鳴。胀滿。心痛。拘急。五臓病各有十八。合为九十病。人又有六微。微有十八病。合为一百八病。五勞。七傷。六極。婦人三十六病。不在其中。清邪居上。濁邪居下。大邪中表。小邪中裏。穀飪之邪。從口入者。宿食也。五邪中人。各有法度。風中於前。寒中於暮。濕傷於下。霧傷於上。風令脉浮。寒令脉急。霧傷皮腠。濕流関節。食傷脾胃。極寒傷經。極熱傷絡。

問曰。病有急当救裏救表者。何谓也。\\
師曰。病。醫下之。續得下利。清穀不止。身体疼痛者。急当救裏。後身体疼痛。清便自調者。急当救表也。

夫病痼疾。加以卒病。当先治其卒病。後乃治其痼疾也。

師曰。五臓病各有所得者愈。五臓病各有所惡。各隨其所不喜者为病。病者素不應食。而反暴思之。必发熱也。

夫諸病在臓。欲攻之。当隨其所得而攻之。如渴者。与豬苓湯。餘皆仿此。

\chapter{痙濕暍
	\footnote{
		本篇見於《金匱要略》吳本、《金匱要略》鄧本、《玉函》、《千金翼》、《脉經》,對比可見前二者为一个系統的傳本,後三者为另一系統,兩系統差別較大。
	}
}
%2021年3月2日,此篇已據吳本为底本重新校訂。

太陽病。发熱。无汗。反惡寒者。名曰剛痙。

太陽病。发熱。汗出。不惡寒者。名曰柔痙。

太陽病。发熱。脉沈細者。名曰痙。为難治。
%	\footnote{
%		「脉沈細者」吳本、鄧本作「脉沈而細者」。
%	}

太陽病。发汗太多。因致痙。
	\footnote{
		《脉經》、《玉函》、《千金翼》「发汗太多」作「发其汗」。
	}

病者身熱足寒。頸項强急。惡寒。时頭熱。面赤。目{\khaai 脉}赤。獨頭動搖。卒口噤。背反张者。痙病也。
	\footnote{
		「目脉赤」同《脉經》、《玉函》、《千金翼》、吳本,鄧本作「目赤」。
	}

痙病。发其汗者。寒濕相得。其表益虗。即惡寒甚。发其汗已。其脉如蛇。暴腹胀大者。为欲觧。脉如故。反伏弦者。痙。
	\footnote{
		「痙病」鄧本作「若」。
	}

夫風病。下之則痙。復发汗。必拘急。

夫痙脉。按之緊如弦。直上下行。
	\footnote{
		鄧本此條末有「一作築築而弦脉經云痙家其脉伏堅直上下」十八字雙行小字註釋。
	}

痙病有灸瘡。難治。

瘡家。雖身疼痛。不可发汗。汗出則痙。85

太陽病。其証備。身体强。几几然。脉反沈遲。此为痙。栝蔞桂枝湯主之。

太陽病。无汗。而小便反少。气上衝胸。口噤不得語。欲作剛痙。葛根湯主之。

剛痙为病。胸滿。口噤。卧不著席。腳攣急。其人必齘齒。可与大承气湯。
	\footnote{
		「剛痙」鄧本作「痙」。
	}

太陽病。関節疼痛而煩。脉沈而細者。此名濕痹。濕痹之{\sungii 𠊱}。其人小便不利。大便反快。但当利其小便。
%	\footnote{
%		「関節疼痛而煩」同
%		「脉沈而細」
%	}

濕家之为病。一身{\sungii 𥁞}疼。发熱。身色如熏黄。
	\footnote{
		鄧本「一身{\sungii 𥁞}疼」下有「一云疼煩」小字注釋。
	}

濕家。其人但頭汗出。背强。欲得被覆向火。若下之早則噦。{\khaai 或}胸滿。小便{\khaai 不}利。舌上如胎。以丹田有熱。胸上有寒。渴欲得飲而不能飲。則口燥煩也。
	\footnote{
		「小便不利」同吳本、鄧本、《玉函》卷二第一、趙本卷二辨痙濕暍,《千金翼》卷九第一、《脉經》卷八第二作「小便利」。
	}

濕家下之。額上汗出。微喘。小便利者死。下利不止者亦死。
	\footnote{
		吳本「小便利」下有「一云不利」小字註釋。
	}

問曰。風濕相摶。一身{\sungii 𥁞}疼痛。法当汗出而觧。值天陰雨不止。醫云此可发汗。汗之病不愈者。何也。\\
答曰。发其汗。汗大出者。但風气去。濕气{\khaai 仍}在。是故不愈也。若治風濕者。发其汗。但微微似欲出汗者。則風濕俱去也。

濕家。病身上疼痛。发熱。面黄而喘。頭痛。鼻塞而煩。其脉大。自能飲食。腹中和。无病。病在頭中寒濕。故鼻塞。内藥鼻中則愈。

濕家。身煩疼。可与麻黄加术湯。发其汗为宜。慎不可以火攻之。
	\footnote{
		「麻黄加术湯」同鄧本,吳本作「麻黄湯加术四兩」。
	}

病者一身{\sungii 𥁞}疼。发熱。日晡所劇者。此名風濕。此病傷於汗出当風。或久傷取冷所致也。可与麻杏薏甘湯。

%\hangindent 1em
%\hangafter=0
%濕家。始得病时。可与薏苡麻黄湯。(外臺)

風濕。脉浮。身重。汗出。惡風者。防己黄耆湯主之。

傷寒八九日。風濕相摶。身体疼煩。不能自轉側。不嘔。不渴。脉浮虗而濇者。桂枝附子湯主之。若其人大便堅。小便自利者。术附子湯主之。174
	\footnote{
		「术附子湯」同吳本,鄧本作「去桂加白术湯」。
	}

風濕相摶。骨節疼煩。掣痛。不得屈伸。近之則痛劇。汗出短气。小便不利。惡風。不欲去衣。或身微腫者。甘草附子湯主之。

太陽中熱者。暍是也。其人汗出。惡寒。身熱而渴。白虎{\khaai 加人参}湯主之。
	\footnote{
		吳本本條末有「一方白虎湯主之」小字註釋。
	}

太陽中暍。身熱疼重。而脉微弱。此以夏月傷冷水。水行皮{\khaai 膚}中所致也。瓜蒂湯主之。

太陽中暍。发熱。惡寒。身重而疼痛。其脉弦細芤遲。小便已。洒洒然毛聳。手足逆冷。小有勞。身即熱。口開。前板齒燥。若发其汗。則惡寒甚。加温針。則发熱甚。數下之。則淋甚。

\chapter{百合狐惑陰陽毒
	\footnote{
		唐弘宇按:百合病篇与其它篇有一个明顯的不同,那就是其它篇章顯然經過了重新整理,而本篇似乎本來就是一个整體。在較早期的文獻如《千金方》、《醫心方》中,《傷寒雜病論》的其它條文往往是散落在全書各処,而本篇卻是作为一个未被拆分的整體而存在。另外,《醫心方》所載的本篇方剂,都是在條文後直接給出方剂的藥物組成,无方名。據此猜測,本篇方剂的名称,「百合知母湯」、「百合滑石代赭湯」、「百合雞子湯」、「百合地黄湯」、「栝蔞牡蛎散」、「百合滑石散」,可能都是後來所起。以《醫心方》所載本篇的條文和方剂相結合的結構來看,《傷寒雜病論》前條文後方剂的結構,不適合本篇。
	}
}

論曰。百合病者。百脉一宗。悉致其病也。意欲食復不能食。常默默。欲得卧復不能卧。欲出行復不能行。飲食或有美时。或有不用聞食飲臭时。如寒无寒。如熱无熱。口苦。小便赤。諸藥不能治。得藥則劇吐利。如有神靈者。身形如和。其脉微數。每尿时頭痛者。六十日乃愈。若尿{\khaai 时}頭不痛。淅然者。四十日愈。若尿快然。但頭眩者。二十日愈。其証或未病而預見。或病四五日而出。或病二十日或一月微見者。各隨証治之。

治百合病。发汗後者。百合知母湯。

治百合病。下之後者。百合滑石代赭湯。

治百合病。吐之後者。百合雞子湯。

治百合病。不經吐下发汗。病形如初者。百合地黄湯。

治百合病。一月不觧。變成渴者。百合洗方。渴不差者。栝蔞牡蛎散。
	\footnote{
		《千金方》「渴不差者栝蔞牡蛎散主之」为註文。
	}

治百合病。變发熱者。百合滑石散。

治百合病。變腹中滿痛者。但取百合根隨多少。熬令黄色。擣篩为散。飲服方寸匕。日三。滿消痛止。
	\footnote{
		此條《金匱要略》吳本、鄧本均无,從《千金方》補入。
	}

百合病。見於陰者。以陽法救之。見於陽者。以陰法救之。見陽攻陰。復发其汗。此为逆。見陰攻陽。乃復下之。此亦为逆。

狐惑之为病。狀如傷寒。默默欲眠。目不得閉。卧起不安。蝕於㗋为惑。蝕於陰为狐。不欲飲食。惡聞食臭。其面目乍赤乍黑乍白。蝕於上部則聲喝。甘草瀉心湯主之。蝕於下部則咽乾。苦参湯洗之。蝕於肛者。雄黄熏之。

病者脉數。无熱。微煩。默默。但欲卧。汗出。初得之三四日。目赤如鳩眼。七八日目四眥黑。若能食者。膿已成也。赤小豆当歸散主之。

陽毒之为病。面赤斑斑如錦文。㗋咽痛。唾膿血。五日可治。七日不可治。陰毒之为病。面目青。身痛。狀如被打。㗋咽痛。死生与陽毒同。升麻鱉甲湯并主之。

\chapter{瘧}
%2021年3月2日,此篇已據吳本为底本重新校訂。

師曰。瘧脉自弦。弦數者多熱。弦遲者多寒。弦小緊者下之差。弦遲者可温之。弦緊者可发汗針灸也。浮大者可吐之。弦數者風疾也。以飲食消息止之。
	\footnote{
		「風疾」鄧本作「風发」。
	}

問曰。瘧以月一日发。当以十五日愈。設不差。当月{\sungii 𥁞}觧也。如其不差。当云何。\\
師曰。此結为癥瘕。名曰瘧母。急治之。宜鱉甲煎丸。

師曰。陰气孤絕。陽气獨发。則熱而少气。煩滿。手足熱而欲嘔。名曰癉瘧。若但熱不寒者。邪气内藏於心。外舍分肉之間。令人消鑠脱肉。
	\footnote{
		「煩滿」鄧本作「煩寃」。
	}

温瘧者。其脉如平。身无寒。但熱。骨節疼煩。时嘔。白虎加桂枝湯主之。

瘧。多寒者。名曰牡瘧。蜀漆散主之。

附方

治牡瘧。牡蛎湯。

瘧病发渴者。与小柴胡去半夏加栝蔞湯。

柴胡桂薑湯。{\scriptsize 此方治寒多微有熱。或但寒不熱。服一剂如神。故錄之。}

\chapter{中風歷節}
%2021年3月2日,此篇已據吳本为底本重新校訂。


夫風之为病。当半身不遂。或但臂不遂者。此为痹。脉微而數。中風使然。

寸口脉浮而緊。緊則为寒。浮則为虗。寒虗相摶。邪在皮膚。浮者血虗。絡脉空虗。賊邪不瀉。或左或右。邪气反緩。正气即急。正气引邪。喎僻不遂。邪在於絡。肌膚不仁。邪在於經。即重不勝。邪入於腑。即不識人。邪入於臓。舌即難言。口吐涎。
	\footnote{
		「口吐涎」同鄧本,吳本作「口吐於涎」。
	}

大風。四肢煩重。心中惡寒不足者。矦氏黑散主之。
	\footnote{
		吳本本條末有「外臺治風癲」小字註釋。
	}

寸口脉遲而緩。遲則为寒。緩則为虗。榮緩則为亡血。衛緩則为中風。邪气中經。則身癢而癮疹。心气不足。邪气入中。則胸滿而短气。
	\footnote{
		此條吳本无。
	}

風引湯。除熱。主癱癇。

病如狂狀。妄行。獨語不休。无寒熱。其脉浮。防己地黄湯主之。

%\hangindent 1em
%\hangafter=0
%言語狂錯。眼目茫茫。或見鬼。精神昏亂。防己地黄湯。{\qianjin}

頭風摩散。

寸口脉沈而弱。沈即主骨。弱即主筋。沈即为腎。弱即为肝。汗出入水中。如水傷心。歷節黄汗出。故曰歷節。

趺陽脉浮而滑。滑則穀气実。浮則汗自出。

少陰脉浮而弱。浮則为風。弱則血不足。風血相摶。即疼痛如掣。

盛人脉濇小。短气。自汗出。歷節疼。不可屈伸。此皆飲酒。汗出当風所致。

諸肢節疼痛。身体魁羸。腳腫如脱。頭眩短气。温温欲吐。桂枝芍藥知母湯主之。
	\footnote{
		唐弘宇按:「魁羸」二字,諸本寫法不同,吳本作「魁瘰」。陸淵雷説:「尪羸是短小瘦弱之意,此非歷節之主証。趙本作魁羸,不誤,狀関節之腫大也。」從。
	}

味酸則傷筋。筋傷則緩。名曰泄。鹹則傷骨。骨傷則痿。名曰枯。枯泄相摶。名曰斷泄。榮气不通。衛不獨行。榮衛俱微。三焦无所御。四屬斷絕。身体羸瘦。獨足腫大。黄汗出。脛冷。假令发熱。便为歷節也。
	\footnote{
		此條吳本无。
	}

病歷節。疼痛。不可屈伸。烏頭湯主之。

烏頭湯。治腳气。疼痛。不可屈伸。
	\footnote{
		此條吳本无。
	}

礬石湯。治腳气衝心。

附方

續命湯。治中風痱。身体不能自收。口不能言。冒昧不知痛処。或拘急。不得轉側。{\scriptsize 姚云。与大續命同。兼治婦人產後去血者。及老人小兒。}
	\footnote{
		「姚云」至「小兒」二十一字,鄧本作小字註釋,吳本作正文。從鄧本。
	}

治中風。手足拘急。百節疼痛。煩熱心亂。惡寒。經日不欲飲食。三黄湯。

治風虗。頭重眩。苦極。不知食味。暖肌補中。益精气。术附子湯。
	\footnote{
		吳本本條末有「方見風濕中見近效」小字註釋。
		《外臺祕要風頭眩方》載:「《近效》术附子湯,有桂心,无生薑、大棗」,并云:「此本仲景《傷寒論》方。」
	}

治腳气上入。少腹不仁。服八味丸。
	\footnote{
		《外臺祕要腳气不隨方》載崔氏方五首,其中第四首云:「若腳气上入少腹,少腹不仁,即服张仲景八味丸。」此方中用山茱萸五兩,澤瀉四兩,桂心三兩,附子二兩,餘与《金匱》同。推測本方当是仲景方,是崔氏將此方用於腳气病,後人便命名此方为崔氏八味丸。
		《金匱玉函要略方論輯義》:「《舊唐書經籍志》云:《崔氏纂要方》十卷,崔知悌撰。《新唐書藝文志》作崔行功撰。所謂崔氏其人也,不知者或以为仲景收錄崔氏之方,故詳及之。」
	}

治肉極。熱則身体津{\khaai 液}脱。腠理開。汗大泄。厉風气。下焦腳弱。越婢加术湯。
	\footnote{
		唐弘宇按:《千金方》有很长一段文字觧釋肉極,此條僅由這段話中不相連的幾句拼接而成。
	}

\chapter{血痹虗勞}
%2021年3月3日,此篇已據吳本为底本重新校訂。

問曰。血痹病從何得之。\\
師曰。夫尊樂人。骨弱肌膚盛。重因疲勞汗出。卧不时動搖。加被微風。遂得之。但以脉自微濇。在寸口。関上小緊。宜針引陽气。令脉和緊去則愈。

血痹。陰陽俱微。寸口関上微。尺中小緊。外証身体不仁。如風狀。黄耆桂枝五物湯主之。
	\footnote{
		「如風狀」同吳本,鄧本作「如風痹狀」。
	}

夫男子平人。脉大为勞。極虗亦为勞。

男子面色薄者。主渴及亡血。卒喘悸。脉浮者。裏虗也。

男子脉虗沈弦。无寒熱。短气。裏急。小便不利。面色白。时目瞑。兼衄。少腹滿。此为勞使之然。

勞之为病。其脉浮大。手足煩。春夏劇。秋冬差。陰寒精自出。痠削不能行。

男子脉浮弱而濇。为无子。精清泠。
	\footnote{
		吳本「精清泠」下有「一作冷」小字註釋。
		「精清泠」同吳本,鄧本作「精气清冷」。
		唐弘宇按:鄧本「精气清冷」義晦。《漢語大字典》:「泠:水清貌。」《漢語大詞典》:「泠:清涼,冷清。」《玉篇·水部》:「泠,清也。」《改併四聲篇海·水部》引《玉篇》:「泠,水清貌。」吳本「精清泠」義勝。
	}

夫失精家。少腹弦急。陰頭寒。目眩。髮落。脉極虗芤遲。为清穀。亡血。失精。脉得諸芤動微緊。男子失精。女子夢交。桂枝加龙骨牡蛎湯主之。天雄散亦主之。
	\footnote{
		吳本「目眩」下有「一作目眶痛」小字註釋。
		「天雄散亦主之」鄧本作「天雄散」,獨立成一條。
	}

男子平人。脉虗弱細微者。善盜汗也。

人年五六十。其病脉大者。痹俠背行。苦腸鳴。馬刀俠癭者。皆为勞得之。
	\footnote{
		吳本本條末有「脉經云人年五十六十其脉浮大者」小字註釋。
	}

脉沈小遲。名脱气。其人疾行則喘喝。手足逆寒。腹滿。甚則溏泄。食不消化也。

脉弦而大。弦則为減。大則为芤。減則为寒。芤則为虗。虗寒相摶。此名为革。婦人則半產漏下。男子則亡血失精。
	\footnote{
		吳遷在本條後有一條校註文字:「右四條。古本并无。鄧氏所編金匱方卻有之。今依補入。并見脉經第八卷虛勞脉証第六。」
		唐弘宇按:此條亦見於驚悸衄吐下血胸滿瘀血篇。
	}

虗勞。裏急。悸。衄。腹中痛。夢失精。四肢痠疼。手足煩熱。咽乾口燥。小建中湯主之。

虗勞。裏急。諸不足。黄耆建中湯主之。

虗勞。腰痛。少腹拘急。小便不利者。八味腎气丸主之。

虗勞。諸不足。風气百疾。薯蕷丸主之。

虗勞。虗煩。不得眠。酸棗{\khaai 仁}湯主之。

\hangindent 1em
\hangafter=0
虗勞。煩。悸。不得眠。酸棗湯主之。{\qianjin}

五勞。虗極。羸瘦。腹滿。不能飲食。食傷。憂傷。飲傷。房室傷。飢傷。勞傷。經絡榮衛气傷。内有乾血。肌膚甲錯。兩目黯黑。緩中補虗。大黄䗪虫丸主之。

附方

虗勞不足。汗出而悶。脉結。心悸。行動如常。不出百日。危急者。十一日死。炙甘草湯主之。

治冷勞。又主鬼疰。一門相染。獺肝散。
	\footnote{
		吳本本條末有「見肘後恐非仲景方」小字註釋。
	}

\chapter{肺痿肺癰欬嗽上气}
%2021年3月3日,此篇已據吳本为底本重新校訂。

問曰。熱在上焦者。因欬为肺痿。肺痿之病。何從得之。\\
師曰。或從汗出。或從嘔吐。或從消渴。小便利數。又被快藥下利。重亡津液。故得之。
	\footnote{
		鄧本「小便利數」後有「或從便難」四字。
	}

問曰。寸口脉數。其人欬。口中反有濁唾涎沫者何。\\
師曰。此为肺痿之病。若口中辟辟燥。欬即胸中隱隱痛。脉反滑數。此为肺癰。欬唾膿血。脉數虗者为肺痿。數実者为肺癰。

問曰。病欬逆。脉之何以知此为肺癰。当有膿血。吐之則死。其脉何類。\\
師曰。寸口脉微而數。微則为風。數則为熱。微則汗出。數則惡寒。風中於衛。呼气不入。熱過於榮。吸而不出。風傷皮毛。熱傷血脉。風舍於肺。其人則欬。口乾。喘滿。咽燥。不渴。时唾濁沫。时时振寒。熱之所過。血为凝滯。畜結癰膿。吐如米粥。始萌可救。膿成則死。

上气。面浮腫。肩息。其脉浮大。不治。又加利尤甚。

上气。躁而喘者。屬肺胀。欲作風水。发汗則愈。
%	\footnote{
%		「喘而躁」吳本作「躁而喘」。
%	}

肺痿。吐涎沫。而不能欬者。其人不渴。必遺尿。小便數。所以然者。以上虗不能制下故也。此为肺中冷。必眩。甘草乾薑湯以温其病。
	\footnote{
		鄧本「必眩」後有「多涎唾」三字。
		鄧本本條末有「若服湯已渴者屬消渴」九字,吳本作「服湯已小温覆之若渴者屬消渴」十三字,且为方後註釋,未入正文,脉經无。
	}

欬而上气。㗋中水雞聲。射干麻黄湯主之。

欬逆。气上衝。唾濁。但坐不得卧。皂莢丸主之。
	\footnote{
		「气上衝」鄧本作「上气时时」。
	}

上气。脉浮者。厚朴麻黄湯主之。脉沈者。澤漆湯主之。
	\footnote{
		「上气」鄧本作「欬而」。
	}

%				校訂紀錄:由於我之前的疏忽,厚朴麻黄湯未加入類聚方。
%				20210102更新:此方類聚方廣義中无,我依據金匱1的排列次序將其列在澤漆湯前。

火逆上气。咽㗋不利。止逆下气者。麥門冬湯主之。
	\footnote{
		「火逆」諸本均作「大逆」,編者改。
	}

肺癰。喘不得卧。葶藶大棗瀉肺湯主之。

欬而胸滿。振寒。脉數。咽乾。不渴。时出濁唾腥臭。久久吐膿如米粥者。为肺癰。桔梗湯主之。

欬逆倚息。此为肺胀。其人喘。目如脱狀。脉浮大者。越婢加半夏湯主之。
	\footnote{
		「欬逆倚息」鄧本作「欬而上气」。
	}

肺胀。欬而上气。煩躁而喘。脉浮者。心下有水。小青龙加石膏湯主之。

附方

肺痿。涎唾多。心中温温液液者。炙甘草湯主之。
	\footnote{
		吳本本條末有「方見虗勞門中。見外臺。」小字註釋。
	}

甘草湯。

肺痿。欬唾涎沫不止。咽燥而渴。生薑甘草湯主之。

肺痿。吐涎沫。桂枝去芍藥加皂莢湯主之。

欬而胸滿。振寒。脉數。咽乾。不渴。时出濁唾腥臭。久久吐膿如米粥者。为肺癰。桔梗白散主之。

治肺癰。葦湯。
	\footnote{
		此條吳本与鄧本差別較大,鄧本作「葦莖湯治欬有微熱煩滿胸中甲錯是为肺癰」,方中吳本用葦枼,鄧本用葦莖。
	}

肺癰。胸滿胀。一身面目浮腫。鼻塞。清涕出。不聞香臭酸辛。欬逆上气。喘鳴迫塞。葶藶大棗瀉肺湯主之。

欬而上气。肺胀。其脉浮。心下有水气。脇下痛引缺盆。小青龙加石膏湯主之。

\chapter{奔豚气吐膿驚怖火邪
	\footnote{
		唐弘宇按:本篇的篇名原为「奔豚气病脉証并治」,僅有奔豚气一証。但本篇第一條卻提到奔豚、吐膿、驚怖、火邪四種疾病,(未完)
	}
}
%2021年3月2日,此篇已據吳本为底本重新校訂。

師曰。病有奔豚。有吐膿。有驚怖。有火邪。此四部病。皆從驚发得之。

師曰。奔豚病者。從少腹起。上衝咽㗋。发作欲死。復還止。皆從驚恐得之。

奔豚。气上衝胸。腹痛。往來寒熱。奔豚湯主之。

燒針令其汗。針処被寒。核起而赤者。必发奔豚。气從少腹上衝心者。灸其核上各一壯。与桂枝加桂湯。117
	\footnote{
		鄧本「燒針」上有「發汗後」三字。
	}

发汗後。其人脐下悸。欲作奔豚。苓桂甘棗湯主之。65

夫嘔家有癰膿。不可治嘔。膿{\sungii 𥁞}自愈。376
	\footnote{
		本條原載於《金匱要略》嘔吐篇。
	}

排膿散。
	\footnote{
		本條原載於《金匱要略》瘡癰篇。
	}

排膿湯。
	\footnote{
		本條原載於《金匱要略》瘡癰篇。
	}

寸口脉動而弱。動則为驚。弱則为悸。
	\footnote{
		本條原載於《金匱要略》驚悸吐衄下血胸滿瘀血篇。
	}

火邪者。桂枝去芍藥加蜀漆牡蛎龙骨救逆湯主之。
	\footnote{
		唐弘宇按:本條原載於《金匱要略》驚悸吐衄下血胸滿瘀血篇,本條実際内容与篇名不符,今移至此。
	}

傷寒。脉浮。醫以火迫劫之。亡陽。{\khaai 必}驚狂。卧起不安。桂枝去芍藥加蜀漆牡蛎龙骨救逆湯主之。112
	\footnote{
		第112至119條原載於《傷寒論》太陽病篇。
	}

傷寒。其脉不弦緊而弱{\khaai 。弱}者必渴。被火必譫語。{\khaai 弱者。发熱。脉浮。觧之当汗出愈。}113

太陽病。以火熏之。不得汗。其人必躁。到經不觧。必清血。114
	\footnote{
		趙本「必清血」下有「名为火邪」四字。
	}
	
\hangindent 1em
\hangafter=0
太陽病。以火蒸之。不得汗者。其人必燥結。若不結。必下清血。其脉躁者。必发黄也。{\shenghui}114

脉浮。熱甚。而反灸之。此为実。実以虗治。因火而動。咽燥。必吐血。115

微數之脉。慎不可灸。因火为邪。則为煩逆。追虗逐実。血散脉中。火气雖微。内攻有力。焦骨傷筋。血難復也。116

\hangindent 1em
\hangafter=0
凡微數之脉。不可灸。因熱为邪。必致煩逆。内有損骨傷筋血枯之患。{\shenghui}116

脉浮。当以汗觧。而反灸之。邪无從出。因火而盛。病從腰以下必重而痹。此为火逆。若欲自觧。当先煩。煩乃有汗。隨汗而觧。何以知之。脉浮。故知汗出当觧。116

\hangindent 1em
\hangafter=0
脉当以汗觧。反以灸之。邪无所去。因火而盛。病当必重。此为逆治。若欲觧者。当发其汗而觧也。{\shenghui}116

燒針令其汗。針処被寒。核起而赤者。必发奔豚。气從少腹上衝心者。灸其核上各一壯。与桂枝加桂湯。117

火逆。下之。因燒針。煩躁者。桂枝甘草龙骨牡蛎湯主之。118

傷寒。加温針必驚。119

\chapter{胸痹心痛短气}
%2021年3月3日,此篇已據吳本为底本重新校訂。

師曰。夫脉当取太過与不及。陽微陰弦。即胸痹而痛。所以然者。責其極虗也。今陽虗。知在上焦。所以胸痹心痛者。以其陰弦故也。

平人无寒熱。短气不足以息者。実也。

胸痹之病。喘息欬唾。胸背痛。短气。寸口脉沈而遲。関上小緊數。栝蔞薤白白酒湯主之。
	\footnote{
		「胸背痛」同鄧本,吳本作「胸苦痛」。
	}

胸痹。不得卧。心痛徹背者。栝蔞薤白半夏湯主之。

胸痹。心中痞。留气結在胸。胸滿。脇下逆搶心。枳実薤白桂枝湯主之。理中湯亦主之。

胸痹。胸中气塞。短气。茯苓杏仁甘草湯主之。橘皮枳実生薑湯亦主之。

胸痹緩急者。薏苡仁附子散主之。

心中痞。諸逆。心懸痛。桂枝生薑枳実湯主之。

心痛徹背。背痛徹心。烏頭赤石脂丸主之。

九痛丸。治九種心痛。{\khaai 一虫心痛。二疰心痛。三風心痛。四悸心痛。五食心痛。六飲心痛。七冷心痛。八熱心痛。九去來心痛。}
	\footnote{
		從「一虫心痛」至「九去來心痛」三十七字,《金匱要略》諸本均无,劇《千金方》補。
	}

\chapter{腹滿寒疝宿食}
%2021年3月2日,此篇已據吳本为底本重新校訂。

趺陽脉微弦。法当腹滿。不滿者必便難。兩胠疼痛。此虗寒從下上也。当以温藥服之。

病者腹滿。按之不痛{\khaai 者}为虗。痛者为実。可下之。舌黄未下者。下之黄自去。

\hangindent 1em
\hangafter=0
傷寒。腹滿。按之不痛者为虗。痛者为実。当下之。舌黄未下者。下之黄自去。宜大承气湯。{\yuhan}

腹滿时減。復如故。此为寒。当与温藥。

病者痿黄。躁而不渴。胸中寒実。而利不止者。死。

寸口脉弦者。即脇下拘急而痛。其人嗇嗇惡寒也。

夫中寒家。喜欠。其人清涕出。发熱。色和者。善嚏。

中寒。其人下利。以裏虗也。欲嚏不能。此人肚中寒。

夫瘦人繞脐痛。必有風冷。穀气不行。而反下之。其气必衝。不衝者。心下則痞。

病腹滿。发熱十日。脉浮而數。飲食如故。厚朴七物湯主之。

%待補充{\maijing}

\hangindent 1em
\hangafter=0
厚朴湯。治腹滿。发數十日。眿浮數。食飲如故。{\yixin}
	\footnote{
		《醫心方》此條中的厚朴湯,其組成与《金匱》厚朴三物湯相同。
	}

腹中寒气。雷鳴。切痛。胸脇逆滿。嘔吐。附子粳米湯主之。

腹滿。脉數。厚朴三物湯主之。
	\footnote{
		此條鄧本作「痛而閉者厚朴三物湯主之」。
	}

病腹中滿痛者。此为実也。当下之。宜大柴胡湯。
	\footnote{
		此條鄧本作「按之心下滿痛者此为実也当下之宜大柴胡湯」。唐弘宇按:此篇为腹滿篇,鄧本卻是「心下滿」,誤。
	}

腹滿不減。減不足言。当下之。宜{\khaai 大}承气湯。255

心胸中大寒痛。嘔。不能飲食。腹中寒。上衝皮起。出見有頭足。上下痛而不可觸近。大建中湯主之。

脇下偏痛。发熱。其脉緊弦。此寒也。以温藥下之。宜大黄附子湯。
	\footnote{
		脉經无「发熱」二字。「緊弦」同鄧本,吳本作「弦緊」。
		唐弘宇按:根據臨床,寒実内結,腹痛便祕証,有时可見发熱症狀,但发熱不一定是全身性的,可以在某一局部出現,故「发熱」亦可与上句連讀为一句。
	}

寒气厥逆。赤丸主之。

寸口脉弦而緊。弦則衛气不行。衛气不行即惡寒。緊則不欲食。弦緊相摶。即为寒疝。寒疝繞脐痛。若发則白汗出。手足厥寒。其脉沈弦者。大烏頭煎主之。

寒疝。腹中痛。及脇痛。裏急者。当歸生薑羊肉湯主之。

寒疝。腹中痛。逆冷。手足不仁。若身疼痛。灸刺諸藥不能治。抵当烏頭桂枝湯主之。
	\footnote{
		「若身疼痛」同鄧本,吳本「若」作「者」,与上句連讀。
	}

附方

烏頭湯。治寒疝。腹中絞痛。賊風入腹。攻五臟。拘急不得轉側。叫呼。发作有时。使人陰縮。手足厥逆。

夫脉浮而緊乃弦。狀如弓弦。按之不移。脉數弦者。当下其寒。脉雙弦而遲者。必心下堅。脉大而緊者。陽中有陰。可下之。
	\footnote{
		鄧本此條的位置与吳本不同。
		「夫脉浮」同吳本,鄧本作「其脉數」。
		「雙弦」鄧本作「緊大」。
	}

寒疝。腹中痛者。柴胡桂枝湯主之。
	\footnote{
		此條鄧本作「柴胡桂枝湯治心腹卒中痛者」。
	}

卒疝。走馬湯主之。
	\footnote{
		此條鄧本作「走馬湯治中惡心痛腹胀大便不通」。
	}

問曰。人病有宿食。何以別之。\\
師曰。寸口脉浮而大。按之反濇。尺中亦微而濇。故知有宿食。大承气湯主之。

脉緊如轉索无常者。有宿食也。

脉緊。頭痛。風寒。腹中有宿食不化也。
	\footnote{
		吳本本條末有「一云寸口脉緊」小字註釋。
	}

脉數而滑者。実也。此有宿食。下之愈。宜大承气湯。

下利。不欲食者。有宿食也。当下之。宜大承气湯。

宿食在上脘。当吐之。宜瓜蒂散。

\chapter{五臓風寒積聚}
%2021年3月3日,此篇已據吳本为底本重新校訂。

肺中風者。口燥而喘。身運而重。冒而腫胀。

肺中寒者。吐濁涕。

肺死臓。浮之虗。按之弱如葱枼。下无根者。死。

肝中風者。頭目瞤。兩脇痛。行常傴。令人嗜甘。

肝中寒者。兩臂不舉。舌本燥。喜太息。胸中痛。不得轉側。食則吐而汗出也。
	\footnote{
		「兩臂」同鄧本,吳本作「兩脇」。
		吳本本條末有「脉經千金云时盜汗飲食已吐其汁」小字註釋。
	}

肝死臓。浮之弱。按之如索不來。或曲如蛇行者。死。

肝著。其人常欲蹈其胸上。先未苦时。但欲飲熱。旋覆花湯主之。
	\footnote{
		臣億等校。諸本旋覆花湯方本闕。
	}

心中風者。翕翕发熱。不能起。心中飢而欲食。食即嘔吐。
	\footnote{
		吳本、《千金方》有「而欲食」三字,鄧本无。
	}

心中寒者。其人苦病心如啖蒜狀。劇者心痛徹背。背痛徹心。譬如蛊注。其脉浮者。自吐乃愈。

心傷者。其人勞倦即頭面赤而下重。心中痛而自煩。发熱。当脐跳。其脉弦。此为心臓傷所致也。

心死臓。浮之実如豆麻。按之益躁疾者。死。

邪哭使魂魄不安者。血气少也。血气少者。屬於心。心气虗者。其人則畏。合目欲眠。夢遠行。而精神離散。魂魄妄行。陰气衰者为癲。陽气衰者为狂。

脾中風者。翕翕发熱。形如醉人。腹中煩重。皮肉瞤瞤而短气。

脾死臓。浮之大堅。按之如覆杯潔潔。狀如搖者。死。
	\footnote{
		臣億等詳。五臟各有中風中寒。今脾只載中風。腎中風中寒俱不載者。古文簡亂。亡失極多。去古既遠。无文可以補綴也。
	}

趺陽脉浮而濇。浮則胃气强。濇則小便數。浮濇相摶。大便則堅。其脾为約。麻子仁丸主之。

腎著之病。其人身体重。腰中冷。如坐水中。形如水狀。反不渴。小便自利。食飲如故。病屬下焦。身勞汗出。衣裏冷濕。久久得之。腰以下冷痛。腹重如帶五千錢。甘草乾薑茯苓白术湯主之。
	\footnote{
		吳本「身勞汗出」前有「從」字。
	}

腎死臓。浮之堅。按之亂如轉丸。益下入尺中者。死。

問曰。三焦竭部。上焦竭善噫。何谓也。\\
師曰。上焦受中焦气未和。不能消穀。故令噫耳。下焦竭。則遺尿。失便。其气不和。不能自禁制。不須治。久自愈。

師曰。熱在上焦者。因欬为肺痿。熱在中焦者。則为堅。熱在下焦者。則尿血。亦令淋閉不通。

大腸有寒者。多鶩溏。有熱者。便腸垢。

小腸有寒者。其人下重。便血。有熱者。必痔。

問曰。病有積。有聚。有䅽气。何谓也。\\
師曰。積者。臓病也。終不移。聚者。腑病也。发作有时。展轉痛移。为可治。䅽气者。脇下痛。按之則愈。復发为䅽气。諸積大法。脉來細而附骨者。乃積也。寸口。積在胸中。微出寸口。積在㗋中。関上。積在脐傍。上関上。積在心下。微下関。積在少腹。尺中。積在气衝。脉出左。積在左。脉出右。積在右。脉兩出。積在中央。各以其部処之。

\chapter{痰飲欬嗽}
%2021年3月4日,此篇已據吳本为底本重新校訂。

問曰。夫飲有四。何谓也。\\
師曰。有痰飲。有懸飲。有溢飲。有支飲。\\
問曰。四飲何以为異。\\
師曰。其人素盛今瘦。水走腸間。瀝瀝有聲。谓之痰飲。飲後水流在脇下。欬唾引痛。谓之懸飲。飲水流行。歸於四肢。当汗出而不汗出。身体疼重。谓之溢飲。其人欬逆。倚息。短气。不得卧。其形如腫。谓之支飲。
	\footnote{
		「瀝瀝」諸病源{\sungii 𠊱}論引作「漉漉」。
	}

水在心。心下堅築{\khaai 築}。短气。惡水。不欲飲。

水在肺。吐涎沫。欲飲水。

水在脾。少气。身重。

水在肝。脇下支滿。嚏而痛。

水在腎。心下悸。

夫心下有留飲。其人背寒冷如手大。

留飲者。脇下痛引缺盆。欬嗽則輒已。

胸中有留飲。其人短气而渴。四肢歷節痛。脉沈者。有留飲。

膈上之病。滿喘欬唾。发則寒熱。背痛。腰疼。目泣自出。其人振振身瞤劇。必有伏飲。
	\footnote{
		「膈上之病」鄧本作「膈上病痰」。
		「欬唾」鄧本作「欬吐」。
	}

夫病人卒飲水多。必暴喘滿。凡食少飲多。水停心下。甚者則悸。微者短气。

脉雙弦者。寒也。皆大下後喜虗。脉偏弦者。飲也。
	\footnote{
		「喜虗」鄧本无「喜」字。
	}

肺飲不弦。但苦喘。短气。

支飲。亦喘而不能卧。加短气。其脉平也。

病痰飲者。当以温藥和之。

心下有痰飲。胸脇支滿。目眩。苓桂术甘湯主之。

夫短气。有微飲。当從小便去之。苓桂术甘湯主之。腎气丸亦主之。

病者脉伏。其人欲自利。利反快。雖利。心下續堅滿。此为留飲欲去故也。甘遂半夏湯主之。
	\footnote{
		「利反快」同鄧本,吳本作「利者反快」。
	}

脉浮而細滑。傷飲。
	\footnote{
		此條吳本无。
	}

脉弦數。有寒飲。冬夏難治。
	\footnote{
		此條吳本无。
	}

脉沈而弦者。懸飲内痛。
	\footnote{
		此條吳本无。
	}

病懸飲者。十棗湯主之。

%病溢飲{\khaai 者}。当发其汗。宜大青龙湯。
病溢飲者。当发其汗。大青龙湯主之。小青龙湯亦主之。
	\footnote{
		此條吳本作「病溢飲当发其汗宜大青龙湯」、「病溢飲者当发其汗宜小青龙湯」兩條。
	}

膈間支飲。其人喘滿。心下痞堅。面色黎黑。其脉沈緊。得之數十日。醫吐下之不愈。木防己湯主之。虗者即愈。実者三日復发。復与不愈者。宜去石膏加茯苓芒硝湯。
	\footnote{
		「虗者」至「芒硝湯」二十五字同吳本,此段文字鄧本位於正文,吳本原位於煎服法之後,編者將其移至此。
		「宜去石膏加茯苓芒硝湯」鄧本作「宜木防己湯去石膏加茯苓芒硝湯主之」。
	}

心下有支飲。其人苦冒眩。澤瀉湯主之。

支飲。胸滿者。厚朴大黄湯主之。

支飲。不得息。葶藶大棗瀉肺湯主之。

嘔家本渴。渴者为欲觧。今反不渴。心下有支飲故也。小半夏湯主之。
	\footnote{
		鄧本本條末有「千金云小半夏加茯苓湯」小字註釋。吳本此註釋位於小半夏湯方剂之後。
	}

腹滿。口舌乾燥。此腸間有水气。防己椒目葶藶大黄丸主之。

卒嘔吐。心下痞。膈間有水。眩悸者。小半夏加茯苓湯主之。

假令瘦人脐下悸。吐涎沫而癲眩。{\khaai 此}水也。五苓散主之。
	\footnote{
		「脐下悸」同《脉經》,吳本、鄧本作「脐下有悸」。
	}

附方

主心胸中有停痰宿水。自吐出水後。心胸間虗。气滿。不能食。消痰气。令能食。茯苓飲。

欬家。其脉弦。为有水。可与十棗湯。

夫有支飲家。欬煩。胸中痛者。不卒死。至一百日一歲。与十棗湯。
%2021.3.4我忘記之前为什麼在「一百日」和「一歲」之間加了个{\khaai 或},查吳本鄧本都沒有「或」,今將其刪除。

久欬數歲。其脉弱者可治。実大數者死。其脉虗者必苦冒。其人本有支飲在胸中故也。治屬飲家。

欬逆倚息。小青龙湯主之。
	\footnote{
		鄧本「倚息」後有「不得卧」三字。
	}
	
青龙湯下已。多唾。口燥。寸脉沈。尺脉微。手足厥逆。气從少腹上衝胸咽。手足痹。其人面翕然如醉。因復下流陰股。小便難。时復冒。可与茯苓桂枝五味子甘草湯。治其气衝。
	\footnote{
		「少腹」鄧本作「小腹」。
		「其人面翕然如醉」鄧本作「其面翕熱如醉狀」。
		「可与」鄧本作「者与」。
	}

衝气即低。而反更欬滿者。用茯苓桂枝五味子甘草湯。去桂加乾薑細辛。以治其欬滿。
	\footnote{
		「欬滿」鄧本作「欬胸滿」。
		「用茯苓桂枝五味子甘草湯」鄧本作「用桂苓五味甘草湯」,吳本作「因茯苓五味子甘草」。
	}

欬滿則止。而復更渴。衝气復发者。以細辛乾薑为熱藥也。此法不当逐渴。而渴反止者。为支飲也。支飲法当冒。冒者必嘔。嘔者復内半夏。以去其水。
	\footnote{
		「此法不当逐渴」同吳本,鄧本作「服之当遂渴」。
	}

水去。嘔則止。其人形腫。可内麻黄。以其欲逐痹。故不内麻黄。乃内杏仁也。若逆而内麻黄者。其人必厥。所以然者。以其人血虗。麻黄发其陽故也。
	\footnote{
		此條鄧本与吳本差異較大,鄧本作「水去。嘔止。其人形腫者。加杏仁主之。其証應内麻黄。以其人遂痹。故不内之。若逆而内之者。必厥。所以然者。以其人血虗。麻黄发其陽故也。」
	}

若面熱如醉狀者。此为胃中熱。上熏其面令熱。加大黄湯和之。
	\footnote{
		此條鄧本与吳本差異較大,鄧本作「若面熱如醉。此为胃熱上衝熏其面。加大黄以利之。」
	}

先渴卻嘔。为水停心下。此屬飲家。小半夏加茯苓湯主之。
	\footnote{
		「先渴卻嘔」鄧本作「先渴後嘔」。
		「小半夏加茯苓湯」鄧本作「小半夏茯苓湯」。
	}

\chapter{消渴小便{\khaai 不}利淋}

厥陰之为病。消渴。气上撞心。心中疼熱。飢而不欲食。{\khaai 甚者}食則吐{\khaai 蛔}。下之利不止。326

寸口脉浮而遲。浮即为虗。遲即为勞。虗則衛气不足。勞則榮气竭。趺陽脉浮而數。浮即为气。數即消穀而大{\khaai 便}堅。气盛則溲數。溲數即堅。堅數相摶。即为消渴。
	\footnote{
		「大便堅」吳本作「矢堅」,鄧本作「大堅」,「便」字为編者所加。
	}

男子消渴。小便反多。以飲一斗。小便一斗。腎气丸主之。

脉浮。小便不利。微熱。消渴者。宜利小便。发汗。五苓散主之。

渴欲飲水。水入則吐者。名曰水逆。五苓散主之。

渴欲飲水不止者。文蛤散主之。

淋之为病。小便如粟狀。小腹弦急。痛引脐中。

趺陽脉數。胃中有熱。即消穀引食。大便必堅。小便即數。

淋家不可发汗。发汗必便血。84

小便不利者。有水气。其人若渴。栝蔞瞿麥丸主之。
	\footnote{
		「若渴」徐本作「苦渴」。
	}

小便不利。蒲灰散主之。滑石白魚散。茯苓戎鹽湯并主之。

渴欲飲水。口乾舌燥者。白虎加人参湯主之。

脉浮。发熱。渴欲飲水。小便不利者。豬苓湯主之。

\chapter{水气}
%2021.3.13已據吳本重新校訂。

師曰。病有風水。有皮水。有正水。有石水。有黄汗。風水。其脉自浮。外証骨節疼痛。{\khaai 其人}惡風。皮水。其脉亦浮。外証胕腫。按之沒指。不惡風。其腹如鼓。不渴。当发其汗。正水。其脉沈遲。外証自喘。石水。其脉自沈。外証腹滿。不喘。黄汗。其脉沈遲。身{\khaai 体}发熱。胸滿。四肢頭面腫。久不愈。必致癰膿。

脉浮而洪。浮則为風。洪則为气。風气相擊。身体洪腫。汗出乃愈。惡風則虗。此为風水。不惡風者。小便通利。上焦有寒。其人多涎。此为黄汗。
	\footnote{
		「風气相摶」吳本「摶」作「擊」,據鄧本改。
		此條鄧本与吳本差別較大,鄧本作「脉浮而洪。浮則为風。洪則为气。風气相摶。風强則为癮疹。身体为癢。癢为泄風。久为痂癩。气强則为水。難以俛仰。風气相擊。身体洪腫。汗出則愈。惡風則虗。此为風水。不惡風者。小便通利。上焦有寒。其口多涎。此为黄汗。」平脉法篇中此條重出,作「脉浮而大。浮为風虗。大为气强。風气相摶。必成癮疹。身体为癢。癢者名泄風。久久为痂癩。」
	}

寸口脉沈滑者。中有水气。面目腫大。有熱。名曰風水。視人之目窠上微擁。如{\khaai 蠶}新卧起狀。其頸脉動。时时欬。按其手足上。陷而不起者。風水。
	\footnote{
		吳本无此條。
	}

太陽病。脉浮而緊。法当骨節疼痛。反不疼。身体反重而痠。其人不渴。汗出即愈。此为風水。惡寒者。此为極虗。发汗得之。渴而不惡寒者。此为皮水。身腫而冷。狀如周痹。胸中窒。不能食。反聚痛。暮躁不眠。此为黄汗。痛在骨節。欬而喘。不渴者。此为脾胀。其狀如腫。发汗即愈。然諸病此者。渴而下利。小便數者。皆不可发汗。

裏水者。一身面目自洪腫。其脉沈。小便不利。故令病水。假如小便自利。亡津液。故令渴也。
	\footnote{
		鄧本此條与上條合为一條。
		「自洪腫」同吳本,鄧本作「黄腫」。
		「此亡津液」吳本无「此」字,據鄧本補。
		鄧本本條末有「越婢加术湯主之」七字,吳本无。
	}

趺陽脉当伏。今反緊。本自有寒疝瘕。腹中痛。醫反下之。下之即胸滿短气。

趺陽脉当伏。今反數。本自有熱。消穀。小便數。今反不利。此欲作水。

寸口脉浮而遲。浮脉則熱。遲脉則潛。熱潛相摶。名曰沈。趺陽脉浮而數。浮脉則熱。數脉則止。熱止相摶。名曰伏。沈伏相摶。名曰水。沈則絡脉虗。伏則小便難。虗難相摶。水走皮膚。即为水矣。

寸口脉弦而緊。弦則衛气不行。衛气不行即惡寒。水不沾流。走於腸間。

少陰脉緊而沈。緊則为痛。沈則为水。小便即難。脉得諸沈。当責有水。身体腫重。水病脉出者死。

夫水病人。目下有卧蠶。面目鲜澤。脉伏。其人消渴。病水。腹大。小便不利。其脉沈絕者。有水。可下之。

問曰。病下利後。渴飲水。小便不利。腹滿。因腫者。何也。\\
答曰。此法当病水。若小便自利。及汗出者。自当愈。

心水者。其身重而少气。不得卧。煩而躁。其人陰腫。

肝水者。其腹大。不能自轉則。脇下腹痛。时时津液微生。小便續通。

肺水者。其身腫。小便難。时时鴨溏。

脾水者。其腹大。四肢苦重。津液不生。但苦少气。小便難。

腎水者。其腹大。脐腫。腰痛。不得尿。陰下濕如牛鼻上汗。其足逆冷。面反瘦。

師曰。諸有水者。腰以下腫。当利小便。腰以上腫。当发汗乃愈。

師曰。寸口脉沈而遲。沈則为水。遲則为寒。寒水相摶。趺陽脉伏。水穀不化。脾气衰則鶩溏。胃气衰則身腫。少陰脉細。男子則小便不利。婦人則經水不通。經为血。血不利則为水。名曰血分。
	\footnote{
		「少陰脉細」上鄧本有「少陽脉卑」四字。
	}

問曰。病者苦水。面目身体四肢皆腫。小便不利。脉之不言水。反言胸中痛。气上衝咽。狀如炙肉。当微欬喘。審如師言。其脉何類。\\
師曰。寸口脉沈而緊。沈为水。緊为寒。沈緊相摶。結在関元。始时当微。年盛不覺。陽衰之後。榮衛相干。陽損陰盛。結寒微動。腎气上衝。㗋咽塞噎。脇下急痛。醫以为留飲。而大下之。气擊不去。其病不除。後重吐之。胃家虗煩。咽燥欲飲水。小便不利。水穀不化。面目手足浮腫。又与葶藶丸下水。当时如小差。食飲過度。腫復如前。胸脇苦痛。象若奔豚。其水揚溢。則浮欬喘逆。当先攻擊衛气令止。乃治欬。欬止。其喘自差。先治新病。病当在後。

風水。脉浮。身重。汗出。惡風者。防己黄耆湯主之。腹痛者。加芍藥。

風水。惡風。一身悉腫。脉浮。不渴。續自汗出。无大熱。越婢湯主之。

皮水为病。四肢腫。水气在皮膚中。四肢聶聶動者。防己茯苓湯主之。

裏水。越婢加术湯主之。甘草麻黄湯亦主之。

水之为病。其脉沈小。屬少陰。浮者为風。无水。虗胀者为气。水。发其汗即已。脉沈者。宜附子麻黄湯。浮者。宜杏子湯。
	\footnote{
		「附子麻黄湯」同吳本,鄧本作「麻黄附子湯」,本方所用藥物、剂量与趙本《傷寒論》「麻黄附子甘草湯」同。
		唐弘宇按:杏子湯方未見,可能是大青龙湯。方名後小字註釋云:「恐是麻黄杏子甘草石膏湯」。
	}

厥而皮水者。蒲灰散主之。

問曰。黄汗之为病。身体腫。发熱。汗出而渴。狀如風水。汗沾衣。色正黄如檗汁。脉自沈。\\
問曰。何從得之。\\
師曰。以汗出入水中浴。水從汗孔入得之。
	\footnote{
		「身体腫」《千金方》作「身体洪腫」。「汗出而渴」千金要方作「汗出不渴」。
	}

黄汗。黄耆芍藥桂枝苦酒湯主之。
	\footnote{
		「黄汗黄耆芍藥桂枝苦酒湯主之」同吳本,鄧本作「宜耆芍桂枝苦酒湯主之」与上條合为一條。
	}

黄汗之病。兩脛自冷。假令发熱。此屬歷節。食已汗出。又身常暮{\khaai 卧}盜汗出者。此勞气也。若汗出已。反发熱者。久久其身必甲錯。发熱不止者。必生惡瘡。若身重。汗出已輒輕者。久久必身瞤瞤。即胸中痛。又從腰以上必汗出。下无汗。腰髖弛痛。如有物在皮中狀。劇者不能食。身疼重。煩躁。小便不利。此为黄汗。桂枝加黄耆湯主之。
	\footnote{
		「桂枝加黄耆湯」同鄧本,吳本作「桂枝加黄耆五兩湯」。
		唐弘宇按:此処可以看出吳本更接近仲景書原本。
	}

師曰。寸口脉遲而濇。遲則为寒。濇为血不足。趺陽脉微而遲。微則为气。遲則为寒。寒气不足。則手足逆冷。手足逆冷。則榮衛不利。榮衛不利。則腹滿脇鳴相逐。气轉膀胱。榮衛俱勞。陽气不通即身冷。陰气不通即骨疼。陽前通則惡寒。陰前通則痹不仁。陰陽相得。其气乃行。大气一轉。其气乃散。実則失气。虗則遺尿。名曰气分。

气分。心下堅。大如盤。邊如旋杯。水飲所作。桂枝去芍藥加麻黄細辛附子湯主之。

心下堅。大如盤。邊如旋盤。水飲所作。枳実术湯主之。

附方

夫風水。脉浮为在表。其人或頭汗出。表无他病。病者但下重。故知從腰以上为和。腰以下当腫及陰。難以屈伸。防己黄耆湯主之。
	\footnote{
		吳本此條末有「方見風濕中。見外臺。出深師。」小字註釋。
	}

\chapter{黄疸}

\hangindent 1em
\hangafter=0
論曰。黄有五種。有黄汗。黄疸。穀疸。酒疸。女勞疸。黄汗者。身体四肢微腫。胸滿。不渴。汗出如黄蘗汁。良由大汗出卒入水中所致。黄疸者。一身面目悉黄如橘。由暴得熱。以冷水洗之。熱因留胃中。食生黄瓜熏上所致。若成黑疸者多死。穀疸者。食畢頭眩。心忪怫鬱不安。而发黄。由失飢大食。胃气衝熏所致。酒疸者。心中懊痛。足脛滿。小便黄。面发赤斑黄黑。由大醉当風入水所致。女勞疸者。身目皆黄。发熱。惡寒。小腹滿急。小便難。由大勞大熱而交接竟入水所致。但依後方治之。{\qianjin}
	\footnote{
		此條《金匱要略》无,從《千金方》補入。
	}

寸口脉浮而緩。浮則为風。緩則为痹。痹非中風。四肢苦煩。脾色必黄。瘀熱以行。

趺陽脉緊而數。數則为熱。熱則消穀。緊則为寒。食即为滿。尺脉浮为傷腎。趺陽脉緊为傷脾。風寒相摶。食穀即眩。穀气不消。胃中苦濁。濁气下流。小便不通。陰被其寒。熱流膀胱。身体{\sungii 𥁞}黄。名曰穀疸。額上黑。微汗出。手足中熱。薄暮即发。膀胱急。小便自利。名曰女勞疸。腹如水狀。不治。心中懊憹而熱。不能食。时欲吐。名曰酒疸。

陽明病。脉遲者。食難用飽。飽則发煩。頭眩。必小便難。此欲作穀疸。雖下之。腹滿如故。所以然者。脉遲故也。

夫病酒黄疸。必小便不利。其{\sungii 𠊱}心中熱。足下熱。是其証也。

酒黄疸者。或无熱。靖言了{\khaai 了}。腹滿欲吐。鼻燥。其脉浮者。先吐之。沈弦者。先下之。

酒疸。心中熱。欲嘔者。吐之愈。

酒疸下之。久久为黑疸。目青。面黑。心中如噉蒜虀狀。大便正黑。皮膚爪之不仁。其脉浮弱。雖黑。微黄。故知之。

師曰。病黄疸。发熱。煩喘。胸滿。口燥者。以病发时火劫其汗。兩熱所得。然黄家所得。從濕得之。一身{\sungii 𥁞}发熱而黄。肚熱。熱在裏。当下之。

脉沈。渴欲飲水。小便不利者。皆发黄。
	\footnote{
		「脉沈」吳本作「脉浮」。
	}

腹滿。舌痿黄燥。不得睡。屬黄家。

黄疸之病。当以十八日为期。治之十日以上差。反劇。为難治。

疸而渴者。其疸難治。疸而不渴者。其疸可治。发於陰部。其人必嘔。{\khaai 发於}陽部。其人振寒而发熱也。

穀疸之为病。寒熱不食。食即頭眩。心胸不安。久久发黄。为穀疸。茵陳蒿湯主之。

黄家。日晡所发熱。而反惡寒。此为女勞得之。膀胱急。少腹滿。身{\sungii 𥁞}黄。額上黑。足下熱。因作黑疸。其腹胀如水狀。大便必黑。时溏。此女勞之病。非水也。腹滿者難治。硝石礬石散主之。

酒黄疸。心中懊憹。或熱痛。栀子{\khaai 枳実豉}大黄湯主之。

諸病黄家。但利其小便。假令脉浮。当以汗觧之。宜桂枝加黄耆湯主之。

諸黄。豬膏髮煎主之。

黄疸病。茵陳五苓散主之。

黄疸。腹滿。小便不利而赤。自汗出。此为表和裏実。当下之。宜大黄{\khaai 黄蘗栀子}硝石湯。

黄疸病。小便色不變。欲自利。腹滿而喘。不可除熱。熱除必噦。噦者。小半夏湯主之。

諸黄。腹痛而嘔者。宜柴胡湯。

男子黄。小便自利。当与虗勞小建中湯。

附方

諸黄。瓜蒂湯主之。

黄疸。麻黄淳酒湯主之。

\chapter{驚悸吐衄下血胸滿瘀血}
%2021.3.5已據吳本重新校訂。

寸口脉動而弱。動則为驚。弱則为悸。

師曰。尺脉浮。目睛暈黄。衄必未止。暈黄去。目睛慧了。知衄今止。

又曰。從春至夏衄者。太陽。從秋至冬衄者。陽明。

衄家不可发汗。汗出必額上促急{\khaai 緊}。直視不能眴。不得眠。86

病人面无色。无寒熱。脉沈弦者。衄。{\khaai 脉}浮弱。手按之絕者。下血。煩欬者。必吐血。

夫吐血。欬逆上气。其脉數而有熱。不得卧者。死。

夫酒客。欬者。必致吐血。此因極飲過度所致也。

寸口脉弦而大。弦則为減。大則为芤。減則为寒。芤則为虗。寒虗相摶。此名为革。婦人則半產漏下。男子則亡血。
	\footnote{
		唐弘宇按:此條亦見於血痹虗勞篇。
	}

亡血不可攻其表。汗出則寒慄而振。87

病人胸滿。唇痿。舌青。口燥。其人但欲嗽水。不欲咽。无寒熱。脉微大來遲。腹不滿。其人言我滿。为有瘀血。

病者如熱狀。煩滿。口乾燥而渴。其脉反无熱。此为陰狀。是瘀血也。当下之。
	\footnote{
		本條下原有「火邪者」一條,此條与本章无関,故移至《奔豚气吐膿驚怖火邪》篇中。
	}

心下悸者。半夏麻黄丸主之。

吐血不止者。柏枼湯主之。

下血。先見血。後見便。此近血也。先見便。後見血。此遠血也。遠血。黄土湯主之。近血。赤小豆当歸散主之。

附方

治心气不足。吐血。衄血。瀉心湯。

\chapter{嘔吐噦下利}

夫嘔家有癰膿。不可治嘔。膿{\sungii 𥁞}自愈。376

先嘔卻渴者。此为欲觧。先渴卻嘔者。为水停心下。此屬飲家。

嘔家本渴。今反不渴者。以心下有支飲故也。此屬支飲。

問曰。病人脉數。數为熱。当消穀引食。而反吐者。何也。\\
師曰。以发其汗。令陽微。膈气虗。脉乃數。數为客熱。不能消穀。胃中虗冷。故吐也。脉弦者虗也。胃气无餘。朝食暮吐。變为胃反。寒在於上。醫反下之。令脉反弦。故名曰虗。
	\footnote{
		「令脉」吳本作「今脉」。
	}

寸口脉微而數。微則无气。无气則榮虗。榮虗則血不足。血不足則胸中冷。

趺陽脉浮而濇。浮則为虗。濇則傷脾。脾傷則不磨。朝食暮吐。暮食朝吐。宿穀不化。名曰胃反。脉緊而濇。其病難治。
	\footnote{
		吳本此條与上條合为一條。
	}

病人欲吐者。不可下之。

噦而腹滿。視其前後。知何部不利。利之即愈。

嘔而胸滿者。吳茱萸湯主之。

乾嘔。吐涎沫。頭痛者。吳茱萸湯主之。378

嘔而腸鳴。心下痞者。半夏瀉心湯主之。

乾嘔而利者。黄芩加半夏生薑湯主之。

諸嘔吐。穀不得下者。小半夏湯主之。

嘔吐。而病在膈上。後思水者觧。急与之。思水者。豬苓散主之。

嘔而脉弱。小便復利。身有微熱。見厥者。難治。四逆湯主之。377

嘔而发熱者。小柴胡湯主之。379

胃反。嘔吐者。大半夏湯主之。

食已即吐者。大黄甘草湯主之。

胃反。吐而渴欲飲水者。茯苓澤瀉湯主之。

吐後。渴欲得飲而貪水者。文蛤湯主之。兼主微風。脉緊。頭痛。

乾嘔。吐{\sungii 𠱘}。吐涎沫。半夏乾薑散主之。

病人胸中似喘不喘。似嘔不嘔。似噦不噦。徹心中憒憒然无奈者。生薑半夏湯主之。

乾嘔。噦。若手足厥{\khaai 冷}者。橘皮湯主之。

噦{\sungii 𠱘}者。橘皮竹茹湯主之。

夫六腑气絕於外者。手足寒。上气。腳縮。五臓气絕於内者。利不禁。下甚者。手足不仁。

下利。脉沈弦者。下重。脉大者。为未止。脉微弱數者。为欲自止。雖发熱。不死。

下利。手足厥{\khaai 冷}。无脉。{\khaai 当灸其厥陰。}灸之不温{\khaai 而脉不還}。反微喘者死。少陰負趺陽者为順。362

下利。有微熱而渴。脉弱者。今自愈。

下利。脉數。有微熱。汗出者。自愈。設{\khaai 脉}復緊。为未觧。361

下利。脉數而渴者。今自愈。設不差。必清膿血。以有熱故也。

下利。脉反弦。发熱。身汗者。自愈。

下利气者。当利其小便。

下利。寸脉反浮數。尺中自濇者。必清膿血。

下利清穀。不可攻其表。汗出必胀滿。

下利。脉沈而遲。其人面少赤。身有微熱。下利清穀者。必鬱冒。汗出而觧。其人必微厥。所以然者。其面戴陽。下虗故也。366
	\footnote{
		「必微厥」鄧本作「必微熱」。
	}

下利後。脉絕。手足厥{\khaai 冷}。晬时脉還。手足温者生。不還{\khaai 不温}者死。368

下利。腹胀滿。身体疼痛者。先温其裏。乃攻其表。温裏宜四逆湯。攻表宜桂枝湯。372

下利。三部脉皆平。按之心下堅者。急下之。宜大承湯。

下利。脉遲而滑者。実也。利未欲止。急下之。宜大承气湯。

下利。脉反滑{\khaai 者}。当有所去。下乃愈。宜大承气湯。

下利已差。至其时復发者。此为病不{\sungii 𥁞}。当{\khaai 復}下之。宜{\khaai 大}承气湯。

下利。譫語者。有燥屎也。宜{\khaai 小}承气湯。374

下利。便膿血者。桃花湯主之。

熱利下重者。白頭翁湯主之。371

下利後更煩。按之心下濡者。为虗煩也。栀子{\khaai 豉}湯主之。375

下利清穀。裏寒外熱。汗出而厥。通脉四逆湯主之。370

下利。肺痛。紫参湯主之。

气利。訶梨勒散主之。

附方

小承气湯。治大便不通。噦。數譫語。

乾嘔。下利。黄芩湯主之。{\scriptsize 玉函經云人参黄芩湯}

\chapter{瘡癰腸癰浸淫}
%2021.3.5已據吳本重新校訂。

諸浮數脉。應当发熱。而反洒淅惡寒。若有痛処。当发其癰。

師曰。諸癰腫。欲知有膿无膿。以手掩腫上。熱者为有膿。不熱者为无膿。

腸癰之为病。其身甲錯。腹皮急。按之濡。如腫狀。腹无積聚。身无熱。脉數。此为腸内有{\khaai 癰}膿。薏苡{\khaai 仁}附子敗醬散主之。
	\footnote{
		「腸内有癰膿」吳本作「腸内有膿」,鄧本作「腹内有癰膿」。
	}

腸癰者。少腹腫痞。按之即痛如淋。小便自調。时时发熱。自汗出。復惡寒。脉遲緊者。膿未成。可下之。当有血。脉洪數者。膿已成。不可下也。大黄牡丹湯主之。

問曰。寸口脉浮微而濇。法当亡血。若汗出。設不汗者云何。\\
答曰。若身有瘡。被刀器所傷。亡血故也。

病金瘡。王不留行散主之。
	\footnote{
		吳本此條与上條合为一條。
	}

排膿散。

排膿湯。

浸淫瘡。從口流向四肢者。可治。從四肢流來入口者。不可治。黄連粉主之。{\scriptsize 方未見。}
	\footnote{
		「黄連粉主之」鄧本作「浸淫瘡黄連粉主之」單獨列为一條。
	}

\chapter{趺蹶手指臂脛轉筋狐疝蛔虫}

師曰。病者趺蹶。其人但能前。不能卻。刺腨入二寸。此太陽經傷也。

病人常以手指臂脛動。此人身体瞤瞤者。藜蘆甘草湯主之。{\scriptsize 方未見。}
	\footnote{
		「臂脛動」鄧本作「臂腫動」。
	}

轉筋之为病。其人臂腳直。脉上下行。微弦。轉筋入腹者。雞屎白散主之。

陰狐疝气者。偏有小大。时时上下。蜘蛛散主之。

問曰。病腹痛。有虫。其脉何以別之。\\
師曰。腹中痛。其脉当沈弦。反洪大。故有蛔虫。
	\footnote{「沈弦」吳本、鄧本均作「沈若弦」,編者改。}

蛔虫之为病。令人吐涎。心痛。发作有时。毒藥不止。甘草粉蜜湯主之。

蛔厥者。其人当吐蛔。今病者靜。而復时煩。此为臓寒。蛔上入膈。故煩。須臾復止。得食而嘔。又煩者。蛔聞食臭出。其人常自吐蛔。蛔厥者。烏梅丸主之。338

\chapter{雜療方
	\footnote{
		唐弘宇按:觀此章文法,不像仲景原文,本來已經刪除,後來我讀到《諸病源{\sungii 𠊱}論》卷六觧散病諸{\sungii 𠊱}第一寒食散发{\sungii 𠊱}引皇甫云:「然寒食藥者。世末知焉。或言華佗。或曰仲景。考之於実。佗之精微。方類單省。而仲景經有矦氏黑散。紫石英方。皆數種相出入。節度略同。然則寒食草石二方。出自仲景。非佗也。且佗之为治。或刳斷腸胃。滌洗五臟。不純任方也。仲景雖精。不及於佗。至於審方物之{\sungii 𠊱}。論草石之宜。亦妙絕乑醫。」據此段引文可知本章中的「紫石寒食散」是仲景方,所以为了穩妥起見,將本章保留。
	}
}

%宣通五臟虗熱。四时加減柴胡飲子。

%長服訶梨勒丸。

%三物備急丸。

%備急散。治人卒上气。呼吸气不得下。喘逆。

%紫石寒食散。治傷寒令已愈不復。

%救卒死方。

\chapter{婦人妊娠病}
%2021年3月4日,此篇已據吳本为底本重新校訂。

師曰。脉婦人得平脉。陰脉小弱。其人渴。不能食。无寒熱。名为軀。桂枝湯主之。法六十日当有娠。設有醫治逆者。卻一月。加吐下者。則絕之。
	\footnote{
		此條鄧本与吳本差異較大,鄧本作「師曰。婦人得平脉。陰脉小弱。其人渴。不能食。无寒熱。名妊娠。桂枝湯主之。於法六十日当有此証。設有醫治逆者。卻一月。加吐下者。則絕之。」
	}

婦人妊娠。經斷三月。而得漏下。下血四十日不止。胎欲動。在於脐上。此为妊娠。六月動者。前三月經水利时。胎也。下血者。後斷三月。衃也。所以下血不止者。其癥不去故也。当下其癥。宜桂枝茯苓丸。
	\footnote{
		此條鄧本与吳本差異較大,鄧本作「婦人宿有癥病。經斷未及三月。而得漏下不止。胎動在脐上者。为癥痼害。妊娠六月動者。前三月經水利时。胎下血者。後斷三月。衃也。所以血不止者。其癥不去故也。当下其癥。桂枝茯苓丸主之。」
	}

婦人懷娠六月七月。脉弦。发熱。其胎愈胀。腹痛。惡寒者。少腹如扇之狀。所以然者。子臓開故也。当以附子湯温其臓。{\scriptsize 方未見。}
	\footnote{
		吳本本條末有「方未見」小字註釋。
		「六月七月」同吳本、《脉經》,鄧本作「六七月」。
		「愈胀」同鄧本,吳本、《脉經》作「逾腹」。
	}

師曰。婦人有漏下者。有半產後因續下血都不絕者。有妊娠下血者。假令妊娠腹中痛。为胞阻。膠艾湯主之。

婦人懷娠。腹中㽲痛。当歸芍藥散主之。

婦人妊娠。嘔吐不止。乾薑人参半夏丸主之。

婦人妊娠。小便難。飲食如故。当歸貝母苦参丸主之。

婦人妊娠。有水气。身重。小便不利。洒淅惡寒。起即頭眩。葵子茯苓散主之。

婦人妊娠。宜服当歸散。
	\footnote{
		「宜服」鄧本作「宜常服」。
	}

附方

妊娠養胎。白术散。

婦人傷胎。懷身。腹滿。不得小便。從腰以下重。如有水气狀。懷身七月。太陰当養不養。此心气実。当刺瀉勞宮及関元。小便利則愈。
	\footnote{
		「傷胎」同鄧本,吳本作「傷寒」。
		「小便利」鄧本作「小便微利」。
	}

\chapter{婦人產後病}
%2021年3月4日,此篇已據吳本为底本重新校訂。

問曰。新產婦人有三病。一者病痙。二者病鬱冒。三者大便難。何谓也。\\
師曰。新產血虗。多汗出。喜中風。故令病痙。\\
{\khaai 問曰。}何故鬱冒。\\
{\khaai 師曰。}亡血復汗。寒多。故令鬱冒。\\
{\khaai 問曰。}何故大便難。\\
{\khaai 師曰。}亡津液。胃燥。故大便難。

產婦鬱{\khaai 冒}。其脉微弱。不能食。大便反堅。但頭汗出。所以然者。血虗而厥。厥而必冒。冒家欲觧。必大汗出。以血虗下厥。孤陽上出。故但頭汗出。所以產婦喜汗出者。亡陰血虗。陽气獨盛。故当汗出。陰陽乃復。所以便堅者。嘔。不能食也。小柴胡湯主之。病觧。能食。七八日。更发熱者。此为胃熱气実。大承气湯主之。
	\footnote{
		「鬱冒」同鄧本,吳本无「冒」字。
		「所以便堅者」鄧本作「大便堅」。
		「不能食也」鄧本无「也」字。
		從「病觧能食」一下,吳本、鄧本均獨立作一條。但根據文意,当与上條合为一條。
	}

婦人產後。腹中㽲痛。当歸生薑羊肉湯主之。并治腹中寒疝。虗勞不足。

婦人產後。腹痛。煩滿。不得卧。枳実芍藥散主之。

師曰。產婦腹痛。法当与枳実芍藥散。假令不愈者。此为腹中有乾血著脐下。与下瘀血湯服之。{\khaai 亦}主經水不利若瘀血。
	\footnote{
		「亦主經水不利若瘀血」吳本作「主經水不利若瘀血」,鄧本作「主經水不利」且为小字註釋。
	}

婦人產後七八日。无太陽証。少腹堅痛。此惡露不{\sungii 𥁞}。不大便四五日。趺陽脉微実。再倍其人发熱。日晡所煩躁者。不食。食即譫語。利之即愈。宜大承气湯。熱在裏。結在膀胱也。
	\footnote{
		此條鄧本与吳本差異較大,鄧本作「產後七八日。无太陽証。少腹堅痛。此惡露不{\sungii 𥁞}。不大便。煩躁。发熱。切脉微実。再倍发熱。日晡时煩躁者。不食。食則譫語。至夜即愈。宜大承气湯主之。熱在裏。結在膀胱也。」
	}

婦人產得風。續之數十日不觧。頭微痛。惡寒。时时有熱。心下堅。乾嘔。汗出。雖久。陽旦証續在耳。可与陽旦湯。
	\footnote{
		「婦人產得風」鄧本作「產後風」。
		「心下堅」鄧本作「心下悶」。
	}

婦人產後。中風。发熱。面{\khaai 正}赤。喘而頭痛。竹枼湯主之。

婦人產中虗。煩亂。嘔{\sungii 𠱘}。安中益气。竹皮大丸主之。
	\footnote{
	「產中虗」同《脉經》,吳本、鄧本作「乳中虗」。
	}

婦人產後。下利。虗極。白頭翁加甘草阿膠湯主之。

附方

婦人多在草蓐得風。四肢苦煩熱。皆自发露所为。頭痛者。与小柴胡湯。頭不痛。但煩者。与三物黄芩湯。
	\footnote{
		此條鄧本与吳本差異較大,鄧本作「千金三物黄芩湯。治婦人在草蓐自发露得風。四肢苦煩熱。頭痛者。与小柴胡湯。頭不痛。但煩者。此湯主之。」
	}

治婦人產後。虗羸不足。腹中刺痛不止。吸吸少气。或苦少腹拘急攣痛引腰背。不能食飲。產後一月。日得服四五剂为善。令人强壯。内補当歸建中湯。
	\footnote{
		此條鄧本与吳本差異較大,鄧本作「千金内補当歸建中湯。治婦人產後。虗羸不足。腹中刺痛不止。吸吸少气。或苦少腹中急摩痛引腰背。不能食飲。產後一月。日得服四五剂为善。令人强壯。」
	}

\chapter{婦人雜病}
%2021.3.10此篇已據吳本重新校訂。

婦人中風七八日。續得寒熱。发作有时。經水適斷。此为熱入血室。其血必結。故使如瘧狀。发作有时。小柴胡湯主之。144

婦人傷寒。发熱。經水適來。晝日明瞭。暮則譫語。如見鬼狀。此为熱入血室。无犯胃气及上二焦。必自愈。145
	\footnote{
		鄧本「无犯」上有「治之」二字。
	}

婦人中風。发熱。惡寒。經水適來。得之七八日。熱除。脉遲。身涼。胸脇下滿。如結胸狀。其人譫語。此为熱入血室。当刺期門。隨其{\khaai 虗}実而取之。143
	\footnote{
		「身涼」同趙本、《千金翼》卷九第七,吳本、鄧本均作「身涼和」。
	}

陽明病。下血。譫語者。此为熱入血室。但頭汗出者。当刺期門。隨其実而瀉之。濈然汗出則愈。216

婦人咽中如有炙臠。半夏厚朴湯主之。

婦人臓躁。喜悲傷。欲哭。象如神靈所作。數欠伸。甘草小麥大棗湯主之。
	\footnote{
		「臟躁」同鄧本,吳本作「臟燥」。
	}

婦人吐涎沫。醫反下之。心下即痞。当先治其吐涎沫。宜小青龙湯。涎沫止。乃治痞。宜瀉心湯。
	\footnote{
		吳本「小青龙湯」下有「方見肺癰中」小字註釋,「瀉心湯」下有「方見驚悸中」小字註釋。
	}

婦人之病。因虗稍入結气。为諸經水斷絕。至有歷年。血寒積結。胞門寒傷。經絡凝堅。在上嘔吐涎唾。久成肺癰。形体損分。在中盤結。繞脐寒疝。或兩脇疼痛。与臓相連。或結熱在中。痛在関元。脉數。无瘡。肌若魚鱗。时著男子。非止女身。在下未多。經{\sungii 𠊱}不勻。令陰掣痛。少腹惡寒。或引腰脊。下根气街。气衝急痛。膝脛疼煩。或奄忽眩冒。狀如厥癲。或有憂惨。悲傷多嗔。此皆帶下。非有鬼神。久則羸瘦。脉虗多寒。三十六病。千變万端。審脉陰陽。虗実緊弦。行其針藥。治危得安。其雖同病。脉各異源。子当辯記。勿谓不然。
	\footnote{
		「稍入」同吳本,鄧本作「積冷」。
	}

問曰。婦人年五十所。病下血。數十日不止。暮即发熱。少腹裏急。腹滿。手掌煩熱。唇口乾燥。何也。\\
師曰。此病屬帶下。何以故。曾經半產。瘀血在少腹不去。何以知之。其証唇口乾燥。故知之。当以温經湯主之。
	\footnote{
		「病下血」諸本均作「病下利」,編者改。
		吳本「少腹裏急」下有「痛」字。
	}

婦人帶下。經水不利。少腹滿痛。經一月再見者。土瓜根散主之。

寸口脉弦而大。弦則为減。大則为芤。減則为寒。芤則为虗。寒虗相摶。脉即名为革。婦人則半產漏下。旋覆花湯主之。

婦人陷經。漏下黑不觧。膠薑湯主之。
	\footnote{
		臣億等按。諸本无膠薑湯{\khaai 方}。恐是前妊娠中膠艾湯也。
	}

婦人少腹滿如敦狀。小便微難而不渴。生後者。此为水与血并結在血室也。大黄甘遂湯主之。
	\footnote{
		「如敦狀」同鄧本,吳本作「如敦敦狀」。
		吳本「敦」字下有「音堆」小字註釋,鄧本无。
	}

婦人經水不利。抵当湯主之。
	\footnote{
		鄧本「經水不利」下有「下」字。
	}

婦人經水閉不利。臓堅癖不止。中有乾血。下白物。礬石丸主之。

治婦人六十二種風。兼主腹中血气刺痛。紅藍花酒。

婦人腹中諸疾痛。当歸芍藥散主之。

婦人腹中痛。小建中湯主之。

問曰。婦人病。食飲如故。煩熱不得卧。而反倚息者。何也。\\
師曰。此病轉胞。不得尿也。以胞系了戾。故致此病。但利小便則愈。宜腎气丸。
	\footnote{
		「宜腎气丸」後,鄧本有「主之」二字,吳本有「以中有茯苓故也」七字。
	}

温陰中坐藥。蛇床子散。

少陰脉滑而數者。陰中即生瘡。
	\footnote{
		吳本此條前有「師曰」二字。
	}

婦人陰中蝕瘡爛。狼牙湯洗之。

胃气下泄。陰吹而正喧。此穀气之実也。膏髮煎導之。

小兒疳虫蝕齒方。

%\part{輯佚}

\part{衍文}

問曰。証象陽旦。按法治之而增劇。厥逆。咽中乾。兩脛拘急而讝語。\\
師曰。言夜半手足当温。兩腳当伸。後如師言。何以知此。\\
答曰。寸口脉浮而大。浮{\khaai 則}为風。大{\khaai 則}为虗。風則生微熱。虗則兩脛攣。病形象桂枝。因加附子参其間。增桂令汗出。附子温經。亡陽故也。厥逆。咽中乾。煩燥。陽明内結。讝語。煩亂。更飲甘草乾薑湯。夜半陽气還。兩足当熱。脛尚微拘急。重与芍藥甘草湯。尔乃脛伸。以承气湯微溏。則止其讝語。故知病可愈。30

太陽病二日。反躁。反熨其背。而大汗出。大熱入胃。胃中水竭。躁煩。必发譫語。十餘日。振慄。自下利者。此为欲觧也。故其汗從腰以下不得汗。欲小便不得。反嘔。欲失溲。足下惡風。大便鞕。小便当數。而反不數及不多。大便已。頭卓然而痛。其人足心必熱。穀气下流故也。110
	\footnote{
		趙本「大熱入胃」下有「一作二日内燒瓦熨背大汗出火气入胃」小字註釋。
	}

太陽病。中風。以火劫发汗。邪風被火熱。血气流溢。失其常度。兩陽相熏灼。其身发黄。陽盛則欲衄。陰虗{\khaai 則}小便難。陰陽俱虗竭。身体則枯燥。但頭汗出。齐頸而還。腹滿微喘。口乾咽爛。或不大便。久則譫語。甚者至噦。手足躁擾。捻衣摸床。小便利者。其人可治。111

脉按之來緩。时一止復來者。名曰結。又脉來動而中止。更來小數。中有還者反動。名曰結。陰也。脉來動而中止。不能自還。因而復動者。名曰代。陰也。得此脉者。必難治。178

問曰。上工治未病。何也。\\
師曰。夫治未病者。見肝之病。知肝傳脾。当先実脾。四季脾王不受邪。即勿補之。中工不曉相傳。見肝之病。不觧実脾。惟治肝也。\\
夫肝之病。補用酸。助用焦苦。益用甘味之藥調之。酸入肝。焦苦入心。甘入脾。脾能傷腎。腎气微弱則水不行。水不行則心火气盛。心火气盛則傷肺。肺被傷則金气不行。金气不行則肝气盛。故実脾則肝自愈。此治肝補脾之要妙也。肝虛則用此法。実則不在用之。經曰。虛虛実実。補不足。損有餘。是其義也。餘藏准此。\\
夫人稟五常。因風气而生長。風气雖能生萬物。亦能害萬物。如水能浮舟。亦能覆舟。若五臟元真通暢。人即安和。客气邪風。中人多死。千般災難。不越三條。一者。經絡受邪。入臟腑。为内所因也。二者。四肢九竅。血脉相傳。壅塞不通。为外皮膚所中也。三者。房室金刃虫獸所傷。以此詳之。病由都{\sungii 𥁞}。若人能養慎。不令邪風干忤經絡。適中經絡。未流傳腑臟。即醫治之。四肢才覺重滯。即導引吐納鍼灸膏摩。勿令九竅閉塞。更能无犯王法。禽獸災傷。房室勿令竭之。服食節其冷熱苦酸辛甘。不遺形体有衰。病則无由入其腠理。腠者。是三焦通會元真之處。为血气所注。理者。是皮膚臟腑之文理也。

\cleardoublepage
\mbox{}
\cleardoublepage

\end{document}

%字形
%
%実为与洒虗术体气処无当脉沈温別卧麥盖内弃泪时𥁞觧发强枼栀鲜个
%覺學舉
%納約結細緣緩縮純紅絕絞縱經絡續綱終繞總綠紙絲綿
%証許譫語設諸診訴謝辯訣記談諺謬誠靄諦訓認譚識説調論議誤訶
%齐脐剂蛴
%腫種
%針鑠鎮鐘銓錯銖鋸錦鋭銅
%門瞤潤悶闔聞問閉開関間闕癇爛
%輿輗輒暈渾漸連軺載暫陳運轉輕輩
%尔弥称
%飲飢飪飽餘蝕餅饐
%龙聋
%万厉蛎
%帶滯
%黄横
%长胀张
%参惨
%專轉傳
%𠊱矦㗋
%蛊虫
%濇喪
